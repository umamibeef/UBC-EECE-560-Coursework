\documentclass[10pt, oneside, letterpaper]{article}
\usepackage[margin=1in]{geometry}
\usepackage[english]{babel}
\usepackage[utf8]{inputenc}
\usepackage{color}
\definecolor{mygreen}{rgb}{0,0.6,0}
\definecolor{mygray}{rgb}{0.5,0.5,0.5}
\definecolor{mymauve}{rgb}{0.58,0,0.82}
\usepackage{listings}
\lstset{
  backgroundcolor=\color{white}, % choose the background color
  basicstyle=\footnotesize\ttfamily, % size of fonts used for the code
  breaklines=true, % automatic line breaking only at whitespace
  frame=single, % add a frame
  captionpos=b, % sets the caption-position to bottom
  commentstyle=\color{mygreen}, % comment style
  escapeinside={\%*}{*)}, % if you want to add LaTeX within your code
  keywordstyle=\color{blue}, % keyword style
  stringstyle=\color{mymauve}, % string literal style
}
\usepackage{enumitem}
\usepackage{blindtext}
\usepackage{datetime2}
\usepackage{fancyhdr}
\usepackage{amsmath}
\usepackage{mathtools}
\usepackage{float}
\usepackage{pgf}
% \usepackage{layouts}
% \printinunitsof{in}\prntlen{\textwidth}
\title{Basic Circuit Discretization}
\author{Assignment 1a}
\date{Due: 2021/01/22}

\pagestyle{fancy}
\setlength{\headheight}{23pt}
\setlength{\parskip}{1em}
\fancyhf{}
\chead{Assignment 1a \\ Basic Circuit Discretization}
\rhead{Michel Kakulphimp \\ Student \#63542880}
\lhead{EECE560 \\ UBC MEng}
\cfoot{\thepage}

\begin{document}
\maketitle
\thispagestyle{fancy}

\section{Introduction}
In this first assignment, we are analyzing the transient behaviour of a series RL circuit as we close its main switch. Since this circuit and its constituent components can be modeled as an easy-to-solve first order differential equation, this allows us to simultaneously solve for its continuous time solution as well as apply some discretization techniques to enable solving the circuit in a step-by-step manner. Comparing the two allows us to analyze how well each discretization technique is able to approximate the true solution.

The following sections will discuss the derivation of the continuous time function as well as the derivation of the discretized functions. The trapezoidal and backward Euler approximations of a differential equation will be employed for this assignment. Following the derivations, we will plot different the approximations and compare their performance with respect to different time steps as well to each other. The continuous solutions will be overlayed to understand how well the approximations come to solving the circuit's solution.

Finally, a discussion of the results as well as improvements will be proposed. A full code listing of the code used in this project is provided as well for review.
\section{Setup}
The following sections go through the derivation of the different solutions used to produce the final plots for this assignment.
\subsection{Homogeneous Plus Steady-State Solutions}
In this section, we will derive the homogeneous plus steady-state solutions for $i(t)$ and $v_L(t)$. These solutions will plot the continuous time value for each of the quantities, and will be used as a baseline comparison to see how well the approximations match the ground truth. Using KVL, the circuit can be represented by the following equation:
\begin{alignat}{2}
10 - Ri(t) & = L\frac{di(t)}{d(t)}
\intertext{Solving this first order differential equation will provide the continuous time system solution for $i(t)$ and $v_L(t)$. First, we will derive the continuous time system solution for $i(t)$in terms of homogeneous plus steady-state equations:}
\frac{10 - Ri(t)}{L} & = \frac{di(t)}{dt} \\
\frac{dt}{L} & = \frac{di(t)}{10-Ri(t)} \\
\frac{1}{L}\int{}\,dt & = \int{}\frac{1}{10-R\,i(t)}di(t) \\
\frac{1}{L}t + K & = -\frac{1}{R}\ln{|10-R\,i(t)|} \\
e^{-\frac{R}{L}-RK} & = 10 - R\,i(t) \\
i(t) & = \frac{10}{R} - A\,e^{-\frac{R}{L}t} & A = e^{-RK} \\
\Aboxed{i(t) & = \frac{10}{R} - A\,e^{-\frac{R}{L}t}}
\intertext{We know from our initial conditions that $i(0) = 0$; this allows us to solve for the steady-state solution by solving for the constant $A$.}
0 & = \frac{10}{R} - A\,e^{-\frac{R}{L}(0)} \\
A & = \frac{10}{R}
\intertext{This gives us our final, continuous time solution for $i(t)$ as follows:}
i(t) & =\frac{10}{R} - \frac{10}{R}\,e^{-\frac{R}{L}t} & i(t)=0 \\
\Aboxed{i(t) & =\frac{10}{R} - \frac{10}{R}\,e^{-\frac{R}{L}t} & i(t)=0}
\intertext{Obtaining our continuous time $v_L(t)$ relies on the formula for the voltage across an inductor. Since we have obtained our $i(t)$, we can take its derivative with respect to time to obtain the equation's $\frac{di(t)}{dt}$ term.}
\frac{di(t)}{dt} & = A\,\frac{R}{L}\,e^{-\frac{R}{L}t}
\intertext{This derivative can then be substituted into the equation for the voltage across our inductor:}
v_L(t) & = L\,\frac{di(t)}{dt} \\
\Aboxed{v_L(t) & = A\,R\,e^{-\frac{R}{L}t}}
\intertext{Knowing our initial conditions, we obtain the following steady-state equation for $v_L(t)$}
\Aboxed{v_L(t) & = 10\,e^{-\frac{R}{L}t}}
\end{alignat}
\subsection{Isolating Components for Discretization}
In order to apply the trapezoidal and backward Euler discretizations used by the step-by-step solution, we must isolate the integration of $i(t)dt$. This is the component that will be approximated by the following discretization techniques:
\begin{alignat}{2}
\intertext{Trapezoidal:}
\int_{t-\Delta{}t}^{t}i(t)dt & \simeq \frac{i(t)+i(t-\Delta{}t)}{2}\Delta{}t
\intertext{Backward Euler:}
\int_{t-\Delta{}t}^{t}i(t)dt & \simeq i(t)\Delta{}t
\end{alignat}
The following steps will manipulate our original KVL relationship into a differential equation that can be discretized, isolating the integration of $i(t)dt$.
\begin{alignat}{2}
10 - Ri(t) & = L\frac{di(t)}{d(t)} \\
\frac{10}{L} - \frac{R}{L}i(t) & = \frac{di(t)}{dt} \\
\frac{10}{L}dt - \frac{R}{L}i(t)dt & = di(t) \\
\frac{10}{L}\int_{t-\Delta{}t}^{t}dt - \frac{R}{L}\int_{t-\Delta{}t}^{t}i(t)dt & = i(t) - i(t-\Delta{}t) \\
\frac{10}{L}[t - (t-\Delta{}t)] - \frac{R}{L}\int_{t-\Delta{}t}^{t}i(t)dt & = i(t) - i(t-\Delta{}t) \\
- \frac{R}{L}\int_{t-\Delta{}t}^{t}i(t)dt & = i(t) - i(t-\Delta{}t) - \frac{10}{L}\Delta{}t \\
\Aboxed{\int_{t-\Delta{}t}^{t}i(t)dt & = -\frac{L}{R}i(t) + \frac{L}{R}i(t-\Delta{}t) + \frac{10}{R}\Delta{}t}
\end{alignat}
With $\int_{t-\Delta{}t}^{t}i(t)dt$ isolated, we can now apply our discretization techniques to obtain our step-by-step solution for the current. Similarly, discretizing $v_L(t)$ can be performed by manipulating the fundamental equation for the voltage across an inductor:
\begin{alignat}{2}
v_L(t) & = L\frac{di(t)}{dt} \\
\int_{t-\Delta{}t}^{t}v_L(t) & = Ldi(t) \\
\Aboxed{\int_{t-\Delta{}t}^{t}v_L(t) & = L[i(t) - i(t-\Delta{}t)]}
\end{alignat}
\subsection{Trapezoidal Discretization}
The following is the derivation of the step-by-step solution for $i(t)$ and $v_L(t)$ using the trapezoidal discretization technique.
\begin{alignat}{2}
\intertext{Current $i(t)$:}
\frac{i(t)+i(t-\Delta{}t)}{2}\Delta{}t & \simeq -\frac{L}{R}i(t) + \frac{L}{R}i(t-\Delta{}t) + \frac{10}{R}\Delta{}t \\
i(t)\Delta{}t + i(t-\Delta{}t)\Delta{}t & \simeq \frac{2L}{R}i(t-\Delta{}t)-\frac{2L}{R}i(t) + \frac{20}{R}\Delta{}t \\
i(t)\Delta{}t + \frac{2L}{R}i(t) & \simeq \frac{2L}{R}i(t-\Delta{}t) - i(t-\Delta{}t)\Delta{}t+20\Delta{}t \\
\Aboxed{i(t) & \simeq \frac{i(t-\Delta{}t)[2L-R\Delta{}t] + 20\Delta{}t}{R\Delta{}t + 2L}}
\intertext{Voltage $v_L(t)$:}
\frac{v_L(t)+v_L(t-\Delta{}t)}{2}\Delta{}t & \simeq L[i(t) - i(t-\Delta{}t)] \\
\Aboxed{v_L(t) & \simeq \frac{2L}{\Delta{}t}[i(t) - i(t-\Delta{}t)] - v_L(t-\Delta{}t)}
\end{alignat}
\subsection{Backward Euler Discretization}
The following is the derivation of the step-by-step solution for $i(t)$ and $v_L(t)$ using the backward Euler discretization technique.
\begin{alignat}{2}
\intertext{Current $i(t)$:}
i(t)\Delta{}t & \simeq -\frac{L}{R}i(t) + \frac{L}{R}i(t-\Delta{}t) + \frac{10}{R}\Delta{}t \\
i(t)R\Delta{}t + Li(t) & \simeq  Li(t - \Delta{t}) + 10\Delta{}t \\
i(t)[R\Delta{}t + L] & \simeq Li(t - \Delta{}t) + 10\Delta{}t \\
\Aboxed{i(t) & \simeq \frac{Li(t - \Delta{}t) + 10\Delta{}t}{R\Delta{}t + L}}
\intertext{Voltage $v_L(t)$:}
v_L(t)\Delta{}t & \simeq L[i(t) - i(t-\Delta{}t)] \\
\Aboxed{v_L(t) & \simeq \frac{L[i(t) - i(t-\Delta{}t)]}{\Delta{}t}}
\end{alignat}
\section{Simulation}
With the equations derived, we can now plot our results to see how they perform with different parameters. The assignment requests the following solutions to be generated using a computer program:
\begin{enumerate}[label=\alph*)]
  \item Using the trapezoidal rule with $\Delta{}t_1 = 0.1ms$
  \item Using the backward Euler rule with $\Delta{}t_1 = 0.1ms$
  \item Using the trapezoidal rule with $\Delta{}t_2 = 0.8ms$
  \item Using the backward Euler rule with $\Delta{}t_2 = 0.8ms$
\end{enumerate}
In all plots, the continuous solution is plotted as a dotted line to show what the exact solution should be. This provides a baseline to compare the performance of the different approximations. Figure \ref{trap_approx}. shows the performance of the Trapezoidal approximation using time steps of $0.1ms$ and $0.8ms$. Figure \ref{back_approx}. shows the performance of the backward Euler approximation using time steps of $\Delta{}t_1 = 0.1ms$ and $\Delta{}t_2 = 0.8ms$. Finally, Figure \ref{approx_comp}. compares both approximation techniques using the time step of $\Delta{}t_2 = 0.8ms$.

As expected, the smaller the time step, the closer the step-by-step solutions approximate the real solution. This makes sense because as the $\Delta{}t$ becomes smaller, the closer it becomes to approximating the integration. What was unexpected, however, is the backward Euler solution more closely approximating the solution numerically versus the trapezoidal one. Intuitively, the trapezoidal solution should approximate the solution better due to its inclusion of a triangle area above the square below. The trapezoidal approximation does seem to follow the shape better than the backward Euler one, however. The backward Euler approximation can be seen to cross the true solution towards the end of the simulation in Figure \ref{approx_comp}. whereas the trapezoidal solution asymptotically approaches the true solution. At small time steps the trapezoidal solution is likely the better one.
\begin{figure}[H]
    \begin{center}
        %% Creator: Matplotlib, PGF backend
%%
%% To include the figure in your LaTeX document, write
%%   \input{<filename>.pgf}
%%
%% Make sure the required packages are loaded in your preamble
%%   \usepackage{pgf}
%%
%% Figures using additional raster images can only be included by \input if
%% they are in the same directory as the main LaTeX file. For loading figures
%% from other directories you can use the `import` package
%%   \usepackage{import}
%% and then include the figures with
%%   \import{<path to file>}{<filename>.pgf}
%%
%% Matplotlib used the following preamble
%%
\begingroup%
\makeatletter%
\begin{pgfpicture}%
\pgfpathrectangle{\pgfpointorigin}{\pgfqpoint{6.500000in}{8.000000in}}%
\pgfusepath{use as bounding box}%
\begin{pgfscope}%
\pgfsetbuttcap%
\pgfsetroundjoin%
\definecolor{currentfill}{rgb}{1.000000,1.000000,1.000000}%
\pgfsetfillcolor{currentfill}%
\pgfsetlinewidth{0.000000pt}%
\definecolor{currentstroke}{rgb}{1.000000,1.000000,1.000000}%
\pgfsetstrokecolor{currentstroke}%
\pgfsetdash{}{0pt}%
\pgfpathmoveto{\pgfqpoint{0.000000in}{0.000000in}}%
\pgfpathlineto{\pgfqpoint{6.500000in}{0.000000in}}%
\pgfpathlineto{\pgfqpoint{6.500000in}{8.000000in}}%
\pgfpathlineto{\pgfqpoint{0.000000in}{8.000000in}}%
\pgfpathclose%
\pgfusepath{fill}%
\end{pgfscope}%
\begin{pgfscope}%
\pgfsetbuttcap%
\pgfsetroundjoin%
\definecolor{currentfill}{rgb}{1.000000,1.000000,1.000000}%
\pgfsetfillcolor{currentfill}%
\pgfsetlinewidth{0.000000pt}%
\definecolor{currentstroke}{rgb}{0.000000,0.000000,0.000000}%
\pgfsetstrokecolor{currentstroke}%
\pgfsetstrokeopacity{0.000000}%
\pgfsetdash{}{0pt}%
\pgfpathmoveto{\pgfqpoint{0.580766in}{4.462900in}}%
\pgfpathlineto{\pgfqpoint{6.192411in}{4.462900in}}%
\pgfpathlineto{\pgfqpoint{6.192411in}{7.656023in}}%
\pgfpathlineto{\pgfqpoint{0.580766in}{7.656023in}}%
\pgfpathclose%
\pgfusepath{fill}%
\end{pgfscope}%
\begin{pgfscope}%
\pgfpathrectangle{\pgfqpoint{0.580766in}{4.462900in}}{\pgfqpoint{5.611646in}{3.193122in}} %
\pgfusepath{clip}%
\pgfsetbuttcap%
\pgfsetroundjoin%
\pgfsetlinewidth{1.003750pt}%
\definecolor{currentstroke}{rgb}{0.000000,0.000000,1.000000}%
\pgfsetstrokecolor{currentstroke}%
\pgfsetdash{{1.000000pt}{3.000000pt}}{0.000000pt}%
\pgfpathmoveto{\pgfqpoint{0.580766in}{4.462900in}}%
\pgfpathlineto{\pgfqpoint{0.633141in}{4.608489in}}%
\pgfpathlineto{\pgfqpoint{0.685517in}{4.747440in}}%
\pgfpathlineto{\pgfqpoint{0.737892in}{4.880055in}}%
\pgfpathlineto{\pgfqpoint{0.790267in}{5.006624in}}%
\pgfpathlineto{\pgfqpoint{0.842643in}{5.127422in}}%
\pgfpathlineto{\pgfqpoint{0.895018in}{5.242713in}}%
\pgfpathlineto{\pgfqpoint{0.947393in}{5.352746in}}%
\pgfpathlineto{\pgfqpoint{0.999769in}{5.457763in}}%
\pgfpathlineto{\pgfqpoint{1.054015in}{5.561485in}}%
\pgfpathlineto{\pgfqpoint{1.108261in}{5.660314in}}%
\pgfpathlineto{\pgfqpoint{1.162506in}{5.754479in}}%
\pgfpathlineto{\pgfqpoint{1.216752in}{5.844201in}}%
\pgfpathlineto{\pgfqpoint{1.270998in}{5.929690in}}%
\pgfpathlineto{\pgfqpoint{1.325244in}{6.011145in}}%
\pgfpathlineto{\pgfqpoint{1.381361in}{6.091367in}}%
\pgfpathlineto{\pgfqpoint{1.437477in}{6.167676in}}%
\pgfpathlineto{\pgfqpoint{1.493594in}{6.240263in}}%
\pgfpathlineto{\pgfqpoint{1.549710in}{6.309311in}}%
\pgfpathlineto{\pgfqpoint{1.607697in}{6.377124in}}%
\pgfpathlineto{\pgfqpoint{1.665684in}{6.441522in}}%
\pgfpathlineto{\pgfqpoint{1.723671in}{6.502678in}}%
\pgfpathlineto{\pgfqpoint{1.783529in}{6.562578in}}%
\pgfpathlineto{\pgfqpoint{1.843386in}{6.619367in}}%
\pgfpathlineto{\pgfqpoint{1.905114in}{6.674844in}}%
\pgfpathlineto{\pgfqpoint{1.966842in}{6.727352in}}%
\pgfpathlineto{\pgfqpoint{2.030441in}{6.778513in}}%
\pgfpathlineto{\pgfqpoint{2.094040in}{6.826856in}}%
\pgfpathlineto{\pgfqpoint{2.159509in}{6.873840in}}%
\pgfpathlineto{\pgfqpoint{2.226849in}{6.919391in}}%
\pgfpathlineto{\pgfqpoint{2.294188in}{6.962289in}}%
\pgfpathlineto{\pgfqpoint{2.363399in}{7.003777in}}%
\pgfpathlineto{\pgfqpoint{2.434479in}{7.043805in}}%
\pgfpathlineto{\pgfqpoint{2.507431in}{7.082333in}}%
\pgfpathlineto{\pgfqpoint{2.582253in}{7.119332in}}%
\pgfpathlineto{\pgfqpoint{2.658945in}{7.154781in}}%
\pgfpathlineto{\pgfqpoint{2.737508in}{7.188668in}}%
\pgfpathlineto{\pgfqpoint{2.819812in}{7.221714in}}%
\pgfpathlineto{\pgfqpoint{2.903987in}{7.253096in}}%
\pgfpathlineto{\pgfqpoint{2.991903in}{7.283454in}}%
\pgfpathlineto{\pgfqpoint{3.081689in}{7.312098in}}%
\pgfpathlineto{\pgfqpoint{3.175217in}{7.339597in}}%
\pgfpathlineto{\pgfqpoint{3.272485in}{7.365866in}}%
\pgfpathlineto{\pgfqpoint{3.373495in}{7.390839in}}%
\pgfpathlineto{\pgfqpoint{3.480116in}{7.414872in}}%
\pgfpathlineto{\pgfqpoint{3.590478in}{7.437457in}}%
\pgfpathlineto{\pgfqpoint{3.706452in}{7.458914in}}%
\pgfpathlineto{\pgfqpoint{3.828038in}{7.479151in}}%
\pgfpathlineto{\pgfqpoint{3.957106in}{7.498366in}}%
\pgfpathlineto{\pgfqpoint{4.093656in}{7.516426in}}%
\pgfpathlineto{\pgfqpoint{4.239559in}{7.533444in}}%
\pgfpathlineto{\pgfqpoint{4.394814in}{7.549280in}}%
\pgfpathlineto{\pgfqpoint{4.561293in}{7.563995in}}%
\pgfpathlineto{\pgfqpoint{4.740866in}{7.577602in}}%
\pgfpathlineto{\pgfqpoint{4.935403in}{7.590082in}}%
\pgfpathlineto{\pgfqpoint{5.148645in}{7.601492in}}%
\pgfpathlineto{\pgfqpoint{5.382464in}{7.611748in}}%
\pgfpathlineto{\pgfqpoint{5.644341in}{7.620962in}}%
\pgfpathlineto{\pgfqpoint{5.939887in}{7.629079in}}%
\pgfpathlineto{\pgfqpoint{6.190541in}{7.634472in}}%
\pgfpathlineto{\pgfqpoint{6.190541in}{7.634472in}}%
\pgfusepath{stroke}%
\end{pgfscope}%
\begin{pgfscope}%
\pgfpathrectangle{\pgfqpoint{0.580766in}{4.462900in}}{\pgfqpoint{5.611646in}{3.193122in}} %
\pgfusepath{clip}%
\pgfsetrectcap%
\pgfsetroundjoin%
\pgfsetlinewidth{1.003750pt}%
\definecolor{currentstroke}{rgb}{0.000000,0.500000,0.000000}%
\pgfsetstrokecolor{currentstroke}%
\pgfsetdash{}{0pt}%
\pgfpathmoveto{\pgfqpoint{0.580766in}{4.618662in}}%
\pgfpathlineto{\pgfqpoint{0.636882in}{4.766826in}}%
\pgfpathlineto{\pgfqpoint{0.692999in}{4.907763in}}%
\pgfpathlineto{\pgfqpoint{0.749115in}{5.041824in}}%
\pgfpathlineto{\pgfqpoint{0.805232in}{5.169346in}}%
\pgfpathlineto{\pgfqpoint{0.861348in}{5.290647in}}%
\pgfpathlineto{\pgfqpoint{0.917465in}{5.406031in}}%
\pgfpathlineto{\pgfqpoint{0.973581in}{5.515787in}}%
\pgfpathlineto{\pgfqpoint{1.029698in}{5.620189in}}%
\pgfpathlineto{\pgfqpoint{1.085814in}{5.719498in}}%
\pgfpathlineto{\pgfqpoint{1.141930in}{5.813962in}}%
\pgfpathlineto{\pgfqpoint{1.198047in}{5.903819in}}%
\pgfpathlineto{\pgfqpoint{1.254163in}{5.989292in}}%
\pgfpathlineto{\pgfqpoint{1.310280in}{6.070596in}}%
\pgfpathlineto{\pgfqpoint{1.366396in}{6.147934in}}%
\pgfpathlineto{\pgfqpoint{1.422513in}{6.221499in}}%
\pgfpathlineto{\pgfqpoint{1.478629in}{6.291476in}}%
\pgfpathlineto{\pgfqpoint{1.534746in}{6.358039in}}%
\pgfpathlineto{\pgfqpoint{1.590862in}{6.421356in}}%
\pgfpathlineto{\pgfqpoint{1.646979in}{6.481583in}}%
\pgfpathlineto{\pgfqpoint{1.703095in}{6.538873in}}%
\pgfpathlineto{\pgfqpoint{1.759211in}{6.593368in}}%
\pgfpathlineto{\pgfqpoint{1.815328in}{6.645205in}}%
\pgfpathlineto{\pgfqpoint{1.871444in}{6.694513in}}%
\pgfpathlineto{\pgfqpoint{1.927561in}{6.741416in}}%
\pgfpathlineto{\pgfqpoint{1.983677in}{6.786031in}}%
\pgfpathlineto{\pgfqpoint{2.039794in}{6.828470in}}%
\pgfpathlineto{\pgfqpoint{2.095910in}{6.868838in}}%
\pgfpathlineto{\pgfqpoint{2.152027in}{6.907237in}}%
\pgfpathlineto{\pgfqpoint{2.208143in}{6.943763in}}%
\pgfpathlineto{\pgfqpoint{2.264260in}{6.978508in}}%
\pgfpathlineto{\pgfqpoint{2.320376in}{7.011557in}}%
\pgfpathlineto{\pgfqpoint{2.376492in}{7.042995in}}%
\pgfpathlineto{\pgfqpoint{2.432609in}{7.072898in}}%
\pgfpathlineto{\pgfqpoint{2.488725in}{7.101343in}}%
\pgfpathlineto{\pgfqpoint{2.544842in}{7.128401in}}%
\pgfpathlineto{\pgfqpoint{2.600958in}{7.154139in}}%
\pgfpathlineto{\pgfqpoint{2.657075in}{7.178621in}}%
\pgfpathlineto{\pgfqpoint{2.713191in}{7.201909in}}%
\pgfpathlineto{\pgfqpoint{2.769308in}{7.224061in}}%
\pgfpathlineto{\pgfqpoint{2.825424in}{7.245132in}}%
\pgfpathlineto{\pgfqpoint{2.881541in}{7.265175in}}%
\pgfpathlineto{\pgfqpoint{2.937657in}{7.284241in}}%
\pgfpathlineto{\pgfqpoint{2.993773in}{7.302377in}}%
\pgfpathlineto{\pgfqpoint{3.049890in}{7.319628in}}%
\pgfpathlineto{\pgfqpoint{3.106006in}{7.336037in}}%
\pgfpathlineto{\pgfqpoint{3.162123in}{7.351646in}}%
\pgfpathlineto{\pgfqpoint{3.218239in}{7.366494in}}%
\pgfpathlineto{\pgfqpoint{3.274356in}{7.380617in}}%
\pgfpathlineto{\pgfqpoint{3.330472in}{7.394052in}}%
\pgfpathlineto{\pgfqpoint{3.386589in}{7.406831in}}%
\pgfpathlineto{\pgfqpoint{3.442705in}{7.418986in}}%
\pgfpathlineto{\pgfqpoint{3.498822in}{7.430549in}}%
\pgfpathlineto{\pgfqpoint{3.554938in}{7.441548in}}%
\pgfpathlineto{\pgfqpoint{3.611055in}{7.452010in}}%
\pgfpathlineto{\pgfqpoint{3.667171in}{7.461962in}}%
\pgfpathlineto{\pgfqpoint{3.723287in}{7.471428in}}%
\pgfpathlineto{\pgfqpoint{3.779404in}{7.480433in}}%
\pgfpathlineto{\pgfqpoint{3.835520in}{7.488998in}}%
\pgfpathlineto{\pgfqpoint{3.891637in}{7.497146in}}%
\pgfpathlineto{\pgfqpoint{3.947753in}{7.504896in}}%
\pgfpathlineto{\pgfqpoint{4.003870in}{7.512268in}}%
\pgfpathlineto{\pgfqpoint{4.059986in}{7.519280in}}%
\pgfpathlineto{\pgfqpoint{4.116103in}{7.525951in}}%
\pgfpathlineto{\pgfqpoint{4.172219in}{7.532296in}}%
\pgfpathlineto{\pgfqpoint{4.228336in}{7.538331in}}%
\pgfpathlineto{\pgfqpoint{4.284452in}{7.544072in}}%
\pgfpathlineto{\pgfqpoint{4.340568in}{7.549533in}}%
\pgfpathlineto{\pgfqpoint{4.396685in}{7.554728in}}%
\pgfpathlineto{\pgfqpoint{4.452801in}{7.559669in}}%
\pgfpathlineto{\pgfqpoint{4.508918in}{7.564369in}}%
\pgfpathlineto{\pgfqpoint{4.565034in}{7.568840in}}%
\pgfpathlineto{\pgfqpoint{4.621151in}{7.573093in}}%
\pgfpathlineto{\pgfqpoint{4.677267in}{7.577138in}}%
\pgfpathlineto{\pgfqpoint{4.733384in}{7.580986in}}%
\pgfpathlineto{\pgfqpoint{4.789500in}{7.584646in}}%
\pgfpathlineto{\pgfqpoint{4.845617in}{7.588128in}}%
\pgfpathlineto{\pgfqpoint{4.901733in}{7.591440in}}%
\pgfpathlineto{\pgfqpoint{4.957849in}{7.594591in}}%
\pgfpathlineto{\pgfqpoint{5.013966in}{7.597587in}}%
\pgfpathlineto{\pgfqpoint{5.070082in}{7.600438in}}%
\pgfpathlineto{\pgfqpoint{5.126199in}{7.603149in}}%
\pgfpathlineto{\pgfqpoint{5.182315in}{7.605728in}}%
\pgfpathlineto{\pgfqpoint{5.238432in}{7.608182in}}%
\pgfpathlineto{\pgfqpoint{5.294548in}{7.610515in}}%
\pgfpathlineto{\pgfqpoint{5.350665in}{7.612735in}}%
\pgfpathlineto{\pgfqpoint{5.406781in}{7.614847in}}%
\pgfpathlineto{\pgfqpoint{5.462898in}{7.616855in}}%
\pgfpathlineto{\pgfqpoint{5.519014in}{7.618766in}}%
\pgfpathlineto{\pgfqpoint{5.575130in}{7.620583in}}%
\pgfpathlineto{\pgfqpoint{5.631247in}{7.622312in}}%
\pgfpathlineto{\pgfqpoint{5.687363in}{7.623957in}}%
\pgfpathlineto{\pgfqpoint{5.743480in}{7.625521in}}%
\pgfpathlineto{\pgfqpoint{5.799596in}{7.627009in}}%
\pgfpathlineto{\pgfqpoint{5.855713in}{7.628424in}}%
\pgfpathlineto{\pgfqpoint{5.911829in}{7.629770in}}%
\pgfpathlineto{\pgfqpoint{5.967946in}{7.631051in}}%
\pgfpathlineto{\pgfqpoint{6.024062in}{7.632269in}}%
\pgfpathlineto{\pgfqpoint{6.080179in}{7.633428in}}%
\pgfpathlineto{\pgfqpoint{6.136295in}{7.634530in}}%
\pgfusepath{stroke}%
\end{pgfscope}%
\begin{pgfscope}%
\pgfpathrectangle{\pgfqpoint{0.580766in}{4.462900in}}{\pgfqpoint{5.611646in}{3.193122in}} %
\pgfusepath{clip}%
\pgfsetrectcap%
\pgfsetroundjoin%
\pgfsetlinewidth{1.003750pt}%
\definecolor{currentstroke}{rgb}{1.000000,0.000000,0.000000}%
\pgfsetstrokecolor{currentstroke}%
\pgfsetdash{}{0pt}%
\pgfpathmoveto{\pgfqpoint{0.580766in}{5.527274in}}%
\pgfpathlineto{\pgfqpoint{1.029698in}{6.236857in}}%
\pgfpathlineto{\pgfqpoint{1.478629in}{6.709912in}}%
\pgfpathlineto{\pgfqpoint{1.927561in}{7.025282in}}%
\pgfpathlineto{\pgfqpoint{2.376492in}{7.235529in}}%
\pgfpathlineto{\pgfqpoint{2.825424in}{7.375694in}}%
\pgfpathlineto{\pgfqpoint{3.274356in}{7.469137in}}%
\pgfpathlineto{\pgfqpoint{3.723287in}{7.531432in}}%
\pgfpathlineto{\pgfqpoint{4.172219in}{7.572962in}}%
\pgfpathlineto{\pgfqpoint{4.621151in}{7.600649in}}%
\pgfpathlineto{\pgfqpoint{5.070082in}{7.619107in}}%
\pgfpathlineto{\pgfqpoint{5.519014in}{7.631412in}}%
\pgfusepath{stroke}%
\end{pgfscope}%
\begin{pgfscope}%
\pgfpathrectangle{\pgfqpoint{0.580766in}{4.462900in}}{\pgfqpoint{5.611646in}{3.193122in}} %
\pgfusepath{clip}%
\pgfsetbuttcap%
\pgfsetroundjoin%
\pgfsetlinewidth{0.501875pt}%
\definecolor{currentstroke}{rgb}{0.000000,0.000000,0.000000}%
\pgfsetstrokecolor{currentstroke}%
\pgfsetdash{{1.000000pt}{3.000000pt}}{0.000000pt}%
\pgfpathmoveto{\pgfqpoint{0.580766in}{4.462900in}}%
\pgfpathlineto{\pgfqpoint{0.580766in}{7.656023in}}%
\pgfusepath{stroke}%
\end{pgfscope}%
\begin{pgfscope}%
\pgfsetbuttcap%
\pgfsetroundjoin%
\definecolor{currentfill}{rgb}{0.000000,0.000000,0.000000}%
\pgfsetfillcolor{currentfill}%
\pgfsetlinewidth{0.501875pt}%
\definecolor{currentstroke}{rgb}{0.000000,0.000000,0.000000}%
\pgfsetstrokecolor{currentstroke}%
\pgfsetdash{}{0pt}%
\pgfsys@defobject{currentmarker}{\pgfqpoint{0.000000in}{0.000000in}}{\pgfqpoint{0.000000in}{0.055556in}}{%
\pgfpathmoveto{\pgfqpoint{0.000000in}{0.000000in}}%
\pgfpathlineto{\pgfqpoint{0.000000in}{0.055556in}}%
\pgfusepath{stroke,fill}%
}%
\begin{pgfscope}%
\pgfsys@transformshift{0.580766in}{4.462900in}%
\pgfsys@useobject{currentmarker}{}%
\end{pgfscope}%
\end{pgfscope}%
\begin{pgfscope}%
\pgfsetbuttcap%
\pgfsetroundjoin%
\definecolor{currentfill}{rgb}{0.000000,0.000000,0.000000}%
\pgfsetfillcolor{currentfill}%
\pgfsetlinewidth{0.501875pt}%
\definecolor{currentstroke}{rgb}{0.000000,0.000000,0.000000}%
\pgfsetstrokecolor{currentstroke}%
\pgfsetdash{}{0pt}%
\pgfsys@defobject{currentmarker}{\pgfqpoint{0.000000in}{-0.055556in}}{\pgfqpoint{0.000000in}{0.000000in}}{%
\pgfpathmoveto{\pgfqpoint{0.000000in}{0.000000in}}%
\pgfpathlineto{\pgfqpoint{0.000000in}{-0.055556in}}%
\pgfusepath{stroke,fill}%
}%
\begin{pgfscope}%
\pgfsys@transformshift{0.580766in}{7.656023in}%
\pgfsys@useobject{currentmarker}{}%
\end{pgfscope}%
\end{pgfscope}%
\begin{pgfscope}%
\pgftext[x=0.580766in,y=4.407345in,,top]{{\rmfamily\fontsize{10.000000}{12.000000}\selectfont \(\displaystyle 0.000\)}}%
\end{pgfscope}%
\begin{pgfscope}%
\pgfpathrectangle{\pgfqpoint{0.580766in}{4.462900in}}{\pgfqpoint{5.611646in}{3.193122in}} %
\pgfusepath{clip}%
\pgfsetbuttcap%
\pgfsetroundjoin%
\pgfsetlinewidth{0.501875pt}%
\definecolor{currentstroke}{rgb}{0.000000,0.000000,0.000000}%
\pgfsetstrokecolor{currentstroke}%
\pgfsetdash{{1.000000pt}{3.000000pt}}{0.000000pt}%
\pgfpathmoveto{\pgfqpoint{1.703095in}{4.462900in}}%
\pgfpathlineto{\pgfqpoint{1.703095in}{7.656023in}}%
\pgfusepath{stroke}%
\end{pgfscope}%
\begin{pgfscope}%
\pgfsetbuttcap%
\pgfsetroundjoin%
\definecolor{currentfill}{rgb}{0.000000,0.000000,0.000000}%
\pgfsetfillcolor{currentfill}%
\pgfsetlinewidth{0.501875pt}%
\definecolor{currentstroke}{rgb}{0.000000,0.000000,0.000000}%
\pgfsetstrokecolor{currentstroke}%
\pgfsetdash{}{0pt}%
\pgfsys@defobject{currentmarker}{\pgfqpoint{0.000000in}{0.000000in}}{\pgfqpoint{0.000000in}{0.055556in}}{%
\pgfpathmoveto{\pgfqpoint{0.000000in}{0.000000in}}%
\pgfpathlineto{\pgfqpoint{0.000000in}{0.055556in}}%
\pgfusepath{stroke,fill}%
}%
\begin{pgfscope}%
\pgfsys@transformshift{1.703095in}{4.462900in}%
\pgfsys@useobject{currentmarker}{}%
\end{pgfscope}%
\end{pgfscope}%
\begin{pgfscope}%
\pgfsetbuttcap%
\pgfsetroundjoin%
\definecolor{currentfill}{rgb}{0.000000,0.000000,0.000000}%
\pgfsetfillcolor{currentfill}%
\pgfsetlinewidth{0.501875pt}%
\definecolor{currentstroke}{rgb}{0.000000,0.000000,0.000000}%
\pgfsetstrokecolor{currentstroke}%
\pgfsetdash{}{0pt}%
\pgfsys@defobject{currentmarker}{\pgfqpoint{0.000000in}{-0.055556in}}{\pgfqpoint{0.000000in}{0.000000in}}{%
\pgfpathmoveto{\pgfqpoint{0.000000in}{0.000000in}}%
\pgfpathlineto{\pgfqpoint{0.000000in}{-0.055556in}}%
\pgfusepath{stroke,fill}%
}%
\begin{pgfscope}%
\pgfsys@transformshift{1.703095in}{7.656023in}%
\pgfsys@useobject{currentmarker}{}%
\end{pgfscope}%
\end{pgfscope}%
\begin{pgfscope}%
\pgftext[x=1.703095in,y=4.407345in,,top]{{\rmfamily\fontsize{10.000000}{12.000000}\selectfont \(\displaystyle 0.002\)}}%
\end{pgfscope}%
\begin{pgfscope}%
\pgfpathrectangle{\pgfqpoint{0.580766in}{4.462900in}}{\pgfqpoint{5.611646in}{3.193122in}} %
\pgfusepath{clip}%
\pgfsetbuttcap%
\pgfsetroundjoin%
\pgfsetlinewidth{0.501875pt}%
\definecolor{currentstroke}{rgb}{0.000000,0.000000,0.000000}%
\pgfsetstrokecolor{currentstroke}%
\pgfsetdash{{1.000000pt}{3.000000pt}}{0.000000pt}%
\pgfpathmoveto{\pgfqpoint{2.825424in}{4.462900in}}%
\pgfpathlineto{\pgfqpoint{2.825424in}{7.656023in}}%
\pgfusepath{stroke}%
\end{pgfscope}%
\begin{pgfscope}%
\pgfsetbuttcap%
\pgfsetroundjoin%
\definecolor{currentfill}{rgb}{0.000000,0.000000,0.000000}%
\pgfsetfillcolor{currentfill}%
\pgfsetlinewidth{0.501875pt}%
\definecolor{currentstroke}{rgb}{0.000000,0.000000,0.000000}%
\pgfsetstrokecolor{currentstroke}%
\pgfsetdash{}{0pt}%
\pgfsys@defobject{currentmarker}{\pgfqpoint{0.000000in}{0.000000in}}{\pgfqpoint{0.000000in}{0.055556in}}{%
\pgfpathmoveto{\pgfqpoint{0.000000in}{0.000000in}}%
\pgfpathlineto{\pgfqpoint{0.000000in}{0.055556in}}%
\pgfusepath{stroke,fill}%
}%
\begin{pgfscope}%
\pgfsys@transformshift{2.825424in}{4.462900in}%
\pgfsys@useobject{currentmarker}{}%
\end{pgfscope}%
\end{pgfscope}%
\begin{pgfscope}%
\pgfsetbuttcap%
\pgfsetroundjoin%
\definecolor{currentfill}{rgb}{0.000000,0.000000,0.000000}%
\pgfsetfillcolor{currentfill}%
\pgfsetlinewidth{0.501875pt}%
\definecolor{currentstroke}{rgb}{0.000000,0.000000,0.000000}%
\pgfsetstrokecolor{currentstroke}%
\pgfsetdash{}{0pt}%
\pgfsys@defobject{currentmarker}{\pgfqpoint{0.000000in}{-0.055556in}}{\pgfqpoint{0.000000in}{0.000000in}}{%
\pgfpathmoveto{\pgfqpoint{0.000000in}{0.000000in}}%
\pgfpathlineto{\pgfqpoint{0.000000in}{-0.055556in}}%
\pgfusepath{stroke,fill}%
}%
\begin{pgfscope}%
\pgfsys@transformshift{2.825424in}{7.656023in}%
\pgfsys@useobject{currentmarker}{}%
\end{pgfscope}%
\end{pgfscope}%
\begin{pgfscope}%
\pgftext[x=2.825424in,y=4.407345in,,top]{{\rmfamily\fontsize{10.000000}{12.000000}\selectfont \(\displaystyle 0.004\)}}%
\end{pgfscope}%
\begin{pgfscope}%
\pgfpathrectangle{\pgfqpoint{0.580766in}{4.462900in}}{\pgfqpoint{5.611646in}{3.193122in}} %
\pgfusepath{clip}%
\pgfsetbuttcap%
\pgfsetroundjoin%
\pgfsetlinewidth{0.501875pt}%
\definecolor{currentstroke}{rgb}{0.000000,0.000000,0.000000}%
\pgfsetstrokecolor{currentstroke}%
\pgfsetdash{{1.000000pt}{3.000000pt}}{0.000000pt}%
\pgfpathmoveto{\pgfqpoint{3.947753in}{4.462900in}}%
\pgfpathlineto{\pgfqpoint{3.947753in}{7.656023in}}%
\pgfusepath{stroke}%
\end{pgfscope}%
\begin{pgfscope}%
\pgfsetbuttcap%
\pgfsetroundjoin%
\definecolor{currentfill}{rgb}{0.000000,0.000000,0.000000}%
\pgfsetfillcolor{currentfill}%
\pgfsetlinewidth{0.501875pt}%
\definecolor{currentstroke}{rgb}{0.000000,0.000000,0.000000}%
\pgfsetstrokecolor{currentstroke}%
\pgfsetdash{}{0pt}%
\pgfsys@defobject{currentmarker}{\pgfqpoint{0.000000in}{0.000000in}}{\pgfqpoint{0.000000in}{0.055556in}}{%
\pgfpathmoveto{\pgfqpoint{0.000000in}{0.000000in}}%
\pgfpathlineto{\pgfqpoint{0.000000in}{0.055556in}}%
\pgfusepath{stroke,fill}%
}%
\begin{pgfscope}%
\pgfsys@transformshift{3.947753in}{4.462900in}%
\pgfsys@useobject{currentmarker}{}%
\end{pgfscope}%
\end{pgfscope}%
\begin{pgfscope}%
\pgfsetbuttcap%
\pgfsetroundjoin%
\definecolor{currentfill}{rgb}{0.000000,0.000000,0.000000}%
\pgfsetfillcolor{currentfill}%
\pgfsetlinewidth{0.501875pt}%
\definecolor{currentstroke}{rgb}{0.000000,0.000000,0.000000}%
\pgfsetstrokecolor{currentstroke}%
\pgfsetdash{}{0pt}%
\pgfsys@defobject{currentmarker}{\pgfqpoint{0.000000in}{-0.055556in}}{\pgfqpoint{0.000000in}{0.000000in}}{%
\pgfpathmoveto{\pgfqpoint{0.000000in}{0.000000in}}%
\pgfpathlineto{\pgfqpoint{0.000000in}{-0.055556in}}%
\pgfusepath{stroke,fill}%
}%
\begin{pgfscope}%
\pgfsys@transformshift{3.947753in}{7.656023in}%
\pgfsys@useobject{currentmarker}{}%
\end{pgfscope}%
\end{pgfscope}%
\begin{pgfscope}%
\pgftext[x=3.947753in,y=4.407345in,,top]{{\rmfamily\fontsize{10.000000}{12.000000}\selectfont \(\displaystyle 0.006\)}}%
\end{pgfscope}%
\begin{pgfscope}%
\pgfpathrectangle{\pgfqpoint{0.580766in}{4.462900in}}{\pgfqpoint{5.611646in}{3.193122in}} %
\pgfusepath{clip}%
\pgfsetbuttcap%
\pgfsetroundjoin%
\pgfsetlinewidth{0.501875pt}%
\definecolor{currentstroke}{rgb}{0.000000,0.000000,0.000000}%
\pgfsetstrokecolor{currentstroke}%
\pgfsetdash{{1.000000pt}{3.000000pt}}{0.000000pt}%
\pgfpathmoveto{\pgfqpoint{5.070082in}{4.462900in}}%
\pgfpathlineto{\pgfqpoint{5.070082in}{7.656023in}}%
\pgfusepath{stroke}%
\end{pgfscope}%
\begin{pgfscope}%
\pgfsetbuttcap%
\pgfsetroundjoin%
\definecolor{currentfill}{rgb}{0.000000,0.000000,0.000000}%
\pgfsetfillcolor{currentfill}%
\pgfsetlinewidth{0.501875pt}%
\definecolor{currentstroke}{rgb}{0.000000,0.000000,0.000000}%
\pgfsetstrokecolor{currentstroke}%
\pgfsetdash{}{0pt}%
\pgfsys@defobject{currentmarker}{\pgfqpoint{0.000000in}{0.000000in}}{\pgfqpoint{0.000000in}{0.055556in}}{%
\pgfpathmoveto{\pgfqpoint{0.000000in}{0.000000in}}%
\pgfpathlineto{\pgfqpoint{0.000000in}{0.055556in}}%
\pgfusepath{stroke,fill}%
}%
\begin{pgfscope}%
\pgfsys@transformshift{5.070082in}{4.462900in}%
\pgfsys@useobject{currentmarker}{}%
\end{pgfscope}%
\end{pgfscope}%
\begin{pgfscope}%
\pgfsetbuttcap%
\pgfsetroundjoin%
\definecolor{currentfill}{rgb}{0.000000,0.000000,0.000000}%
\pgfsetfillcolor{currentfill}%
\pgfsetlinewidth{0.501875pt}%
\definecolor{currentstroke}{rgb}{0.000000,0.000000,0.000000}%
\pgfsetstrokecolor{currentstroke}%
\pgfsetdash{}{0pt}%
\pgfsys@defobject{currentmarker}{\pgfqpoint{0.000000in}{-0.055556in}}{\pgfqpoint{0.000000in}{0.000000in}}{%
\pgfpathmoveto{\pgfqpoint{0.000000in}{0.000000in}}%
\pgfpathlineto{\pgfqpoint{0.000000in}{-0.055556in}}%
\pgfusepath{stroke,fill}%
}%
\begin{pgfscope}%
\pgfsys@transformshift{5.070082in}{7.656023in}%
\pgfsys@useobject{currentmarker}{}%
\end{pgfscope}%
\end{pgfscope}%
\begin{pgfscope}%
\pgftext[x=5.070082in,y=4.407345in,,top]{{\rmfamily\fontsize{10.000000}{12.000000}\selectfont \(\displaystyle 0.008\)}}%
\end{pgfscope}%
\begin{pgfscope}%
\pgfpathrectangle{\pgfqpoint{0.580766in}{4.462900in}}{\pgfqpoint{5.611646in}{3.193122in}} %
\pgfusepath{clip}%
\pgfsetbuttcap%
\pgfsetroundjoin%
\pgfsetlinewidth{0.501875pt}%
\definecolor{currentstroke}{rgb}{0.000000,0.000000,0.000000}%
\pgfsetstrokecolor{currentstroke}%
\pgfsetdash{{1.000000pt}{3.000000pt}}{0.000000pt}%
\pgfpathmoveto{\pgfqpoint{6.192411in}{4.462900in}}%
\pgfpathlineto{\pgfqpoint{6.192411in}{7.656023in}}%
\pgfusepath{stroke}%
\end{pgfscope}%
\begin{pgfscope}%
\pgfsetbuttcap%
\pgfsetroundjoin%
\definecolor{currentfill}{rgb}{0.000000,0.000000,0.000000}%
\pgfsetfillcolor{currentfill}%
\pgfsetlinewidth{0.501875pt}%
\definecolor{currentstroke}{rgb}{0.000000,0.000000,0.000000}%
\pgfsetstrokecolor{currentstroke}%
\pgfsetdash{}{0pt}%
\pgfsys@defobject{currentmarker}{\pgfqpoint{0.000000in}{0.000000in}}{\pgfqpoint{0.000000in}{0.055556in}}{%
\pgfpathmoveto{\pgfqpoint{0.000000in}{0.000000in}}%
\pgfpathlineto{\pgfqpoint{0.000000in}{0.055556in}}%
\pgfusepath{stroke,fill}%
}%
\begin{pgfscope}%
\pgfsys@transformshift{6.192411in}{4.462900in}%
\pgfsys@useobject{currentmarker}{}%
\end{pgfscope}%
\end{pgfscope}%
\begin{pgfscope}%
\pgfsetbuttcap%
\pgfsetroundjoin%
\definecolor{currentfill}{rgb}{0.000000,0.000000,0.000000}%
\pgfsetfillcolor{currentfill}%
\pgfsetlinewidth{0.501875pt}%
\definecolor{currentstroke}{rgb}{0.000000,0.000000,0.000000}%
\pgfsetstrokecolor{currentstroke}%
\pgfsetdash{}{0pt}%
\pgfsys@defobject{currentmarker}{\pgfqpoint{0.000000in}{-0.055556in}}{\pgfqpoint{0.000000in}{0.000000in}}{%
\pgfpathmoveto{\pgfqpoint{0.000000in}{0.000000in}}%
\pgfpathlineto{\pgfqpoint{0.000000in}{-0.055556in}}%
\pgfusepath{stroke,fill}%
}%
\begin{pgfscope}%
\pgfsys@transformshift{6.192411in}{7.656023in}%
\pgfsys@useobject{currentmarker}{}%
\end{pgfscope}%
\end{pgfscope}%
\begin{pgfscope}%
\pgftext[x=6.192411in,y=4.407345in,,top]{{\rmfamily\fontsize{10.000000}{12.000000}\selectfont \(\displaystyle 0.010\)}}%
\end{pgfscope}%
\begin{pgfscope}%
\pgftext[x=3.386589in,y=4.214443in,,top]{{\rmfamily\fontsize{10.000000}{12.000000}\selectfont time (s)}}%
\end{pgfscope}%
\begin{pgfscope}%
\pgfpathrectangle{\pgfqpoint{0.580766in}{4.462900in}}{\pgfqpoint{5.611646in}{3.193122in}} %
\pgfusepath{clip}%
\pgfsetbuttcap%
\pgfsetroundjoin%
\pgfsetlinewidth{0.501875pt}%
\definecolor{currentstroke}{rgb}{0.000000,0.000000,0.000000}%
\pgfsetstrokecolor{currentstroke}%
\pgfsetdash{{1.000000pt}{3.000000pt}}{0.000000pt}%
\pgfpathmoveto{\pgfqpoint{0.580766in}{4.462900in}}%
\pgfpathlineto{\pgfqpoint{6.192411in}{4.462900in}}%
\pgfusepath{stroke}%
\end{pgfscope}%
\begin{pgfscope}%
\pgfsetbuttcap%
\pgfsetroundjoin%
\definecolor{currentfill}{rgb}{0.000000,0.000000,0.000000}%
\pgfsetfillcolor{currentfill}%
\pgfsetlinewidth{0.501875pt}%
\definecolor{currentstroke}{rgb}{0.000000,0.000000,0.000000}%
\pgfsetstrokecolor{currentstroke}%
\pgfsetdash{}{0pt}%
\pgfsys@defobject{currentmarker}{\pgfqpoint{0.000000in}{0.000000in}}{\pgfqpoint{0.055556in}{0.000000in}}{%
\pgfpathmoveto{\pgfqpoint{0.000000in}{0.000000in}}%
\pgfpathlineto{\pgfqpoint{0.055556in}{0.000000in}}%
\pgfusepath{stroke,fill}%
}%
\begin{pgfscope}%
\pgfsys@transformshift{0.580766in}{4.462900in}%
\pgfsys@useobject{currentmarker}{}%
\end{pgfscope}%
\end{pgfscope}%
\begin{pgfscope}%
\pgfsetbuttcap%
\pgfsetroundjoin%
\definecolor{currentfill}{rgb}{0.000000,0.000000,0.000000}%
\pgfsetfillcolor{currentfill}%
\pgfsetlinewidth{0.501875pt}%
\definecolor{currentstroke}{rgb}{0.000000,0.000000,0.000000}%
\pgfsetstrokecolor{currentstroke}%
\pgfsetdash{}{0pt}%
\pgfsys@defobject{currentmarker}{\pgfqpoint{-0.055556in}{0.000000in}}{\pgfqpoint{0.000000in}{0.000000in}}{%
\pgfpathmoveto{\pgfqpoint{0.000000in}{0.000000in}}%
\pgfpathlineto{\pgfqpoint{-0.055556in}{0.000000in}}%
\pgfusepath{stroke,fill}%
}%
\begin{pgfscope}%
\pgfsys@transformshift{6.192411in}{4.462900in}%
\pgfsys@useobject{currentmarker}{}%
\end{pgfscope}%
\end{pgfscope}%
\begin{pgfscope}%
\pgftext[x=0.525210in,y=4.462900in,right,]{{\rmfamily\fontsize{10.000000}{12.000000}\selectfont \(\displaystyle 0.0\)}}%
\end{pgfscope}%
\begin{pgfscope}%
\pgfpathrectangle{\pgfqpoint{0.580766in}{4.462900in}}{\pgfqpoint{5.611646in}{3.193122in}} %
\pgfusepath{clip}%
\pgfsetbuttcap%
\pgfsetroundjoin%
\pgfsetlinewidth{0.501875pt}%
\definecolor{currentstroke}{rgb}{0.000000,0.000000,0.000000}%
\pgfsetstrokecolor{currentstroke}%
\pgfsetdash{{1.000000pt}{3.000000pt}}{0.000000pt}%
\pgfpathmoveto{\pgfqpoint{0.580766in}{5.101525in}}%
\pgfpathlineto{\pgfqpoint{6.192411in}{5.101525in}}%
\pgfusepath{stroke}%
\end{pgfscope}%
\begin{pgfscope}%
\pgfsetbuttcap%
\pgfsetroundjoin%
\definecolor{currentfill}{rgb}{0.000000,0.000000,0.000000}%
\pgfsetfillcolor{currentfill}%
\pgfsetlinewidth{0.501875pt}%
\definecolor{currentstroke}{rgb}{0.000000,0.000000,0.000000}%
\pgfsetstrokecolor{currentstroke}%
\pgfsetdash{}{0pt}%
\pgfsys@defobject{currentmarker}{\pgfqpoint{0.000000in}{0.000000in}}{\pgfqpoint{0.055556in}{0.000000in}}{%
\pgfpathmoveto{\pgfqpoint{0.000000in}{0.000000in}}%
\pgfpathlineto{\pgfqpoint{0.055556in}{0.000000in}}%
\pgfusepath{stroke,fill}%
}%
\begin{pgfscope}%
\pgfsys@transformshift{0.580766in}{5.101525in}%
\pgfsys@useobject{currentmarker}{}%
\end{pgfscope}%
\end{pgfscope}%
\begin{pgfscope}%
\pgfsetbuttcap%
\pgfsetroundjoin%
\definecolor{currentfill}{rgb}{0.000000,0.000000,0.000000}%
\pgfsetfillcolor{currentfill}%
\pgfsetlinewidth{0.501875pt}%
\definecolor{currentstroke}{rgb}{0.000000,0.000000,0.000000}%
\pgfsetstrokecolor{currentstroke}%
\pgfsetdash{}{0pt}%
\pgfsys@defobject{currentmarker}{\pgfqpoint{-0.055556in}{0.000000in}}{\pgfqpoint{0.000000in}{0.000000in}}{%
\pgfpathmoveto{\pgfqpoint{0.000000in}{0.000000in}}%
\pgfpathlineto{\pgfqpoint{-0.055556in}{0.000000in}}%
\pgfusepath{stroke,fill}%
}%
\begin{pgfscope}%
\pgfsys@transformshift{6.192411in}{5.101525in}%
\pgfsys@useobject{currentmarker}{}%
\end{pgfscope}%
\end{pgfscope}%
\begin{pgfscope}%
\pgftext[x=0.525210in,y=5.101525in,right,]{{\rmfamily\fontsize{10.000000}{12.000000}\selectfont \(\displaystyle 0.2\)}}%
\end{pgfscope}%
\begin{pgfscope}%
\pgfpathrectangle{\pgfqpoint{0.580766in}{4.462900in}}{\pgfqpoint{5.611646in}{3.193122in}} %
\pgfusepath{clip}%
\pgfsetbuttcap%
\pgfsetroundjoin%
\pgfsetlinewidth{0.501875pt}%
\definecolor{currentstroke}{rgb}{0.000000,0.000000,0.000000}%
\pgfsetstrokecolor{currentstroke}%
\pgfsetdash{{1.000000pt}{3.000000pt}}{0.000000pt}%
\pgfpathmoveto{\pgfqpoint{0.580766in}{5.740149in}}%
\pgfpathlineto{\pgfqpoint{6.192411in}{5.740149in}}%
\pgfusepath{stroke}%
\end{pgfscope}%
\begin{pgfscope}%
\pgfsetbuttcap%
\pgfsetroundjoin%
\definecolor{currentfill}{rgb}{0.000000,0.000000,0.000000}%
\pgfsetfillcolor{currentfill}%
\pgfsetlinewidth{0.501875pt}%
\definecolor{currentstroke}{rgb}{0.000000,0.000000,0.000000}%
\pgfsetstrokecolor{currentstroke}%
\pgfsetdash{}{0pt}%
\pgfsys@defobject{currentmarker}{\pgfqpoint{0.000000in}{0.000000in}}{\pgfqpoint{0.055556in}{0.000000in}}{%
\pgfpathmoveto{\pgfqpoint{0.000000in}{0.000000in}}%
\pgfpathlineto{\pgfqpoint{0.055556in}{0.000000in}}%
\pgfusepath{stroke,fill}%
}%
\begin{pgfscope}%
\pgfsys@transformshift{0.580766in}{5.740149in}%
\pgfsys@useobject{currentmarker}{}%
\end{pgfscope}%
\end{pgfscope}%
\begin{pgfscope}%
\pgfsetbuttcap%
\pgfsetroundjoin%
\definecolor{currentfill}{rgb}{0.000000,0.000000,0.000000}%
\pgfsetfillcolor{currentfill}%
\pgfsetlinewidth{0.501875pt}%
\definecolor{currentstroke}{rgb}{0.000000,0.000000,0.000000}%
\pgfsetstrokecolor{currentstroke}%
\pgfsetdash{}{0pt}%
\pgfsys@defobject{currentmarker}{\pgfqpoint{-0.055556in}{0.000000in}}{\pgfqpoint{0.000000in}{0.000000in}}{%
\pgfpathmoveto{\pgfqpoint{0.000000in}{0.000000in}}%
\pgfpathlineto{\pgfqpoint{-0.055556in}{0.000000in}}%
\pgfusepath{stroke,fill}%
}%
\begin{pgfscope}%
\pgfsys@transformshift{6.192411in}{5.740149in}%
\pgfsys@useobject{currentmarker}{}%
\end{pgfscope}%
\end{pgfscope}%
\begin{pgfscope}%
\pgftext[x=0.525210in,y=5.740149in,right,]{{\rmfamily\fontsize{10.000000}{12.000000}\selectfont \(\displaystyle 0.4\)}}%
\end{pgfscope}%
\begin{pgfscope}%
\pgfpathrectangle{\pgfqpoint{0.580766in}{4.462900in}}{\pgfqpoint{5.611646in}{3.193122in}} %
\pgfusepath{clip}%
\pgfsetbuttcap%
\pgfsetroundjoin%
\pgfsetlinewidth{0.501875pt}%
\definecolor{currentstroke}{rgb}{0.000000,0.000000,0.000000}%
\pgfsetstrokecolor{currentstroke}%
\pgfsetdash{{1.000000pt}{3.000000pt}}{0.000000pt}%
\pgfpathmoveto{\pgfqpoint{0.580766in}{6.378774in}}%
\pgfpathlineto{\pgfqpoint{6.192411in}{6.378774in}}%
\pgfusepath{stroke}%
\end{pgfscope}%
\begin{pgfscope}%
\pgfsetbuttcap%
\pgfsetroundjoin%
\definecolor{currentfill}{rgb}{0.000000,0.000000,0.000000}%
\pgfsetfillcolor{currentfill}%
\pgfsetlinewidth{0.501875pt}%
\definecolor{currentstroke}{rgb}{0.000000,0.000000,0.000000}%
\pgfsetstrokecolor{currentstroke}%
\pgfsetdash{}{0pt}%
\pgfsys@defobject{currentmarker}{\pgfqpoint{0.000000in}{0.000000in}}{\pgfqpoint{0.055556in}{0.000000in}}{%
\pgfpathmoveto{\pgfqpoint{0.000000in}{0.000000in}}%
\pgfpathlineto{\pgfqpoint{0.055556in}{0.000000in}}%
\pgfusepath{stroke,fill}%
}%
\begin{pgfscope}%
\pgfsys@transformshift{0.580766in}{6.378774in}%
\pgfsys@useobject{currentmarker}{}%
\end{pgfscope}%
\end{pgfscope}%
\begin{pgfscope}%
\pgfsetbuttcap%
\pgfsetroundjoin%
\definecolor{currentfill}{rgb}{0.000000,0.000000,0.000000}%
\pgfsetfillcolor{currentfill}%
\pgfsetlinewidth{0.501875pt}%
\definecolor{currentstroke}{rgb}{0.000000,0.000000,0.000000}%
\pgfsetstrokecolor{currentstroke}%
\pgfsetdash{}{0pt}%
\pgfsys@defobject{currentmarker}{\pgfqpoint{-0.055556in}{0.000000in}}{\pgfqpoint{0.000000in}{0.000000in}}{%
\pgfpathmoveto{\pgfqpoint{0.000000in}{0.000000in}}%
\pgfpathlineto{\pgfqpoint{-0.055556in}{0.000000in}}%
\pgfusepath{stroke,fill}%
}%
\begin{pgfscope}%
\pgfsys@transformshift{6.192411in}{6.378774in}%
\pgfsys@useobject{currentmarker}{}%
\end{pgfscope}%
\end{pgfscope}%
\begin{pgfscope}%
\pgftext[x=0.525210in,y=6.378774in,right,]{{\rmfamily\fontsize{10.000000}{12.000000}\selectfont \(\displaystyle 0.6\)}}%
\end{pgfscope}%
\begin{pgfscope}%
\pgfpathrectangle{\pgfqpoint{0.580766in}{4.462900in}}{\pgfqpoint{5.611646in}{3.193122in}} %
\pgfusepath{clip}%
\pgfsetbuttcap%
\pgfsetroundjoin%
\pgfsetlinewidth{0.501875pt}%
\definecolor{currentstroke}{rgb}{0.000000,0.000000,0.000000}%
\pgfsetstrokecolor{currentstroke}%
\pgfsetdash{{1.000000pt}{3.000000pt}}{0.000000pt}%
\pgfpathmoveto{\pgfqpoint{0.580766in}{7.017398in}}%
\pgfpathlineto{\pgfqpoint{6.192411in}{7.017398in}}%
\pgfusepath{stroke}%
\end{pgfscope}%
\begin{pgfscope}%
\pgfsetbuttcap%
\pgfsetroundjoin%
\definecolor{currentfill}{rgb}{0.000000,0.000000,0.000000}%
\pgfsetfillcolor{currentfill}%
\pgfsetlinewidth{0.501875pt}%
\definecolor{currentstroke}{rgb}{0.000000,0.000000,0.000000}%
\pgfsetstrokecolor{currentstroke}%
\pgfsetdash{}{0pt}%
\pgfsys@defobject{currentmarker}{\pgfqpoint{0.000000in}{0.000000in}}{\pgfqpoint{0.055556in}{0.000000in}}{%
\pgfpathmoveto{\pgfqpoint{0.000000in}{0.000000in}}%
\pgfpathlineto{\pgfqpoint{0.055556in}{0.000000in}}%
\pgfusepath{stroke,fill}%
}%
\begin{pgfscope}%
\pgfsys@transformshift{0.580766in}{7.017398in}%
\pgfsys@useobject{currentmarker}{}%
\end{pgfscope}%
\end{pgfscope}%
\begin{pgfscope}%
\pgfsetbuttcap%
\pgfsetroundjoin%
\definecolor{currentfill}{rgb}{0.000000,0.000000,0.000000}%
\pgfsetfillcolor{currentfill}%
\pgfsetlinewidth{0.501875pt}%
\definecolor{currentstroke}{rgb}{0.000000,0.000000,0.000000}%
\pgfsetstrokecolor{currentstroke}%
\pgfsetdash{}{0pt}%
\pgfsys@defobject{currentmarker}{\pgfqpoint{-0.055556in}{0.000000in}}{\pgfqpoint{0.000000in}{0.000000in}}{%
\pgfpathmoveto{\pgfqpoint{0.000000in}{0.000000in}}%
\pgfpathlineto{\pgfqpoint{-0.055556in}{0.000000in}}%
\pgfusepath{stroke,fill}%
}%
\begin{pgfscope}%
\pgfsys@transformshift{6.192411in}{7.017398in}%
\pgfsys@useobject{currentmarker}{}%
\end{pgfscope}%
\end{pgfscope}%
\begin{pgfscope}%
\pgftext[x=0.525210in,y=7.017398in,right,]{{\rmfamily\fontsize{10.000000}{12.000000}\selectfont \(\displaystyle 0.8\)}}%
\end{pgfscope}%
\begin{pgfscope}%
\pgfpathrectangle{\pgfqpoint{0.580766in}{4.462900in}}{\pgfqpoint{5.611646in}{3.193122in}} %
\pgfusepath{clip}%
\pgfsetbuttcap%
\pgfsetroundjoin%
\pgfsetlinewidth{0.501875pt}%
\definecolor{currentstroke}{rgb}{0.000000,0.000000,0.000000}%
\pgfsetstrokecolor{currentstroke}%
\pgfsetdash{{1.000000pt}{3.000000pt}}{0.000000pt}%
\pgfpathmoveto{\pgfqpoint{0.580766in}{7.656023in}}%
\pgfpathlineto{\pgfqpoint{6.192411in}{7.656023in}}%
\pgfusepath{stroke}%
\end{pgfscope}%
\begin{pgfscope}%
\pgfsetbuttcap%
\pgfsetroundjoin%
\definecolor{currentfill}{rgb}{0.000000,0.000000,0.000000}%
\pgfsetfillcolor{currentfill}%
\pgfsetlinewidth{0.501875pt}%
\definecolor{currentstroke}{rgb}{0.000000,0.000000,0.000000}%
\pgfsetstrokecolor{currentstroke}%
\pgfsetdash{}{0pt}%
\pgfsys@defobject{currentmarker}{\pgfqpoint{0.000000in}{0.000000in}}{\pgfqpoint{0.055556in}{0.000000in}}{%
\pgfpathmoveto{\pgfqpoint{0.000000in}{0.000000in}}%
\pgfpathlineto{\pgfqpoint{0.055556in}{0.000000in}}%
\pgfusepath{stroke,fill}%
}%
\begin{pgfscope}%
\pgfsys@transformshift{0.580766in}{7.656023in}%
\pgfsys@useobject{currentmarker}{}%
\end{pgfscope}%
\end{pgfscope}%
\begin{pgfscope}%
\pgfsetbuttcap%
\pgfsetroundjoin%
\definecolor{currentfill}{rgb}{0.000000,0.000000,0.000000}%
\pgfsetfillcolor{currentfill}%
\pgfsetlinewidth{0.501875pt}%
\definecolor{currentstroke}{rgb}{0.000000,0.000000,0.000000}%
\pgfsetstrokecolor{currentstroke}%
\pgfsetdash{}{0pt}%
\pgfsys@defobject{currentmarker}{\pgfqpoint{-0.055556in}{0.000000in}}{\pgfqpoint{0.000000in}{0.000000in}}{%
\pgfpathmoveto{\pgfqpoint{0.000000in}{0.000000in}}%
\pgfpathlineto{\pgfqpoint{-0.055556in}{0.000000in}}%
\pgfusepath{stroke,fill}%
}%
\begin{pgfscope}%
\pgfsys@transformshift{6.192411in}{7.656023in}%
\pgfsys@useobject{currentmarker}{}%
\end{pgfscope}%
\end{pgfscope}%
\begin{pgfscope}%
\pgftext[x=0.525210in,y=7.656023in,right,]{{\rmfamily\fontsize{10.000000}{12.000000}\selectfont \(\displaystyle 1.0\)}}%
\end{pgfscope}%
\begin{pgfscope}%
\pgftext[x=0.278296in,y=6.059461in,,bottom,rotate=90.000000]{{\rmfamily\fontsize{10.000000}{12.000000}\selectfont current (A)}}%
\end{pgfscope}%
\begin{pgfscope}%
\pgfsetbuttcap%
\pgfsetroundjoin%
\pgfsetlinewidth{1.003750pt}%
\definecolor{currentstroke}{rgb}{0.000000,0.000000,0.000000}%
\pgfsetstrokecolor{currentstroke}%
\pgfsetdash{}{0pt}%
\pgfpathmoveto{\pgfqpoint{0.580766in}{7.656023in}}%
\pgfpathlineto{\pgfqpoint{6.192411in}{7.656023in}}%
\pgfusepath{stroke}%
\end{pgfscope}%
\begin{pgfscope}%
\pgfsetbuttcap%
\pgfsetroundjoin%
\pgfsetlinewidth{1.003750pt}%
\definecolor{currentstroke}{rgb}{0.000000,0.000000,0.000000}%
\pgfsetstrokecolor{currentstroke}%
\pgfsetdash{}{0pt}%
\pgfpathmoveto{\pgfqpoint{6.192411in}{4.462900in}}%
\pgfpathlineto{\pgfqpoint{6.192411in}{7.656023in}}%
\pgfusepath{stroke}%
\end{pgfscope}%
\begin{pgfscope}%
\pgfsetbuttcap%
\pgfsetroundjoin%
\pgfsetlinewidth{1.003750pt}%
\definecolor{currentstroke}{rgb}{0.000000,0.000000,0.000000}%
\pgfsetstrokecolor{currentstroke}%
\pgfsetdash{}{0pt}%
\pgfpathmoveto{\pgfqpoint{0.580766in}{4.462900in}}%
\pgfpathlineto{\pgfqpoint{6.192411in}{4.462900in}}%
\pgfusepath{stroke}%
\end{pgfscope}%
\begin{pgfscope}%
\pgfsetbuttcap%
\pgfsetroundjoin%
\pgfsetlinewidth{1.003750pt}%
\definecolor{currentstroke}{rgb}{0.000000,0.000000,0.000000}%
\pgfsetstrokecolor{currentstroke}%
\pgfsetdash{}{0pt}%
\pgfpathmoveto{\pgfqpoint{0.580766in}{4.462900in}}%
\pgfpathlineto{\pgfqpoint{0.580766in}{7.656023in}}%
\pgfusepath{stroke}%
\end{pgfscope}%
\begin{pgfscope}%
\pgftext[x=3.386589in,y=7.725467in,,base]{{\rmfamily\fontsize{12.000000}{14.400000}\selectfont Step-By-Step Approximations of \(\displaystyle i(t)\)}}%
\end{pgfscope}%
\begin{pgfscope}%
\pgfsetbuttcap%
\pgfsetroundjoin%
\definecolor{currentfill}{rgb}{0.300000,0.300000,0.300000}%
\pgfsetfillcolor{currentfill}%
\pgfsetfillopacity{0.500000}%
\pgfsetlinewidth{1.003750pt}%
\definecolor{currentstroke}{rgb}{0.300000,0.300000,0.300000}%
\pgfsetstrokecolor{currentstroke}%
\pgfsetstrokeopacity{0.500000}%
\pgfsetdash{}{0pt}%
\pgfpathmoveto{\pgfqpoint{3.610140in}{4.518456in}}%
\pgfpathlineto{\pgfqpoint{6.103523in}{4.518456in}}%
\pgfpathquadraticcurveto{\pgfqpoint{6.136856in}{4.518456in}}{\pgfqpoint{6.136856in}{4.551789in}}%
\pgfpathlineto{\pgfqpoint{6.136856in}{5.267529in}}%
\pgfpathquadraticcurveto{\pgfqpoint{6.136856in}{5.300863in}}{\pgfqpoint{6.103523in}{5.300863in}}%
\pgfpathlineto{\pgfqpoint{3.610140in}{5.300863in}}%
\pgfpathquadraticcurveto{\pgfqpoint{3.576807in}{5.300863in}}{\pgfqpoint{3.576807in}{5.267529in}}%
\pgfpathlineto{\pgfqpoint{3.576807in}{4.551789in}}%
\pgfpathquadraticcurveto{\pgfqpoint{3.576807in}{4.518456in}}{\pgfqpoint{3.610140in}{4.518456in}}%
\pgfpathclose%
\pgfusepath{stroke,fill}%
\end{pgfscope}%
\begin{pgfscope}%
\pgfsetbuttcap%
\pgfsetroundjoin%
\definecolor{currentfill}{rgb}{1.000000,1.000000,1.000000}%
\pgfsetfillcolor{currentfill}%
\pgfsetlinewidth{1.003750pt}%
\definecolor{currentstroke}{rgb}{0.000000,0.000000,0.000000}%
\pgfsetstrokecolor{currentstroke}%
\pgfsetdash{}{0pt}%
\pgfpathmoveto{\pgfqpoint{3.582363in}{4.546233in}}%
\pgfpathlineto{\pgfqpoint{6.075745in}{4.546233in}}%
\pgfpathquadraticcurveto{\pgfqpoint{6.109078in}{4.546233in}}{\pgfqpoint{6.109078in}{4.579567in}}%
\pgfpathlineto{\pgfqpoint{6.109078in}{5.295307in}}%
\pgfpathquadraticcurveto{\pgfqpoint{6.109078in}{5.328641in}}{\pgfqpoint{6.075745in}{5.328641in}}%
\pgfpathlineto{\pgfqpoint{3.582363in}{5.328641in}}%
\pgfpathquadraticcurveto{\pgfqpoint{3.549029in}{5.328641in}}{\pgfqpoint{3.549029in}{5.295307in}}%
\pgfpathlineto{\pgfqpoint{3.549029in}{4.579567in}}%
\pgfpathquadraticcurveto{\pgfqpoint{3.549029in}{4.546233in}}{\pgfqpoint{3.582363in}{4.546233in}}%
\pgfpathclose%
\pgfusepath{stroke,fill}%
\end{pgfscope}%
\begin{pgfscope}%
\pgfsetbuttcap%
\pgfsetroundjoin%
\pgfsetlinewidth{1.003750pt}%
\definecolor{currentstroke}{rgb}{0.000000,0.000000,1.000000}%
\pgfsetstrokecolor{currentstroke}%
\pgfsetdash{{1.000000pt}{3.000000pt}}{0.000000pt}%
\pgfpathmoveto{\pgfqpoint{3.665696in}{5.203641in}}%
\pgfpathlineto{\pgfqpoint{3.899029in}{5.203641in}}%
\pgfusepath{stroke}%
\end{pgfscope}%
\begin{pgfscope}%
\pgftext[x=4.082363in,y=5.145307in,left,base]{{\rmfamily\fontsize{12.000000}{14.400000}\selectfont Continuous}}%
\end{pgfscope}%
\begin{pgfscope}%
\pgfsetrectcap%
\pgfsetroundjoin%
\pgfsetlinewidth{1.003750pt}%
\definecolor{currentstroke}{rgb}{0.000000,0.500000,0.000000}%
\pgfsetstrokecolor{currentstroke}%
\pgfsetdash{}{0pt}%
\pgfpathmoveto{\pgfqpoint{3.665696in}{4.962900in}}%
\pgfpathlineto{\pgfqpoint{3.899029in}{4.962900in}}%
\pgfusepath{stroke}%
\end{pgfscope}%
\begin{pgfscope}%
\pgftext[x=4.082363in,y=4.904567in,left,base]{{\rmfamily\fontsize{12.000000}{14.400000}\selectfont Trapezoidal (\(\displaystyle \Delta{}t = 0.0001 s\))}}%
\end{pgfscope}%
\begin{pgfscope}%
\pgfsetrectcap%
\pgfsetroundjoin%
\pgfsetlinewidth{1.003750pt}%
\definecolor{currentstroke}{rgb}{1.000000,0.000000,0.000000}%
\pgfsetstrokecolor{currentstroke}%
\pgfsetdash{}{0pt}%
\pgfpathmoveto{\pgfqpoint{3.665696in}{4.712900in}}%
\pgfpathlineto{\pgfqpoint{3.899029in}{4.712900in}}%
\pgfusepath{stroke}%
\end{pgfscope}%
\begin{pgfscope}%
\pgftext[x=4.082363in,y=4.654567in,left,base]{{\rmfamily\fontsize{12.000000}{14.400000}\selectfont Trapezoidal (\(\displaystyle \Delta{}t = 0.0008 s\))}}%
\end{pgfscope}%
\begin{pgfscope}%
\pgfsetbuttcap%
\pgfsetroundjoin%
\definecolor{currentfill}{rgb}{1.000000,1.000000,1.000000}%
\pgfsetfillcolor{currentfill}%
\pgfsetlinewidth{0.000000pt}%
\definecolor{currentstroke}{rgb}{0.000000,0.000000,0.000000}%
\pgfsetstrokecolor{currentstroke}%
\pgfsetstrokeopacity{0.000000}%
\pgfsetdash{}{0pt}%
\pgfpathmoveto{\pgfqpoint{0.580766in}{0.532919in}}%
\pgfpathlineto{\pgfqpoint{6.192411in}{0.532919in}}%
\pgfpathlineto{\pgfqpoint{6.192411in}{3.726041in}}%
\pgfpathlineto{\pgfqpoint{0.580766in}{3.726041in}}%
\pgfpathclose%
\pgfusepath{fill}%
\end{pgfscope}%
\begin{pgfscope}%
\pgfpathrectangle{\pgfqpoint{0.580766in}{0.532919in}}{\pgfqpoint{5.611646in}{3.193122in}} %
\pgfusepath{clip}%
\pgfsetbuttcap%
\pgfsetroundjoin%
\pgfsetlinewidth{1.003750pt}%
\definecolor{currentstroke}{rgb}{0.000000,0.000000,1.000000}%
\pgfsetstrokecolor{currentstroke}%
\pgfsetdash{{1.000000pt}{3.000000pt}}{0.000000pt}%
\pgfpathmoveto{\pgfqpoint{0.580766in}{3.726041in}}%
\pgfpathlineto{\pgfqpoint{0.633141in}{3.580452in}}%
\pgfpathlineto{\pgfqpoint{0.685517in}{3.441501in}}%
\pgfpathlineto{\pgfqpoint{0.737892in}{3.308886in}}%
\pgfpathlineto{\pgfqpoint{0.790267in}{3.182317in}}%
\pgfpathlineto{\pgfqpoint{0.842643in}{3.061519in}}%
\pgfpathlineto{\pgfqpoint{0.895018in}{2.946229in}}%
\pgfpathlineto{\pgfqpoint{0.947393in}{2.836195in}}%
\pgfpathlineto{\pgfqpoint{0.999769in}{2.731178in}}%
\pgfpathlineto{\pgfqpoint{1.054015in}{2.627456in}}%
\pgfpathlineto{\pgfqpoint{1.108261in}{2.528627in}}%
\pgfpathlineto{\pgfqpoint{1.162506in}{2.434462in}}%
\pgfpathlineto{\pgfqpoint{1.216752in}{2.344740in}}%
\pgfpathlineto{\pgfqpoint{1.270998in}{2.259251in}}%
\pgfpathlineto{\pgfqpoint{1.325244in}{2.177796in}}%
\pgfpathlineto{\pgfqpoint{1.381361in}{2.097574in}}%
\pgfpathlineto{\pgfqpoint{1.437477in}{2.021265in}}%
\pgfpathlineto{\pgfqpoint{1.493594in}{1.948678in}}%
\pgfpathlineto{\pgfqpoint{1.549710in}{1.879630in}}%
\pgfpathlineto{\pgfqpoint{1.607697in}{1.811817in}}%
\pgfpathlineto{\pgfqpoint{1.665684in}{1.747419in}}%
\pgfpathlineto{\pgfqpoint{1.723671in}{1.686263in}}%
\pgfpathlineto{\pgfqpoint{1.783529in}{1.626363in}}%
\pgfpathlineto{\pgfqpoint{1.843386in}{1.569574in}}%
\pgfpathlineto{\pgfqpoint{1.905114in}{1.514097in}}%
\pgfpathlineto{\pgfqpoint{1.966842in}{1.461590in}}%
\pgfpathlineto{\pgfqpoint{2.030441in}{1.410428in}}%
\pgfpathlineto{\pgfqpoint{2.094040in}{1.362085in}}%
\pgfpathlineto{\pgfqpoint{2.159509in}{1.315101in}}%
\pgfpathlineto{\pgfqpoint{2.226849in}{1.269550in}}%
\pgfpathlineto{\pgfqpoint{2.294188in}{1.226652in}}%
\pgfpathlineto{\pgfqpoint{2.363399in}{1.185164in}}%
\pgfpathlineto{\pgfqpoint{2.434479in}{1.145136in}}%
\pgfpathlineto{\pgfqpoint{2.507431in}{1.106608in}}%
\pgfpathlineto{\pgfqpoint{2.582253in}{1.069609in}}%
\pgfpathlineto{\pgfqpoint{2.658945in}{1.034160in}}%
\pgfpathlineto{\pgfqpoint{2.737508in}{1.000273in}}%
\pgfpathlineto{\pgfqpoint{2.819812in}{0.967227in}}%
\pgfpathlineto{\pgfqpoint{2.903987in}{0.935845in}}%
\pgfpathlineto{\pgfqpoint{2.991903in}{0.905487in}}%
\pgfpathlineto{\pgfqpoint{3.081689in}{0.876843in}}%
\pgfpathlineto{\pgfqpoint{3.175217in}{0.849344in}}%
\pgfpathlineto{\pgfqpoint{3.272485in}{0.823075in}}%
\pgfpathlineto{\pgfqpoint{3.373495in}{0.798102in}}%
\pgfpathlineto{\pgfqpoint{3.480116in}{0.774069in}}%
\pgfpathlineto{\pgfqpoint{3.590478in}{0.751484in}}%
\pgfpathlineto{\pgfqpoint{3.706452in}{0.730027in}}%
\pgfpathlineto{\pgfqpoint{3.828038in}{0.709790in}}%
\pgfpathlineto{\pgfqpoint{3.957106in}{0.690575in}}%
\pgfpathlineto{\pgfqpoint{4.093656in}{0.672515in}}%
\pgfpathlineto{\pgfqpoint{4.239559in}{0.655497in}}%
\pgfpathlineto{\pgfqpoint{4.394814in}{0.639661in}}%
\pgfpathlineto{\pgfqpoint{4.561293in}{0.624946in}}%
\pgfpathlineto{\pgfqpoint{4.740866in}{0.611339in}}%
\pgfpathlineto{\pgfqpoint{4.935403in}{0.598859in}}%
\pgfpathlineto{\pgfqpoint{5.148645in}{0.587449in}}%
\pgfpathlineto{\pgfqpoint{5.382464in}{0.577194in}}%
\pgfpathlineto{\pgfqpoint{5.644341in}{0.567979in}}%
\pgfpathlineto{\pgfqpoint{5.939887in}{0.559862in}}%
\pgfpathlineto{\pgfqpoint{6.190541in}{0.554469in}}%
\pgfpathlineto{\pgfqpoint{6.190541in}{0.554469in}}%
\pgfusepath{stroke}%
\end{pgfscope}%
\begin{pgfscope}%
\pgfpathrectangle{\pgfqpoint{0.580766in}{0.532919in}}{\pgfqpoint{5.611646in}{3.193122in}} %
\pgfusepath{clip}%
\pgfsetrectcap%
\pgfsetroundjoin%
\pgfsetlinewidth{1.003750pt}%
\definecolor{currentstroke}{rgb}{0.000000,0.500000,0.000000}%
\pgfsetstrokecolor{currentstroke}%
\pgfsetdash{}{0pt}%
\pgfpathmoveto{\pgfqpoint{0.580766in}{3.570279in}}%
\pgfpathlineto{\pgfqpoint{0.636882in}{3.422115in}}%
\pgfpathlineto{\pgfqpoint{0.692999in}{3.281179in}}%
\pgfpathlineto{\pgfqpoint{0.749115in}{3.147117in}}%
\pgfpathlineto{\pgfqpoint{0.805232in}{3.019595in}}%
\pgfpathlineto{\pgfqpoint{0.861348in}{2.898294in}}%
\pgfpathlineto{\pgfqpoint{0.917465in}{2.782910in}}%
\pgfpathlineto{\pgfqpoint{0.973581in}{2.673154in}}%
\pgfpathlineto{\pgfqpoint{1.029698in}{2.568752in}}%
\pgfpathlineto{\pgfqpoint{1.085814in}{2.469443in}}%
\pgfpathlineto{\pgfqpoint{1.141930in}{2.374979in}}%
\pgfpathlineto{\pgfqpoint{1.198047in}{2.285122in}}%
\pgfpathlineto{\pgfqpoint{1.254163in}{2.199649in}}%
\pgfpathlineto{\pgfqpoint{1.310280in}{2.118345in}}%
\pgfpathlineto{\pgfqpoint{1.366396in}{2.041007in}}%
\pgfpathlineto{\pgfqpoint{1.422513in}{1.967442in}}%
\pgfpathlineto{\pgfqpoint{1.478629in}{1.897465in}}%
\pgfpathlineto{\pgfqpoint{1.534746in}{1.830902in}}%
\pgfpathlineto{\pgfqpoint{1.590862in}{1.767585in}}%
\pgfpathlineto{\pgfqpoint{1.646979in}{1.707358in}}%
\pgfpathlineto{\pgfqpoint{1.703095in}{1.650068in}}%
\pgfpathlineto{\pgfqpoint{1.759211in}{1.595573in}}%
\pgfpathlineto{\pgfqpoint{1.815328in}{1.543736in}}%
\pgfpathlineto{\pgfqpoint{1.871444in}{1.494428in}}%
\pgfpathlineto{\pgfqpoint{1.927561in}{1.447525in}}%
\pgfpathlineto{\pgfqpoint{1.983677in}{1.402910in}}%
\pgfpathlineto{\pgfqpoint{2.039794in}{1.360472in}}%
\pgfpathlineto{\pgfqpoint{2.095910in}{1.320103in}}%
\pgfpathlineto{\pgfqpoint{2.152027in}{1.281704in}}%
\pgfpathlineto{\pgfqpoint{2.208143in}{1.245178in}}%
\pgfpathlineto{\pgfqpoint{2.264260in}{1.210433in}}%
\pgfpathlineto{\pgfqpoint{2.320376in}{1.177384in}}%
\pgfpathlineto{\pgfqpoint{2.376492in}{1.145947in}}%
\pgfpathlineto{\pgfqpoint{2.432609in}{1.116043in}}%
\pgfpathlineto{\pgfqpoint{2.488725in}{1.087598in}}%
\pgfpathlineto{\pgfqpoint{2.544842in}{1.060540in}}%
\pgfpathlineto{\pgfqpoint{2.600958in}{1.034802in}}%
\pgfpathlineto{\pgfqpoint{2.657075in}{1.010320in}}%
\pgfpathlineto{\pgfqpoint{2.713191in}{0.987032in}}%
\pgfpathlineto{\pgfqpoint{2.769308in}{0.964881in}}%
\pgfpathlineto{\pgfqpoint{2.825424in}{0.943809in}}%
\pgfpathlineto{\pgfqpoint{2.881541in}{0.923766in}}%
\pgfpathlineto{\pgfqpoint{2.937657in}{0.904700in}}%
\pgfpathlineto{\pgfqpoint{2.993773in}{0.886564in}}%
\pgfpathlineto{\pgfqpoint{3.049890in}{0.869313in}}%
\pgfpathlineto{\pgfqpoint{3.106006in}{0.852904in}}%
\pgfpathlineto{\pgfqpoint{3.162123in}{0.837295in}}%
\pgfpathlineto{\pgfqpoint{3.218239in}{0.822447in}}%
\pgfpathlineto{\pgfqpoint{3.274356in}{0.808324in}}%
\pgfpathlineto{\pgfqpoint{3.330472in}{0.794889in}}%
\pgfpathlineto{\pgfqpoint{3.386589in}{0.782110in}}%
\pgfpathlineto{\pgfqpoint{3.442705in}{0.769955in}}%
\pgfpathlineto{\pgfqpoint{3.498822in}{0.758392in}}%
\pgfpathlineto{\pgfqpoint{3.554938in}{0.747393in}}%
\pgfpathlineto{\pgfqpoint{3.611055in}{0.736931in}}%
\pgfpathlineto{\pgfqpoint{3.667171in}{0.726979in}}%
\pgfpathlineto{\pgfqpoint{3.723287in}{0.717513in}}%
\pgfpathlineto{\pgfqpoint{3.779404in}{0.708508in}}%
\pgfpathlineto{\pgfqpoint{3.835520in}{0.699943in}}%
\pgfpathlineto{\pgfqpoint{3.891637in}{0.691795in}}%
\pgfpathlineto{\pgfqpoint{3.947753in}{0.684045in}}%
\pgfpathlineto{\pgfqpoint{4.003870in}{0.676673in}}%
\pgfpathlineto{\pgfqpoint{4.059986in}{0.669661in}}%
\pgfpathlineto{\pgfqpoint{4.116103in}{0.662990in}}%
\pgfpathlineto{\pgfqpoint{4.172219in}{0.656645in}}%
\pgfpathlineto{\pgfqpoint{4.228336in}{0.650610in}}%
\pgfpathlineto{\pgfqpoint{4.284452in}{0.644869in}}%
\pgfpathlineto{\pgfqpoint{4.340568in}{0.639408in}}%
\pgfpathlineto{\pgfqpoint{4.396685in}{0.634213in}}%
\pgfpathlineto{\pgfqpoint{4.452801in}{0.629272in}}%
\pgfpathlineto{\pgfqpoint{4.508918in}{0.624572in}}%
\pgfpathlineto{\pgfqpoint{4.565034in}{0.620101in}}%
\pgfpathlineto{\pgfqpoint{4.621151in}{0.615848in}}%
\pgfpathlineto{\pgfqpoint{4.677267in}{0.611803in}}%
\pgfpathlineto{\pgfqpoint{4.733384in}{0.607955in}}%
\pgfpathlineto{\pgfqpoint{4.789500in}{0.604295in}}%
\pgfpathlineto{\pgfqpoint{4.845617in}{0.600813in}}%
\pgfpathlineto{\pgfqpoint{4.901733in}{0.597501in}}%
\pgfpathlineto{\pgfqpoint{4.957849in}{0.594351in}}%
\pgfpathlineto{\pgfqpoint{5.013966in}{0.591354in}}%
\pgfpathlineto{\pgfqpoint{5.070082in}{0.588503in}}%
\pgfpathlineto{\pgfqpoint{5.126199in}{0.585792in}}%
\pgfpathlineto{\pgfqpoint{5.182315in}{0.583213in}}%
\pgfpathlineto{\pgfqpoint{5.238432in}{0.580759in}}%
\pgfpathlineto{\pgfqpoint{5.294548in}{0.578426in}}%
\pgfpathlineto{\pgfqpoint{5.350665in}{0.576206in}}%
\pgfpathlineto{\pgfqpoint{5.406781in}{0.574094in}}%
\pgfpathlineto{\pgfqpoint{5.462898in}{0.572086in}}%
\pgfpathlineto{\pgfqpoint{5.519014in}{0.570175in}}%
\pgfpathlineto{\pgfqpoint{5.575130in}{0.568358in}}%
\pgfpathlineto{\pgfqpoint{5.631247in}{0.566629in}}%
\pgfpathlineto{\pgfqpoint{5.687363in}{0.564984in}}%
\pgfpathlineto{\pgfqpoint{5.743480in}{0.563420in}}%
\pgfpathlineto{\pgfqpoint{5.799596in}{0.561932in}}%
\pgfpathlineto{\pgfqpoint{5.855713in}{0.560517in}}%
\pgfpathlineto{\pgfqpoint{5.911829in}{0.559171in}}%
\pgfpathlineto{\pgfqpoint{5.967946in}{0.557890in}}%
\pgfpathlineto{\pgfqpoint{6.024062in}{0.556672in}}%
\pgfpathlineto{\pgfqpoint{6.080179in}{0.555513in}}%
\pgfpathlineto{\pgfqpoint{6.136295in}{0.554411in}}%
\pgfusepath{stroke}%
\end{pgfscope}%
\begin{pgfscope}%
\pgfpathrectangle{\pgfqpoint{0.580766in}{0.532919in}}{\pgfqpoint{5.611646in}{3.193122in}} %
\pgfusepath{clip}%
\pgfsetrectcap%
\pgfsetroundjoin%
\pgfsetlinewidth{1.003750pt}%
\definecolor{currentstroke}{rgb}{1.000000,0.000000,0.000000}%
\pgfsetstrokecolor{currentstroke}%
\pgfsetdash{}{0pt}%
\pgfpathmoveto{\pgfqpoint{0.580766in}{2.661667in}}%
\pgfpathlineto{\pgfqpoint{1.029698in}{1.952084in}}%
\pgfpathlineto{\pgfqpoint{1.478629in}{1.479029in}}%
\pgfpathlineto{\pgfqpoint{1.927561in}{1.163659in}}%
\pgfpathlineto{\pgfqpoint{2.376492in}{0.953412in}}%
\pgfpathlineto{\pgfqpoint{2.825424in}{0.813247in}}%
\pgfpathlineto{\pgfqpoint{3.274356in}{0.719805in}}%
\pgfpathlineto{\pgfqpoint{3.723287in}{0.657509in}}%
\pgfpathlineto{\pgfqpoint{4.172219in}{0.615979in}}%
\pgfpathlineto{\pgfqpoint{4.621151in}{0.588292in}}%
\pgfpathlineto{\pgfqpoint{5.070082in}{0.569834in}}%
\pgfpathlineto{\pgfqpoint{5.519014in}{0.557529in}}%
\pgfusepath{stroke}%
\end{pgfscope}%
\begin{pgfscope}%
\pgfpathrectangle{\pgfqpoint{0.580766in}{0.532919in}}{\pgfqpoint{5.611646in}{3.193122in}} %
\pgfusepath{clip}%
\pgfsetbuttcap%
\pgfsetroundjoin%
\pgfsetlinewidth{0.501875pt}%
\definecolor{currentstroke}{rgb}{0.000000,0.000000,0.000000}%
\pgfsetstrokecolor{currentstroke}%
\pgfsetdash{{1.000000pt}{3.000000pt}}{0.000000pt}%
\pgfpathmoveto{\pgfqpoint{0.580766in}{0.532919in}}%
\pgfpathlineto{\pgfqpoint{0.580766in}{3.726041in}}%
\pgfusepath{stroke}%
\end{pgfscope}%
\begin{pgfscope}%
\pgfsetbuttcap%
\pgfsetroundjoin%
\definecolor{currentfill}{rgb}{0.000000,0.000000,0.000000}%
\pgfsetfillcolor{currentfill}%
\pgfsetlinewidth{0.501875pt}%
\definecolor{currentstroke}{rgb}{0.000000,0.000000,0.000000}%
\pgfsetstrokecolor{currentstroke}%
\pgfsetdash{}{0pt}%
\pgfsys@defobject{currentmarker}{\pgfqpoint{0.000000in}{0.000000in}}{\pgfqpoint{0.000000in}{0.055556in}}{%
\pgfpathmoveto{\pgfqpoint{0.000000in}{0.000000in}}%
\pgfpathlineto{\pgfqpoint{0.000000in}{0.055556in}}%
\pgfusepath{stroke,fill}%
}%
\begin{pgfscope}%
\pgfsys@transformshift{0.580766in}{0.532919in}%
\pgfsys@useobject{currentmarker}{}%
\end{pgfscope}%
\end{pgfscope}%
\begin{pgfscope}%
\pgfsetbuttcap%
\pgfsetroundjoin%
\definecolor{currentfill}{rgb}{0.000000,0.000000,0.000000}%
\pgfsetfillcolor{currentfill}%
\pgfsetlinewidth{0.501875pt}%
\definecolor{currentstroke}{rgb}{0.000000,0.000000,0.000000}%
\pgfsetstrokecolor{currentstroke}%
\pgfsetdash{}{0pt}%
\pgfsys@defobject{currentmarker}{\pgfqpoint{0.000000in}{-0.055556in}}{\pgfqpoint{0.000000in}{0.000000in}}{%
\pgfpathmoveto{\pgfqpoint{0.000000in}{0.000000in}}%
\pgfpathlineto{\pgfqpoint{0.000000in}{-0.055556in}}%
\pgfusepath{stroke,fill}%
}%
\begin{pgfscope}%
\pgfsys@transformshift{0.580766in}{3.726041in}%
\pgfsys@useobject{currentmarker}{}%
\end{pgfscope}%
\end{pgfscope}%
\begin{pgfscope}%
\pgftext[x=0.580766in,y=0.477363in,,top]{{\rmfamily\fontsize{10.000000}{12.000000}\selectfont \(\displaystyle 0.000\)}}%
\end{pgfscope}%
\begin{pgfscope}%
\pgfpathrectangle{\pgfqpoint{0.580766in}{0.532919in}}{\pgfqpoint{5.611646in}{3.193122in}} %
\pgfusepath{clip}%
\pgfsetbuttcap%
\pgfsetroundjoin%
\pgfsetlinewidth{0.501875pt}%
\definecolor{currentstroke}{rgb}{0.000000,0.000000,0.000000}%
\pgfsetstrokecolor{currentstroke}%
\pgfsetdash{{1.000000pt}{3.000000pt}}{0.000000pt}%
\pgfpathmoveto{\pgfqpoint{1.703095in}{0.532919in}}%
\pgfpathlineto{\pgfqpoint{1.703095in}{3.726041in}}%
\pgfusepath{stroke}%
\end{pgfscope}%
\begin{pgfscope}%
\pgfsetbuttcap%
\pgfsetroundjoin%
\definecolor{currentfill}{rgb}{0.000000,0.000000,0.000000}%
\pgfsetfillcolor{currentfill}%
\pgfsetlinewidth{0.501875pt}%
\definecolor{currentstroke}{rgb}{0.000000,0.000000,0.000000}%
\pgfsetstrokecolor{currentstroke}%
\pgfsetdash{}{0pt}%
\pgfsys@defobject{currentmarker}{\pgfqpoint{0.000000in}{0.000000in}}{\pgfqpoint{0.000000in}{0.055556in}}{%
\pgfpathmoveto{\pgfqpoint{0.000000in}{0.000000in}}%
\pgfpathlineto{\pgfqpoint{0.000000in}{0.055556in}}%
\pgfusepath{stroke,fill}%
}%
\begin{pgfscope}%
\pgfsys@transformshift{1.703095in}{0.532919in}%
\pgfsys@useobject{currentmarker}{}%
\end{pgfscope}%
\end{pgfscope}%
\begin{pgfscope}%
\pgfsetbuttcap%
\pgfsetroundjoin%
\definecolor{currentfill}{rgb}{0.000000,0.000000,0.000000}%
\pgfsetfillcolor{currentfill}%
\pgfsetlinewidth{0.501875pt}%
\definecolor{currentstroke}{rgb}{0.000000,0.000000,0.000000}%
\pgfsetstrokecolor{currentstroke}%
\pgfsetdash{}{0pt}%
\pgfsys@defobject{currentmarker}{\pgfqpoint{0.000000in}{-0.055556in}}{\pgfqpoint{0.000000in}{0.000000in}}{%
\pgfpathmoveto{\pgfqpoint{0.000000in}{0.000000in}}%
\pgfpathlineto{\pgfqpoint{0.000000in}{-0.055556in}}%
\pgfusepath{stroke,fill}%
}%
\begin{pgfscope}%
\pgfsys@transformshift{1.703095in}{3.726041in}%
\pgfsys@useobject{currentmarker}{}%
\end{pgfscope}%
\end{pgfscope}%
\begin{pgfscope}%
\pgftext[x=1.703095in,y=0.477363in,,top]{{\rmfamily\fontsize{10.000000}{12.000000}\selectfont \(\displaystyle 0.002\)}}%
\end{pgfscope}%
\begin{pgfscope}%
\pgfpathrectangle{\pgfqpoint{0.580766in}{0.532919in}}{\pgfqpoint{5.611646in}{3.193122in}} %
\pgfusepath{clip}%
\pgfsetbuttcap%
\pgfsetroundjoin%
\pgfsetlinewidth{0.501875pt}%
\definecolor{currentstroke}{rgb}{0.000000,0.000000,0.000000}%
\pgfsetstrokecolor{currentstroke}%
\pgfsetdash{{1.000000pt}{3.000000pt}}{0.000000pt}%
\pgfpathmoveto{\pgfqpoint{2.825424in}{0.532919in}}%
\pgfpathlineto{\pgfqpoint{2.825424in}{3.726041in}}%
\pgfusepath{stroke}%
\end{pgfscope}%
\begin{pgfscope}%
\pgfsetbuttcap%
\pgfsetroundjoin%
\definecolor{currentfill}{rgb}{0.000000,0.000000,0.000000}%
\pgfsetfillcolor{currentfill}%
\pgfsetlinewidth{0.501875pt}%
\definecolor{currentstroke}{rgb}{0.000000,0.000000,0.000000}%
\pgfsetstrokecolor{currentstroke}%
\pgfsetdash{}{0pt}%
\pgfsys@defobject{currentmarker}{\pgfqpoint{0.000000in}{0.000000in}}{\pgfqpoint{0.000000in}{0.055556in}}{%
\pgfpathmoveto{\pgfqpoint{0.000000in}{0.000000in}}%
\pgfpathlineto{\pgfqpoint{0.000000in}{0.055556in}}%
\pgfusepath{stroke,fill}%
}%
\begin{pgfscope}%
\pgfsys@transformshift{2.825424in}{0.532919in}%
\pgfsys@useobject{currentmarker}{}%
\end{pgfscope}%
\end{pgfscope}%
\begin{pgfscope}%
\pgfsetbuttcap%
\pgfsetroundjoin%
\definecolor{currentfill}{rgb}{0.000000,0.000000,0.000000}%
\pgfsetfillcolor{currentfill}%
\pgfsetlinewidth{0.501875pt}%
\definecolor{currentstroke}{rgb}{0.000000,0.000000,0.000000}%
\pgfsetstrokecolor{currentstroke}%
\pgfsetdash{}{0pt}%
\pgfsys@defobject{currentmarker}{\pgfqpoint{0.000000in}{-0.055556in}}{\pgfqpoint{0.000000in}{0.000000in}}{%
\pgfpathmoveto{\pgfqpoint{0.000000in}{0.000000in}}%
\pgfpathlineto{\pgfqpoint{0.000000in}{-0.055556in}}%
\pgfusepath{stroke,fill}%
}%
\begin{pgfscope}%
\pgfsys@transformshift{2.825424in}{3.726041in}%
\pgfsys@useobject{currentmarker}{}%
\end{pgfscope}%
\end{pgfscope}%
\begin{pgfscope}%
\pgftext[x=2.825424in,y=0.477363in,,top]{{\rmfamily\fontsize{10.000000}{12.000000}\selectfont \(\displaystyle 0.004\)}}%
\end{pgfscope}%
\begin{pgfscope}%
\pgfpathrectangle{\pgfqpoint{0.580766in}{0.532919in}}{\pgfqpoint{5.611646in}{3.193122in}} %
\pgfusepath{clip}%
\pgfsetbuttcap%
\pgfsetroundjoin%
\pgfsetlinewidth{0.501875pt}%
\definecolor{currentstroke}{rgb}{0.000000,0.000000,0.000000}%
\pgfsetstrokecolor{currentstroke}%
\pgfsetdash{{1.000000pt}{3.000000pt}}{0.000000pt}%
\pgfpathmoveto{\pgfqpoint{3.947753in}{0.532919in}}%
\pgfpathlineto{\pgfqpoint{3.947753in}{3.726041in}}%
\pgfusepath{stroke}%
\end{pgfscope}%
\begin{pgfscope}%
\pgfsetbuttcap%
\pgfsetroundjoin%
\definecolor{currentfill}{rgb}{0.000000,0.000000,0.000000}%
\pgfsetfillcolor{currentfill}%
\pgfsetlinewidth{0.501875pt}%
\definecolor{currentstroke}{rgb}{0.000000,0.000000,0.000000}%
\pgfsetstrokecolor{currentstroke}%
\pgfsetdash{}{0pt}%
\pgfsys@defobject{currentmarker}{\pgfqpoint{0.000000in}{0.000000in}}{\pgfqpoint{0.000000in}{0.055556in}}{%
\pgfpathmoveto{\pgfqpoint{0.000000in}{0.000000in}}%
\pgfpathlineto{\pgfqpoint{0.000000in}{0.055556in}}%
\pgfusepath{stroke,fill}%
}%
\begin{pgfscope}%
\pgfsys@transformshift{3.947753in}{0.532919in}%
\pgfsys@useobject{currentmarker}{}%
\end{pgfscope}%
\end{pgfscope}%
\begin{pgfscope}%
\pgfsetbuttcap%
\pgfsetroundjoin%
\definecolor{currentfill}{rgb}{0.000000,0.000000,0.000000}%
\pgfsetfillcolor{currentfill}%
\pgfsetlinewidth{0.501875pt}%
\definecolor{currentstroke}{rgb}{0.000000,0.000000,0.000000}%
\pgfsetstrokecolor{currentstroke}%
\pgfsetdash{}{0pt}%
\pgfsys@defobject{currentmarker}{\pgfqpoint{0.000000in}{-0.055556in}}{\pgfqpoint{0.000000in}{0.000000in}}{%
\pgfpathmoveto{\pgfqpoint{0.000000in}{0.000000in}}%
\pgfpathlineto{\pgfqpoint{0.000000in}{-0.055556in}}%
\pgfusepath{stroke,fill}%
}%
\begin{pgfscope}%
\pgfsys@transformshift{3.947753in}{3.726041in}%
\pgfsys@useobject{currentmarker}{}%
\end{pgfscope}%
\end{pgfscope}%
\begin{pgfscope}%
\pgftext[x=3.947753in,y=0.477363in,,top]{{\rmfamily\fontsize{10.000000}{12.000000}\selectfont \(\displaystyle 0.006\)}}%
\end{pgfscope}%
\begin{pgfscope}%
\pgfpathrectangle{\pgfqpoint{0.580766in}{0.532919in}}{\pgfqpoint{5.611646in}{3.193122in}} %
\pgfusepath{clip}%
\pgfsetbuttcap%
\pgfsetroundjoin%
\pgfsetlinewidth{0.501875pt}%
\definecolor{currentstroke}{rgb}{0.000000,0.000000,0.000000}%
\pgfsetstrokecolor{currentstroke}%
\pgfsetdash{{1.000000pt}{3.000000pt}}{0.000000pt}%
\pgfpathmoveto{\pgfqpoint{5.070082in}{0.532919in}}%
\pgfpathlineto{\pgfqpoint{5.070082in}{3.726041in}}%
\pgfusepath{stroke}%
\end{pgfscope}%
\begin{pgfscope}%
\pgfsetbuttcap%
\pgfsetroundjoin%
\definecolor{currentfill}{rgb}{0.000000,0.000000,0.000000}%
\pgfsetfillcolor{currentfill}%
\pgfsetlinewidth{0.501875pt}%
\definecolor{currentstroke}{rgb}{0.000000,0.000000,0.000000}%
\pgfsetstrokecolor{currentstroke}%
\pgfsetdash{}{0pt}%
\pgfsys@defobject{currentmarker}{\pgfqpoint{0.000000in}{0.000000in}}{\pgfqpoint{0.000000in}{0.055556in}}{%
\pgfpathmoveto{\pgfqpoint{0.000000in}{0.000000in}}%
\pgfpathlineto{\pgfqpoint{0.000000in}{0.055556in}}%
\pgfusepath{stroke,fill}%
}%
\begin{pgfscope}%
\pgfsys@transformshift{5.070082in}{0.532919in}%
\pgfsys@useobject{currentmarker}{}%
\end{pgfscope}%
\end{pgfscope}%
\begin{pgfscope}%
\pgfsetbuttcap%
\pgfsetroundjoin%
\definecolor{currentfill}{rgb}{0.000000,0.000000,0.000000}%
\pgfsetfillcolor{currentfill}%
\pgfsetlinewidth{0.501875pt}%
\definecolor{currentstroke}{rgb}{0.000000,0.000000,0.000000}%
\pgfsetstrokecolor{currentstroke}%
\pgfsetdash{}{0pt}%
\pgfsys@defobject{currentmarker}{\pgfqpoint{0.000000in}{-0.055556in}}{\pgfqpoint{0.000000in}{0.000000in}}{%
\pgfpathmoveto{\pgfqpoint{0.000000in}{0.000000in}}%
\pgfpathlineto{\pgfqpoint{0.000000in}{-0.055556in}}%
\pgfusepath{stroke,fill}%
}%
\begin{pgfscope}%
\pgfsys@transformshift{5.070082in}{3.726041in}%
\pgfsys@useobject{currentmarker}{}%
\end{pgfscope}%
\end{pgfscope}%
\begin{pgfscope}%
\pgftext[x=5.070082in,y=0.477363in,,top]{{\rmfamily\fontsize{10.000000}{12.000000}\selectfont \(\displaystyle 0.008\)}}%
\end{pgfscope}%
\begin{pgfscope}%
\pgfpathrectangle{\pgfqpoint{0.580766in}{0.532919in}}{\pgfqpoint{5.611646in}{3.193122in}} %
\pgfusepath{clip}%
\pgfsetbuttcap%
\pgfsetroundjoin%
\pgfsetlinewidth{0.501875pt}%
\definecolor{currentstroke}{rgb}{0.000000,0.000000,0.000000}%
\pgfsetstrokecolor{currentstroke}%
\pgfsetdash{{1.000000pt}{3.000000pt}}{0.000000pt}%
\pgfpathmoveto{\pgfqpoint{6.192411in}{0.532919in}}%
\pgfpathlineto{\pgfqpoint{6.192411in}{3.726041in}}%
\pgfusepath{stroke}%
\end{pgfscope}%
\begin{pgfscope}%
\pgfsetbuttcap%
\pgfsetroundjoin%
\definecolor{currentfill}{rgb}{0.000000,0.000000,0.000000}%
\pgfsetfillcolor{currentfill}%
\pgfsetlinewidth{0.501875pt}%
\definecolor{currentstroke}{rgb}{0.000000,0.000000,0.000000}%
\pgfsetstrokecolor{currentstroke}%
\pgfsetdash{}{0pt}%
\pgfsys@defobject{currentmarker}{\pgfqpoint{0.000000in}{0.000000in}}{\pgfqpoint{0.000000in}{0.055556in}}{%
\pgfpathmoveto{\pgfqpoint{0.000000in}{0.000000in}}%
\pgfpathlineto{\pgfqpoint{0.000000in}{0.055556in}}%
\pgfusepath{stroke,fill}%
}%
\begin{pgfscope}%
\pgfsys@transformshift{6.192411in}{0.532919in}%
\pgfsys@useobject{currentmarker}{}%
\end{pgfscope}%
\end{pgfscope}%
\begin{pgfscope}%
\pgfsetbuttcap%
\pgfsetroundjoin%
\definecolor{currentfill}{rgb}{0.000000,0.000000,0.000000}%
\pgfsetfillcolor{currentfill}%
\pgfsetlinewidth{0.501875pt}%
\definecolor{currentstroke}{rgb}{0.000000,0.000000,0.000000}%
\pgfsetstrokecolor{currentstroke}%
\pgfsetdash{}{0pt}%
\pgfsys@defobject{currentmarker}{\pgfqpoint{0.000000in}{-0.055556in}}{\pgfqpoint{0.000000in}{0.000000in}}{%
\pgfpathmoveto{\pgfqpoint{0.000000in}{0.000000in}}%
\pgfpathlineto{\pgfqpoint{0.000000in}{-0.055556in}}%
\pgfusepath{stroke,fill}%
}%
\begin{pgfscope}%
\pgfsys@transformshift{6.192411in}{3.726041in}%
\pgfsys@useobject{currentmarker}{}%
\end{pgfscope}%
\end{pgfscope}%
\begin{pgfscope}%
\pgftext[x=6.192411in,y=0.477363in,,top]{{\rmfamily\fontsize{10.000000}{12.000000}\selectfont \(\displaystyle 0.010\)}}%
\end{pgfscope}%
\begin{pgfscope}%
\pgftext[x=3.386589in,y=0.284462in,,top]{{\rmfamily\fontsize{10.000000}{12.000000}\selectfont time (s)}}%
\end{pgfscope}%
\begin{pgfscope}%
\pgfpathrectangle{\pgfqpoint{0.580766in}{0.532919in}}{\pgfqpoint{5.611646in}{3.193122in}} %
\pgfusepath{clip}%
\pgfsetbuttcap%
\pgfsetroundjoin%
\pgfsetlinewidth{0.501875pt}%
\definecolor{currentstroke}{rgb}{0.000000,0.000000,0.000000}%
\pgfsetstrokecolor{currentstroke}%
\pgfsetdash{{1.000000pt}{3.000000pt}}{0.000000pt}%
\pgfpathmoveto{\pgfqpoint{0.580766in}{0.532919in}}%
\pgfpathlineto{\pgfqpoint{6.192411in}{0.532919in}}%
\pgfusepath{stroke}%
\end{pgfscope}%
\begin{pgfscope}%
\pgfsetbuttcap%
\pgfsetroundjoin%
\definecolor{currentfill}{rgb}{0.000000,0.000000,0.000000}%
\pgfsetfillcolor{currentfill}%
\pgfsetlinewidth{0.501875pt}%
\definecolor{currentstroke}{rgb}{0.000000,0.000000,0.000000}%
\pgfsetstrokecolor{currentstroke}%
\pgfsetdash{}{0pt}%
\pgfsys@defobject{currentmarker}{\pgfqpoint{0.000000in}{0.000000in}}{\pgfqpoint{0.055556in}{0.000000in}}{%
\pgfpathmoveto{\pgfqpoint{0.000000in}{0.000000in}}%
\pgfpathlineto{\pgfqpoint{0.055556in}{0.000000in}}%
\pgfusepath{stroke,fill}%
}%
\begin{pgfscope}%
\pgfsys@transformshift{0.580766in}{0.532919in}%
\pgfsys@useobject{currentmarker}{}%
\end{pgfscope}%
\end{pgfscope}%
\begin{pgfscope}%
\pgfsetbuttcap%
\pgfsetroundjoin%
\definecolor{currentfill}{rgb}{0.000000,0.000000,0.000000}%
\pgfsetfillcolor{currentfill}%
\pgfsetlinewidth{0.501875pt}%
\definecolor{currentstroke}{rgb}{0.000000,0.000000,0.000000}%
\pgfsetstrokecolor{currentstroke}%
\pgfsetdash{}{0pt}%
\pgfsys@defobject{currentmarker}{\pgfqpoint{-0.055556in}{0.000000in}}{\pgfqpoint{0.000000in}{0.000000in}}{%
\pgfpathmoveto{\pgfqpoint{0.000000in}{0.000000in}}%
\pgfpathlineto{\pgfqpoint{-0.055556in}{0.000000in}}%
\pgfusepath{stroke,fill}%
}%
\begin{pgfscope}%
\pgfsys@transformshift{6.192411in}{0.532919in}%
\pgfsys@useobject{currentmarker}{}%
\end{pgfscope}%
\end{pgfscope}%
\begin{pgfscope}%
\pgftext[x=0.525210in,y=0.532919in,right,]{{\rmfamily\fontsize{10.000000}{12.000000}\selectfont \(\displaystyle 0\)}}%
\end{pgfscope}%
\begin{pgfscope}%
\pgfpathrectangle{\pgfqpoint{0.580766in}{0.532919in}}{\pgfqpoint{5.611646in}{3.193122in}} %
\pgfusepath{clip}%
\pgfsetbuttcap%
\pgfsetroundjoin%
\pgfsetlinewidth{0.501875pt}%
\definecolor{currentstroke}{rgb}{0.000000,0.000000,0.000000}%
\pgfsetstrokecolor{currentstroke}%
\pgfsetdash{{1.000000pt}{3.000000pt}}{0.000000pt}%
\pgfpathmoveto{\pgfqpoint{0.580766in}{1.171543in}}%
\pgfpathlineto{\pgfqpoint{6.192411in}{1.171543in}}%
\pgfusepath{stroke}%
\end{pgfscope}%
\begin{pgfscope}%
\pgfsetbuttcap%
\pgfsetroundjoin%
\definecolor{currentfill}{rgb}{0.000000,0.000000,0.000000}%
\pgfsetfillcolor{currentfill}%
\pgfsetlinewidth{0.501875pt}%
\definecolor{currentstroke}{rgb}{0.000000,0.000000,0.000000}%
\pgfsetstrokecolor{currentstroke}%
\pgfsetdash{}{0pt}%
\pgfsys@defobject{currentmarker}{\pgfqpoint{0.000000in}{0.000000in}}{\pgfqpoint{0.055556in}{0.000000in}}{%
\pgfpathmoveto{\pgfqpoint{0.000000in}{0.000000in}}%
\pgfpathlineto{\pgfqpoint{0.055556in}{0.000000in}}%
\pgfusepath{stroke,fill}%
}%
\begin{pgfscope}%
\pgfsys@transformshift{0.580766in}{1.171543in}%
\pgfsys@useobject{currentmarker}{}%
\end{pgfscope}%
\end{pgfscope}%
\begin{pgfscope}%
\pgfsetbuttcap%
\pgfsetroundjoin%
\definecolor{currentfill}{rgb}{0.000000,0.000000,0.000000}%
\pgfsetfillcolor{currentfill}%
\pgfsetlinewidth{0.501875pt}%
\definecolor{currentstroke}{rgb}{0.000000,0.000000,0.000000}%
\pgfsetstrokecolor{currentstroke}%
\pgfsetdash{}{0pt}%
\pgfsys@defobject{currentmarker}{\pgfqpoint{-0.055556in}{0.000000in}}{\pgfqpoint{0.000000in}{0.000000in}}{%
\pgfpathmoveto{\pgfqpoint{0.000000in}{0.000000in}}%
\pgfpathlineto{\pgfqpoint{-0.055556in}{0.000000in}}%
\pgfusepath{stroke,fill}%
}%
\begin{pgfscope}%
\pgfsys@transformshift{6.192411in}{1.171543in}%
\pgfsys@useobject{currentmarker}{}%
\end{pgfscope}%
\end{pgfscope}%
\begin{pgfscope}%
\pgftext[x=0.525210in,y=1.171543in,right,]{{\rmfamily\fontsize{10.000000}{12.000000}\selectfont \(\displaystyle 2\)}}%
\end{pgfscope}%
\begin{pgfscope}%
\pgfpathrectangle{\pgfqpoint{0.580766in}{0.532919in}}{\pgfqpoint{5.611646in}{3.193122in}} %
\pgfusepath{clip}%
\pgfsetbuttcap%
\pgfsetroundjoin%
\pgfsetlinewidth{0.501875pt}%
\definecolor{currentstroke}{rgb}{0.000000,0.000000,0.000000}%
\pgfsetstrokecolor{currentstroke}%
\pgfsetdash{{1.000000pt}{3.000000pt}}{0.000000pt}%
\pgfpathmoveto{\pgfqpoint{0.580766in}{1.810167in}}%
\pgfpathlineto{\pgfqpoint{6.192411in}{1.810167in}}%
\pgfusepath{stroke}%
\end{pgfscope}%
\begin{pgfscope}%
\pgfsetbuttcap%
\pgfsetroundjoin%
\definecolor{currentfill}{rgb}{0.000000,0.000000,0.000000}%
\pgfsetfillcolor{currentfill}%
\pgfsetlinewidth{0.501875pt}%
\definecolor{currentstroke}{rgb}{0.000000,0.000000,0.000000}%
\pgfsetstrokecolor{currentstroke}%
\pgfsetdash{}{0pt}%
\pgfsys@defobject{currentmarker}{\pgfqpoint{0.000000in}{0.000000in}}{\pgfqpoint{0.055556in}{0.000000in}}{%
\pgfpathmoveto{\pgfqpoint{0.000000in}{0.000000in}}%
\pgfpathlineto{\pgfqpoint{0.055556in}{0.000000in}}%
\pgfusepath{stroke,fill}%
}%
\begin{pgfscope}%
\pgfsys@transformshift{0.580766in}{1.810167in}%
\pgfsys@useobject{currentmarker}{}%
\end{pgfscope}%
\end{pgfscope}%
\begin{pgfscope}%
\pgfsetbuttcap%
\pgfsetroundjoin%
\definecolor{currentfill}{rgb}{0.000000,0.000000,0.000000}%
\pgfsetfillcolor{currentfill}%
\pgfsetlinewidth{0.501875pt}%
\definecolor{currentstroke}{rgb}{0.000000,0.000000,0.000000}%
\pgfsetstrokecolor{currentstroke}%
\pgfsetdash{}{0pt}%
\pgfsys@defobject{currentmarker}{\pgfqpoint{-0.055556in}{0.000000in}}{\pgfqpoint{0.000000in}{0.000000in}}{%
\pgfpathmoveto{\pgfqpoint{0.000000in}{0.000000in}}%
\pgfpathlineto{\pgfqpoint{-0.055556in}{0.000000in}}%
\pgfusepath{stroke,fill}%
}%
\begin{pgfscope}%
\pgfsys@transformshift{6.192411in}{1.810167in}%
\pgfsys@useobject{currentmarker}{}%
\end{pgfscope}%
\end{pgfscope}%
\begin{pgfscope}%
\pgftext[x=0.525210in,y=1.810167in,right,]{{\rmfamily\fontsize{10.000000}{12.000000}\selectfont \(\displaystyle 4\)}}%
\end{pgfscope}%
\begin{pgfscope}%
\pgfpathrectangle{\pgfqpoint{0.580766in}{0.532919in}}{\pgfqpoint{5.611646in}{3.193122in}} %
\pgfusepath{clip}%
\pgfsetbuttcap%
\pgfsetroundjoin%
\pgfsetlinewidth{0.501875pt}%
\definecolor{currentstroke}{rgb}{0.000000,0.000000,0.000000}%
\pgfsetstrokecolor{currentstroke}%
\pgfsetdash{{1.000000pt}{3.000000pt}}{0.000000pt}%
\pgfpathmoveto{\pgfqpoint{0.580766in}{2.448792in}}%
\pgfpathlineto{\pgfqpoint{6.192411in}{2.448792in}}%
\pgfusepath{stroke}%
\end{pgfscope}%
\begin{pgfscope}%
\pgfsetbuttcap%
\pgfsetroundjoin%
\definecolor{currentfill}{rgb}{0.000000,0.000000,0.000000}%
\pgfsetfillcolor{currentfill}%
\pgfsetlinewidth{0.501875pt}%
\definecolor{currentstroke}{rgb}{0.000000,0.000000,0.000000}%
\pgfsetstrokecolor{currentstroke}%
\pgfsetdash{}{0pt}%
\pgfsys@defobject{currentmarker}{\pgfqpoint{0.000000in}{0.000000in}}{\pgfqpoint{0.055556in}{0.000000in}}{%
\pgfpathmoveto{\pgfqpoint{0.000000in}{0.000000in}}%
\pgfpathlineto{\pgfqpoint{0.055556in}{0.000000in}}%
\pgfusepath{stroke,fill}%
}%
\begin{pgfscope}%
\pgfsys@transformshift{0.580766in}{2.448792in}%
\pgfsys@useobject{currentmarker}{}%
\end{pgfscope}%
\end{pgfscope}%
\begin{pgfscope}%
\pgfsetbuttcap%
\pgfsetroundjoin%
\definecolor{currentfill}{rgb}{0.000000,0.000000,0.000000}%
\pgfsetfillcolor{currentfill}%
\pgfsetlinewidth{0.501875pt}%
\definecolor{currentstroke}{rgb}{0.000000,0.000000,0.000000}%
\pgfsetstrokecolor{currentstroke}%
\pgfsetdash{}{0pt}%
\pgfsys@defobject{currentmarker}{\pgfqpoint{-0.055556in}{0.000000in}}{\pgfqpoint{0.000000in}{0.000000in}}{%
\pgfpathmoveto{\pgfqpoint{0.000000in}{0.000000in}}%
\pgfpathlineto{\pgfqpoint{-0.055556in}{0.000000in}}%
\pgfusepath{stroke,fill}%
}%
\begin{pgfscope}%
\pgfsys@transformshift{6.192411in}{2.448792in}%
\pgfsys@useobject{currentmarker}{}%
\end{pgfscope}%
\end{pgfscope}%
\begin{pgfscope}%
\pgftext[x=0.525210in,y=2.448792in,right,]{{\rmfamily\fontsize{10.000000}{12.000000}\selectfont \(\displaystyle 6\)}}%
\end{pgfscope}%
\begin{pgfscope}%
\pgfpathrectangle{\pgfqpoint{0.580766in}{0.532919in}}{\pgfqpoint{5.611646in}{3.193122in}} %
\pgfusepath{clip}%
\pgfsetbuttcap%
\pgfsetroundjoin%
\pgfsetlinewidth{0.501875pt}%
\definecolor{currentstroke}{rgb}{0.000000,0.000000,0.000000}%
\pgfsetstrokecolor{currentstroke}%
\pgfsetdash{{1.000000pt}{3.000000pt}}{0.000000pt}%
\pgfpathmoveto{\pgfqpoint{0.580766in}{3.087416in}}%
\pgfpathlineto{\pgfqpoint{6.192411in}{3.087416in}}%
\pgfusepath{stroke}%
\end{pgfscope}%
\begin{pgfscope}%
\pgfsetbuttcap%
\pgfsetroundjoin%
\definecolor{currentfill}{rgb}{0.000000,0.000000,0.000000}%
\pgfsetfillcolor{currentfill}%
\pgfsetlinewidth{0.501875pt}%
\definecolor{currentstroke}{rgb}{0.000000,0.000000,0.000000}%
\pgfsetstrokecolor{currentstroke}%
\pgfsetdash{}{0pt}%
\pgfsys@defobject{currentmarker}{\pgfqpoint{0.000000in}{0.000000in}}{\pgfqpoint{0.055556in}{0.000000in}}{%
\pgfpathmoveto{\pgfqpoint{0.000000in}{0.000000in}}%
\pgfpathlineto{\pgfqpoint{0.055556in}{0.000000in}}%
\pgfusepath{stroke,fill}%
}%
\begin{pgfscope}%
\pgfsys@transformshift{0.580766in}{3.087416in}%
\pgfsys@useobject{currentmarker}{}%
\end{pgfscope}%
\end{pgfscope}%
\begin{pgfscope}%
\pgfsetbuttcap%
\pgfsetroundjoin%
\definecolor{currentfill}{rgb}{0.000000,0.000000,0.000000}%
\pgfsetfillcolor{currentfill}%
\pgfsetlinewidth{0.501875pt}%
\definecolor{currentstroke}{rgb}{0.000000,0.000000,0.000000}%
\pgfsetstrokecolor{currentstroke}%
\pgfsetdash{}{0pt}%
\pgfsys@defobject{currentmarker}{\pgfqpoint{-0.055556in}{0.000000in}}{\pgfqpoint{0.000000in}{0.000000in}}{%
\pgfpathmoveto{\pgfqpoint{0.000000in}{0.000000in}}%
\pgfpathlineto{\pgfqpoint{-0.055556in}{0.000000in}}%
\pgfusepath{stroke,fill}%
}%
\begin{pgfscope}%
\pgfsys@transformshift{6.192411in}{3.087416in}%
\pgfsys@useobject{currentmarker}{}%
\end{pgfscope}%
\end{pgfscope}%
\begin{pgfscope}%
\pgftext[x=0.525210in,y=3.087416in,right,]{{\rmfamily\fontsize{10.000000}{12.000000}\selectfont \(\displaystyle 8\)}}%
\end{pgfscope}%
\begin{pgfscope}%
\pgfpathrectangle{\pgfqpoint{0.580766in}{0.532919in}}{\pgfqpoint{5.611646in}{3.193122in}} %
\pgfusepath{clip}%
\pgfsetbuttcap%
\pgfsetroundjoin%
\pgfsetlinewidth{0.501875pt}%
\definecolor{currentstroke}{rgb}{0.000000,0.000000,0.000000}%
\pgfsetstrokecolor{currentstroke}%
\pgfsetdash{{1.000000pt}{3.000000pt}}{0.000000pt}%
\pgfpathmoveto{\pgfqpoint{0.580766in}{3.726041in}}%
\pgfpathlineto{\pgfqpoint{6.192411in}{3.726041in}}%
\pgfusepath{stroke}%
\end{pgfscope}%
\begin{pgfscope}%
\pgfsetbuttcap%
\pgfsetroundjoin%
\definecolor{currentfill}{rgb}{0.000000,0.000000,0.000000}%
\pgfsetfillcolor{currentfill}%
\pgfsetlinewidth{0.501875pt}%
\definecolor{currentstroke}{rgb}{0.000000,0.000000,0.000000}%
\pgfsetstrokecolor{currentstroke}%
\pgfsetdash{}{0pt}%
\pgfsys@defobject{currentmarker}{\pgfqpoint{0.000000in}{0.000000in}}{\pgfqpoint{0.055556in}{0.000000in}}{%
\pgfpathmoveto{\pgfqpoint{0.000000in}{0.000000in}}%
\pgfpathlineto{\pgfqpoint{0.055556in}{0.000000in}}%
\pgfusepath{stroke,fill}%
}%
\begin{pgfscope}%
\pgfsys@transformshift{0.580766in}{3.726041in}%
\pgfsys@useobject{currentmarker}{}%
\end{pgfscope}%
\end{pgfscope}%
\begin{pgfscope}%
\pgfsetbuttcap%
\pgfsetroundjoin%
\definecolor{currentfill}{rgb}{0.000000,0.000000,0.000000}%
\pgfsetfillcolor{currentfill}%
\pgfsetlinewidth{0.501875pt}%
\definecolor{currentstroke}{rgb}{0.000000,0.000000,0.000000}%
\pgfsetstrokecolor{currentstroke}%
\pgfsetdash{}{0pt}%
\pgfsys@defobject{currentmarker}{\pgfqpoint{-0.055556in}{0.000000in}}{\pgfqpoint{0.000000in}{0.000000in}}{%
\pgfpathmoveto{\pgfqpoint{0.000000in}{0.000000in}}%
\pgfpathlineto{\pgfqpoint{-0.055556in}{0.000000in}}%
\pgfusepath{stroke,fill}%
}%
\begin{pgfscope}%
\pgfsys@transformshift{6.192411in}{3.726041in}%
\pgfsys@useobject{currentmarker}{}%
\end{pgfscope}%
\end{pgfscope}%
\begin{pgfscope}%
\pgftext[x=0.525210in,y=3.726041in,right,]{{\rmfamily\fontsize{10.000000}{12.000000}\selectfont \(\displaystyle 10\)}}%
\end{pgfscope}%
\begin{pgfscope}%
\pgftext[x=0.316877in,y=2.129480in,,bottom,rotate=90.000000]{{\rmfamily\fontsize{10.000000}{12.000000}\selectfont voltage (V)}}%
\end{pgfscope}%
\begin{pgfscope}%
\pgfsetbuttcap%
\pgfsetroundjoin%
\pgfsetlinewidth{1.003750pt}%
\definecolor{currentstroke}{rgb}{0.000000,0.000000,0.000000}%
\pgfsetstrokecolor{currentstroke}%
\pgfsetdash{}{0pt}%
\pgfpathmoveto{\pgfqpoint{0.580766in}{3.726041in}}%
\pgfpathlineto{\pgfqpoint{6.192411in}{3.726041in}}%
\pgfusepath{stroke}%
\end{pgfscope}%
\begin{pgfscope}%
\pgfsetbuttcap%
\pgfsetroundjoin%
\pgfsetlinewidth{1.003750pt}%
\definecolor{currentstroke}{rgb}{0.000000,0.000000,0.000000}%
\pgfsetstrokecolor{currentstroke}%
\pgfsetdash{}{0pt}%
\pgfpathmoveto{\pgfqpoint{6.192411in}{0.532919in}}%
\pgfpathlineto{\pgfqpoint{6.192411in}{3.726041in}}%
\pgfusepath{stroke}%
\end{pgfscope}%
\begin{pgfscope}%
\pgfsetbuttcap%
\pgfsetroundjoin%
\pgfsetlinewidth{1.003750pt}%
\definecolor{currentstroke}{rgb}{0.000000,0.000000,0.000000}%
\pgfsetstrokecolor{currentstroke}%
\pgfsetdash{}{0pt}%
\pgfpathmoveto{\pgfqpoint{0.580766in}{0.532919in}}%
\pgfpathlineto{\pgfqpoint{6.192411in}{0.532919in}}%
\pgfusepath{stroke}%
\end{pgfscope}%
\begin{pgfscope}%
\pgfsetbuttcap%
\pgfsetroundjoin%
\pgfsetlinewidth{1.003750pt}%
\definecolor{currentstroke}{rgb}{0.000000,0.000000,0.000000}%
\pgfsetstrokecolor{currentstroke}%
\pgfsetdash{}{0pt}%
\pgfpathmoveto{\pgfqpoint{0.580766in}{0.532919in}}%
\pgfpathlineto{\pgfqpoint{0.580766in}{3.726041in}}%
\pgfusepath{stroke}%
\end{pgfscope}%
\begin{pgfscope}%
\pgftext[x=3.386589in,y=3.795485in,,base]{{\rmfamily\fontsize{12.000000}{14.400000}\selectfont Step-By-Step Approximations of \(\displaystyle v_L(t)\)}}%
\end{pgfscope}%
\begin{pgfscope}%
\pgfsetbuttcap%
\pgfsetroundjoin%
\definecolor{currentfill}{rgb}{0.300000,0.300000,0.300000}%
\pgfsetfillcolor{currentfill}%
\pgfsetfillopacity{0.500000}%
\pgfsetlinewidth{1.003750pt}%
\definecolor{currentstroke}{rgb}{0.300000,0.300000,0.300000}%
\pgfsetstrokecolor{currentstroke}%
\pgfsetstrokeopacity{0.500000}%
\pgfsetdash{}{0pt}%
\pgfpathmoveto{\pgfqpoint{3.610140in}{2.832523in}}%
\pgfpathlineto{\pgfqpoint{6.103523in}{2.832523in}}%
\pgfpathquadraticcurveto{\pgfqpoint{6.136856in}{2.832523in}}{\pgfqpoint{6.136856in}{2.865856in}}%
\pgfpathlineto{\pgfqpoint{6.136856in}{3.581596in}}%
\pgfpathquadraticcurveto{\pgfqpoint{6.136856in}{3.614930in}}{\pgfqpoint{6.103523in}{3.614930in}}%
\pgfpathlineto{\pgfqpoint{3.610140in}{3.614930in}}%
\pgfpathquadraticcurveto{\pgfqpoint{3.576807in}{3.614930in}}{\pgfqpoint{3.576807in}{3.581596in}}%
\pgfpathlineto{\pgfqpoint{3.576807in}{2.865856in}}%
\pgfpathquadraticcurveto{\pgfqpoint{3.576807in}{2.832523in}}{\pgfqpoint{3.610140in}{2.832523in}}%
\pgfpathclose%
\pgfusepath{stroke,fill}%
\end{pgfscope}%
\begin{pgfscope}%
\pgfsetbuttcap%
\pgfsetroundjoin%
\definecolor{currentfill}{rgb}{1.000000,1.000000,1.000000}%
\pgfsetfillcolor{currentfill}%
\pgfsetlinewidth{1.003750pt}%
\definecolor{currentstroke}{rgb}{0.000000,0.000000,0.000000}%
\pgfsetstrokecolor{currentstroke}%
\pgfsetdash{}{0pt}%
\pgfpathmoveto{\pgfqpoint{3.582363in}{2.860300in}}%
\pgfpathlineto{\pgfqpoint{6.075745in}{2.860300in}}%
\pgfpathquadraticcurveto{\pgfqpoint{6.109078in}{2.860300in}}{\pgfqpoint{6.109078in}{2.893634in}}%
\pgfpathlineto{\pgfqpoint{6.109078in}{3.609374in}}%
\pgfpathquadraticcurveto{\pgfqpoint{6.109078in}{3.642708in}}{\pgfqpoint{6.075745in}{3.642708in}}%
\pgfpathlineto{\pgfqpoint{3.582363in}{3.642708in}}%
\pgfpathquadraticcurveto{\pgfqpoint{3.549029in}{3.642708in}}{\pgfqpoint{3.549029in}{3.609374in}}%
\pgfpathlineto{\pgfqpoint{3.549029in}{2.893634in}}%
\pgfpathquadraticcurveto{\pgfqpoint{3.549029in}{2.860300in}}{\pgfqpoint{3.582363in}{2.860300in}}%
\pgfpathclose%
\pgfusepath{stroke,fill}%
\end{pgfscope}%
\begin{pgfscope}%
\pgfsetbuttcap%
\pgfsetroundjoin%
\pgfsetlinewidth{1.003750pt}%
\definecolor{currentstroke}{rgb}{0.000000,0.000000,1.000000}%
\pgfsetstrokecolor{currentstroke}%
\pgfsetdash{{1.000000pt}{3.000000pt}}{0.000000pt}%
\pgfpathmoveto{\pgfqpoint{3.665696in}{3.517708in}}%
\pgfpathlineto{\pgfqpoint{3.899029in}{3.517708in}}%
\pgfusepath{stroke}%
\end{pgfscope}%
\begin{pgfscope}%
\pgftext[x=4.082363in,y=3.459374in,left,base]{{\rmfamily\fontsize{12.000000}{14.400000}\selectfont Continuous}}%
\end{pgfscope}%
\begin{pgfscope}%
\pgfsetrectcap%
\pgfsetroundjoin%
\pgfsetlinewidth{1.003750pt}%
\definecolor{currentstroke}{rgb}{0.000000,0.500000,0.000000}%
\pgfsetstrokecolor{currentstroke}%
\pgfsetdash{}{0pt}%
\pgfpathmoveto{\pgfqpoint{3.665696in}{3.276967in}}%
\pgfpathlineto{\pgfqpoint{3.899029in}{3.276967in}}%
\pgfusepath{stroke}%
\end{pgfscope}%
\begin{pgfscope}%
\pgftext[x=4.082363in,y=3.218634in,left,base]{{\rmfamily\fontsize{12.000000}{14.400000}\selectfont Trapezoidal (\(\displaystyle \Delta{}t = 0.0001 s\))}}%
\end{pgfscope}%
\begin{pgfscope}%
\pgfsetrectcap%
\pgfsetroundjoin%
\pgfsetlinewidth{1.003750pt}%
\definecolor{currentstroke}{rgb}{1.000000,0.000000,0.000000}%
\pgfsetstrokecolor{currentstroke}%
\pgfsetdash{}{0pt}%
\pgfpathmoveto{\pgfqpoint{3.665696in}{3.026967in}}%
\pgfpathlineto{\pgfqpoint{3.899029in}{3.026967in}}%
\pgfusepath{stroke}%
\end{pgfscope}%
\begin{pgfscope}%
\pgftext[x=4.082363in,y=2.968634in,left,base]{{\rmfamily\fontsize{12.000000}{14.400000}\selectfont Trapezoidal (\(\displaystyle \Delta{}t = 0.0008 s\))}}%
\end{pgfscope}%
\end{pgfpicture}%
\makeatother%
\endgroup%

    \end{center}
    \caption{Trapezoidal Approximations at $\Delta{}t_1 = 0.1 ms, \Delta{}t_2 = 0.8 ms$}
    \label{trap_approx}
\end{figure}

\begin{figure}[H]
    \begin{center}
        %% Creator: Matplotlib, PGF backend
%%
%% To include the figure in your LaTeX document, write
%%   \input{<filename>.pgf}
%%
%% Make sure the required packages are loaded in your preamble
%%   \usepackage{pgf}
%%
%% Figures using additional raster images can only be included by \input if
%% they are in the same directory as the main LaTeX file. For loading figures
%% from other directories you can use the `import` package
%%   \usepackage{import}
%% and then include the figures with
%%   \import{<path to file>}{<filename>.pgf}
%%
%% Matplotlib used the following preamble
%%
\begingroup%
\makeatletter%
\begin{pgfpicture}%
\pgfpathrectangle{\pgfpointorigin}{\pgfqpoint{6.500000in}{8.000000in}}%
\pgfusepath{use as bounding box}%
\begin{pgfscope}%
\pgfsetbuttcap%
\pgfsetroundjoin%
\definecolor{currentfill}{rgb}{1.000000,1.000000,1.000000}%
\pgfsetfillcolor{currentfill}%
\pgfsetlinewidth{0.000000pt}%
\definecolor{currentstroke}{rgb}{1.000000,1.000000,1.000000}%
\pgfsetstrokecolor{currentstroke}%
\pgfsetdash{}{0pt}%
\pgfpathmoveto{\pgfqpoint{0.000000in}{0.000000in}}%
\pgfpathlineto{\pgfqpoint{6.500000in}{0.000000in}}%
\pgfpathlineto{\pgfqpoint{6.500000in}{8.000000in}}%
\pgfpathlineto{\pgfqpoint{0.000000in}{8.000000in}}%
\pgfpathclose%
\pgfusepath{fill}%
\end{pgfscope}%
\begin{pgfscope}%
\pgfsetbuttcap%
\pgfsetroundjoin%
\definecolor{currentfill}{rgb}{1.000000,1.000000,1.000000}%
\pgfsetfillcolor{currentfill}%
\pgfsetlinewidth{0.000000pt}%
\definecolor{currentstroke}{rgb}{0.000000,0.000000,0.000000}%
\pgfsetstrokecolor{currentstroke}%
\pgfsetstrokeopacity{0.000000}%
\pgfsetdash{}{0pt}%
\pgfpathmoveto{\pgfqpoint{0.580766in}{4.462900in}}%
\pgfpathlineto{\pgfqpoint{6.192411in}{4.462900in}}%
\pgfpathlineto{\pgfqpoint{6.192411in}{7.656023in}}%
\pgfpathlineto{\pgfqpoint{0.580766in}{7.656023in}}%
\pgfpathclose%
\pgfusepath{fill}%
\end{pgfscope}%
\begin{pgfscope}%
\pgfpathrectangle{\pgfqpoint{0.580766in}{4.462900in}}{\pgfqpoint{5.611646in}{3.193122in}} %
\pgfusepath{clip}%
\pgfsetbuttcap%
\pgfsetroundjoin%
\pgfsetlinewidth{1.003750pt}%
\definecolor{currentstroke}{rgb}{0.000000,0.000000,1.000000}%
\pgfsetstrokecolor{currentstroke}%
\pgfsetdash{{1.000000pt}{3.000000pt}}{0.000000pt}%
\pgfpathmoveto{\pgfqpoint{0.580766in}{4.462900in}}%
\pgfpathlineto{\pgfqpoint{0.633141in}{4.608489in}}%
\pgfpathlineto{\pgfqpoint{0.685517in}{4.747440in}}%
\pgfpathlineto{\pgfqpoint{0.737892in}{4.880055in}}%
\pgfpathlineto{\pgfqpoint{0.790267in}{5.006624in}}%
\pgfpathlineto{\pgfqpoint{0.842643in}{5.127422in}}%
\pgfpathlineto{\pgfqpoint{0.895018in}{5.242713in}}%
\pgfpathlineto{\pgfqpoint{0.947393in}{5.352746in}}%
\pgfpathlineto{\pgfqpoint{0.999769in}{5.457763in}}%
\pgfpathlineto{\pgfqpoint{1.054015in}{5.561485in}}%
\pgfpathlineto{\pgfqpoint{1.108261in}{5.660314in}}%
\pgfpathlineto{\pgfqpoint{1.162506in}{5.754479in}}%
\pgfpathlineto{\pgfqpoint{1.216752in}{5.844201in}}%
\pgfpathlineto{\pgfqpoint{1.270998in}{5.929690in}}%
\pgfpathlineto{\pgfqpoint{1.325244in}{6.011145in}}%
\pgfpathlineto{\pgfqpoint{1.381361in}{6.091367in}}%
\pgfpathlineto{\pgfqpoint{1.437477in}{6.167676in}}%
\pgfpathlineto{\pgfqpoint{1.493594in}{6.240263in}}%
\pgfpathlineto{\pgfqpoint{1.549710in}{6.309311in}}%
\pgfpathlineto{\pgfqpoint{1.607697in}{6.377124in}}%
\pgfpathlineto{\pgfqpoint{1.665684in}{6.441522in}}%
\pgfpathlineto{\pgfqpoint{1.723671in}{6.502678in}}%
\pgfpathlineto{\pgfqpoint{1.783529in}{6.562578in}}%
\pgfpathlineto{\pgfqpoint{1.843386in}{6.619367in}}%
\pgfpathlineto{\pgfqpoint{1.905114in}{6.674844in}}%
\pgfpathlineto{\pgfqpoint{1.966842in}{6.727352in}}%
\pgfpathlineto{\pgfqpoint{2.030441in}{6.778513in}}%
\pgfpathlineto{\pgfqpoint{2.094040in}{6.826856in}}%
\pgfpathlineto{\pgfqpoint{2.159509in}{6.873840in}}%
\pgfpathlineto{\pgfqpoint{2.226849in}{6.919391in}}%
\pgfpathlineto{\pgfqpoint{2.294188in}{6.962289in}}%
\pgfpathlineto{\pgfqpoint{2.363399in}{7.003777in}}%
\pgfpathlineto{\pgfqpoint{2.434479in}{7.043805in}}%
\pgfpathlineto{\pgfqpoint{2.507431in}{7.082333in}}%
\pgfpathlineto{\pgfqpoint{2.582253in}{7.119332in}}%
\pgfpathlineto{\pgfqpoint{2.658945in}{7.154781in}}%
\pgfpathlineto{\pgfqpoint{2.737508in}{7.188668in}}%
\pgfpathlineto{\pgfqpoint{2.819812in}{7.221714in}}%
\pgfpathlineto{\pgfqpoint{2.903987in}{7.253096in}}%
\pgfpathlineto{\pgfqpoint{2.991903in}{7.283454in}}%
\pgfpathlineto{\pgfqpoint{3.081689in}{7.312098in}}%
\pgfpathlineto{\pgfqpoint{3.175217in}{7.339597in}}%
\pgfpathlineto{\pgfqpoint{3.272485in}{7.365866in}}%
\pgfpathlineto{\pgfqpoint{3.373495in}{7.390839in}}%
\pgfpathlineto{\pgfqpoint{3.480116in}{7.414872in}}%
\pgfpathlineto{\pgfqpoint{3.590478in}{7.437457in}}%
\pgfpathlineto{\pgfqpoint{3.706452in}{7.458914in}}%
\pgfpathlineto{\pgfqpoint{3.828038in}{7.479151in}}%
\pgfpathlineto{\pgfqpoint{3.957106in}{7.498366in}}%
\pgfpathlineto{\pgfqpoint{4.093656in}{7.516426in}}%
\pgfpathlineto{\pgfqpoint{4.239559in}{7.533444in}}%
\pgfpathlineto{\pgfqpoint{4.394814in}{7.549280in}}%
\pgfpathlineto{\pgfqpoint{4.561293in}{7.563995in}}%
\pgfpathlineto{\pgfqpoint{4.740866in}{7.577602in}}%
\pgfpathlineto{\pgfqpoint{4.935403in}{7.590082in}}%
\pgfpathlineto{\pgfqpoint{5.148645in}{7.601492in}}%
\pgfpathlineto{\pgfqpoint{5.382464in}{7.611748in}}%
\pgfpathlineto{\pgfqpoint{5.644341in}{7.620962in}}%
\pgfpathlineto{\pgfqpoint{5.939887in}{7.629079in}}%
\pgfpathlineto{\pgfqpoint{6.190541in}{7.634472in}}%
\pgfpathlineto{\pgfqpoint{6.190541in}{7.634472in}}%
\pgfusepath{stroke}%
\end{pgfscope}%
\begin{pgfscope}%
\pgfpathrectangle{\pgfqpoint{0.580766in}{4.462900in}}{\pgfqpoint{5.611646in}{3.193122in}} %
\pgfusepath{clip}%
\pgfsetrectcap%
\pgfsetroundjoin%
\pgfsetlinewidth{1.003750pt}%
\definecolor{currentstroke}{rgb}{0.000000,0.500000,0.000000}%
\pgfsetstrokecolor{currentstroke}%
\pgfsetdash{}{0pt}%
\pgfpathmoveto{\pgfqpoint{0.580766in}{4.614954in}}%
\pgfpathlineto{\pgfqpoint{0.636882in}{4.759766in}}%
\pgfpathlineto{\pgfqpoint{0.692999in}{4.897683in}}%
\pgfpathlineto{\pgfqpoint{0.749115in}{5.029033in}}%
\pgfpathlineto{\pgfqpoint{0.805232in}{5.154128in}}%
\pgfpathlineto{\pgfqpoint{0.861348in}{5.273265in}}%
\pgfpathlineto{\pgfqpoint{0.917465in}{5.386730in}}%
\pgfpathlineto{\pgfqpoint{0.973581in}{5.494792in}}%
\pgfpathlineto{\pgfqpoint{1.029698in}{5.597707in}}%
\pgfpathlineto{\pgfqpoint{1.085814in}{5.695722in}}%
\pgfpathlineto{\pgfqpoint{1.141930in}{5.789070in}}%
\pgfpathlineto{\pgfqpoint{1.198047in}{5.877973in}}%
\pgfpathlineto{\pgfqpoint{1.254163in}{5.962642in}}%
\pgfpathlineto{\pgfqpoint{1.310280in}{6.043279in}}%
\pgfpathlineto{\pgfqpoint{1.366396in}{6.120076in}}%
\pgfpathlineto{\pgfqpoint{1.422513in}{6.193216in}}%
\pgfpathlineto{\pgfqpoint{1.478629in}{6.262874in}}%
\pgfpathlineto{\pgfqpoint{1.534746in}{6.329214in}}%
\pgfpathlineto{\pgfqpoint{1.590862in}{6.392396in}}%
\pgfpathlineto{\pgfqpoint{1.646979in}{6.452568in}}%
\pgfpathlineto{\pgfqpoint{1.703095in}{6.509876in}}%
\pgfpathlineto{\pgfqpoint{1.759211in}{6.564454in}}%
\pgfpathlineto{\pgfqpoint{1.815328in}{6.616434in}}%
\pgfpathlineto{\pgfqpoint{1.871444in}{6.665938in}}%
\pgfpathlineto{\pgfqpoint{1.927561in}{6.713085in}}%
\pgfpathlineto{\pgfqpoint{1.983677in}{6.757986in}}%
\pgfpathlineto{\pgfqpoint{2.039794in}{6.800750in}}%
\pgfpathlineto{\pgfqpoint{2.095910in}{6.841477in}}%
\pgfpathlineto{\pgfqpoint{2.152027in}{6.880265in}}%
\pgfpathlineto{\pgfqpoint{2.208143in}{6.917206in}}%
\pgfpathlineto{\pgfqpoint{2.264260in}{6.952388in}}%
\pgfpathlineto{\pgfqpoint{2.320376in}{6.985894in}}%
\pgfpathlineto{\pgfqpoint{2.376492in}{7.017805in}}%
\pgfpathlineto{\pgfqpoint{2.432609in}{7.048196in}}%
\pgfpathlineto{\pgfqpoint{2.488725in}{7.077140in}}%
\pgfpathlineto{\pgfqpoint{2.544842in}{7.104706in}}%
\pgfpathlineto{\pgfqpoint{2.600958in}{7.130959in}}%
\pgfpathlineto{\pgfqpoint{2.657075in}{7.155962in}}%
\pgfpathlineto{\pgfqpoint{2.713191in}{7.179775in}}%
\pgfpathlineto{\pgfqpoint{2.769308in}{7.202453in}}%
\pgfpathlineto{\pgfqpoint{2.825424in}{7.224052in}}%
\pgfpathlineto{\pgfqpoint{2.881541in}{7.244622in}}%
\pgfpathlineto{\pgfqpoint{2.937657in}{7.264212in}}%
\pgfpathlineto{\pgfqpoint{2.993773in}{7.282870in}}%
\pgfpathlineto{\pgfqpoint{3.049890in}{7.300639in}}%
\pgfpathlineto{\pgfqpoint{3.106006in}{7.317562in}}%
\pgfpathlineto{\pgfqpoint{3.162123in}{7.333679in}}%
\pgfpathlineto{\pgfqpoint{3.218239in}{7.349029in}}%
\pgfpathlineto{\pgfqpoint{3.274356in}{7.363648in}}%
\pgfpathlineto{\pgfqpoint{3.330472in}{7.377570in}}%
\pgfpathlineto{\pgfqpoint{3.386589in}{7.390830in}}%
\pgfpathlineto{\pgfqpoint{3.442705in}{7.403458in}}%
\pgfpathlineto{\pgfqpoint{3.498822in}{7.415485in}}%
\pgfpathlineto{\pgfqpoint{3.554938in}{7.426939in}}%
\pgfpathlineto{\pgfqpoint{3.611055in}{7.437848in}}%
\pgfpathlineto{\pgfqpoint{3.667171in}{7.448237in}}%
\pgfpathlineto{\pgfqpoint{3.723287in}{7.458132in}}%
\pgfpathlineto{\pgfqpoint{3.779404in}{7.467555in}}%
\pgfpathlineto{\pgfqpoint{3.835520in}{7.476530in}}%
\pgfpathlineto{\pgfqpoint{3.891637in}{7.485077in}}%
\pgfpathlineto{\pgfqpoint{3.947753in}{7.493217in}}%
\pgfpathlineto{\pgfqpoint{4.003870in}{7.500970in}}%
\pgfpathlineto{\pgfqpoint{4.059986in}{7.508353in}}%
\pgfpathlineto{\pgfqpoint{4.116103in}{7.515385in}}%
\pgfpathlineto{\pgfqpoint{4.172219in}{7.522082in}}%
\pgfpathlineto{\pgfqpoint{4.228336in}{7.528460in}}%
\pgfpathlineto{\pgfqpoint{4.284452in}{7.534535in}}%
\pgfpathlineto{\pgfqpoint{4.340568in}{7.540320in}}%
\pgfpathlineto{\pgfqpoint{4.396685in}{7.545830in}}%
\pgfpathlineto{\pgfqpoint{4.452801in}{7.551077in}}%
\pgfpathlineto{\pgfqpoint{4.508918in}{7.556074in}}%
\pgfpathlineto{\pgfqpoint{4.565034in}{7.560834in}}%
\pgfpathlineto{\pgfqpoint{4.621151in}{7.565367in}}%
\pgfpathlineto{\pgfqpoint{4.677267in}{7.569683in}}%
\pgfpathlineto{\pgfqpoint{4.733384in}{7.573795in}}%
\pgfpathlineto{\pgfqpoint{4.789500in}{7.577710in}}%
\pgfpathlineto{\pgfqpoint{4.845617in}{7.581440in}}%
\pgfpathlineto{\pgfqpoint{4.901733in}{7.584991in}}%
\pgfpathlineto{\pgfqpoint{4.957849in}{7.588374in}}%
\pgfpathlineto{\pgfqpoint{5.013966in}{7.591595in}}%
\pgfpathlineto{\pgfqpoint{5.070082in}{7.594663in}}%
\pgfpathlineto{\pgfqpoint{5.126199in}{7.597585in}}%
\pgfpathlineto{\pgfqpoint{5.182315in}{7.600368in}}%
\pgfpathlineto{\pgfqpoint{5.238432in}{7.603018in}}%
\pgfpathlineto{\pgfqpoint{5.294548in}{7.605542in}}%
\pgfpathlineto{\pgfqpoint{5.350665in}{7.607946in}}%
\pgfpathlineto{\pgfqpoint{5.406781in}{7.610235in}}%
\pgfpathlineto{\pgfqpoint{5.462898in}{7.612415in}}%
\pgfpathlineto{\pgfqpoint{5.519014in}{7.614492in}}%
\pgfpathlineto{\pgfqpoint{5.575130in}{7.616470in}}%
\pgfpathlineto{\pgfqpoint{5.631247in}{7.618353in}}%
\pgfpathlineto{\pgfqpoint{5.687363in}{7.620147in}}%
\pgfpathlineto{\pgfqpoint{5.743480in}{7.621855in}}%
\pgfpathlineto{\pgfqpoint{5.799596in}{7.623482in}}%
\pgfpathlineto{\pgfqpoint{5.855713in}{7.625032in}}%
\pgfpathlineto{\pgfqpoint{5.911829in}{7.626508in}}%
\pgfpathlineto{\pgfqpoint{5.967946in}{7.627913in}}%
\pgfpathlineto{\pgfqpoint{6.024062in}{7.629252in}}%
\pgfpathlineto{\pgfqpoint{6.080179in}{7.630526in}}%
\pgfpathlineto{\pgfqpoint{6.136295in}{7.631740in}}%
\pgfusepath{stroke}%
\end{pgfscope}%
\begin{pgfscope}%
\pgfpathrectangle{\pgfqpoint{0.580766in}{4.462900in}}{\pgfqpoint{5.611646in}{3.193122in}} %
\pgfusepath{clip}%
\pgfsetrectcap%
\pgfsetroundjoin%
\pgfsetlinewidth{1.003750pt}%
\definecolor{currentstroke}{rgb}{1.000000,0.000000,0.000000}%
\pgfsetstrokecolor{currentstroke}%
\pgfsetdash{}{0pt}%
\pgfpathmoveto{\pgfqpoint{0.580766in}{5.375221in}}%
\pgfpathlineto{\pgfqpoint{1.029698in}{6.026878in}}%
\pgfpathlineto{\pgfqpoint{1.478629in}{6.492348in}}%
\pgfpathlineto{\pgfqpoint{1.927561in}{6.824827in}}%
\pgfpathlineto{\pgfqpoint{2.376492in}{7.062311in}}%
\pgfpathlineto{\pgfqpoint{2.825424in}{7.231943in}}%
\pgfpathlineto{\pgfqpoint{3.274356in}{7.353109in}}%
\pgfpathlineto{\pgfqpoint{3.723287in}{7.439655in}}%
\pgfpathlineto{\pgfqpoint{4.172219in}{7.501475in}}%
\pgfpathlineto{\pgfqpoint{4.621151in}{7.545631in}}%
\pgfpathlineto{\pgfqpoint{5.070082in}{7.577172in}}%
\pgfpathlineto{\pgfqpoint{5.519014in}{7.599700in}}%
\pgfusepath{stroke}%
\end{pgfscope}%
\begin{pgfscope}%
\pgfpathrectangle{\pgfqpoint{0.580766in}{4.462900in}}{\pgfqpoint{5.611646in}{3.193122in}} %
\pgfusepath{clip}%
\pgfsetbuttcap%
\pgfsetroundjoin%
\pgfsetlinewidth{0.501875pt}%
\definecolor{currentstroke}{rgb}{0.000000,0.000000,0.000000}%
\pgfsetstrokecolor{currentstroke}%
\pgfsetdash{{1.000000pt}{3.000000pt}}{0.000000pt}%
\pgfpathmoveto{\pgfqpoint{0.580766in}{4.462900in}}%
\pgfpathlineto{\pgfqpoint{0.580766in}{7.656023in}}%
\pgfusepath{stroke}%
\end{pgfscope}%
\begin{pgfscope}%
\pgfsetbuttcap%
\pgfsetroundjoin%
\definecolor{currentfill}{rgb}{0.000000,0.000000,0.000000}%
\pgfsetfillcolor{currentfill}%
\pgfsetlinewidth{0.501875pt}%
\definecolor{currentstroke}{rgb}{0.000000,0.000000,0.000000}%
\pgfsetstrokecolor{currentstroke}%
\pgfsetdash{}{0pt}%
\pgfsys@defobject{currentmarker}{\pgfqpoint{0.000000in}{0.000000in}}{\pgfqpoint{0.000000in}{0.055556in}}{%
\pgfpathmoveto{\pgfqpoint{0.000000in}{0.000000in}}%
\pgfpathlineto{\pgfqpoint{0.000000in}{0.055556in}}%
\pgfusepath{stroke,fill}%
}%
\begin{pgfscope}%
\pgfsys@transformshift{0.580766in}{4.462900in}%
\pgfsys@useobject{currentmarker}{}%
\end{pgfscope}%
\end{pgfscope}%
\begin{pgfscope}%
\pgfsetbuttcap%
\pgfsetroundjoin%
\definecolor{currentfill}{rgb}{0.000000,0.000000,0.000000}%
\pgfsetfillcolor{currentfill}%
\pgfsetlinewidth{0.501875pt}%
\definecolor{currentstroke}{rgb}{0.000000,0.000000,0.000000}%
\pgfsetstrokecolor{currentstroke}%
\pgfsetdash{}{0pt}%
\pgfsys@defobject{currentmarker}{\pgfqpoint{0.000000in}{-0.055556in}}{\pgfqpoint{0.000000in}{0.000000in}}{%
\pgfpathmoveto{\pgfqpoint{0.000000in}{0.000000in}}%
\pgfpathlineto{\pgfqpoint{0.000000in}{-0.055556in}}%
\pgfusepath{stroke,fill}%
}%
\begin{pgfscope}%
\pgfsys@transformshift{0.580766in}{7.656023in}%
\pgfsys@useobject{currentmarker}{}%
\end{pgfscope}%
\end{pgfscope}%
\begin{pgfscope}%
\pgftext[x=0.580766in,y=4.407345in,,top]{{\rmfamily\fontsize{10.000000}{12.000000}\selectfont \(\displaystyle 0.000\)}}%
\end{pgfscope}%
\begin{pgfscope}%
\pgfpathrectangle{\pgfqpoint{0.580766in}{4.462900in}}{\pgfqpoint{5.611646in}{3.193122in}} %
\pgfusepath{clip}%
\pgfsetbuttcap%
\pgfsetroundjoin%
\pgfsetlinewidth{0.501875pt}%
\definecolor{currentstroke}{rgb}{0.000000,0.000000,0.000000}%
\pgfsetstrokecolor{currentstroke}%
\pgfsetdash{{1.000000pt}{3.000000pt}}{0.000000pt}%
\pgfpathmoveto{\pgfqpoint{1.703095in}{4.462900in}}%
\pgfpathlineto{\pgfqpoint{1.703095in}{7.656023in}}%
\pgfusepath{stroke}%
\end{pgfscope}%
\begin{pgfscope}%
\pgfsetbuttcap%
\pgfsetroundjoin%
\definecolor{currentfill}{rgb}{0.000000,0.000000,0.000000}%
\pgfsetfillcolor{currentfill}%
\pgfsetlinewidth{0.501875pt}%
\definecolor{currentstroke}{rgb}{0.000000,0.000000,0.000000}%
\pgfsetstrokecolor{currentstroke}%
\pgfsetdash{}{0pt}%
\pgfsys@defobject{currentmarker}{\pgfqpoint{0.000000in}{0.000000in}}{\pgfqpoint{0.000000in}{0.055556in}}{%
\pgfpathmoveto{\pgfqpoint{0.000000in}{0.000000in}}%
\pgfpathlineto{\pgfqpoint{0.000000in}{0.055556in}}%
\pgfusepath{stroke,fill}%
}%
\begin{pgfscope}%
\pgfsys@transformshift{1.703095in}{4.462900in}%
\pgfsys@useobject{currentmarker}{}%
\end{pgfscope}%
\end{pgfscope}%
\begin{pgfscope}%
\pgfsetbuttcap%
\pgfsetroundjoin%
\definecolor{currentfill}{rgb}{0.000000,0.000000,0.000000}%
\pgfsetfillcolor{currentfill}%
\pgfsetlinewidth{0.501875pt}%
\definecolor{currentstroke}{rgb}{0.000000,0.000000,0.000000}%
\pgfsetstrokecolor{currentstroke}%
\pgfsetdash{}{0pt}%
\pgfsys@defobject{currentmarker}{\pgfqpoint{0.000000in}{-0.055556in}}{\pgfqpoint{0.000000in}{0.000000in}}{%
\pgfpathmoveto{\pgfqpoint{0.000000in}{0.000000in}}%
\pgfpathlineto{\pgfqpoint{0.000000in}{-0.055556in}}%
\pgfusepath{stroke,fill}%
}%
\begin{pgfscope}%
\pgfsys@transformshift{1.703095in}{7.656023in}%
\pgfsys@useobject{currentmarker}{}%
\end{pgfscope}%
\end{pgfscope}%
\begin{pgfscope}%
\pgftext[x=1.703095in,y=4.407345in,,top]{{\rmfamily\fontsize{10.000000}{12.000000}\selectfont \(\displaystyle 0.002\)}}%
\end{pgfscope}%
\begin{pgfscope}%
\pgfpathrectangle{\pgfqpoint{0.580766in}{4.462900in}}{\pgfqpoint{5.611646in}{3.193122in}} %
\pgfusepath{clip}%
\pgfsetbuttcap%
\pgfsetroundjoin%
\pgfsetlinewidth{0.501875pt}%
\definecolor{currentstroke}{rgb}{0.000000,0.000000,0.000000}%
\pgfsetstrokecolor{currentstroke}%
\pgfsetdash{{1.000000pt}{3.000000pt}}{0.000000pt}%
\pgfpathmoveto{\pgfqpoint{2.825424in}{4.462900in}}%
\pgfpathlineto{\pgfqpoint{2.825424in}{7.656023in}}%
\pgfusepath{stroke}%
\end{pgfscope}%
\begin{pgfscope}%
\pgfsetbuttcap%
\pgfsetroundjoin%
\definecolor{currentfill}{rgb}{0.000000,0.000000,0.000000}%
\pgfsetfillcolor{currentfill}%
\pgfsetlinewidth{0.501875pt}%
\definecolor{currentstroke}{rgb}{0.000000,0.000000,0.000000}%
\pgfsetstrokecolor{currentstroke}%
\pgfsetdash{}{0pt}%
\pgfsys@defobject{currentmarker}{\pgfqpoint{0.000000in}{0.000000in}}{\pgfqpoint{0.000000in}{0.055556in}}{%
\pgfpathmoveto{\pgfqpoint{0.000000in}{0.000000in}}%
\pgfpathlineto{\pgfqpoint{0.000000in}{0.055556in}}%
\pgfusepath{stroke,fill}%
}%
\begin{pgfscope}%
\pgfsys@transformshift{2.825424in}{4.462900in}%
\pgfsys@useobject{currentmarker}{}%
\end{pgfscope}%
\end{pgfscope}%
\begin{pgfscope}%
\pgfsetbuttcap%
\pgfsetroundjoin%
\definecolor{currentfill}{rgb}{0.000000,0.000000,0.000000}%
\pgfsetfillcolor{currentfill}%
\pgfsetlinewidth{0.501875pt}%
\definecolor{currentstroke}{rgb}{0.000000,0.000000,0.000000}%
\pgfsetstrokecolor{currentstroke}%
\pgfsetdash{}{0pt}%
\pgfsys@defobject{currentmarker}{\pgfqpoint{0.000000in}{-0.055556in}}{\pgfqpoint{0.000000in}{0.000000in}}{%
\pgfpathmoveto{\pgfqpoint{0.000000in}{0.000000in}}%
\pgfpathlineto{\pgfqpoint{0.000000in}{-0.055556in}}%
\pgfusepath{stroke,fill}%
}%
\begin{pgfscope}%
\pgfsys@transformshift{2.825424in}{7.656023in}%
\pgfsys@useobject{currentmarker}{}%
\end{pgfscope}%
\end{pgfscope}%
\begin{pgfscope}%
\pgftext[x=2.825424in,y=4.407345in,,top]{{\rmfamily\fontsize{10.000000}{12.000000}\selectfont \(\displaystyle 0.004\)}}%
\end{pgfscope}%
\begin{pgfscope}%
\pgfpathrectangle{\pgfqpoint{0.580766in}{4.462900in}}{\pgfqpoint{5.611646in}{3.193122in}} %
\pgfusepath{clip}%
\pgfsetbuttcap%
\pgfsetroundjoin%
\pgfsetlinewidth{0.501875pt}%
\definecolor{currentstroke}{rgb}{0.000000,0.000000,0.000000}%
\pgfsetstrokecolor{currentstroke}%
\pgfsetdash{{1.000000pt}{3.000000pt}}{0.000000pt}%
\pgfpathmoveto{\pgfqpoint{3.947753in}{4.462900in}}%
\pgfpathlineto{\pgfqpoint{3.947753in}{7.656023in}}%
\pgfusepath{stroke}%
\end{pgfscope}%
\begin{pgfscope}%
\pgfsetbuttcap%
\pgfsetroundjoin%
\definecolor{currentfill}{rgb}{0.000000,0.000000,0.000000}%
\pgfsetfillcolor{currentfill}%
\pgfsetlinewidth{0.501875pt}%
\definecolor{currentstroke}{rgb}{0.000000,0.000000,0.000000}%
\pgfsetstrokecolor{currentstroke}%
\pgfsetdash{}{0pt}%
\pgfsys@defobject{currentmarker}{\pgfqpoint{0.000000in}{0.000000in}}{\pgfqpoint{0.000000in}{0.055556in}}{%
\pgfpathmoveto{\pgfqpoint{0.000000in}{0.000000in}}%
\pgfpathlineto{\pgfqpoint{0.000000in}{0.055556in}}%
\pgfusepath{stroke,fill}%
}%
\begin{pgfscope}%
\pgfsys@transformshift{3.947753in}{4.462900in}%
\pgfsys@useobject{currentmarker}{}%
\end{pgfscope}%
\end{pgfscope}%
\begin{pgfscope}%
\pgfsetbuttcap%
\pgfsetroundjoin%
\definecolor{currentfill}{rgb}{0.000000,0.000000,0.000000}%
\pgfsetfillcolor{currentfill}%
\pgfsetlinewidth{0.501875pt}%
\definecolor{currentstroke}{rgb}{0.000000,0.000000,0.000000}%
\pgfsetstrokecolor{currentstroke}%
\pgfsetdash{}{0pt}%
\pgfsys@defobject{currentmarker}{\pgfqpoint{0.000000in}{-0.055556in}}{\pgfqpoint{0.000000in}{0.000000in}}{%
\pgfpathmoveto{\pgfqpoint{0.000000in}{0.000000in}}%
\pgfpathlineto{\pgfqpoint{0.000000in}{-0.055556in}}%
\pgfusepath{stroke,fill}%
}%
\begin{pgfscope}%
\pgfsys@transformshift{3.947753in}{7.656023in}%
\pgfsys@useobject{currentmarker}{}%
\end{pgfscope}%
\end{pgfscope}%
\begin{pgfscope}%
\pgftext[x=3.947753in,y=4.407345in,,top]{{\rmfamily\fontsize{10.000000}{12.000000}\selectfont \(\displaystyle 0.006\)}}%
\end{pgfscope}%
\begin{pgfscope}%
\pgfpathrectangle{\pgfqpoint{0.580766in}{4.462900in}}{\pgfqpoint{5.611646in}{3.193122in}} %
\pgfusepath{clip}%
\pgfsetbuttcap%
\pgfsetroundjoin%
\pgfsetlinewidth{0.501875pt}%
\definecolor{currentstroke}{rgb}{0.000000,0.000000,0.000000}%
\pgfsetstrokecolor{currentstroke}%
\pgfsetdash{{1.000000pt}{3.000000pt}}{0.000000pt}%
\pgfpathmoveto{\pgfqpoint{5.070082in}{4.462900in}}%
\pgfpathlineto{\pgfqpoint{5.070082in}{7.656023in}}%
\pgfusepath{stroke}%
\end{pgfscope}%
\begin{pgfscope}%
\pgfsetbuttcap%
\pgfsetroundjoin%
\definecolor{currentfill}{rgb}{0.000000,0.000000,0.000000}%
\pgfsetfillcolor{currentfill}%
\pgfsetlinewidth{0.501875pt}%
\definecolor{currentstroke}{rgb}{0.000000,0.000000,0.000000}%
\pgfsetstrokecolor{currentstroke}%
\pgfsetdash{}{0pt}%
\pgfsys@defobject{currentmarker}{\pgfqpoint{0.000000in}{0.000000in}}{\pgfqpoint{0.000000in}{0.055556in}}{%
\pgfpathmoveto{\pgfqpoint{0.000000in}{0.000000in}}%
\pgfpathlineto{\pgfqpoint{0.000000in}{0.055556in}}%
\pgfusepath{stroke,fill}%
}%
\begin{pgfscope}%
\pgfsys@transformshift{5.070082in}{4.462900in}%
\pgfsys@useobject{currentmarker}{}%
\end{pgfscope}%
\end{pgfscope}%
\begin{pgfscope}%
\pgfsetbuttcap%
\pgfsetroundjoin%
\definecolor{currentfill}{rgb}{0.000000,0.000000,0.000000}%
\pgfsetfillcolor{currentfill}%
\pgfsetlinewidth{0.501875pt}%
\definecolor{currentstroke}{rgb}{0.000000,0.000000,0.000000}%
\pgfsetstrokecolor{currentstroke}%
\pgfsetdash{}{0pt}%
\pgfsys@defobject{currentmarker}{\pgfqpoint{0.000000in}{-0.055556in}}{\pgfqpoint{0.000000in}{0.000000in}}{%
\pgfpathmoveto{\pgfqpoint{0.000000in}{0.000000in}}%
\pgfpathlineto{\pgfqpoint{0.000000in}{-0.055556in}}%
\pgfusepath{stroke,fill}%
}%
\begin{pgfscope}%
\pgfsys@transformshift{5.070082in}{7.656023in}%
\pgfsys@useobject{currentmarker}{}%
\end{pgfscope}%
\end{pgfscope}%
\begin{pgfscope}%
\pgftext[x=5.070082in,y=4.407345in,,top]{{\rmfamily\fontsize{10.000000}{12.000000}\selectfont \(\displaystyle 0.008\)}}%
\end{pgfscope}%
\begin{pgfscope}%
\pgfpathrectangle{\pgfqpoint{0.580766in}{4.462900in}}{\pgfqpoint{5.611646in}{3.193122in}} %
\pgfusepath{clip}%
\pgfsetbuttcap%
\pgfsetroundjoin%
\pgfsetlinewidth{0.501875pt}%
\definecolor{currentstroke}{rgb}{0.000000,0.000000,0.000000}%
\pgfsetstrokecolor{currentstroke}%
\pgfsetdash{{1.000000pt}{3.000000pt}}{0.000000pt}%
\pgfpathmoveto{\pgfqpoint{6.192411in}{4.462900in}}%
\pgfpathlineto{\pgfqpoint{6.192411in}{7.656023in}}%
\pgfusepath{stroke}%
\end{pgfscope}%
\begin{pgfscope}%
\pgfsetbuttcap%
\pgfsetroundjoin%
\definecolor{currentfill}{rgb}{0.000000,0.000000,0.000000}%
\pgfsetfillcolor{currentfill}%
\pgfsetlinewidth{0.501875pt}%
\definecolor{currentstroke}{rgb}{0.000000,0.000000,0.000000}%
\pgfsetstrokecolor{currentstroke}%
\pgfsetdash{}{0pt}%
\pgfsys@defobject{currentmarker}{\pgfqpoint{0.000000in}{0.000000in}}{\pgfqpoint{0.000000in}{0.055556in}}{%
\pgfpathmoveto{\pgfqpoint{0.000000in}{0.000000in}}%
\pgfpathlineto{\pgfqpoint{0.000000in}{0.055556in}}%
\pgfusepath{stroke,fill}%
}%
\begin{pgfscope}%
\pgfsys@transformshift{6.192411in}{4.462900in}%
\pgfsys@useobject{currentmarker}{}%
\end{pgfscope}%
\end{pgfscope}%
\begin{pgfscope}%
\pgfsetbuttcap%
\pgfsetroundjoin%
\definecolor{currentfill}{rgb}{0.000000,0.000000,0.000000}%
\pgfsetfillcolor{currentfill}%
\pgfsetlinewidth{0.501875pt}%
\definecolor{currentstroke}{rgb}{0.000000,0.000000,0.000000}%
\pgfsetstrokecolor{currentstroke}%
\pgfsetdash{}{0pt}%
\pgfsys@defobject{currentmarker}{\pgfqpoint{0.000000in}{-0.055556in}}{\pgfqpoint{0.000000in}{0.000000in}}{%
\pgfpathmoveto{\pgfqpoint{0.000000in}{0.000000in}}%
\pgfpathlineto{\pgfqpoint{0.000000in}{-0.055556in}}%
\pgfusepath{stroke,fill}%
}%
\begin{pgfscope}%
\pgfsys@transformshift{6.192411in}{7.656023in}%
\pgfsys@useobject{currentmarker}{}%
\end{pgfscope}%
\end{pgfscope}%
\begin{pgfscope}%
\pgftext[x=6.192411in,y=4.407345in,,top]{{\rmfamily\fontsize{10.000000}{12.000000}\selectfont \(\displaystyle 0.010\)}}%
\end{pgfscope}%
\begin{pgfscope}%
\pgftext[x=3.386589in,y=4.214443in,,top]{{\rmfamily\fontsize{10.000000}{12.000000}\selectfont time (s)}}%
\end{pgfscope}%
\begin{pgfscope}%
\pgfpathrectangle{\pgfqpoint{0.580766in}{4.462900in}}{\pgfqpoint{5.611646in}{3.193122in}} %
\pgfusepath{clip}%
\pgfsetbuttcap%
\pgfsetroundjoin%
\pgfsetlinewidth{0.501875pt}%
\definecolor{currentstroke}{rgb}{0.000000,0.000000,0.000000}%
\pgfsetstrokecolor{currentstroke}%
\pgfsetdash{{1.000000pt}{3.000000pt}}{0.000000pt}%
\pgfpathmoveto{\pgfqpoint{0.580766in}{4.462900in}}%
\pgfpathlineto{\pgfqpoint{6.192411in}{4.462900in}}%
\pgfusepath{stroke}%
\end{pgfscope}%
\begin{pgfscope}%
\pgfsetbuttcap%
\pgfsetroundjoin%
\definecolor{currentfill}{rgb}{0.000000,0.000000,0.000000}%
\pgfsetfillcolor{currentfill}%
\pgfsetlinewidth{0.501875pt}%
\definecolor{currentstroke}{rgb}{0.000000,0.000000,0.000000}%
\pgfsetstrokecolor{currentstroke}%
\pgfsetdash{}{0pt}%
\pgfsys@defobject{currentmarker}{\pgfqpoint{0.000000in}{0.000000in}}{\pgfqpoint{0.055556in}{0.000000in}}{%
\pgfpathmoveto{\pgfqpoint{0.000000in}{0.000000in}}%
\pgfpathlineto{\pgfqpoint{0.055556in}{0.000000in}}%
\pgfusepath{stroke,fill}%
}%
\begin{pgfscope}%
\pgfsys@transformshift{0.580766in}{4.462900in}%
\pgfsys@useobject{currentmarker}{}%
\end{pgfscope}%
\end{pgfscope}%
\begin{pgfscope}%
\pgfsetbuttcap%
\pgfsetroundjoin%
\definecolor{currentfill}{rgb}{0.000000,0.000000,0.000000}%
\pgfsetfillcolor{currentfill}%
\pgfsetlinewidth{0.501875pt}%
\definecolor{currentstroke}{rgb}{0.000000,0.000000,0.000000}%
\pgfsetstrokecolor{currentstroke}%
\pgfsetdash{}{0pt}%
\pgfsys@defobject{currentmarker}{\pgfqpoint{-0.055556in}{0.000000in}}{\pgfqpoint{0.000000in}{0.000000in}}{%
\pgfpathmoveto{\pgfqpoint{0.000000in}{0.000000in}}%
\pgfpathlineto{\pgfqpoint{-0.055556in}{0.000000in}}%
\pgfusepath{stroke,fill}%
}%
\begin{pgfscope}%
\pgfsys@transformshift{6.192411in}{4.462900in}%
\pgfsys@useobject{currentmarker}{}%
\end{pgfscope}%
\end{pgfscope}%
\begin{pgfscope}%
\pgftext[x=0.525210in,y=4.462900in,right,]{{\rmfamily\fontsize{10.000000}{12.000000}\selectfont \(\displaystyle 0.0\)}}%
\end{pgfscope}%
\begin{pgfscope}%
\pgfpathrectangle{\pgfqpoint{0.580766in}{4.462900in}}{\pgfqpoint{5.611646in}{3.193122in}} %
\pgfusepath{clip}%
\pgfsetbuttcap%
\pgfsetroundjoin%
\pgfsetlinewidth{0.501875pt}%
\definecolor{currentstroke}{rgb}{0.000000,0.000000,0.000000}%
\pgfsetstrokecolor{currentstroke}%
\pgfsetdash{{1.000000pt}{3.000000pt}}{0.000000pt}%
\pgfpathmoveto{\pgfqpoint{0.580766in}{5.101525in}}%
\pgfpathlineto{\pgfqpoint{6.192411in}{5.101525in}}%
\pgfusepath{stroke}%
\end{pgfscope}%
\begin{pgfscope}%
\pgfsetbuttcap%
\pgfsetroundjoin%
\definecolor{currentfill}{rgb}{0.000000,0.000000,0.000000}%
\pgfsetfillcolor{currentfill}%
\pgfsetlinewidth{0.501875pt}%
\definecolor{currentstroke}{rgb}{0.000000,0.000000,0.000000}%
\pgfsetstrokecolor{currentstroke}%
\pgfsetdash{}{0pt}%
\pgfsys@defobject{currentmarker}{\pgfqpoint{0.000000in}{0.000000in}}{\pgfqpoint{0.055556in}{0.000000in}}{%
\pgfpathmoveto{\pgfqpoint{0.000000in}{0.000000in}}%
\pgfpathlineto{\pgfqpoint{0.055556in}{0.000000in}}%
\pgfusepath{stroke,fill}%
}%
\begin{pgfscope}%
\pgfsys@transformshift{0.580766in}{5.101525in}%
\pgfsys@useobject{currentmarker}{}%
\end{pgfscope}%
\end{pgfscope}%
\begin{pgfscope}%
\pgfsetbuttcap%
\pgfsetroundjoin%
\definecolor{currentfill}{rgb}{0.000000,0.000000,0.000000}%
\pgfsetfillcolor{currentfill}%
\pgfsetlinewidth{0.501875pt}%
\definecolor{currentstroke}{rgb}{0.000000,0.000000,0.000000}%
\pgfsetstrokecolor{currentstroke}%
\pgfsetdash{}{0pt}%
\pgfsys@defobject{currentmarker}{\pgfqpoint{-0.055556in}{0.000000in}}{\pgfqpoint{0.000000in}{0.000000in}}{%
\pgfpathmoveto{\pgfqpoint{0.000000in}{0.000000in}}%
\pgfpathlineto{\pgfqpoint{-0.055556in}{0.000000in}}%
\pgfusepath{stroke,fill}%
}%
\begin{pgfscope}%
\pgfsys@transformshift{6.192411in}{5.101525in}%
\pgfsys@useobject{currentmarker}{}%
\end{pgfscope}%
\end{pgfscope}%
\begin{pgfscope}%
\pgftext[x=0.525210in,y=5.101525in,right,]{{\rmfamily\fontsize{10.000000}{12.000000}\selectfont \(\displaystyle 0.2\)}}%
\end{pgfscope}%
\begin{pgfscope}%
\pgfpathrectangle{\pgfqpoint{0.580766in}{4.462900in}}{\pgfqpoint{5.611646in}{3.193122in}} %
\pgfusepath{clip}%
\pgfsetbuttcap%
\pgfsetroundjoin%
\pgfsetlinewidth{0.501875pt}%
\definecolor{currentstroke}{rgb}{0.000000,0.000000,0.000000}%
\pgfsetstrokecolor{currentstroke}%
\pgfsetdash{{1.000000pt}{3.000000pt}}{0.000000pt}%
\pgfpathmoveto{\pgfqpoint{0.580766in}{5.740149in}}%
\pgfpathlineto{\pgfqpoint{6.192411in}{5.740149in}}%
\pgfusepath{stroke}%
\end{pgfscope}%
\begin{pgfscope}%
\pgfsetbuttcap%
\pgfsetroundjoin%
\definecolor{currentfill}{rgb}{0.000000,0.000000,0.000000}%
\pgfsetfillcolor{currentfill}%
\pgfsetlinewidth{0.501875pt}%
\definecolor{currentstroke}{rgb}{0.000000,0.000000,0.000000}%
\pgfsetstrokecolor{currentstroke}%
\pgfsetdash{}{0pt}%
\pgfsys@defobject{currentmarker}{\pgfqpoint{0.000000in}{0.000000in}}{\pgfqpoint{0.055556in}{0.000000in}}{%
\pgfpathmoveto{\pgfqpoint{0.000000in}{0.000000in}}%
\pgfpathlineto{\pgfqpoint{0.055556in}{0.000000in}}%
\pgfusepath{stroke,fill}%
}%
\begin{pgfscope}%
\pgfsys@transformshift{0.580766in}{5.740149in}%
\pgfsys@useobject{currentmarker}{}%
\end{pgfscope}%
\end{pgfscope}%
\begin{pgfscope}%
\pgfsetbuttcap%
\pgfsetroundjoin%
\definecolor{currentfill}{rgb}{0.000000,0.000000,0.000000}%
\pgfsetfillcolor{currentfill}%
\pgfsetlinewidth{0.501875pt}%
\definecolor{currentstroke}{rgb}{0.000000,0.000000,0.000000}%
\pgfsetstrokecolor{currentstroke}%
\pgfsetdash{}{0pt}%
\pgfsys@defobject{currentmarker}{\pgfqpoint{-0.055556in}{0.000000in}}{\pgfqpoint{0.000000in}{0.000000in}}{%
\pgfpathmoveto{\pgfqpoint{0.000000in}{0.000000in}}%
\pgfpathlineto{\pgfqpoint{-0.055556in}{0.000000in}}%
\pgfusepath{stroke,fill}%
}%
\begin{pgfscope}%
\pgfsys@transformshift{6.192411in}{5.740149in}%
\pgfsys@useobject{currentmarker}{}%
\end{pgfscope}%
\end{pgfscope}%
\begin{pgfscope}%
\pgftext[x=0.525210in,y=5.740149in,right,]{{\rmfamily\fontsize{10.000000}{12.000000}\selectfont \(\displaystyle 0.4\)}}%
\end{pgfscope}%
\begin{pgfscope}%
\pgfpathrectangle{\pgfqpoint{0.580766in}{4.462900in}}{\pgfqpoint{5.611646in}{3.193122in}} %
\pgfusepath{clip}%
\pgfsetbuttcap%
\pgfsetroundjoin%
\pgfsetlinewidth{0.501875pt}%
\definecolor{currentstroke}{rgb}{0.000000,0.000000,0.000000}%
\pgfsetstrokecolor{currentstroke}%
\pgfsetdash{{1.000000pt}{3.000000pt}}{0.000000pt}%
\pgfpathmoveto{\pgfqpoint{0.580766in}{6.378774in}}%
\pgfpathlineto{\pgfqpoint{6.192411in}{6.378774in}}%
\pgfusepath{stroke}%
\end{pgfscope}%
\begin{pgfscope}%
\pgfsetbuttcap%
\pgfsetroundjoin%
\definecolor{currentfill}{rgb}{0.000000,0.000000,0.000000}%
\pgfsetfillcolor{currentfill}%
\pgfsetlinewidth{0.501875pt}%
\definecolor{currentstroke}{rgb}{0.000000,0.000000,0.000000}%
\pgfsetstrokecolor{currentstroke}%
\pgfsetdash{}{0pt}%
\pgfsys@defobject{currentmarker}{\pgfqpoint{0.000000in}{0.000000in}}{\pgfqpoint{0.055556in}{0.000000in}}{%
\pgfpathmoveto{\pgfqpoint{0.000000in}{0.000000in}}%
\pgfpathlineto{\pgfqpoint{0.055556in}{0.000000in}}%
\pgfusepath{stroke,fill}%
}%
\begin{pgfscope}%
\pgfsys@transformshift{0.580766in}{6.378774in}%
\pgfsys@useobject{currentmarker}{}%
\end{pgfscope}%
\end{pgfscope}%
\begin{pgfscope}%
\pgfsetbuttcap%
\pgfsetroundjoin%
\definecolor{currentfill}{rgb}{0.000000,0.000000,0.000000}%
\pgfsetfillcolor{currentfill}%
\pgfsetlinewidth{0.501875pt}%
\definecolor{currentstroke}{rgb}{0.000000,0.000000,0.000000}%
\pgfsetstrokecolor{currentstroke}%
\pgfsetdash{}{0pt}%
\pgfsys@defobject{currentmarker}{\pgfqpoint{-0.055556in}{0.000000in}}{\pgfqpoint{0.000000in}{0.000000in}}{%
\pgfpathmoveto{\pgfqpoint{0.000000in}{0.000000in}}%
\pgfpathlineto{\pgfqpoint{-0.055556in}{0.000000in}}%
\pgfusepath{stroke,fill}%
}%
\begin{pgfscope}%
\pgfsys@transformshift{6.192411in}{6.378774in}%
\pgfsys@useobject{currentmarker}{}%
\end{pgfscope}%
\end{pgfscope}%
\begin{pgfscope}%
\pgftext[x=0.525210in,y=6.378774in,right,]{{\rmfamily\fontsize{10.000000}{12.000000}\selectfont \(\displaystyle 0.6\)}}%
\end{pgfscope}%
\begin{pgfscope}%
\pgfpathrectangle{\pgfqpoint{0.580766in}{4.462900in}}{\pgfqpoint{5.611646in}{3.193122in}} %
\pgfusepath{clip}%
\pgfsetbuttcap%
\pgfsetroundjoin%
\pgfsetlinewidth{0.501875pt}%
\definecolor{currentstroke}{rgb}{0.000000,0.000000,0.000000}%
\pgfsetstrokecolor{currentstroke}%
\pgfsetdash{{1.000000pt}{3.000000pt}}{0.000000pt}%
\pgfpathmoveto{\pgfqpoint{0.580766in}{7.017398in}}%
\pgfpathlineto{\pgfqpoint{6.192411in}{7.017398in}}%
\pgfusepath{stroke}%
\end{pgfscope}%
\begin{pgfscope}%
\pgfsetbuttcap%
\pgfsetroundjoin%
\definecolor{currentfill}{rgb}{0.000000,0.000000,0.000000}%
\pgfsetfillcolor{currentfill}%
\pgfsetlinewidth{0.501875pt}%
\definecolor{currentstroke}{rgb}{0.000000,0.000000,0.000000}%
\pgfsetstrokecolor{currentstroke}%
\pgfsetdash{}{0pt}%
\pgfsys@defobject{currentmarker}{\pgfqpoint{0.000000in}{0.000000in}}{\pgfqpoint{0.055556in}{0.000000in}}{%
\pgfpathmoveto{\pgfqpoint{0.000000in}{0.000000in}}%
\pgfpathlineto{\pgfqpoint{0.055556in}{0.000000in}}%
\pgfusepath{stroke,fill}%
}%
\begin{pgfscope}%
\pgfsys@transformshift{0.580766in}{7.017398in}%
\pgfsys@useobject{currentmarker}{}%
\end{pgfscope}%
\end{pgfscope}%
\begin{pgfscope}%
\pgfsetbuttcap%
\pgfsetroundjoin%
\definecolor{currentfill}{rgb}{0.000000,0.000000,0.000000}%
\pgfsetfillcolor{currentfill}%
\pgfsetlinewidth{0.501875pt}%
\definecolor{currentstroke}{rgb}{0.000000,0.000000,0.000000}%
\pgfsetstrokecolor{currentstroke}%
\pgfsetdash{}{0pt}%
\pgfsys@defobject{currentmarker}{\pgfqpoint{-0.055556in}{0.000000in}}{\pgfqpoint{0.000000in}{0.000000in}}{%
\pgfpathmoveto{\pgfqpoint{0.000000in}{0.000000in}}%
\pgfpathlineto{\pgfqpoint{-0.055556in}{0.000000in}}%
\pgfusepath{stroke,fill}%
}%
\begin{pgfscope}%
\pgfsys@transformshift{6.192411in}{7.017398in}%
\pgfsys@useobject{currentmarker}{}%
\end{pgfscope}%
\end{pgfscope}%
\begin{pgfscope}%
\pgftext[x=0.525210in,y=7.017398in,right,]{{\rmfamily\fontsize{10.000000}{12.000000}\selectfont \(\displaystyle 0.8\)}}%
\end{pgfscope}%
\begin{pgfscope}%
\pgfpathrectangle{\pgfqpoint{0.580766in}{4.462900in}}{\pgfqpoint{5.611646in}{3.193122in}} %
\pgfusepath{clip}%
\pgfsetbuttcap%
\pgfsetroundjoin%
\pgfsetlinewidth{0.501875pt}%
\definecolor{currentstroke}{rgb}{0.000000,0.000000,0.000000}%
\pgfsetstrokecolor{currentstroke}%
\pgfsetdash{{1.000000pt}{3.000000pt}}{0.000000pt}%
\pgfpathmoveto{\pgfqpoint{0.580766in}{7.656023in}}%
\pgfpathlineto{\pgfqpoint{6.192411in}{7.656023in}}%
\pgfusepath{stroke}%
\end{pgfscope}%
\begin{pgfscope}%
\pgfsetbuttcap%
\pgfsetroundjoin%
\definecolor{currentfill}{rgb}{0.000000,0.000000,0.000000}%
\pgfsetfillcolor{currentfill}%
\pgfsetlinewidth{0.501875pt}%
\definecolor{currentstroke}{rgb}{0.000000,0.000000,0.000000}%
\pgfsetstrokecolor{currentstroke}%
\pgfsetdash{}{0pt}%
\pgfsys@defobject{currentmarker}{\pgfqpoint{0.000000in}{0.000000in}}{\pgfqpoint{0.055556in}{0.000000in}}{%
\pgfpathmoveto{\pgfqpoint{0.000000in}{0.000000in}}%
\pgfpathlineto{\pgfqpoint{0.055556in}{0.000000in}}%
\pgfusepath{stroke,fill}%
}%
\begin{pgfscope}%
\pgfsys@transformshift{0.580766in}{7.656023in}%
\pgfsys@useobject{currentmarker}{}%
\end{pgfscope}%
\end{pgfscope}%
\begin{pgfscope}%
\pgfsetbuttcap%
\pgfsetroundjoin%
\definecolor{currentfill}{rgb}{0.000000,0.000000,0.000000}%
\pgfsetfillcolor{currentfill}%
\pgfsetlinewidth{0.501875pt}%
\definecolor{currentstroke}{rgb}{0.000000,0.000000,0.000000}%
\pgfsetstrokecolor{currentstroke}%
\pgfsetdash{}{0pt}%
\pgfsys@defobject{currentmarker}{\pgfqpoint{-0.055556in}{0.000000in}}{\pgfqpoint{0.000000in}{0.000000in}}{%
\pgfpathmoveto{\pgfqpoint{0.000000in}{0.000000in}}%
\pgfpathlineto{\pgfqpoint{-0.055556in}{0.000000in}}%
\pgfusepath{stroke,fill}%
}%
\begin{pgfscope}%
\pgfsys@transformshift{6.192411in}{7.656023in}%
\pgfsys@useobject{currentmarker}{}%
\end{pgfscope}%
\end{pgfscope}%
\begin{pgfscope}%
\pgftext[x=0.525210in,y=7.656023in,right,]{{\rmfamily\fontsize{10.000000}{12.000000}\selectfont \(\displaystyle 1.0\)}}%
\end{pgfscope}%
\begin{pgfscope}%
\pgftext[x=0.278296in,y=6.059461in,,bottom,rotate=90.000000]{{\rmfamily\fontsize{10.000000}{12.000000}\selectfont current (A)}}%
\end{pgfscope}%
\begin{pgfscope}%
\pgfsetbuttcap%
\pgfsetroundjoin%
\pgfsetlinewidth{1.003750pt}%
\definecolor{currentstroke}{rgb}{0.000000,0.000000,0.000000}%
\pgfsetstrokecolor{currentstroke}%
\pgfsetdash{}{0pt}%
\pgfpathmoveto{\pgfqpoint{0.580766in}{7.656023in}}%
\pgfpathlineto{\pgfqpoint{6.192411in}{7.656023in}}%
\pgfusepath{stroke}%
\end{pgfscope}%
\begin{pgfscope}%
\pgfsetbuttcap%
\pgfsetroundjoin%
\pgfsetlinewidth{1.003750pt}%
\definecolor{currentstroke}{rgb}{0.000000,0.000000,0.000000}%
\pgfsetstrokecolor{currentstroke}%
\pgfsetdash{}{0pt}%
\pgfpathmoveto{\pgfqpoint{6.192411in}{4.462900in}}%
\pgfpathlineto{\pgfqpoint{6.192411in}{7.656023in}}%
\pgfusepath{stroke}%
\end{pgfscope}%
\begin{pgfscope}%
\pgfsetbuttcap%
\pgfsetroundjoin%
\pgfsetlinewidth{1.003750pt}%
\definecolor{currentstroke}{rgb}{0.000000,0.000000,0.000000}%
\pgfsetstrokecolor{currentstroke}%
\pgfsetdash{}{0pt}%
\pgfpathmoveto{\pgfqpoint{0.580766in}{4.462900in}}%
\pgfpathlineto{\pgfqpoint{6.192411in}{4.462900in}}%
\pgfusepath{stroke}%
\end{pgfscope}%
\begin{pgfscope}%
\pgfsetbuttcap%
\pgfsetroundjoin%
\pgfsetlinewidth{1.003750pt}%
\definecolor{currentstroke}{rgb}{0.000000,0.000000,0.000000}%
\pgfsetstrokecolor{currentstroke}%
\pgfsetdash{}{0pt}%
\pgfpathmoveto{\pgfqpoint{0.580766in}{4.462900in}}%
\pgfpathlineto{\pgfqpoint{0.580766in}{7.656023in}}%
\pgfusepath{stroke}%
\end{pgfscope}%
\begin{pgfscope}%
\pgftext[x=3.386589in,y=7.725467in,,base]{{\rmfamily\fontsize{12.000000}{14.400000}\selectfont Step-By-Step Approximations of \(\displaystyle i(t)\)}}%
\end{pgfscope}%
\begin{pgfscope}%
\pgfsetbuttcap%
\pgfsetroundjoin%
\definecolor{currentfill}{rgb}{0.300000,0.300000,0.300000}%
\pgfsetfillcolor{currentfill}%
\pgfsetfillopacity{0.500000}%
\pgfsetlinewidth{1.003750pt}%
\definecolor{currentstroke}{rgb}{0.300000,0.300000,0.300000}%
\pgfsetstrokecolor{currentstroke}%
\pgfsetstrokeopacity{0.500000}%
\pgfsetdash{}{0pt}%
\pgfpathmoveto{\pgfqpoint{3.306478in}{4.518456in}}%
\pgfpathlineto{\pgfqpoint{6.103523in}{4.518456in}}%
\pgfpathquadraticcurveto{\pgfqpoint{6.136856in}{4.518456in}}{\pgfqpoint{6.136856in}{4.551789in}}%
\pgfpathlineto{\pgfqpoint{6.136856in}{5.267529in}}%
\pgfpathquadraticcurveto{\pgfqpoint{6.136856in}{5.300863in}}{\pgfqpoint{6.103523in}{5.300863in}}%
\pgfpathlineto{\pgfqpoint{3.306478in}{5.300863in}}%
\pgfpathquadraticcurveto{\pgfqpoint{3.273145in}{5.300863in}}{\pgfqpoint{3.273145in}{5.267529in}}%
\pgfpathlineto{\pgfqpoint{3.273145in}{4.551789in}}%
\pgfpathquadraticcurveto{\pgfqpoint{3.273145in}{4.518456in}}{\pgfqpoint{3.306478in}{4.518456in}}%
\pgfpathclose%
\pgfusepath{stroke,fill}%
\end{pgfscope}%
\begin{pgfscope}%
\pgfsetbuttcap%
\pgfsetroundjoin%
\definecolor{currentfill}{rgb}{1.000000,1.000000,1.000000}%
\pgfsetfillcolor{currentfill}%
\pgfsetlinewidth{1.003750pt}%
\definecolor{currentstroke}{rgb}{0.000000,0.000000,0.000000}%
\pgfsetstrokecolor{currentstroke}%
\pgfsetdash{}{0pt}%
\pgfpathmoveto{\pgfqpoint{3.278700in}{4.546233in}}%
\pgfpathlineto{\pgfqpoint{6.075745in}{4.546233in}}%
\pgfpathquadraticcurveto{\pgfqpoint{6.109078in}{4.546233in}}{\pgfqpoint{6.109078in}{4.579567in}}%
\pgfpathlineto{\pgfqpoint{6.109078in}{5.295307in}}%
\pgfpathquadraticcurveto{\pgfqpoint{6.109078in}{5.328641in}}{\pgfqpoint{6.075745in}{5.328641in}}%
\pgfpathlineto{\pgfqpoint{3.278700in}{5.328641in}}%
\pgfpathquadraticcurveto{\pgfqpoint{3.245367in}{5.328641in}}{\pgfqpoint{3.245367in}{5.295307in}}%
\pgfpathlineto{\pgfqpoint{3.245367in}{4.579567in}}%
\pgfpathquadraticcurveto{\pgfqpoint{3.245367in}{4.546233in}}{\pgfqpoint{3.278700in}{4.546233in}}%
\pgfpathclose%
\pgfusepath{stroke,fill}%
\end{pgfscope}%
\begin{pgfscope}%
\pgfsetbuttcap%
\pgfsetroundjoin%
\pgfsetlinewidth{1.003750pt}%
\definecolor{currentstroke}{rgb}{0.000000,0.000000,1.000000}%
\pgfsetstrokecolor{currentstroke}%
\pgfsetdash{{1.000000pt}{3.000000pt}}{0.000000pt}%
\pgfpathmoveto{\pgfqpoint{3.362034in}{5.203641in}}%
\pgfpathlineto{\pgfqpoint{3.595367in}{5.203641in}}%
\pgfusepath{stroke}%
\end{pgfscope}%
\begin{pgfscope}%
\pgftext[x=3.778700in,y=5.145307in,left,base]{{\rmfamily\fontsize{12.000000}{14.400000}\selectfont Continuous}}%
\end{pgfscope}%
\begin{pgfscope}%
\pgfsetrectcap%
\pgfsetroundjoin%
\pgfsetlinewidth{1.003750pt}%
\definecolor{currentstroke}{rgb}{0.000000,0.500000,0.000000}%
\pgfsetstrokecolor{currentstroke}%
\pgfsetdash{}{0pt}%
\pgfpathmoveto{\pgfqpoint{3.362034in}{4.962900in}}%
\pgfpathlineto{\pgfqpoint{3.595367in}{4.962900in}}%
\pgfusepath{stroke}%
\end{pgfscope}%
\begin{pgfscope}%
\pgftext[x=3.778700in,y=4.904567in,left,base]{{\rmfamily\fontsize{12.000000}{14.400000}\selectfont Backward Euler (\(\displaystyle \Delta{}t = 0.0001 s\))}}%
\end{pgfscope}%
\begin{pgfscope}%
\pgfsetrectcap%
\pgfsetroundjoin%
\pgfsetlinewidth{1.003750pt}%
\definecolor{currentstroke}{rgb}{1.000000,0.000000,0.000000}%
\pgfsetstrokecolor{currentstroke}%
\pgfsetdash{}{0pt}%
\pgfpathmoveto{\pgfqpoint{3.362034in}{4.712900in}}%
\pgfpathlineto{\pgfqpoint{3.595367in}{4.712900in}}%
\pgfusepath{stroke}%
\end{pgfscope}%
\begin{pgfscope}%
\pgftext[x=3.778700in,y=4.654567in,left,base]{{\rmfamily\fontsize{12.000000}{14.400000}\selectfont Backward Euler (\(\displaystyle \Delta{}t = 0.0008 s\))}}%
\end{pgfscope}%
\begin{pgfscope}%
\pgfsetbuttcap%
\pgfsetroundjoin%
\definecolor{currentfill}{rgb}{1.000000,1.000000,1.000000}%
\pgfsetfillcolor{currentfill}%
\pgfsetlinewidth{0.000000pt}%
\definecolor{currentstroke}{rgb}{0.000000,0.000000,0.000000}%
\pgfsetstrokecolor{currentstroke}%
\pgfsetstrokeopacity{0.000000}%
\pgfsetdash{}{0pt}%
\pgfpathmoveto{\pgfqpoint{0.580766in}{0.532919in}}%
\pgfpathlineto{\pgfqpoint{6.192411in}{0.532919in}}%
\pgfpathlineto{\pgfqpoint{6.192411in}{3.726041in}}%
\pgfpathlineto{\pgfqpoint{0.580766in}{3.726041in}}%
\pgfpathclose%
\pgfusepath{fill}%
\end{pgfscope}%
\begin{pgfscope}%
\pgfpathrectangle{\pgfqpoint{0.580766in}{0.532919in}}{\pgfqpoint{5.611646in}{3.193122in}} %
\pgfusepath{clip}%
\pgfsetbuttcap%
\pgfsetroundjoin%
\pgfsetlinewidth{1.003750pt}%
\definecolor{currentstroke}{rgb}{0.000000,0.000000,1.000000}%
\pgfsetstrokecolor{currentstroke}%
\pgfsetdash{{1.000000pt}{3.000000pt}}{0.000000pt}%
\pgfpathmoveto{\pgfqpoint{0.580766in}{3.726041in}}%
\pgfpathlineto{\pgfqpoint{0.633141in}{3.580452in}}%
\pgfpathlineto{\pgfqpoint{0.685517in}{3.441501in}}%
\pgfpathlineto{\pgfqpoint{0.737892in}{3.308886in}}%
\pgfpathlineto{\pgfqpoint{0.790267in}{3.182317in}}%
\pgfpathlineto{\pgfqpoint{0.842643in}{3.061519in}}%
\pgfpathlineto{\pgfqpoint{0.895018in}{2.946229in}}%
\pgfpathlineto{\pgfqpoint{0.947393in}{2.836195in}}%
\pgfpathlineto{\pgfqpoint{0.999769in}{2.731178in}}%
\pgfpathlineto{\pgfqpoint{1.054015in}{2.627456in}}%
\pgfpathlineto{\pgfqpoint{1.108261in}{2.528627in}}%
\pgfpathlineto{\pgfqpoint{1.162506in}{2.434462in}}%
\pgfpathlineto{\pgfqpoint{1.216752in}{2.344740in}}%
\pgfpathlineto{\pgfqpoint{1.270998in}{2.259251in}}%
\pgfpathlineto{\pgfqpoint{1.325244in}{2.177796in}}%
\pgfpathlineto{\pgfqpoint{1.381361in}{2.097574in}}%
\pgfpathlineto{\pgfqpoint{1.437477in}{2.021265in}}%
\pgfpathlineto{\pgfqpoint{1.493594in}{1.948678in}}%
\pgfpathlineto{\pgfqpoint{1.549710in}{1.879630in}}%
\pgfpathlineto{\pgfqpoint{1.607697in}{1.811817in}}%
\pgfpathlineto{\pgfqpoint{1.665684in}{1.747419in}}%
\pgfpathlineto{\pgfqpoint{1.723671in}{1.686263in}}%
\pgfpathlineto{\pgfqpoint{1.783529in}{1.626363in}}%
\pgfpathlineto{\pgfqpoint{1.843386in}{1.569574in}}%
\pgfpathlineto{\pgfqpoint{1.905114in}{1.514097in}}%
\pgfpathlineto{\pgfqpoint{1.966842in}{1.461590in}}%
\pgfpathlineto{\pgfqpoint{2.030441in}{1.410428in}}%
\pgfpathlineto{\pgfqpoint{2.094040in}{1.362085in}}%
\pgfpathlineto{\pgfqpoint{2.159509in}{1.315101in}}%
\pgfpathlineto{\pgfqpoint{2.226849in}{1.269550in}}%
\pgfpathlineto{\pgfqpoint{2.294188in}{1.226652in}}%
\pgfpathlineto{\pgfqpoint{2.363399in}{1.185164in}}%
\pgfpathlineto{\pgfqpoint{2.434479in}{1.145136in}}%
\pgfpathlineto{\pgfqpoint{2.507431in}{1.106608in}}%
\pgfpathlineto{\pgfqpoint{2.582253in}{1.069609in}}%
\pgfpathlineto{\pgfqpoint{2.658945in}{1.034160in}}%
\pgfpathlineto{\pgfqpoint{2.737508in}{1.000273in}}%
\pgfpathlineto{\pgfqpoint{2.819812in}{0.967227in}}%
\pgfpathlineto{\pgfqpoint{2.903987in}{0.935845in}}%
\pgfpathlineto{\pgfqpoint{2.991903in}{0.905487in}}%
\pgfpathlineto{\pgfqpoint{3.081689in}{0.876843in}}%
\pgfpathlineto{\pgfqpoint{3.175217in}{0.849344in}}%
\pgfpathlineto{\pgfqpoint{3.272485in}{0.823075in}}%
\pgfpathlineto{\pgfqpoint{3.373495in}{0.798102in}}%
\pgfpathlineto{\pgfqpoint{3.480116in}{0.774069in}}%
\pgfpathlineto{\pgfqpoint{3.590478in}{0.751484in}}%
\pgfpathlineto{\pgfqpoint{3.706452in}{0.730027in}}%
\pgfpathlineto{\pgfqpoint{3.828038in}{0.709790in}}%
\pgfpathlineto{\pgfqpoint{3.957106in}{0.690575in}}%
\pgfpathlineto{\pgfqpoint{4.093656in}{0.672515in}}%
\pgfpathlineto{\pgfqpoint{4.239559in}{0.655497in}}%
\pgfpathlineto{\pgfqpoint{4.394814in}{0.639661in}}%
\pgfpathlineto{\pgfqpoint{4.561293in}{0.624946in}}%
\pgfpathlineto{\pgfqpoint{4.740866in}{0.611339in}}%
\pgfpathlineto{\pgfqpoint{4.935403in}{0.598859in}}%
\pgfpathlineto{\pgfqpoint{5.148645in}{0.587449in}}%
\pgfpathlineto{\pgfqpoint{5.382464in}{0.577194in}}%
\pgfpathlineto{\pgfqpoint{5.644341in}{0.567979in}}%
\pgfpathlineto{\pgfqpoint{5.939887in}{0.559862in}}%
\pgfpathlineto{\pgfqpoint{6.190541in}{0.554469in}}%
\pgfpathlineto{\pgfqpoint{6.190541in}{0.554469in}}%
\pgfusepath{stroke}%
\end{pgfscope}%
\begin{pgfscope}%
\pgfpathrectangle{\pgfqpoint{0.580766in}{0.532919in}}{\pgfqpoint{5.611646in}{3.193122in}} %
\pgfusepath{clip}%
\pgfsetrectcap%
\pgfsetroundjoin%
\pgfsetlinewidth{1.003750pt}%
\definecolor{currentstroke}{rgb}{0.000000,0.500000,0.000000}%
\pgfsetstrokecolor{currentstroke}%
\pgfsetdash{}{0pt}%
\pgfpathmoveto{\pgfqpoint{0.580766in}{3.573987in}}%
\pgfpathlineto{\pgfqpoint{0.636882in}{3.429175in}}%
\pgfpathlineto{\pgfqpoint{0.692999in}{3.291258in}}%
\pgfpathlineto{\pgfqpoint{0.749115in}{3.159908in}}%
\pgfpathlineto{\pgfqpoint{0.805232in}{3.034813in}}%
\pgfpathlineto{\pgfqpoint{0.861348in}{2.915676in}}%
\pgfpathlineto{\pgfqpoint{0.917465in}{2.802211in}}%
\pgfpathlineto{\pgfqpoint{0.973581in}{2.694149in}}%
\pgfpathlineto{\pgfqpoint{1.029698in}{2.591234in}}%
\pgfpathlineto{\pgfqpoint{1.085814in}{2.493219in}}%
\pgfpathlineto{\pgfqpoint{1.141930in}{2.399871in}}%
\pgfpathlineto{\pgfqpoint{1.198047in}{2.310969in}}%
\pgfpathlineto{\pgfqpoint{1.254163in}{2.226299in}}%
\pgfpathlineto{\pgfqpoint{1.310280in}{2.145662in}}%
\pgfpathlineto{\pgfqpoint{1.366396in}{2.068865in}}%
\pgfpathlineto{\pgfqpoint{1.422513in}{1.995725in}}%
\pgfpathlineto{\pgfqpoint{1.478629in}{1.926067in}}%
\pgfpathlineto{\pgfqpoint{1.534746in}{1.859727in}}%
\pgfpathlineto{\pgfqpoint{1.590862in}{1.796545in}}%
\pgfpathlineto{\pgfqpoint{1.646979in}{1.736373in}}%
\pgfpathlineto{\pgfqpoint{1.703095in}{1.679065in}}%
\pgfpathlineto{\pgfqpoint{1.759211in}{1.624487in}}%
\pgfpathlineto{\pgfqpoint{1.815328in}{1.572508in}}%
\pgfpathlineto{\pgfqpoint{1.871444in}{1.523003in}}%
\pgfpathlineto{\pgfqpoint{1.927561in}{1.475856in}}%
\pgfpathlineto{\pgfqpoint{1.983677in}{1.430955in}}%
\pgfpathlineto{\pgfqpoint{2.039794in}{1.388191in}}%
\pgfpathlineto{\pgfqpoint{2.095910in}{1.347464in}}%
\pgfpathlineto{\pgfqpoint{2.152027in}{1.308676in}}%
\pgfpathlineto{\pgfqpoint{2.208143in}{1.271735in}}%
\pgfpathlineto{\pgfqpoint{2.264260in}{1.236553in}}%
\pgfpathlineto{\pgfqpoint{2.320376in}{1.203047in}}%
\pgfpathlineto{\pgfqpoint{2.376492in}{1.171136in}}%
\pgfpathlineto{\pgfqpoint{2.432609in}{1.140745in}}%
\pgfpathlineto{\pgfqpoint{2.488725in}{1.111801in}}%
\pgfpathlineto{\pgfqpoint{2.544842in}{1.084235in}}%
\pgfpathlineto{\pgfqpoint{2.600958in}{1.057982in}}%
\pgfpathlineto{\pgfqpoint{2.657075in}{1.032979in}}%
\pgfpathlineto{\pgfqpoint{2.713191in}{1.009166in}}%
\pgfpathlineto{\pgfqpoint{2.769308in}{0.986488in}}%
\pgfpathlineto{\pgfqpoint{2.825424in}{0.964889in}}%
\pgfpathlineto{\pgfqpoint{2.881541in}{0.944319in}}%
\pgfpathlineto{\pgfqpoint{2.937657in}{0.924729in}}%
\pgfpathlineto{\pgfqpoint{2.993773in}{0.906071in}}%
\pgfpathlineto{\pgfqpoint{3.049890in}{0.888302in}}%
\pgfpathlineto{\pgfqpoint{3.106006in}{0.871379in}}%
\pgfpathlineto{\pgfqpoint{3.162123in}{0.855262in}}%
\pgfpathlineto{\pgfqpoint{3.218239in}{0.839912in}}%
\pgfpathlineto{\pgfqpoint{3.274356in}{0.825293in}}%
\pgfpathlineto{\pgfqpoint{3.330472in}{0.811371in}}%
\pgfpathlineto{\pgfqpoint{3.386589in}{0.798111in}}%
\pgfpathlineto{\pgfqpoint{3.442705in}{0.785483in}}%
\pgfpathlineto{\pgfqpoint{3.498822in}{0.773456in}}%
\pgfpathlineto{\pgfqpoint{3.554938in}{0.762002in}}%
\pgfpathlineto{\pgfqpoint{3.611055in}{0.751093in}}%
\pgfpathlineto{\pgfqpoint{3.667171in}{0.740704in}}%
\pgfpathlineto{\pgfqpoint{3.723287in}{0.730809in}}%
\pgfpathlineto{\pgfqpoint{3.779404in}{0.721386in}}%
\pgfpathlineto{\pgfqpoint{3.835520in}{0.712411in}}%
\pgfpathlineto{\pgfqpoint{3.891637in}{0.703864in}}%
\pgfpathlineto{\pgfqpoint{3.947753in}{0.695724in}}%
\pgfpathlineto{\pgfqpoint{4.003870in}{0.687971in}}%
\pgfpathlineto{\pgfqpoint{4.059986in}{0.680588in}}%
\pgfpathlineto{\pgfqpoint{4.116103in}{0.673556in}}%
\pgfpathlineto{\pgfqpoint{4.172219in}{0.666859in}}%
\pgfpathlineto{\pgfqpoint{4.228336in}{0.660481in}}%
\pgfpathlineto{\pgfqpoint{4.284452in}{0.654406in}}%
\pgfpathlineto{\pgfqpoint{4.340568in}{0.648621in}}%
\pgfpathlineto{\pgfqpoint{4.396685in}{0.643111in}}%
\pgfpathlineto{\pgfqpoint{4.452801in}{0.637864in}}%
\pgfpathlineto{\pgfqpoint{4.508918in}{0.632867in}}%
\pgfpathlineto{\pgfqpoint{4.565034in}{0.628107in}}%
\pgfpathlineto{\pgfqpoint{4.621151in}{0.623575in}}%
\pgfpathlineto{\pgfqpoint{4.677267in}{0.619258in}}%
\pgfpathlineto{\pgfqpoint{4.733384in}{0.615146in}}%
\pgfpathlineto{\pgfqpoint{4.789500in}{0.611231in}}%
\pgfpathlineto{\pgfqpoint{4.845617in}{0.607501in}}%
\pgfpathlineto{\pgfqpoint{4.901733in}{0.603950in}}%
\pgfpathlineto{\pgfqpoint{4.957849in}{0.600567in}}%
\pgfpathlineto{\pgfqpoint{5.013966in}{0.597346in}}%
\pgfpathlineto{\pgfqpoint{5.070082in}{0.594278in}}%
\pgfpathlineto{\pgfqpoint{5.126199in}{0.591356in}}%
\pgfpathlineto{\pgfqpoint{5.182315in}{0.588573in}}%
\pgfpathlineto{\pgfqpoint{5.238432in}{0.585923in}}%
\pgfpathlineto{\pgfqpoint{5.294548in}{0.583399in}}%
\pgfpathlineto{\pgfqpoint{5.350665in}{0.580995in}}%
\pgfpathlineto{\pgfqpoint{5.406781in}{0.578706in}}%
\pgfpathlineto{\pgfqpoint{5.462898in}{0.576526in}}%
\pgfpathlineto{\pgfqpoint{5.519014in}{0.574449in}}%
\pgfpathlineto{\pgfqpoint{5.575130in}{0.572471in}}%
\pgfpathlineto{\pgfqpoint{5.631247in}{0.570588in}}%
\pgfpathlineto{\pgfqpoint{5.687363in}{0.568794in}}%
\pgfpathlineto{\pgfqpoint{5.743480in}{0.567086in}}%
\pgfpathlineto{\pgfqpoint{5.799596in}{0.565459in}}%
\pgfpathlineto{\pgfqpoint{5.855713in}{0.563909in}}%
\pgfpathlineto{\pgfqpoint{5.911829in}{0.562434in}}%
\pgfpathlineto{\pgfqpoint{5.967946in}{0.561028in}}%
\pgfpathlineto{\pgfqpoint{6.024062in}{0.559689in}}%
\pgfpathlineto{\pgfqpoint{6.080179in}{0.558415in}}%
\pgfpathlineto{\pgfqpoint{6.136295in}{0.557201in}}%
\pgfusepath{stroke}%
\end{pgfscope}%
\begin{pgfscope}%
\pgfpathrectangle{\pgfqpoint{0.580766in}{0.532919in}}{\pgfqpoint{5.611646in}{3.193122in}} %
\pgfusepath{clip}%
\pgfsetrectcap%
\pgfsetroundjoin%
\pgfsetlinewidth{1.003750pt}%
\definecolor{currentstroke}{rgb}{1.000000,0.000000,0.000000}%
\pgfsetstrokecolor{currentstroke}%
\pgfsetdash{}{0pt}%
\pgfpathmoveto{\pgfqpoint{0.580766in}{2.813720in}}%
\pgfpathlineto{\pgfqpoint{1.029698in}{2.162063in}}%
\pgfpathlineto{\pgfqpoint{1.478629in}{1.696593in}}%
\pgfpathlineto{\pgfqpoint{1.927561in}{1.364114in}}%
\pgfpathlineto{\pgfqpoint{2.376492in}{1.126630in}}%
\pgfpathlineto{\pgfqpoint{2.825424in}{0.956998in}}%
\pgfpathlineto{\pgfqpoint{3.274356in}{0.835832in}}%
\pgfpathlineto{\pgfqpoint{3.723287in}{0.749286in}}%
\pgfpathlineto{\pgfqpoint{4.172219in}{0.687466in}}%
\pgfpathlineto{\pgfqpoint{4.621151in}{0.643310in}}%
\pgfpathlineto{\pgfqpoint{5.070082in}{0.611770in}}%
\pgfpathlineto{\pgfqpoint{5.519014in}{0.589241in}}%
\pgfusepath{stroke}%
\end{pgfscope}%
\begin{pgfscope}%
\pgfpathrectangle{\pgfqpoint{0.580766in}{0.532919in}}{\pgfqpoint{5.611646in}{3.193122in}} %
\pgfusepath{clip}%
\pgfsetbuttcap%
\pgfsetroundjoin%
\pgfsetlinewidth{0.501875pt}%
\definecolor{currentstroke}{rgb}{0.000000,0.000000,0.000000}%
\pgfsetstrokecolor{currentstroke}%
\pgfsetdash{{1.000000pt}{3.000000pt}}{0.000000pt}%
\pgfpathmoveto{\pgfqpoint{0.580766in}{0.532919in}}%
\pgfpathlineto{\pgfqpoint{0.580766in}{3.726041in}}%
\pgfusepath{stroke}%
\end{pgfscope}%
\begin{pgfscope}%
\pgfsetbuttcap%
\pgfsetroundjoin%
\definecolor{currentfill}{rgb}{0.000000,0.000000,0.000000}%
\pgfsetfillcolor{currentfill}%
\pgfsetlinewidth{0.501875pt}%
\definecolor{currentstroke}{rgb}{0.000000,0.000000,0.000000}%
\pgfsetstrokecolor{currentstroke}%
\pgfsetdash{}{0pt}%
\pgfsys@defobject{currentmarker}{\pgfqpoint{0.000000in}{0.000000in}}{\pgfqpoint{0.000000in}{0.055556in}}{%
\pgfpathmoveto{\pgfqpoint{0.000000in}{0.000000in}}%
\pgfpathlineto{\pgfqpoint{0.000000in}{0.055556in}}%
\pgfusepath{stroke,fill}%
}%
\begin{pgfscope}%
\pgfsys@transformshift{0.580766in}{0.532919in}%
\pgfsys@useobject{currentmarker}{}%
\end{pgfscope}%
\end{pgfscope}%
\begin{pgfscope}%
\pgfsetbuttcap%
\pgfsetroundjoin%
\definecolor{currentfill}{rgb}{0.000000,0.000000,0.000000}%
\pgfsetfillcolor{currentfill}%
\pgfsetlinewidth{0.501875pt}%
\definecolor{currentstroke}{rgb}{0.000000,0.000000,0.000000}%
\pgfsetstrokecolor{currentstroke}%
\pgfsetdash{}{0pt}%
\pgfsys@defobject{currentmarker}{\pgfqpoint{0.000000in}{-0.055556in}}{\pgfqpoint{0.000000in}{0.000000in}}{%
\pgfpathmoveto{\pgfqpoint{0.000000in}{0.000000in}}%
\pgfpathlineto{\pgfqpoint{0.000000in}{-0.055556in}}%
\pgfusepath{stroke,fill}%
}%
\begin{pgfscope}%
\pgfsys@transformshift{0.580766in}{3.726041in}%
\pgfsys@useobject{currentmarker}{}%
\end{pgfscope}%
\end{pgfscope}%
\begin{pgfscope}%
\pgftext[x=0.580766in,y=0.477363in,,top]{{\rmfamily\fontsize{10.000000}{12.000000}\selectfont \(\displaystyle 0.000\)}}%
\end{pgfscope}%
\begin{pgfscope}%
\pgfpathrectangle{\pgfqpoint{0.580766in}{0.532919in}}{\pgfqpoint{5.611646in}{3.193122in}} %
\pgfusepath{clip}%
\pgfsetbuttcap%
\pgfsetroundjoin%
\pgfsetlinewidth{0.501875pt}%
\definecolor{currentstroke}{rgb}{0.000000,0.000000,0.000000}%
\pgfsetstrokecolor{currentstroke}%
\pgfsetdash{{1.000000pt}{3.000000pt}}{0.000000pt}%
\pgfpathmoveto{\pgfqpoint{1.703095in}{0.532919in}}%
\pgfpathlineto{\pgfqpoint{1.703095in}{3.726041in}}%
\pgfusepath{stroke}%
\end{pgfscope}%
\begin{pgfscope}%
\pgfsetbuttcap%
\pgfsetroundjoin%
\definecolor{currentfill}{rgb}{0.000000,0.000000,0.000000}%
\pgfsetfillcolor{currentfill}%
\pgfsetlinewidth{0.501875pt}%
\definecolor{currentstroke}{rgb}{0.000000,0.000000,0.000000}%
\pgfsetstrokecolor{currentstroke}%
\pgfsetdash{}{0pt}%
\pgfsys@defobject{currentmarker}{\pgfqpoint{0.000000in}{0.000000in}}{\pgfqpoint{0.000000in}{0.055556in}}{%
\pgfpathmoveto{\pgfqpoint{0.000000in}{0.000000in}}%
\pgfpathlineto{\pgfqpoint{0.000000in}{0.055556in}}%
\pgfusepath{stroke,fill}%
}%
\begin{pgfscope}%
\pgfsys@transformshift{1.703095in}{0.532919in}%
\pgfsys@useobject{currentmarker}{}%
\end{pgfscope}%
\end{pgfscope}%
\begin{pgfscope}%
\pgfsetbuttcap%
\pgfsetroundjoin%
\definecolor{currentfill}{rgb}{0.000000,0.000000,0.000000}%
\pgfsetfillcolor{currentfill}%
\pgfsetlinewidth{0.501875pt}%
\definecolor{currentstroke}{rgb}{0.000000,0.000000,0.000000}%
\pgfsetstrokecolor{currentstroke}%
\pgfsetdash{}{0pt}%
\pgfsys@defobject{currentmarker}{\pgfqpoint{0.000000in}{-0.055556in}}{\pgfqpoint{0.000000in}{0.000000in}}{%
\pgfpathmoveto{\pgfqpoint{0.000000in}{0.000000in}}%
\pgfpathlineto{\pgfqpoint{0.000000in}{-0.055556in}}%
\pgfusepath{stroke,fill}%
}%
\begin{pgfscope}%
\pgfsys@transformshift{1.703095in}{3.726041in}%
\pgfsys@useobject{currentmarker}{}%
\end{pgfscope}%
\end{pgfscope}%
\begin{pgfscope}%
\pgftext[x=1.703095in,y=0.477363in,,top]{{\rmfamily\fontsize{10.000000}{12.000000}\selectfont \(\displaystyle 0.002\)}}%
\end{pgfscope}%
\begin{pgfscope}%
\pgfpathrectangle{\pgfqpoint{0.580766in}{0.532919in}}{\pgfqpoint{5.611646in}{3.193122in}} %
\pgfusepath{clip}%
\pgfsetbuttcap%
\pgfsetroundjoin%
\pgfsetlinewidth{0.501875pt}%
\definecolor{currentstroke}{rgb}{0.000000,0.000000,0.000000}%
\pgfsetstrokecolor{currentstroke}%
\pgfsetdash{{1.000000pt}{3.000000pt}}{0.000000pt}%
\pgfpathmoveto{\pgfqpoint{2.825424in}{0.532919in}}%
\pgfpathlineto{\pgfqpoint{2.825424in}{3.726041in}}%
\pgfusepath{stroke}%
\end{pgfscope}%
\begin{pgfscope}%
\pgfsetbuttcap%
\pgfsetroundjoin%
\definecolor{currentfill}{rgb}{0.000000,0.000000,0.000000}%
\pgfsetfillcolor{currentfill}%
\pgfsetlinewidth{0.501875pt}%
\definecolor{currentstroke}{rgb}{0.000000,0.000000,0.000000}%
\pgfsetstrokecolor{currentstroke}%
\pgfsetdash{}{0pt}%
\pgfsys@defobject{currentmarker}{\pgfqpoint{0.000000in}{0.000000in}}{\pgfqpoint{0.000000in}{0.055556in}}{%
\pgfpathmoveto{\pgfqpoint{0.000000in}{0.000000in}}%
\pgfpathlineto{\pgfqpoint{0.000000in}{0.055556in}}%
\pgfusepath{stroke,fill}%
}%
\begin{pgfscope}%
\pgfsys@transformshift{2.825424in}{0.532919in}%
\pgfsys@useobject{currentmarker}{}%
\end{pgfscope}%
\end{pgfscope}%
\begin{pgfscope}%
\pgfsetbuttcap%
\pgfsetroundjoin%
\definecolor{currentfill}{rgb}{0.000000,0.000000,0.000000}%
\pgfsetfillcolor{currentfill}%
\pgfsetlinewidth{0.501875pt}%
\definecolor{currentstroke}{rgb}{0.000000,0.000000,0.000000}%
\pgfsetstrokecolor{currentstroke}%
\pgfsetdash{}{0pt}%
\pgfsys@defobject{currentmarker}{\pgfqpoint{0.000000in}{-0.055556in}}{\pgfqpoint{0.000000in}{0.000000in}}{%
\pgfpathmoveto{\pgfqpoint{0.000000in}{0.000000in}}%
\pgfpathlineto{\pgfqpoint{0.000000in}{-0.055556in}}%
\pgfusepath{stroke,fill}%
}%
\begin{pgfscope}%
\pgfsys@transformshift{2.825424in}{3.726041in}%
\pgfsys@useobject{currentmarker}{}%
\end{pgfscope}%
\end{pgfscope}%
\begin{pgfscope}%
\pgftext[x=2.825424in,y=0.477363in,,top]{{\rmfamily\fontsize{10.000000}{12.000000}\selectfont \(\displaystyle 0.004\)}}%
\end{pgfscope}%
\begin{pgfscope}%
\pgfpathrectangle{\pgfqpoint{0.580766in}{0.532919in}}{\pgfqpoint{5.611646in}{3.193122in}} %
\pgfusepath{clip}%
\pgfsetbuttcap%
\pgfsetroundjoin%
\pgfsetlinewidth{0.501875pt}%
\definecolor{currentstroke}{rgb}{0.000000,0.000000,0.000000}%
\pgfsetstrokecolor{currentstroke}%
\pgfsetdash{{1.000000pt}{3.000000pt}}{0.000000pt}%
\pgfpathmoveto{\pgfqpoint{3.947753in}{0.532919in}}%
\pgfpathlineto{\pgfqpoint{3.947753in}{3.726041in}}%
\pgfusepath{stroke}%
\end{pgfscope}%
\begin{pgfscope}%
\pgfsetbuttcap%
\pgfsetroundjoin%
\definecolor{currentfill}{rgb}{0.000000,0.000000,0.000000}%
\pgfsetfillcolor{currentfill}%
\pgfsetlinewidth{0.501875pt}%
\definecolor{currentstroke}{rgb}{0.000000,0.000000,0.000000}%
\pgfsetstrokecolor{currentstroke}%
\pgfsetdash{}{0pt}%
\pgfsys@defobject{currentmarker}{\pgfqpoint{0.000000in}{0.000000in}}{\pgfqpoint{0.000000in}{0.055556in}}{%
\pgfpathmoveto{\pgfqpoint{0.000000in}{0.000000in}}%
\pgfpathlineto{\pgfqpoint{0.000000in}{0.055556in}}%
\pgfusepath{stroke,fill}%
}%
\begin{pgfscope}%
\pgfsys@transformshift{3.947753in}{0.532919in}%
\pgfsys@useobject{currentmarker}{}%
\end{pgfscope}%
\end{pgfscope}%
\begin{pgfscope}%
\pgfsetbuttcap%
\pgfsetroundjoin%
\definecolor{currentfill}{rgb}{0.000000,0.000000,0.000000}%
\pgfsetfillcolor{currentfill}%
\pgfsetlinewidth{0.501875pt}%
\definecolor{currentstroke}{rgb}{0.000000,0.000000,0.000000}%
\pgfsetstrokecolor{currentstroke}%
\pgfsetdash{}{0pt}%
\pgfsys@defobject{currentmarker}{\pgfqpoint{0.000000in}{-0.055556in}}{\pgfqpoint{0.000000in}{0.000000in}}{%
\pgfpathmoveto{\pgfqpoint{0.000000in}{0.000000in}}%
\pgfpathlineto{\pgfqpoint{0.000000in}{-0.055556in}}%
\pgfusepath{stroke,fill}%
}%
\begin{pgfscope}%
\pgfsys@transformshift{3.947753in}{3.726041in}%
\pgfsys@useobject{currentmarker}{}%
\end{pgfscope}%
\end{pgfscope}%
\begin{pgfscope}%
\pgftext[x=3.947753in,y=0.477363in,,top]{{\rmfamily\fontsize{10.000000}{12.000000}\selectfont \(\displaystyle 0.006\)}}%
\end{pgfscope}%
\begin{pgfscope}%
\pgfpathrectangle{\pgfqpoint{0.580766in}{0.532919in}}{\pgfqpoint{5.611646in}{3.193122in}} %
\pgfusepath{clip}%
\pgfsetbuttcap%
\pgfsetroundjoin%
\pgfsetlinewidth{0.501875pt}%
\definecolor{currentstroke}{rgb}{0.000000,0.000000,0.000000}%
\pgfsetstrokecolor{currentstroke}%
\pgfsetdash{{1.000000pt}{3.000000pt}}{0.000000pt}%
\pgfpathmoveto{\pgfqpoint{5.070082in}{0.532919in}}%
\pgfpathlineto{\pgfqpoint{5.070082in}{3.726041in}}%
\pgfusepath{stroke}%
\end{pgfscope}%
\begin{pgfscope}%
\pgfsetbuttcap%
\pgfsetroundjoin%
\definecolor{currentfill}{rgb}{0.000000,0.000000,0.000000}%
\pgfsetfillcolor{currentfill}%
\pgfsetlinewidth{0.501875pt}%
\definecolor{currentstroke}{rgb}{0.000000,0.000000,0.000000}%
\pgfsetstrokecolor{currentstroke}%
\pgfsetdash{}{0pt}%
\pgfsys@defobject{currentmarker}{\pgfqpoint{0.000000in}{0.000000in}}{\pgfqpoint{0.000000in}{0.055556in}}{%
\pgfpathmoveto{\pgfqpoint{0.000000in}{0.000000in}}%
\pgfpathlineto{\pgfqpoint{0.000000in}{0.055556in}}%
\pgfusepath{stroke,fill}%
}%
\begin{pgfscope}%
\pgfsys@transformshift{5.070082in}{0.532919in}%
\pgfsys@useobject{currentmarker}{}%
\end{pgfscope}%
\end{pgfscope}%
\begin{pgfscope}%
\pgfsetbuttcap%
\pgfsetroundjoin%
\definecolor{currentfill}{rgb}{0.000000,0.000000,0.000000}%
\pgfsetfillcolor{currentfill}%
\pgfsetlinewidth{0.501875pt}%
\definecolor{currentstroke}{rgb}{0.000000,0.000000,0.000000}%
\pgfsetstrokecolor{currentstroke}%
\pgfsetdash{}{0pt}%
\pgfsys@defobject{currentmarker}{\pgfqpoint{0.000000in}{-0.055556in}}{\pgfqpoint{0.000000in}{0.000000in}}{%
\pgfpathmoveto{\pgfqpoint{0.000000in}{0.000000in}}%
\pgfpathlineto{\pgfqpoint{0.000000in}{-0.055556in}}%
\pgfusepath{stroke,fill}%
}%
\begin{pgfscope}%
\pgfsys@transformshift{5.070082in}{3.726041in}%
\pgfsys@useobject{currentmarker}{}%
\end{pgfscope}%
\end{pgfscope}%
\begin{pgfscope}%
\pgftext[x=5.070082in,y=0.477363in,,top]{{\rmfamily\fontsize{10.000000}{12.000000}\selectfont \(\displaystyle 0.008\)}}%
\end{pgfscope}%
\begin{pgfscope}%
\pgfpathrectangle{\pgfqpoint{0.580766in}{0.532919in}}{\pgfqpoint{5.611646in}{3.193122in}} %
\pgfusepath{clip}%
\pgfsetbuttcap%
\pgfsetroundjoin%
\pgfsetlinewidth{0.501875pt}%
\definecolor{currentstroke}{rgb}{0.000000,0.000000,0.000000}%
\pgfsetstrokecolor{currentstroke}%
\pgfsetdash{{1.000000pt}{3.000000pt}}{0.000000pt}%
\pgfpathmoveto{\pgfqpoint{6.192411in}{0.532919in}}%
\pgfpathlineto{\pgfqpoint{6.192411in}{3.726041in}}%
\pgfusepath{stroke}%
\end{pgfscope}%
\begin{pgfscope}%
\pgfsetbuttcap%
\pgfsetroundjoin%
\definecolor{currentfill}{rgb}{0.000000,0.000000,0.000000}%
\pgfsetfillcolor{currentfill}%
\pgfsetlinewidth{0.501875pt}%
\definecolor{currentstroke}{rgb}{0.000000,0.000000,0.000000}%
\pgfsetstrokecolor{currentstroke}%
\pgfsetdash{}{0pt}%
\pgfsys@defobject{currentmarker}{\pgfqpoint{0.000000in}{0.000000in}}{\pgfqpoint{0.000000in}{0.055556in}}{%
\pgfpathmoveto{\pgfqpoint{0.000000in}{0.000000in}}%
\pgfpathlineto{\pgfqpoint{0.000000in}{0.055556in}}%
\pgfusepath{stroke,fill}%
}%
\begin{pgfscope}%
\pgfsys@transformshift{6.192411in}{0.532919in}%
\pgfsys@useobject{currentmarker}{}%
\end{pgfscope}%
\end{pgfscope}%
\begin{pgfscope}%
\pgfsetbuttcap%
\pgfsetroundjoin%
\definecolor{currentfill}{rgb}{0.000000,0.000000,0.000000}%
\pgfsetfillcolor{currentfill}%
\pgfsetlinewidth{0.501875pt}%
\definecolor{currentstroke}{rgb}{0.000000,0.000000,0.000000}%
\pgfsetstrokecolor{currentstroke}%
\pgfsetdash{}{0pt}%
\pgfsys@defobject{currentmarker}{\pgfqpoint{0.000000in}{-0.055556in}}{\pgfqpoint{0.000000in}{0.000000in}}{%
\pgfpathmoveto{\pgfqpoint{0.000000in}{0.000000in}}%
\pgfpathlineto{\pgfqpoint{0.000000in}{-0.055556in}}%
\pgfusepath{stroke,fill}%
}%
\begin{pgfscope}%
\pgfsys@transformshift{6.192411in}{3.726041in}%
\pgfsys@useobject{currentmarker}{}%
\end{pgfscope}%
\end{pgfscope}%
\begin{pgfscope}%
\pgftext[x=6.192411in,y=0.477363in,,top]{{\rmfamily\fontsize{10.000000}{12.000000}\selectfont \(\displaystyle 0.010\)}}%
\end{pgfscope}%
\begin{pgfscope}%
\pgftext[x=3.386589in,y=0.284462in,,top]{{\rmfamily\fontsize{10.000000}{12.000000}\selectfont time (s)}}%
\end{pgfscope}%
\begin{pgfscope}%
\pgfpathrectangle{\pgfqpoint{0.580766in}{0.532919in}}{\pgfqpoint{5.611646in}{3.193122in}} %
\pgfusepath{clip}%
\pgfsetbuttcap%
\pgfsetroundjoin%
\pgfsetlinewidth{0.501875pt}%
\definecolor{currentstroke}{rgb}{0.000000,0.000000,0.000000}%
\pgfsetstrokecolor{currentstroke}%
\pgfsetdash{{1.000000pt}{3.000000pt}}{0.000000pt}%
\pgfpathmoveto{\pgfqpoint{0.580766in}{0.532919in}}%
\pgfpathlineto{\pgfqpoint{6.192411in}{0.532919in}}%
\pgfusepath{stroke}%
\end{pgfscope}%
\begin{pgfscope}%
\pgfsetbuttcap%
\pgfsetroundjoin%
\definecolor{currentfill}{rgb}{0.000000,0.000000,0.000000}%
\pgfsetfillcolor{currentfill}%
\pgfsetlinewidth{0.501875pt}%
\definecolor{currentstroke}{rgb}{0.000000,0.000000,0.000000}%
\pgfsetstrokecolor{currentstroke}%
\pgfsetdash{}{0pt}%
\pgfsys@defobject{currentmarker}{\pgfqpoint{0.000000in}{0.000000in}}{\pgfqpoint{0.055556in}{0.000000in}}{%
\pgfpathmoveto{\pgfqpoint{0.000000in}{0.000000in}}%
\pgfpathlineto{\pgfqpoint{0.055556in}{0.000000in}}%
\pgfusepath{stroke,fill}%
}%
\begin{pgfscope}%
\pgfsys@transformshift{0.580766in}{0.532919in}%
\pgfsys@useobject{currentmarker}{}%
\end{pgfscope}%
\end{pgfscope}%
\begin{pgfscope}%
\pgfsetbuttcap%
\pgfsetroundjoin%
\definecolor{currentfill}{rgb}{0.000000,0.000000,0.000000}%
\pgfsetfillcolor{currentfill}%
\pgfsetlinewidth{0.501875pt}%
\definecolor{currentstroke}{rgb}{0.000000,0.000000,0.000000}%
\pgfsetstrokecolor{currentstroke}%
\pgfsetdash{}{0pt}%
\pgfsys@defobject{currentmarker}{\pgfqpoint{-0.055556in}{0.000000in}}{\pgfqpoint{0.000000in}{0.000000in}}{%
\pgfpathmoveto{\pgfqpoint{0.000000in}{0.000000in}}%
\pgfpathlineto{\pgfqpoint{-0.055556in}{0.000000in}}%
\pgfusepath{stroke,fill}%
}%
\begin{pgfscope}%
\pgfsys@transformshift{6.192411in}{0.532919in}%
\pgfsys@useobject{currentmarker}{}%
\end{pgfscope}%
\end{pgfscope}%
\begin{pgfscope}%
\pgftext[x=0.525210in,y=0.532919in,right,]{{\rmfamily\fontsize{10.000000}{12.000000}\selectfont \(\displaystyle 0\)}}%
\end{pgfscope}%
\begin{pgfscope}%
\pgfpathrectangle{\pgfqpoint{0.580766in}{0.532919in}}{\pgfqpoint{5.611646in}{3.193122in}} %
\pgfusepath{clip}%
\pgfsetbuttcap%
\pgfsetroundjoin%
\pgfsetlinewidth{0.501875pt}%
\definecolor{currentstroke}{rgb}{0.000000,0.000000,0.000000}%
\pgfsetstrokecolor{currentstroke}%
\pgfsetdash{{1.000000pt}{3.000000pt}}{0.000000pt}%
\pgfpathmoveto{\pgfqpoint{0.580766in}{1.171543in}}%
\pgfpathlineto{\pgfqpoint{6.192411in}{1.171543in}}%
\pgfusepath{stroke}%
\end{pgfscope}%
\begin{pgfscope}%
\pgfsetbuttcap%
\pgfsetroundjoin%
\definecolor{currentfill}{rgb}{0.000000,0.000000,0.000000}%
\pgfsetfillcolor{currentfill}%
\pgfsetlinewidth{0.501875pt}%
\definecolor{currentstroke}{rgb}{0.000000,0.000000,0.000000}%
\pgfsetstrokecolor{currentstroke}%
\pgfsetdash{}{0pt}%
\pgfsys@defobject{currentmarker}{\pgfqpoint{0.000000in}{0.000000in}}{\pgfqpoint{0.055556in}{0.000000in}}{%
\pgfpathmoveto{\pgfqpoint{0.000000in}{0.000000in}}%
\pgfpathlineto{\pgfqpoint{0.055556in}{0.000000in}}%
\pgfusepath{stroke,fill}%
}%
\begin{pgfscope}%
\pgfsys@transformshift{0.580766in}{1.171543in}%
\pgfsys@useobject{currentmarker}{}%
\end{pgfscope}%
\end{pgfscope}%
\begin{pgfscope}%
\pgfsetbuttcap%
\pgfsetroundjoin%
\definecolor{currentfill}{rgb}{0.000000,0.000000,0.000000}%
\pgfsetfillcolor{currentfill}%
\pgfsetlinewidth{0.501875pt}%
\definecolor{currentstroke}{rgb}{0.000000,0.000000,0.000000}%
\pgfsetstrokecolor{currentstroke}%
\pgfsetdash{}{0pt}%
\pgfsys@defobject{currentmarker}{\pgfqpoint{-0.055556in}{0.000000in}}{\pgfqpoint{0.000000in}{0.000000in}}{%
\pgfpathmoveto{\pgfqpoint{0.000000in}{0.000000in}}%
\pgfpathlineto{\pgfqpoint{-0.055556in}{0.000000in}}%
\pgfusepath{stroke,fill}%
}%
\begin{pgfscope}%
\pgfsys@transformshift{6.192411in}{1.171543in}%
\pgfsys@useobject{currentmarker}{}%
\end{pgfscope}%
\end{pgfscope}%
\begin{pgfscope}%
\pgftext[x=0.525210in,y=1.171543in,right,]{{\rmfamily\fontsize{10.000000}{12.000000}\selectfont \(\displaystyle 2\)}}%
\end{pgfscope}%
\begin{pgfscope}%
\pgfpathrectangle{\pgfqpoint{0.580766in}{0.532919in}}{\pgfqpoint{5.611646in}{3.193122in}} %
\pgfusepath{clip}%
\pgfsetbuttcap%
\pgfsetroundjoin%
\pgfsetlinewidth{0.501875pt}%
\definecolor{currentstroke}{rgb}{0.000000,0.000000,0.000000}%
\pgfsetstrokecolor{currentstroke}%
\pgfsetdash{{1.000000pt}{3.000000pt}}{0.000000pt}%
\pgfpathmoveto{\pgfqpoint{0.580766in}{1.810167in}}%
\pgfpathlineto{\pgfqpoint{6.192411in}{1.810167in}}%
\pgfusepath{stroke}%
\end{pgfscope}%
\begin{pgfscope}%
\pgfsetbuttcap%
\pgfsetroundjoin%
\definecolor{currentfill}{rgb}{0.000000,0.000000,0.000000}%
\pgfsetfillcolor{currentfill}%
\pgfsetlinewidth{0.501875pt}%
\definecolor{currentstroke}{rgb}{0.000000,0.000000,0.000000}%
\pgfsetstrokecolor{currentstroke}%
\pgfsetdash{}{0pt}%
\pgfsys@defobject{currentmarker}{\pgfqpoint{0.000000in}{0.000000in}}{\pgfqpoint{0.055556in}{0.000000in}}{%
\pgfpathmoveto{\pgfqpoint{0.000000in}{0.000000in}}%
\pgfpathlineto{\pgfqpoint{0.055556in}{0.000000in}}%
\pgfusepath{stroke,fill}%
}%
\begin{pgfscope}%
\pgfsys@transformshift{0.580766in}{1.810167in}%
\pgfsys@useobject{currentmarker}{}%
\end{pgfscope}%
\end{pgfscope}%
\begin{pgfscope}%
\pgfsetbuttcap%
\pgfsetroundjoin%
\definecolor{currentfill}{rgb}{0.000000,0.000000,0.000000}%
\pgfsetfillcolor{currentfill}%
\pgfsetlinewidth{0.501875pt}%
\definecolor{currentstroke}{rgb}{0.000000,0.000000,0.000000}%
\pgfsetstrokecolor{currentstroke}%
\pgfsetdash{}{0pt}%
\pgfsys@defobject{currentmarker}{\pgfqpoint{-0.055556in}{0.000000in}}{\pgfqpoint{0.000000in}{0.000000in}}{%
\pgfpathmoveto{\pgfqpoint{0.000000in}{0.000000in}}%
\pgfpathlineto{\pgfqpoint{-0.055556in}{0.000000in}}%
\pgfusepath{stroke,fill}%
}%
\begin{pgfscope}%
\pgfsys@transformshift{6.192411in}{1.810167in}%
\pgfsys@useobject{currentmarker}{}%
\end{pgfscope}%
\end{pgfscope}%
\begin{pgfscope}%
\pgftext[x=0.525210in,y=1.810167in,right,]{{\rmfamily\fontsize{10.000000}{12.000000}\selectfont \(\displaystyle 4\)}}%
\end{pgfscope}%
\begin{pgfscope}%
\pgfpathrectangle{\pgfqpoint{0.580766in}{0.532919in}}{\pgfqpoint{5.611646in}{3.193122in}} %
\pgfusepath{clip}%
\pgfsetbuttcap%
\pgfsetroundjoin%
\pgfsetlinewidth{0.501875pt}%
\definecolor{currentstroke}{rgb}{0.000000,0.000000,0.000000}%
\pgfsetstrokecolor{currentstroke}%
\pgfsetdash{{1.000000pt}{3.000000pt}}{0.000000pt}%
\pgfpathmoveto{\pgfqpoint{0.580766in}{2.448792in}}%
\pgfpathlineto{\pgfqpoint{6.192411in}{2.448792in}}%
\pgfusepath{stroke}%
\end{pgfscope}%
\begin{pgfscope}%
\pgfsetbuttcap%
\pgfsetroundjoin%
\definecolor{currentfill}{rgb}{0.000000,0.000000,0.000000}%
\pgfsetfillcolor{currentfill}%
\pgfsetlinewidth{0.501875pt}%
\definecolor{currentstroke}{rgb}{0.000000,0.000000,0.000000}%
\pgfsetstrokecolor{currentstroke}%
\pgfsetdash{}{0pt}%
\pgfsys@defobject{currentmarker}{\pgfqpoint{0.000000in}{0.000000in}}{\pgfqpoint{0.055556in}{0.000000in}}{%
\pgfpathmoveto{\pgfqpoint{0.000000in}{0.000000in}}%
\pgfpathlineto{\pgfqpoint{0.055556in}{0.000000in}}%
\pgfusepath{stroke,fill}%
}%
\begin{pgfscope}%
\pgfsys@transformshift{0.580766in}{2.448792in}%
\pgfsys@useobject{currentmarker}{}%
\end{pgfscope}%
\end{pgfscope}%
\begin{pgfscope}%
\pgfsetbuttcap%
\pgfsetroundjoin%
\definecolor{currentfill}{rgb}{0.000000,0.000000,0.000000}%
\pgfsetfillcolor{currentfill}%
\pgfsetlinewidth{0.501875pt}%
\definecolor{currentstroke}{rgb}{0.000000,0.000000,0.000000}%
\pgfsetstrokecolor{currentstroke}%
\pgfsetdash{}{0pt}%
\pgfsys@defobject{currentmarker}{\pgfqpoint{-0.055556in}{0.000000in}}{\pgfqpoint{0.000000in}{0.000000in}}{%
\pgfpathmoveto{\pgfqpoint{0.000000in}{0.000000in}}%
\pgfpathlineto{\pgfqpoint{-0.055556in}{0.000000in}}%
\pgfusepath{stroke,fill}%
}%
\begin{pgfscope}%
\pgfsys@transformshift{6.192411in}{2.448792in}%
\pgfsys@useobject{currentmarker}{}%
\end{pgfscope}%
\end{pgfscope}%
\begin{pgfscope}%
\pgftext[x=0.525210in,y=2.448792in,right,]{{\rmfamily\fontsize{10.000000}{12.000000}\selectfont \(\displaystyle 6\)}}%
\end{pgfscope}%
\begin{pgfscope}%
\pgfpathrectangle{\pgfqpoint{0.580766in}{0.532919in}}{\pgfqpoint{5.611646in}{3.193122in}} %
\pgfusepath{clip}%
\pgfsetbuttcap%
\pgfsetroundjoin%
\pgfsetlinewidth{0.501875pt}%
\definecolor{currentstroke}{rgb}{0.000000,0.000000,0.000000}%
\pgfsetstrokecolor{currentstroke}%
\pgfsetdash{{1.000000pt}{3.000000pt}}{0.000000pt}%
\pgfpathmoveto{\pgfqpoint{0.580766in}{3.087416in}}%
\pgfpathlineto{\pgfqpoint{6.192411in}{3.087416in}}%
\pgfusepath{stroke}%
\end{pgfscope}%
\begin{pgfscope}%
\pgfsetbuttcap%
\pgfsetroundjoin%
\definecolor{currentfill}{rgb}{0.000000,0.000000,0.000000}%
\pgfsetfillcolor{currentfill}%
\pgfsetlinewidth{0.501875pt}%
\definecolor{currentstroke}{rgb}{0.000000,0.000000,0.000000}%
\pgfsetstrokecolor{currentstroke}%
\pgfsetdash{}{0pt}%
\pgfsys@defobject{currentmarker}{\pgfqpoint{0.000000in}{0.000000in}}{\pgfqpoint{0.055556in}{0.000000in}}{%
\pgfpathmoveto{\pgfqpoint{0.000000in}{0.000000in}}%
\pgfpathlineto{\pgfqpoint{0.055556in}{0.000000in}}%
\pgfusepath{stroke,fill}%
}%
\begin{pgfscope}%
\pgfsys@transformshift{0.580766in}{3.087416in}%
\pgfsys@useobject{currentmarker}{}%
\end{pgfscope}%
\end{pgfscope}%
\begin{pgfscope}%
\pgfsetbuttcap%
\pgfsetroundjoin%
\definecolor{currentfill}{rgb}{0.000000,0.000000,0.000000}%
\pgfsetfillcolor{currentfill}%
\pgfsetlinewidth{0.501875pt}%
\definecolor{currentstroke}{rgb}{0.000000,0.000000,0.000000}%
\pgfsetstrokecolor{currentstroke}%
\pgfsetdash{}{0pt}%
\pgfsys@defobject{currentmarker}{\pgfqpoint{-0.055556in}{0.000000in}}{\pgfqpoint{0.000000in}{0.000000in}}{%
\pgfpathmoveto{\pgfqpoint{0.000000in}{0.000000in}}%
\pgfpathlineto{\pgfqpoint{-0.055556in}{0.000000in}}%
\pgfusepath{stroke,fill}%
}%
\begin{pgfscope}%
\pgfsys@transformshift{6.192411in}{3.087416in}%
\pgfsys@useobject{currentmarker}{}%
\end{pgfscope}%
\end{pgfscope}%
\begin{pgfscope}%
\pgftext[x=0.525210in,y=3.087416in,right,]{{\rmfamily\fontsize{10.000000}{12.000000}\selectfont \(\displaystyle 8\)}}%
\end{pgfscope}%
\begin{pgfscope}%
\pgfpathrectangle{\pgfqpoint{0.580766in}{0.532919in}}{\pgfqpoint{5.611646in}{3.193122in}} %
\pgfusepath{clip}%
\pgfsetbuttcap%
\pgfsetroundjoin%
\pgfsetlinewidth{0.501875pt}%
\definecolor{currentstroke}{rgb}{0.000000,0.000000,0.000000}%
\pgfsetstrokecolor{currentstroke}%
\pgfsetdash{{1.000000pt}{3.000000pt}}{0.000000pt}%
\pgfpathmoveto{\pgfqpoint{0.580766in}{3.726041in}}%
\pgfpathlineto{\pgfqpoint{6.192411in}{3.726041in}}%
\pgfusepath{stroke}%
\end{pgfscope}%
\begin{pgfscope}%
\pgfsetbuttcap%
\pgfsetroundjoin%
\definecolor{currentfill}{rgb}{0.000000,0.000000,0.000000}%
\pgfsetfillcolor{currentfill}%
\pgfsetlinewidth{0.501875pt}%
\definecolor{currentstroke}{rgb}{0.000000,0.000000,0.000000}%
\pgfsetstrokecolor{currentstroke}%
\pgfsetdash{}{0pt}%
\pgfsys@defobject{currentmarker}{\pgfqpoint{0.000000in}{0.000000in}}{\pgfqpoint{0.055556in}{0.000000in}}{%
\pgfpathmoveto{\pgfqpoint{0.000000in}{0.000000in}}%
\pgfpathlineto{\pgfqpoint{0.055556in}{0.000000in}}%
\pgfusepath{stroke,fill}%
}%
\begin{pgfscope}%
\pgfsys@transformshift{0.580766in}{3.726041in}%
\pgfsys@useobject{currentmarker}{}%
\end{pgfscope}%
\end{pgfscope}%
\begin{pgfscope}%
\pgfsetbuttcap%
\pgfsetroundjoin%
\definecolor{currentfill}{rgb}{0.000000,0.000000,0.000000}%
\pgfsetfillcolor{currentfill}%
\pgfsetlinewidth{0.501875pt}%
\definecolor{currentstroke}{rgb}{0.000000,0.000000,0.000000}%
\pgfsetstrokecolor{currentstroke}%
\pgfsetdash{}{0pt}%
\pgfsys@defobject{currentmarker}{\pgfqpoint{-0.055556in}{0.000000in}}{\pgfqpoint{0.000000in}{0.000000in}}{%
\pgfpathmoveto{\pgfqpoint{0.000000in}{0.000000in}}%
\pgfpathlineto{\pgfqpoint{-0.055556in}{0.000000in}}%
\pgfusepath{stroke,fill}%
}%
\begin{pgfscope}%
\pgfsys@transformshift{6.192411in}{3.726041in}%
\pgfsys@useobject{currentmarker}{}%
\end{pgfscope}%
\end{pgfscope}%
\begin{pgfscope}%
\pgftext[x=0.525210in,y=3.726041in,right,]{{\rmfamily\fontsize{10.000000}{12.000000}\selectfont \(\displaystyle 10\)}}%
\end{pgfscope}%
\begin{pgfscope}%
\pgftext[x=0.316877in,y=2.129480in,,bottom,rotate=90.000000]{{\rmfamily\fontsize{10.000000}{12.000000}\selectfont voltage (V)}}%
\end{pgfscope}%
\begin{pgfscope}%
\pgfsetbuttcap%
\pgfsetroundjoin%
\pgfsetlinewidth{1.003750pt}%
\definecolor{currentstroke}{rgb}{0.000000,0.000000,0.000000}%
\pgfsetstrokecolor{currentstroke}%
\pgfsetdash{}{0pt}%
\pgfpathmoveto{\pgfqpoint{0.580766in}{3.726041in}}%
\pgfpathlineto{\pgfqpoint{6.192411in}{3.726041in}}%
\pgfusepath{stroke}%
\end{pgfscope}%
\begin{pgfscope}%
\pgfsetbuttcap%
\pgfsetroundjoin%
\pgfsetlinewidth{1.003750pt}%
\definecolor{currentstroke}{rgb}{0.000000,0.000000,0.000000}%
\pgfsetstrokecolor{currentstroke}%
\pgfsetdash{}{0pt}%
\pgfpathmoveto{\pgfqpoint{6.192411in}{0.532919in}}%
\pgfpathlineto{\pgfqpoint{6.192411in}{3.726041in}}%
\pgfusepath{stroke}%
\end{pgfscope}%
\begin{pgfscope}%
\pgfsetbuttcap%
\pgfsetroundjoin%
\pgfsetlinewidth{1.003750pt}%
\definecolor{currentstroke}{rgb}{0.000000,0.000000,0.000000}%
\pgfsetstrokecolor{currentstroke}%
\pgfsetdash{}{0pt}%
\pgfpathmoveto{\pgfqpoint{0.580766in}{0.532919in}}%
\pgfpathlineto{\pgfqpoint{6.192411in}{0.532919in}}%
\pgfusepath{stroke}%
\end{pgfscope}%
\begin{pgfscope}%
\pgfsetbuttcap%
\pgfsetroundjoin%
\pgfsetlinewidth{1.003750pt}%
\definecolor{currentstroke}{rgb}{0.000000,0.000000,0.000000}%
\pgfsetstrokecolor{currentstroke}%
\pgfsetdash{}{0pt}%
\pgfpathmoveto{\pgfqpoint{0.580766in}{0.532919in}}%
\pgfpathlineto{\pgfqpoint{0.580766in}{3.726041in}}%
\pgfusepath{stroke}%
\end{pgfscope}%
\begin{pgfscope}%
\pgftext[x=3.386589in,y=3.795485in,,base]{{\rmfamily\fontsize{12.000000}{14.400000}\selectfont Step-By-Step Approximations of \(\displaystyle v_L(t)\)}}%
\end{pgfscope}%
\begin{pgfscope}%
\pgfsetbuttcap%
\pgfsetroundjoin%
\definecolor{currentfill}{rgb}{0.300000,0.300000,0.300000}%
\pgfsetfillcolor{currentfill}%
\pgfsetfillopacity{0.500000}%
\pgfsetlinewidth{1.003750pt}%
\definecolor{currentstroke}{rgb}{0.300000,0.300000,0.300000}%
\pgfsetstrokecolor{currentstroke}%
\pgfsetstrokeopacity{0.500000}%
\pgfsetdash{}{0pt}%
\pgfpathmoveto{\pgfqpoint{3.306478in}{2.832523in}}%
\pgfpathlineto{\pgfqpoint{6.103523in}{2.832523in}}%
\pgfpathquadraticcurveto{\pgfqpoint{6.136856in}{2.832523in}}{\pgfqpoint{6.136856in}{2.865856in}}%
\pgfpathlineto{\pgfqpoint{6.136856in}{3.581596in}}%
\pgfpathquadraticcurveto{\pgfqpoint{6.136856in}{3.614930in}}{\pgfqpoint{6.103523in}{3.614930in}}%
\pgfpathlineto{\pgfqpoint{3.306478in}{3.614930in}}%
\pgfpathquadraticcurveto{\pgfqpoint{3.273145in}{3.614930in}}{\pgfqpoint{3.273145in}{3.581596in}}%
\pgfpathlineto{\pgfqpoint{3.273145in}{2.865856in}}%
\pgfpathquadraticcurveto{\pgfqpoint{3.273145in}{2.832523in}}{\pgfqpoint{3.306478in}{2.832523in}}%
\pgfpathclose%
\pgfusepath{stroke,fill}%
\end{pgfscope}%
\begin{pgfscope}%
\pgfsetbuttcap%
\pgfsetroundjoin%
\definecolor{currentfill}{rgb}{1.000000,1.000000,1.000000}%
\pgfsetfillcolor{currentfill}%
\pgfsetlinewidth{1.003750pt}%
\definecolor{currentstroke}{rgb}{0.000000,0.000000,0.000000}%
\pgfsetstrokecolor{currentstroke}%
\pgfsetdash{}{0pt}%
\pgfpathmoveto{\pgfqpoint{3.278700in}{2.860300in}}%
\pgfpathlineto{\pgfqpoint{6.075745in}{2.860300in}}%
\pgfpathquadraticcurveto{\pgfqpoint{6.109078in}{2.860300in}}{\pgfqpoint{6.109078in}{2.893634in}}%
\pgfpathlineto{\pgfqpoint{6.109078in}{3.609374in}}%
\pgfpathquadraticcurveto{\pgfqpoint{6.109078in}{3.642708in}}{\pgfqpoint{6.075745in}{3.642708in}}%
\pgfpathlineto{\pgfqpoint{3.278700in}{3.642708in}}%
\pgfpathquadraticcurveto{\pgfqpoint{3.245367in}{3.642708in}}{\pgfqpoint{3.245367in}{3.609374in}}%
\pgfpathlineto{\pgfqpoint{3.245367in}{2.893634in}}%
\pgfpathquadraticcurveto{\pgfqpoint{3.245367in}{2.860300in}}{\pgfqpoint{3.278700in}{2.860300in}}%
\pgfpathclose%
\pgfusepath{stroke,fill}%
\end{pgfscope}%
\begin{pgfscope}%
\pgfsetbuttcap%
\pgfsetroundjoin%
\pgfsetlinewidth{1.003750pt}%
\definecolor{currentstroke}{rgb}{0.000000,0.000000,1.000000}%
\pgfsetstrokecolor{currentstroke}%
\pgfsetdash{{1.000000pt}{3.000000pt}}{0.000000pt}%
\pgfpathmoveto{\pgfqpoint{3.362034in}{3.517708in}}%
\pgfpathlineto{\pgfqpoint{3.595367in}{3.517708in}}%
\pgfusepath{stroke}%
\end{pgfscope}%
\begin{pgfscope}%
\pgftext[x=3.778700in,y=3.459374in,left,base]{{\rmfamily\fontsize{12.000000}{14.400000}\selectfont Continuous}}%
\end{pgfscope}%
\begin{pgfscope}%
\pgfsetrectcap%
\pgfsetroundjoin%
\pgfsetlinewidth{1.003750pt}%
\definecolor{currentstroke}{rgb}{0.000000,0.500000,0.000000}%
\pgfsetstrokecolor{currentstroke}%
\pgfsetdash{}{0pt}%
\pgfpathmoveto{\pgfqpoint{3.362034in}{3.276967in}}%
\pgfpathlineto{\pgfqpoint{3.595367in}{3.276967in}}%
\pgfusepath{stroke}%
\end{pgfscope}%
\begin{pgfscope}%
\pgftext[x=3.778700in,y=3.218634in,left,base]{{\rmfamily\fontsize{12.000000}{14.400000}\selectfont Backward Euler (\(\displaystyle \Delta{}t = 0.0001 s\))}}%
\end{pgfscope}%
\begin{pgfscope}%
\pgfsetrectcap%
\pgfsetroundjoin%
\pgfsetlinewidth{1.003750pt}%
\definecolor{currentstroke}{rgb}{1.000000,0.000000,0.000000}%
\pgfsetstrokecolor{currentstroke}%
\pgfsetdash{}{0pt}%
\pgfpathmoveto{\pgfqpoint{3.362034in}{3.026967in}}%
\pgfpathlineto{\pgfqpoint{3.595367in}{3.026967in}}%
\pgfusepath{stroke}%
\end{pgfscope}%
\begin{pgfscope}%
\pgftext[x=3.778700in,y=2.968634in,left,base]{{\rmfamily\fontsize{12.000000}{14.400000}\selectfont Backward Euler (\(\displaystyle \Delta{}t = 0.0008 s\))}}%
\end{pgfscope}%
\end{pgfpicture}%
\makeatother%
\endgroup%

    \end{center}
    \caption{Backward Euler Approximations at $\Delta{}t_1 = 0.1 ms, \Delta{}t_2 = 0.8 ms$}
    \label{back_approx}
\end{figure}

\begin{figure}[H]
    \begin{center}
        %% Creator: Matplotlib, PGF backend
%%
%% To include the figure in your LaTeX document, write
%%   \input{<filename>.pgf}
%%
%% Make sure the required packages are loaded in your preamble
%%   \usepackage{pgf}
%%
%% Figures using additional raster images can only be included by \input if
%% they are in the same directory as the main LaTeX file. For loading figures
%% from other directories you can use the `import` package
%%   \usepackage{import}
%% and then include the figures with
%%   \import{<path to file>}{<filename>.pgf}
%%
%% Matplotlib used the following preamble
%%
\begingroup%
\makeatletter%
\begin{pgfpicture}%
\pgfpathrectangle{\pgfpointorigin}{\pgfqpoint{6.500000in}{8.000000in}}%
\pgfusepath{use as bounding box}%
\begin{pgfscope}%
\pgfsetbuttcap%
\pgfsetroundjoin%
\definecolor{currentfill}{rgb}{1.000000,1.000000,1.000000}%
\pgfsetfillcolor{currentfill}%
\pgfsetlinewidth{0.000000pt}%
\definecolor{currentstroke}{rgb}{1.000000,1.000000,1.000000}%
\pgfsetstrokecolor{currentstroke}%
\pgfsetdash{}{0pt}%
\pgfpathmoveto{\pgfqpoint{0.000000in}{0.000000in}}%
\pgfpathlineto{\pgfqpoint{6.500000in}{0.000000in}}%
\pgfpathlineto{\pgfqpoint{6.500000in}{8.000000in}}%
\pgfpathlineto{\pgfqpoint{0.000000in}{8.000000in}}%
\pgfpathclose%
\pgfusepath{fill}%
\end{pgfscope}%
\begin{pgfscope}%
\pgfsetbuttcap%
\pgfsetroundjoin%
\definecolor{currentfill}{rgb}{1.000000,1.000000,1.000000}%
\pgfsetfillcolor{currentfill}%
\pgfsetlinewidth{0.000000pt}%
\definecolor{currentstroke}{rgb}{0.000000,0.000000,0.000000}%
\pgfsetstrokecolor{currentstroke}%
\pgfsetstrokeopacity{0.000000}%
\pgfsetdash{}{0pt}%
\pgfpathmoveto{\pgfqpoint{0.580766in}{4.462900in}}%
\pgfpathlineto{\pgfqpoint{6.192411in}{4.462900in}}%
\pgfpathlineto{\pgfqpoint{6.192411in}{7.656023in}}%
\pgfpathlineto{\pgfqpoint{0.580766in}{7.656023in}}%
\pgfpathclose%
\pgfusepath{fill}%
\end{pgfscope}%
\begin{pgfscope}%
\pgfpathrectangle{\pgfqpoint{0.580766in}{4.462900in}}{\pgfqpoint{5.611646in}{3.193122in}} %
\pgfusepath{clip}%
\pgfsetbuttcap%
\pgfsetroundjoin%
\pgfsetlinewidth{1.003750pt}%
\definecolor{currentstroke}{rgb}{0.000000,0.000000,1.000000}%
\pgfsetstrokecolor{currentstroke}%
\pgfsetdash{{1.000000pt}{3.000000pt}}{0.000000pt}%
\pgfpathmoveto{\pgfqpoint{0.580766in}{4.462900in}}%
\pgfpathlineto{\pgfqpoint{0.633141in}{4.608489in}}%
\pgfpathlineto{\pgfqpoint{0.685517in}{4.747440in}}%
\pgfpathlineto{\pgfqpoint{0.737892in}{4.880055in}}%
\pgfpathlineto{\pgfqpoint{0.790267in}{5.006624in}}%
\pgfpathlineto{\pgfqpoint{0.842643in}{5.127422in}}%
\pgfpathlineto{\pgfqpoint{0.895018in}{5.242713in}}%
\pgfpathlineto{\pgfqpoint{0.947393in}{5.352746in}}%
\pgfpathlineto{\pgfqpoint{0.999769in}{5.457763in}}%
\pgfpathlineto{\pgfqpoint{1.054015in}{5.561485in}}%
\pgfpathlineto{\pgfqpoint{1.108261in}{5.660314in}}%
\pgfpathlineto{\pgfqpoint{1.162506in}{5.754479in}}%
\pgfpathlineto{\pgfqpoint{1.216752in}{5.844201in}}%
\pgfpathlineto{\pgfqpoint{1.270998in}{5.929690in}}%
\pgfpathlineto{\pgfqpoint{1.325244in}{6.011145in}}%
\pgfpathlineto{\pgfqpoint{1.381361in}{6.091367in}}%
\pgfpathlineto{\pgfqpoint{1.437477in}{6.167676in}}%
\pgfpathlineto{\pgfqpoint{1.493594in}{6.240263in}}%
\pgfpathlineto{\pgfqpoint{1.549710in}{6.309311in}}%
\pgfpathlineto{\pgfqpoint{1.607697in}{6.377124in}}%
\pgfpathlineto{\pgfqpoint{1.665684in}{6.441522in}}%
\pgfpathlineto{\pgfqpoint{1.723671in}{6.502678in}}%
\pgfpathlineto{\pgfqpoint{1.783529in}{6.562578in}}%
\pgfpathlineto{\pgfqpoint{1.843386in}{6.619367in}}%
\pgfpathlineto{\pgfqpoint{1.905114in}{6.674844in}}%
\pgfpathlineto{\pgfqpoint{1.966842in}{6.727352in}}%
\pgfpathlineto{\pgfqpoint{2.030441in}{6.778513in}}%
\pgfpathlineto{\pgfqpoint{2.094040in}{6.826856in}}%
\pgfpathlineto{\pgfqpoint{2.159509in}{6.873840in}}%
\pgfpathlineto{\pgfqpoint{2.226849in}{6.919391in}}%
\pgfpathlineto{\pgfqpoint{2.294188in}{6.962289in}}%
\pgfpathlineto{\pgfqpoint{2.363399in}{7.003777in}}%
\pgfpathlineto{\pgfqpoint{2.434479in}{7.043805in}}%
\pgfpathlineto{\pgfqpoint{2.507431in}{7.082333in}}%
\pgfpathlineto{\pgfqpoint{2.582253in}{7.119332in}}%
\pgfpathlineto{\pgfqpoint{2.658945in}{7.154781in}}%
\pgfpathlineto{\pgfqpoint{2.737508in}{7.188668in}}%
\pgfpathlineto{\pgfqpoint{2.819812in}{7.221714in}}%
\pgfpathlineto{\pgfqpoint{2.903987in}{7.253096in}}%
\pgfpathlineto{\pgfqpoint{2.991903in}{7.283454in}}%
\pgfpathlineto{\pgfqpoint{3.081689in}{7.312098in}}%
\pgfpathlineto{\pgfqpoint{3.175217in}{7.339597in}}%
\pgfpathlineto{\pgfqpoint{3.272485in}{7.365866in}}%
\pgfpathlineto{\pgfqpoint{3.373495in}{7.390839in}}%
\pgfpathlineto{\pgfqpoint{3.480116in}{7.414872in}}%
\pgfpathlineto{\pgfqpoint{3.590478in}{7.437457in}}%
\pgfpathlineto{\pgfqpoint{3.706452in}{7.458914in}}%
\pgfpathlineto{\pgfqpoint{3.828038in}{7.479151in}}%
\pgfpathlineto{\pgfqpoint{3.957106in}{7.498366in}}%
\pgfpathlineto{\pgfqpoint{4.093656in}{7.516426in}}%
\pgfpathlineto{\pgfqpoint{4.239559in}{7.533444in}}%
\pgfpathlineto{\pgfqpoint{4.394814in}{7.549280in}}%
\pgfpathlineto{\pgfqpoint{4.561293in}{7.563995in}}%
\pgfpathlineto{\pgfqpoint{4.740866in}{7.577602in}}%
\pgfpathlineto{\pgfqpoint{4.935403in}{7.590082in}}%
\pgfpathlineto{\pgfqpoint{5.148645in}{7.601492in}}%
\pgfpathlineto{\pgfqpoint{5.382464in}{7.611748in}}%
\pgfpathlineto{\pgfqpoint{5.644341in}{7.620962in}}%
\pgfpathlineto{\pgfqpoint{5.939887in}{7.629079in}}%
\pgfpathlineto{\pgfqpoint{6.190541in}{7.634472in}}%
\pgfpathlineto{\pgfqpoint{6.190541in}{7.634472in}}%
\pgfusepath{stroke}%
\end{pgfscope}%
\begin{pgfscope}%
\pgfpathrectangle{\pgfqpoint{0.580766in}{4.462900in}}{\pgfqpoint{5.611646in}{3.193122in}} %
\pgfusepath{clip}%
\pgfsetrectcap%
\pgfsetroundjoin%
\pgfsetlinewidth{1.003750pt}%
\definecolor{currentstroke}{rgb}{0.000000,0.500000,0.000000}%
\pgfsetstrokecolor{currentstroke}%
\pgfsetdash{}{0pt}%
\pgfpathmoveto{\pgfqpoint{0.580766in}{5.527274in}}%
\pgfpathlineto{\pgfqpoint{1.029698in}{6.236857in}}%
\pgfpathlineto{\pgfqpoint{1.478629in}{6.709912in}}%
\pgfpathlineto{\pgfqpoint{1.927561in}{7.025282in}}%
\pgfpathlineto{\pgfqpoint{2.376492in}{7.235529in}}%
\pgfpathlineto{\pgfqpoint{2.825424in}{7.375694in}}%
\pgfpathlineto{\pgfqpoint{3.274356in}{7.469137in}}%
\pgfpathlineto{\pgfqpoint{3.723287in}{7.531432in}}%
\pgfpathlineto{\pgfqpoint{4.172219in}{7.572962in}}%
\pgfpathlineto{\pgfqpoint{4.621151in}{7.600649in}}%
\pgfpathlineto{\pgfqpoint{5.070082in}{7.619107in}}%
\pgfpathlineto{\pgfqpoint{5.519014in}{7.631412in}}%
\pgfpathlineto{\pgfqpoint{5.967946in}{7.639616in}}%
\pgfusepath{stroke}%
\end{pgfscope}%
\begin{pgfscope}%
\pgfpathrectangle{\pgfqpoint{0.580766in}{4.462900in}}{\pgfqpoint{5.611646in}{3.193122in}} %
\pgfusepath{clip}%
\pgfsetrectcap%
\pgfsetroundjoin%
\pgfsetlinewidth{1.003750pt}%
\definecolor{currentstroke}{rgb}{1.000000,0.000000,0.000000}%
\pgfsetstrokecolor{currentstroke}%
\pgfsetdash{}{0pt}%
\pgfpathmoveto{\pgfqpoint{0.580766in}{5.375221in}}%
\pgfpathlineto{\pgfqpoint{1.029698in}{6.026878in}}%
\pgfpathlineto{\pgfqpoint{1.478629in}{6.492348in}}%
\pgfpathlineto{\pgfqpoint{1.927561in}{6.824827in}}%
\pgfpathlineto{\pgfqpoint{2.376492in}{7.062311in}}%
\pgfpathlineto{\pgfqpoint{2.825424in}{7.231943in}}%
\pgfpathlineto{\pgfqpoint{3.274356in}{7.353109in}}%
\pgfpathlineto{\pgfqpoint{3.723287in}{7.439655in}}%
\pgfpathlineto{\pgfqpoint{4.172219in}{7.501475in}}%
\pgfpathlineto{\pgfqpoint{4.621151in}{7.545631in}}%
\pgfpathlineto{\pgfqpoint{5.070082in}{7.577172in}}%
\pgfpathlineto{\pgfqpoint{5.519014in}{7.599700in}}%
\pgfpathlineto{\pgfqpoint{5.967946in}{7.615792in}}%
\pgfusepath{stroke}%
\end{pgfscope}%
\begin{pgfscope}%
\pgfpathrectangle{\pgfqpoint{0.580766in}{4.462900in}}{\pgfqpoint{5.611646in}{3.193122in}} %
\pgfusepath{clip}%
\pgfsetbuttcap%
\pgfsetroundjoin%
\pgfsetlinewidth{0.501875pt}%
\definecolor{currentstroke}{rgb}{0.000000,0.000000,0.000000}%
\pgfsetstrokecolor{currentstroke}%
\pgfsetdash{{1.000000pt}{3.000000pt}}{0.000000pt}%
\pgfpathmoveto{\pgfqpoint{0.580766in}{4.462900in}}%
\pgfpathlineto{\pgfqpoint{0.580766in}{7.656023in}}%
\pgfusepath{stroke}%
\end{pgfscope}%
\begin{pgfscope}%
\pgfsetbuttcap%
\pgfsetroundjoin%
\definecolor{currentfill}{rgb}{0.000000,0.000000,0.000000}%
\pgfsetfillcolor{currentfill}%
\pgfsetlinewidth{0.501875pt}%
\definecolor{currentstroke}{rgb}{0.000000,0.000000,0.000000}%
\pgfsetstrokecolor{currentstroke}%
\pgfsetdash{}{0pt}%
\pgfsys@defobject{currentmarker}{\pgfqpoint{0.000000in}{0.000000in}}{\pgfqpoint{0.000000in}{0.055556in}}{%
\pgfpathmoveto{\pgfqpoint{0.000000in}{0.000000in}}%
\pgfpathlineto{\pgfqpoint{0.000000in}{0.055556in}}%
\pgfusepath{stroke,fill}%
}%
\begin{pgfscope}%
\pgfsys@transformshift{0.580766in}{4.462900in}%
\pgfsys@useobject{currentmarker}{}%
\end{pgfscope}%
\end{pgfscope}%
\begin{pgfscope}%
\pgfsetbuttcap%
\pgfsetroundjoin%
\definecolor{currentfill}{rgb}{0.000000,0.000000,0.000000}%
\pgfsetfillcolor{currentfill}%
\pgfsetlinewidth{0.501875pt}%
\definecolor{currentstroke}{rgb}{0.000000,0.000000,0.000000}%
\pgfsetstrokecolor{currentstroke}%
\pgfsetdash{}{0pt}%
\pgfsys@defobject{currentmarker}{\pgfqpoint{0.000000in}{-0.055556in}}{\pgfqpoint{0.000000in}{0.000000in}}{%
\pgfpathmoveto{\pgfqpoint{0.000000in}{0.000000in}}%
\pgfpathlineto{\pgfqpoint{0.000000in}{-0.055556in}}%
\pgfusepath{stroke,fill}%
}%
\begin{pgfscope}%
\pgfsys@transformshift{0.580766in}{7.656023in}%
\pgfsys@useobject{currentmarker}{}%
\end{pgfscope}%
\end{pgfscope}%
\begin{pgfscope}%
\pgftext[x=0.580766in,y=4.407345in,,top]{{\rmfamily\fontsize{10.000000}{12.000000}\selectfont \(\displaystyle 0.000\)}}%
\end{pgfscope}%
\begin{pgfscope}%
\pgfpathrectangle{\pgfqpoint{0.580766in}{4.462900in}}{\pgfqpoint{5.611646in}{3.193122in}} %
\pgfusepath{clip}%
\pgfsetbuttcap%
\pgfsetroundjoin%
\pgfsetlinewidth{0.501875pt}%
\definecolor{currentstroke}{rgb}{0.000000,0.000000,0.000000}%
\pgfsetstrokecolor{currentstroke}%
\pgfsetdash{{1.000000pt}{3.000000pt}}{0.000000pt}%
\pgfpathmoveto{\pgfqpoint{1.703095in}{4.462900in}}%
\pgfpathlineto{\pgfqpoint{1.703095in}{7.656023in}}%
\pgfusepath{stroke}%
\end{pgfscope}%
\begin{pgfscope}%
\pgfsetbuttcap%
\pgfsetroundjoin%
\definecolor{currentfill}{rgb}{0.000000,0.000000,0.000000}%
\pgfsetfillcolor{currentfill}%
\pgfsetlinewidth{0.501875pt}%
\definecolor{currentstroke}{rgb}{0.000000,0.000000,0.000000}%
\pgfsetstrokecolor{currentstroke}%
\pgfsetdash{}{0pt}%
\pgfsys@defobject{currentmarker}{\pgfqpoint{0.000000in}{0.000000in}}{\pgfqpoint{0.000000in}{0.055556in}}{%
\pgfpathmoveto{\pgfqpoint{0.000000in}{0.000000in}}%
\pgfpathlineto{\pgfqpoint{0.000000in}{0.055556in}}%
\pgfusepath{stroke,fill}%
}%
\begin{pgfscope}%
\pgfsys@transformshift{1.703095in}{4.462900in}%
\pgfsys@useobject{currentmarker}{}%
\end{pgfscope}%
\end{pgfscope}%
\begin{pgfscope}%
\pgfsetbuttcap%
\pgfsetroundjoin%
\definecolor{currentfill}{rgb}{0.000000,0.000000,0.000000}%
\pgfsetfillcolor{currentfill}%
\pgfsetlinewidth{0.501875pt}%
\definecolor{currentstroke}{rgb}{0.000000,0.000000,0.000000}%
\pgfsetstrokecolor{currentstroke}%
\pgfsetdash{}{0pt}%
\pgfsys@defobject{currentmarker}{\pgfqpoint{0.000000in}{-0.055556in}}{\pgfqpoint{0.000000in}{0.000000in}}{%
\pgfpathmoveto{\pgfqpoint{0.000000in}{0.000000in}}%
\pgfpathlineto{\pgfqpoint{0.000000in}{-0.055556in}}%
\pgfusepath{stroke,fill}%
}%
\begin{pgfscope}%
\pgfsys@transformshift{1.703095in}{7.656023in}%
\pgfsys@useobject{currentmarker}{}%
\end{pgfscope}%
\end{pgfscope}%
\begin{pgfscope}%
\pgftext[x=1.703095in,y=4.407345in,,top]{{\rmfamily\fontsize{10.000000}{12.000000}\selectfont \(\displaystyle 0.002\)}}%
\end{pgfscope}%
\begin{pgfscope}%
\pgfpathrectangle{\pgfqpoint{0.580766in}{4.462900in}}{\pgfqpoint{5.611646in}{3.193122in}} %
\pgfusepath{clip}%
\pgfsetbuttcap%
\pgfsetroundjoin%
\pgfsetlinewidth{0.501875pt}%
\definecolor{currentstroke}{rgb}{0.000000,0.000000,0.000000}%
\pgfsetstrokecolor{currentstroke}%
\pgfsetdash{{1.000000pt}{3.000000pt}}{0.000000pt}%
\pgfpathmoveto{\pgfqpoint{2.825424in}{4.462900in}}%
\pgfpathlineto{\pgfqpoint{2.825424in}{7.656023in}}%
\pgfusepath{stroke}%
\end{pgfscope}%
\begin{pgfscope}%
\pgfsetbuttcap%
\pgfsetroundjoin%
\definecolor{currentfill}{rgb}{0.000000,0.000000,0.000000}%
\pgfsetfillcolor{currentfill}%
\pgfsetlinewidth{0.501875pt}%
\definecolor{currentstroke}{rgb}{0.000000,0.000000,0.000000}%
\pgfsetstrokecolor{currentstroke}%
\pgfsetdash{}{0pt}%
\pgfsys@defobject{currentmarker}{\pgfqpoint{0.000000in}{0.000000in}}{\pgfqpoint{0.000000in}{0.055556in}}{%
\pgfpathmoveto{\pgfqpoint{0.000000in}{0.000000in}}%
\pgfpathlineto{\pgfqpoint{0.000000in}{0.055556in}}%
\pgfusepath{stroke,fill}%
}%
\begin{pgfscope}%
\pgfsys@transformshift{2.825424in}{4.462900in}%
\pgfsys@useobject{currentmarker}{}%
\end{pgfscope}%
\end{pgfscope}%
\begin{pgfscope}%
\pgfsetbuttcap%
\pgfsetroundjoin%
\definecolor{currentfill}{rgb}{0.000000,0.000000,0.000000}%
\pgfsetfillcolor{currentfill}%
\pgfsetlinewidth{0.501875pt}%
\definecolor{currentstroke}{rgb}{0.000000,0.000000,0.000000}%
\pgfsetstrokecolor{currentstroke}%
\pgfsetdash{}{0pt}%
\pgfsys@defobject{currentmarker}{\pgfqpoint{0.000000in}{-0.055556in}}{\pgfqpoint{0.000000in}{0.000000in}}{%
\pgfpathmoveto{\pgfqpoint{0.000000in}{0.000000in}}%
\pgfpathlineto{\pgfqpoint{0.000000in}{-0.055556in}}%
\pgfusepath{stroke,fill}%
}%
\begin{pgfscope}%
\pgfsys@transformshift{2.825424in}{7.656023in}%
\pgfsys@useobject{currentmarker}{}%
\end{pgfscope}%
\end{pgfscope}%
\begin{pgfscope}%
\pgftext[x=2.825424in,y=4.407345in,,top]{{\rmfamily\fontsize{10.000000}{12.000000}\selectfont \(\displaystyle 0.004\)}}%
\end{pgfscope}%
\begin{pgfscope}%
\pgfpathrectangle{\pgfqpoint{0.580766in}{4.462900in}}{\pgfqpoint{5.611646in}{3.193122in}} %
\pgfusepath{clip}%
\pgfsetbuttcap%
\pgfsetroundjoin%
\pgfsetlinewidth{0.501875pt}%
\definecolor{currentstroke}{rgb}{0.000000,0.000000,0.000000}%
\pgfsetstrokecolor{currentstroke}%
\pgfsetdash{{1.000000pt}{3.000000pt}}{0.000000pt}%
\pgfpathmoveto{\pgfqpoint{3.947753in}{4.462900in}}%
\pgfpathlineto{\pgfqpoint{3.947753in}{7.656023in}}%
\pgfusepath{stroke}%
\end{pgfscope}%
\begin{pgfscope}%
\pgfsetbuttcap%
\pgfsetroundjoin%
\definecolor{currentfill}{rgb}{0.000000,0.000000,0.000000}%
\pgfsetfillcolor{currentfill}%
\pgfsetlinewidth{0.501875pt}%
\definecolor{currentstroke}{rgb}{0.000000,0.000000,0.000000}%
\pgfsetstrokecolor{currentstroke}%
\pgfsetdash{}{0pt}%
\pgfsys@defobject{currentmarker}{\pgfqpoint{0.000000in}{0.000000in}}{\pgfqpoint{0.000000in}{0.055556in}}{%
\pgfpathmoveto{\pgfqpoint{0.000000in}{0.000000in}}%
\pgfpathlineto{\pgfqpoint{0.000000in}{0.055556in}}%
\pgfusepath{stroke,fill}%
}%
\begin{pgfscope}%
\pgfsys@transformshift{3.947753in}{4.462900in}%
\pgfsys@useobject{currentmarker}{}%
\end{pgfscope}%
\end{pgfscope}%
\begin{pgfscope}%
\pgfsetbuttcap%
\pgfsetroundjoin%
\definecolor{currentfill}{rgb}{0.000000,0.000000,0.000000}%
\pgfsetfillcolor{currentfill}%
\pgfsetlinewidth{0.501875pt}%
\definecolor{currentstroke}{rgb}{0.000000,0.000000,0.000000}%
\pgfsetstrokecolor{currentstroke}%
\pgfsetdash{}{0pt}%
\pgfsys@defobject{currentmarker}{\pgfqpoint{0.000000in}{-0.055556in}}{\pgfqpoint{0.000000in}{0.000000in}}{%
\pgfpathmoveto{\pgfqpoint{0.000000in}{0.000000in}}%
\pgfpathlineto{\pgfqpoint{0.000000in}{-0.055556in}}%
\pgfusepath{stroke,fill}%
}%
\begin{pgfscope}%
\pgfsys@transformshift{3.947753in}{7.656023in}%
\pgfsys@useobject{currentmarker}{}%
\end{pgfscope}%
\end{pgfscope}%
\begin{pgfscope}%
\pgftext[x=3.947753in,y=4.407345in,,top]{{\rmfamily\fontsize{10.000000}{12.000000}\selectfont \(\displaystyle 0.006\)}}%
\end{pgfscope}%
\begin{pgfscope}%
\pgfpathrectangle{\pgfqpoint{0.580766in}{4.462900in}}{\pgfqpoint{5.611646in}{3.193122in}} %
\pgfusepath{clip}%
\pgfsetbuttcap%
\pgfsetroundjoin%
\pgfsetlinewidth{0.501875pt}%
\definecolor{currentstroke}{rgb}{0.000000,0.000000,0.000000}%
\pgfsetstrokecolor{currentstroke}%
\pgfsetdash{{1.000000pt}{3.000000pt}}{0.000000pt}%
\pgfpathmoveto{\pgfqpoint{5.070082in}{4.462900in}}%
\pgfpathlineto{\pgfqpoint{5.070082in}{7.656023in}}%
\pgfusepath{stroke}%
\end{pgfscope}%
\begin{pgfscope}%
\pgfsetbuttcap%
\pgfsetroundjoin%
\definecolor{currentfill}{rgb}{0.000000,0.000000,0.000000}%
\pgfsetfillcolor{currentfill}%
\pgfsetlinewidth{0.501875pt}%
\definecolor{currentstroke}{rgb}{0.000000,0.000000,0.000000}%
\pgfsetstrokecolor{currentstroke}%
\pgfsetdash{}{0pt}%
\pgfsys@defobject{currentmarker}{\pgfqpoint{0.000000in}{0.000000in}}{\pgfqpoint{0.000000in}{0.055556in}}{%
\pgfpathmoveto{\pgfqpoint{0.000000in}{0.000000in}}%
\pgfpathlineto{\pgfqpoint{0.000000in}{0.055556in}}%
\pgfusepath{stroke,fill}%
}%
\begin{pgfscope}%
\pgfsys@transformshift{5.070082in}{4.462900in}%
\pgfsys@useobject{currentmarker}{}%
\end{pgfscope}%
\end{pgfscope}%
\begin{pgfscope}%
\pgfsetbuttcap%
\pgfsetroundjoin%
\definecolor{currentfill}{rgb}{0.000000,0.000000,0.000000}%
\pgfsetfillcolor{currentfill}%
\pgfsetlinewidth{0.501875pt}%
\definecolor{currentstroke}{rgb}{0.000000,0.000000,0.000000}%
\pgfsetstrokecolor{currentstroke}%
\pgfsetdash{}{0pt}%
\pgfsys@defobject{currentmarker}{\pgfqpoint{0.000000in}{-0.055556in}}{\pgfqpoint{0.000000in}{0.000000in}}{%
\pgfpathmoveto{\pgfqpoint{0.000000in}{0.000000in}}%
\pgfpathlineto{\pgfqpoint{0.000000in}{-0.055556in}}%
\pgfusepath{stroke,fill}%
}%
\begin{pgfscope}%
\pgfsys@transformshift{5.070082in}{7.656023in}%
\pgfsys@useobject{currentmarker}{}%
\end{pgfscope}%
\end{pgfscope}%
\begin{pgfscope}%
\pgftext[x=5.070082in,y=4.407345in,,top]{{\rmfamily\fontsize{10.000000}{12.000000}\selectfont \(\displaystyle 0.008\)}}%
\end{pgfscope}%
\begin{pgfscope}%
\pgfpathrectangle{\pgfqpoint{0.580766in}{4.462900in}}{\pgfqpoint{5.611646in}{3.193122in}} %
\pgfusepath{clip}%
\pgfsetbuttcap%
\pgfsetroundjoin%
\pgfsetlinewidth{0.501875pt}%
\definecolor{currentstroke}{rgb}{0.000000,0.000000,0.000000}%
\pgfsetstrokecolor{currentstroke}%
\pgfsetdash{{1.000000pt}{3.000000pt}}{0.000000pt}%
\pgfpathmoveto{\pgfqpoint{6.192411in}{4.462900in}}%
\pgfpathlineto{\pgfqpoint{6.192411in}{7.656023in}}%
\pgfusepath{stroke}%
\end{pgfscope}%
\begin{pgfscope}%
\pgfsetbuttcap%
\pgfsetroundjoin%
\definecolor{currentfill}{rgb}{0.000000,0.000000,0.000000}%
\pgfsetfillcolor{currentfill}%
\pgfsetlinewidth{0.501875pt}%
\definecolor{currentstroke}{rgb}{0.000000,0.000000,0.000000}%
\pgfsetstrokecolor{currentstroke}%
\pgfsetdash{}{0pt}%
\pgfsys@defobject{currentmarker}{\pgfqpoint{0.000000in}{0.000000in}}{\pgfqpoint{0.000000in}{0.055556in}}{%
\pgfpathmoveto{\pgfqpoint{0.000000in}{0.000000in}}%
\pgfpathlineto{\pgfqpoint{0.000000in}{0.055556in}}%
\pgfusepath{stroke,fill}%
}%
\begin{pgfscope}%
\pgfsys@transformshift{6.192411in}{4.462900in}%
\pgfsys@useobject{currentmarker}{}%
\end{pgfscope}%
\end{pgfscope}%
\begin{pgfscope}%
\pgfsetbuttcap%
\pgfsetroundjoin%
\definecolor{currentfill}{rgb}{0.000000,0.000000,0.000000}%
\pgfsetfillcolor{currentfill}%
\pgfsetlinewidth{0.501875pt}%
\definecolor{currentstroke}{rgb}{0.000000,0.000000,0.000000}%
\pgfsetstrokecolor{currentstroke}%
\pgfsetdash{}{0pt}%
\pgfsys@defobject{currentmarker}{\pgfqpoint{0.000000in}{-0.055556in}}{\pgfqpoint{0.000000in}{0.000000in}}{%
\pgfpathmoveto{\pgfqpoint{0.000000in}{0.000000in}}%
\pgfpathlineto{\pgfqpoint{0.000000in}{-0.055556in}}%
\pgfusepath{stroke,fill}%
}%
\begin{pgfscope}%
\pgfsys@transformshift{6.192411in}{7.656023in}%
\pgfsys@useobject{currentmarker}{}%
\end{pgfscope}%
\end{pgfscope}%
\begin{pgfscope}%
\pgftext[x=6.192411in,y=4.407345in,,top]{{\rmfamily\fontsize{10.000000}{12.000000}\selectfont \(\displaystyle 0.010\)}}%
\end{pgfscope}%
\begin{pgfscope}%
\pgftext[x=3.386589in,y=4.214443in,,top]{{\rmfamily\fontsize{10.000000}{12.000000}\selectfont time (s)}}%
\end{pgfscope}%
\begin{pgfscope}%
\pgfpathrectangle{\pgfqpoint{0.580766in}{4.462900in}}{\pgfqpoint{5.611646in}{3.193122in}} %
\pgfusepath{clip}%
\pgfsetbuttcap%
\pgfsetroundjoin%
\pgfsetlinewidth{0.501875pt}%
\definecolor{currentstroke}{rgb}{0.000000,0.000000,0.000000}%
\pgfsetstrokecolor{currentstroke}%
\pgfsetdash{{1.000000pt}{3.000000pt}}{0.000000pt}%
\pgfpathmoveto{\pgfqpoint{0.580766in}{4.462900in}}%
\pgfpathlineto{\pgfqpoint{6.192411in}{4.462900in}}%
\pgfusepath{stroke}%
\end{pgfscope}%
\begin{pgfscope}%
\pgfsetbuttcap%
\pgfsetroundjoin%
\definecolor{currentfill}{rgb}{0.000000,0.000000,0.000000}%
\pgfsetfillcolor{currentfill}%
\pgfsetlinewidth{0.501875pt}%
\definecolor{currentstroke}{rgb}{0.000000,0.000000,0.000000}%
\pgfsetstrokecolor{currentstroke}%
\pgfsetdash{}{0pt}%
\pgfsys@defobject{currentmarker}{\pgfqpoint{0.000000in}{0.000000in}}{\pgfqpoint{0.055556in}{0.000000in}}{%
\pgfpathmoveto{\pgfqpoint{0.000000in}{0.000000in}}%
\pgfpathlineto{\pgfqpoint{0.055556in}{0.000000in}}%
\pgfusepath{stroke,fill}%
}%
\begin{pgfscope}%
\pgfsys@transformshift{0.580766in}{4.462900in}%
\pgfsys@useobject{currentmarker}{}%
\end{pgfscope}%
\end{pgfscope}%
\begin{pgfscope}%
\pgfsetbuttcap%
\pgfsetroundjoin%
\definecolor{currentfill}{rgb}{0.000000,0.000000,0.000000}%
\pgfsetfillcolor{currentfill}%
\pgfsetlinewidth{0.501875pt}%
\definecolor{currentstroke}{rgb}{0.000000,0.000000,0.000000}%
\pgfsetstrokecolor{currentstroke}%
\pgfsetdash{}{0pt}%
\pgfsys@defobject{currentmarker}{\pgfqpoint{-0.055556in}{0.000000in}}{\pgfqpoint{0.000000in}{0.000000in}}{%
\pgfpathmoveto{\pgfqpoint{0.000000in}{0.000000in}}%
\pgfpathlineto{\pgfqpoint{-0.055556in}{0.000000in}}%
\pgfusepath{stroke,fill}%
}%
\begin{pgfscope}%
\pgfsys@transformshift{6.192411in}{4.462900in}%
\pgfsys@useobject{currentmarker}{}%
\end{pgfscope}%
\end{pgfscope}%
\begin{pgfscope}%
\pgftext[x=0.525210in,y=4.462900in,right,]{{\rmfamily\fontsize{10.000000}{12.000000}\selectfont \(\displaystyle 0.0\)}}%
\end{pgfscope}%
\begin{pgfscope}%
\pgfpathrectangle{\pgfqpoint{0.580766in}{4.462900in}}{\pgfqpoint{5.611646in}{3.193122in}} %
\pgfusepath{clip}%
\pgfsetbuttcap%
\pgfsetroundjoin%
\pgfsetlinewidth{0.501875pt}%
\definecolor{currentstroke}{rgb}{0.000000,0.000000,0.000000}%
\pgfsetstrokecolor{currentstroke}%
\pgfsetdash{{1.000000pt}{3.000000pt}}{0.000000pt}%
\pgfpathmoveto{\pgfqpoint{0.580766in}{5.101525in}}%
\pgfpathlineto{\pgfqpoint{6.192411in}{5.101525in}}%
\pgfusepath{stroke}%
\end{pgfscope}%
\begin{pgfscope}%
\pgfsetbuttcap%
\pgfsetroundjoin%
\definecolor{currentfill}{rgb}{0.000000,0.000000,0.000000}%
\pgfsetfillcolor{currentfill}%
\pgfsetlinewidth{0.501875pt}%
\definecolor{currentstroke}{rgb}{0.000000,0.000000,0.000000}%
\pgfsetstrokecolor{currentstroke}%
\pgfsetdash{}{0pt}%
\pgfsys@defobject{currentmarker}{\pgfqpoint{0.000000in}{0.000000in}}{\pgfqpoint{0.055556in}{0.000000in}}{%
\pgfpathmoveto{\pgfqpoint{0.000000in}{0.000000in}}%
\pgfpathlineto{\pgfqpoint{0.055556in}{0.000000in}}%
\pgfusepath{stroke,fill}%
}%
\begin{pgfscope}%
\pgfsys@transformshift{0.580766in}{5.101525in}%
\pgfsys@useobject{currentmarker}{}%
\end{pgfscope}%
\end{pgfscope}%
\begin{pgfscope}%
\pgfsetbuttcap%
\pgfsetroundjoin%
\definecolor{currentfill}{rgb}{0.000000,0.000000,0.000000}%
\pgfsetfillcolor{currentfill}%
\pgfsetlinewidth{0.501875pt}%
\definecolor{currentstroke}{rgb}{0.000000,0.000000,0.000000}%
\pgfsetstrokecolor{currentstroke}%
\pgfsetdash{}{0pt}%
\pgfsys@defobject{currentmarker}{\pgfqpoint{-0.055556in}{0.000000in}}{\pgfqpoint{0.000000in}{0.000000in}}{%
\pgfpathmoveto{\pgfqpoint{0.000000in}{0.000000in}}%
\pgfpathlineto{\pgfqpoint{-0.055556in}{0.000000in}}%
\pgfusepath{stroke,fill}%
}%
\begin{pgfscope}%
\pgfsys@transformshift{6.192411in}{5.101525in}%
\pgfsys@useobject{currentmarker}{}%
\end{pgfscope}%
\end{pgfscope}%
\begin{pgfscope}%
\pgftext[x=0.525210in,y=5.101525in,right,]{{\rmfamily\fontsize{10.000000}{12.000000}\selectfont \(\displaystyle 0.2\)}}%
\end{pgfscope}%
\begin{pgfscope}%
\pgfpathrectangle{\pgfqpoint{0.580766in}{4.462900in}}{\pgfqpoint{5.611646in}{3.193122in}} %
\pgfusepath{clip}%
\pgfsetbuttcap%
\pgfsetroundjoin%
\pgfsetlinewidth{0.501875pt}%
\definecolor{currentstroke}{rgb}{0.000000,0.000000,0.000000}%
\pgfsetstrokecolor{currentstroke}%
\pgfsetdash{{1.000000pt}{3.000000pt}}{0.000000pt}%
\pgfpathmoveto{\pgfqpoint{0.580766in}{5.740149in}}%
\pgfpathlineto{\pgfqpoint{6.192411in}{5.740149in}}%
\pgfusepath{stroke}%
\end{pgfscope}%
\begin{pgfscope}%
\pgfsetbuttcap%
\pgfsetroundjoin%
\definecolor{currentfill}{rgb}{0.000000,0.000000,0.000000}%
\pgfsetfillcolor{currentfill}%
\pgfsetlinewidth{0.501875pt}%
\definecolor{currentstroke}{rgb}{0.000000,0.000000,0.000000}%
\pgfsetstrokecolor{currentstroke}%
\pgfsetdash{}{0pt}%
\pgfsys@defobject{currentmarker}{\pgfqpoint{0.000000in}{0.000000in}}{\pgfqpoint{0.055556in}{0.000000in}}{%
\pgfpathmoveto{\pgfqpoint{0.000000in}{0.000000in}}%
\pgfpathlineto{\pgfqpoint{0.055556in}{0.000000in}}%
\pgfusepath{stroke,fill}%
}%
\begin{pgfscope}%
\pgfsys@transformshift{0.580766in}{5.740149in}%
\pgfsys@useobject{currentmarker}{}%
\end{pgfscope}%
\end{pgfscope}%
\begin{pgfscope}%
\pgfsetbuttcap%
\pgfsetroundjoin%
\definecolor{currentfill}{rgb}{0.000000,0.000000,0.000000}%
\pgfsetfillcolor{currentfill}%
\pgfsetlinewidth{0.501875pt}%
\definecolor{currentstroke}{rgb}{0.000000,0.000000,0.000000}%
\pgfsetstrokecolor{currentstroke}%
\pgfsetdash{}{0pt}%
\pgfsys@defobject{currentmarker}{\pgfqpoint{-0.055556in}{0.000000in}}{\pgfqpoint{0.000000in}{0.000000in}}{%
\pgfpathmoveto{\pgfqpoint{0.000000in}{0.000000in}}%
\pgfpathlineto{\pgfqpoint{-0.055556in}{0.000000in}}%
\pgfusepath{stroke,fill}%
}%
\begin{pgfscope}%
\pgfsys@transformshift{6.192411in}{5.740149in}%
\pgfsys@useobject{currentmarker}{}%
\end{pgfscope}%
\end{pgfscope}%
\begin{pgfscope}%
\pgftext[x=0.525210in,y=5.740149in,right,]{{\rmfamily\fontsize{10.000000}{12.000000}\selectfont \(\displaystyle 0.4\)}}%
\end{pgfscope}%
\begin{pgfscope}%
\pgfpathrectangle{\pgfqpoint{0.580766in}{4.462900in}}{\pgfqpoint{5.611646in}{3.193122in}} %
\pgfusepath{clip}%
\pgfsetbuttcap%
\pgfsetroundjoin%
\pgfsetlinewidth{0.501875pt}%
\definecolor{currentstroke}{rgb}{0.000000,0.000000,0.000000}%
\pgfsetstrokecolor{currentstroke}%
\pgfsetdash{{1.000000pt}{3.000000pt}}{0.000000pt}%
\pgfpathmoveto{\pgfqpoint{0.580766in}{6.378774in}}%
\pgfpathlineto{\pgfqpoint{6.192411in}{6.378774in}}%
\pgfusepath{stroke}%
\end{pgfscope}%
\begin{pgfscope}%
\pgfsetbuttcap%
\pgfsetroundjoin%
\definecolor{currentfill}{rgb}{0.000000,0.000000,0.000000}%
\pgfsetfillcolor{currentfill}%
\pgfsetlinewidth{0.501875pt}%
\definecolor{currentstroke}{rgb}{0.000000,0.000000,0.000000}%
\pgfsetstrokecolor{currentstroke}%
\pgfsetdash{}{0pt}%
\pgfsys@defobject{currentmarker}{\pgfqpoint{0.000000in}{0.000000in}}{\pgfqpoint{0.055556in}{0.000000in}}{%
\pgfpathmoveto{\pgfqpoint{0.000000in}{0.000000in}}%
\pgfpathlineto{\pgfqpoint{0.055556in}{0.000000in}}%
\pgfusepath{stroke,fill}%
}%
\begin{pgfscope}%
\pgfsys@transformshift{0.580766in}{6.378774in}%
\pgfsys@useobject{currentmarker}{}%
\end{pgfscope}%
\end{pgfscope}%
\begin{pgfscope}%
\pgfsetbuttcap%
\pgfsetroundjoin%
\definecolor{currentfill}{rgb}{0.000000,0.000000,0.000000}%
\pgfsetfillcolor{currentfill}%
\pgfsetlinewidth{0.501875pt}%
\definecolor{currentstroke}{rgb}{0.000000,0.000000,0.000000}%
\pgfsetstrokecolor{currentstroke}%
\pgfsetdash{}{0pt}%
\pgfsys@defobject{currentmarker}{\pgfqpoint{-0.055556in}{0.000000in}}{\pgfqpoint{0.000000in}{0.000000in}}{%
\pgfpathmoveto{\pgfqpoint{0.000000in}{0.000000in}}%
\pgfpathlineto{\pgfqpoint{-0.055556in}{0.000000in}}%
\pgfusepath{stroke,fill}%
}%
\begin{pgfscope}%
\pgfsys@transformshift{6.192411in}{6.378774in}%
\pgfsys@useobject{currentmarker}{}%
\end{pgfscope}%
\end{pgfscope}%
\begin{pgfscope}%
\pgftext[x=0.525210in,y=6.378774in,right,]{{\rmfamily\fontsize{10.000000}{12.000000}\selectfont \(\displaystyle 0.6\)}}%
\end{pgfscope}%
\begin{pgfscope}%
\pgfpathrectangle{\pgfqpoint{0.580766in}{4.462900in}}{\pgfqpoint{5.611646in}{3.193122in}} %
\pgfusepath{clip}%
\pgfsetbuttcap%
\pgfsetroundjoin%
\pgfsetlinewidth{0.501875pt}%
\definecolor{currentstroke}{rgb}{0.000000,0.000000,0.000000}%
\pgfsetstrokecolor{currentstroke}%
\pgfsetdash{{1.000000pt}{3.000000pt}}{0.000000pt}%
\pgfpathmoveto{\pgfqpoint{0.580766in}{7.017398in}}%
\pgfpathlineto{\pgfqpoint{6.192411in}{7.017398in}}%
\pgfusepath{stroke}%
\end{pgfscope}%
\begin{pgfscope}%
\pgfsetbuttcap%
\pgfsetroundjoin%
\definecolor{currentfill}{rgb}{0.000000,0.000000,0.000000}%
\pgfsetfillcolor{currentfill}%
\pgfsetlinewidth{0.501875pt}%
\definecolor{currentstroke}{rgb}{0.000000,0.000000,0.000000}%
\pgfsetstrokecolor{currentstroke}%
\pgfsetdash{}{0pt}%
\pgfsys@defobject{currentmarker}{\pgfqpoint{0.000000in}{0.000000in}}{\pgfqpoint{0.055556in}{0.000000in}}{%
\pgfpathmoveto{\pgfqpoint{0.000000in}{0.000000in}}%
\pgfpathlineto{\pgfqpoint{0.055556in}{0.000000in}}%
\pgfusepath{stroke,fill}%
}%
\begin{pgfscope}%
\pgfsys@transformshift{0.580766in}{7.017398in}%
\pgfsys@useobject{currentmarker}{}%
\end{pgfscope}%
\end{pgfscope}%
\begin{pgfscope}%
\pgfsetbuttcap%
\pgfsetroundjoin%
\definecolor{currentfill}{rgb}{0.000000,0.000000,0.000000}%
\pgfsetfillcolor{currentfill}%
\pgfsetlinewidth{0.501875pt}%
\definecolor{currentstroke}{rgb}{0.000000,0.000000,0.000000}%
\pgfsetstrokecolor{currentstroke}%
\pgfsetdash{}{0pt}%
\pgfsys@defobject{currentmarker}{\pgfqpoint{-0.055556in}{0.000000in}}{\pgfqpoint{0.000000in}{0.000000in}}{%
\pgfpathmoveto{\pgfqpoint{0.000000in}{0.000000in}}%
\pgfpathlineto{\pgfqpoint{-0.055556in}{0.000000in}}%
\pgfusepath{stroke,fill}%
}%
\begin{pgfscope}%
\pgfsys@transformshift{6.192411in}{7.017398in}%
\pgfsys@useobject{currentmarker}{}%
\end{pgfscope}%
\end{pgfscope}%
\begin{pgfscope}%
\pgftext[x=0.525210in,y=7.017398in,right,]{{\rmfamily\fontsize{10.000000}{12.000000}\selectfont \(\displaystyle 0.8\)}}%
\end{pgfscope}%
\begin{pgfscope}%
\pgfpathrectangle{\pgfqpoint{0.580766in}{4.462900in}}{\pgfqpoint{5.611646in}{3.193122in}} %
\pgfusepath{clip}%
\pgfsetbuttcap%
\pgfsetroundjoin%
\pgfsetlinewidth{0.501875pt}%
\definecolor{currentstroke}{rgb}{0.000000,0.000000,0.000000}%
\pgfsetstrokecolor{currentstroke}%
\pgfsetdash{{1.000000pt}{3.000000pt}}{0.000000pt}%
\pgfpathmoveto{\pgfqpoint{0.580766in}{7.656023in}}%
\pgfpathlineto{\pgfqpoint{6.192411in}{7.656023in}}%
\pgfusepath{stroke}%
\end{pgfscope}%
\begin{pgfscope}%
\pgfsetbuttcap%
\pgfsetroundjoin%
\definecolor{currentfill}{rgb}{0.000000,0.000000,0.000000}%
\pgfsetfillcolor{currentfill}%
\pgfsetlinewidth{0.501875pt}%
\definecolor{currentstroke}{rgb}{0.000000,0.000000,0.000000}%
\pgfsetstrokecolor{currentstroke}%
\pgfsetdash{}{0pt}%
\pgfsys@defobject{currentmarker}{\pgfqpoint{0.000000in}{0.000000in}}{\pgfqpoint{0.055556in}{0.000000in}}{%
\pgfpathmoveto{\pgfqpoint{0.000000in}{0.000000in}}%
\pgfpathlineto{\pgfqpoint{0.055556in}{0.000000in}}%
\pgfusepath{stroke,fill}%
}%
\begin{pgfscope}%
\pgfsys@transformshift{0.580766in}{7.656023in}%
\pgfsys@useobject{currentmarker}{}%
\end{pgfscope}%
\end{pgfscope}%
\begin{pgfscope}%
\pgfsetbuttcap%
\pgfsetroundjoin%
\definecolor{currentfill}{rgb}{0.000000,0.000000,0.000000}%
\pgfsetfillcolor{currentfill}%
\pgfsetlinewidth{0.501875pt}%
\definecolor{currentstroke}{rgb}{0.000000,0.000000,0.000000}%
\pgfsetstrokecolor{currentstroke}%
\pgfsetdash{}{0pt}%
\pgfsys@defobject{currentmarker}{\pgfqpoint{-0.055556in}{0.000000in}}{\pgfqpoint{0.000000in}{0.000000in}}{%
\pgfpathmoveto{\pgfqpoint{0.000000in}{0.000000in}}%
\pgfpathlineto{\pgfqpoint{-0.055556in}{0.000000in}}%
\pgfusepath{stroke,fill}%
}%
\begin{pgfscope}%
\pgfsys@transformshift{6.192411in}{7.656023in}%
\pgfsys@useobject{currentmarker}{}%
\end{pgfscope}%
\end{pgfscope}%
\begin{pgfscope}%
\pgftext[x=0.525210in,y=7.656023in,right,]{{\rmfamily\fontsize{10.000000}{12.000000}\selectfont \(\displaystyle 1.0\)}}%
\end{pgfscope}%
\begin{pgfscope}%
\pgftext[x=0.278296in,y=6.059461in,,bottom,rotate=90.000000]{{\rmfamily\fontsize{10.000000}{12.000000}\selectfont current (A)}}%
\end{pgfscope}%
\begin{pgfscope}%
\pgfsetbuttcap%
\pgfsetroundjoin%
\pgfsetlinewidth{1.003750pt}%
\definecolor{currentstroke}{rgb}{0.000000,0.000000,0.000000}%
\pgfsetstrokecolor{currentstroke}%
\pgfsetdash{}{0pt}%
\pgfpathmoveto{\pgfqpoint{0.580766in}{7.656023in}}%
\pgfpathlineto{\pgfqpoint{6.192411in}{7.656023in}}%
\pgfusepath{stroke}%
\end{pgfscope}%
\begin{pgfscope}%
\pgfsetbuttcap%
\pgfsetroundjoin%
\pgfsetlinewidth{1.003750pt}%
\definecolor{currentstroke}{rgb}{0.000000,0.000000,0.000000}%
\pgfsetstrokecolor{currentstroke}%
\pgfsetdash{}{0pt}%
\pgfpathmoveto{\pgfqpoint{6.192411in}{4.462900in}}%
\pgfpathlineto{\pgfqpoint{6.192411in}{7.656023in}}%
\pgfusepath{stroke}%
\end{pgfscope}%
\begin{pgfscope}%
\pgfsetbuttcap%
\pgfsetroundjoin%
\pgfsetlinewidth{1.003750pt}%
\definecolor{currentstroke}{rgb}{0.000000,0.000000,0.000000}%
\pgfsetstrokecolor{currentstroke}%
\pgfsetdash{}{0pt}%
\pgfpathmoveto{\pgfqpoint{0.580766in}{4.462900in}}%
\pgfpathlineto{\pgfqpoint{6.192411in}{4.462900in}}%
\pgfusepath{stroke}%
\end{pgfscope}%
\begin{pgfscope}%
\pgfsetbuttcap%
\pgfsetroundjoin%
\pgfsetlinewidth{1.003750pt}%
\definecolor{currentstroke}{rgb}{0.000000,0.000000,0.000000}%
\pgfsetstrokecolor{currentstroke}%
\pgfsetdash{}{0pt}%
\pgfpathmoveto{\pgfqpoint{0.580766in}{4.462900in}}%
\pgfpathlineto{\pgfqpoint{0.580766in}{7.656023in}}%
\pgfusepath{stroke}%
\end{pgfscope}%
\begin{pgfscope}%
\pgftext[x=3.386589in,y=7.725467in,,base]{{\rmfamily\fontsize{12.000000}{14.400000}\selectfont Step-By-Step Approximations of \(\displaystyle i(t)\)}}%
\end{pgfscope}%
\begin{pgfscope}%
\pgfsetbuttcap%
\pgfsetroundjoin%
\definecolor{currentfill}{rgb}{0.300000,0.300000,0.300000}%
\pgfsetfillcolor{currentfill}%
\pgfsetfillopacity{0.500000}%
\pgfsetlinewidth{1.003750pt}%
\definecolor{currentstroke}{rgb}{0.300000,0.300000,0.300000}%
\pgfsetstrokecolor{currentstroke}%
\pgfsetstrokeopacity{0.500000}%
\pgfsetdash{}{0pt}%
\pgfpathmoveto{\pgfqpoint{4.378033in}{4.518456in}}%
\pgfpathlineto{\pgfqpoint{6.103523in}{4.518456in}}%
\pgfpathquadraticcurveto{\pgfqpoint{6.136856in}{4.518456in}}{\pgfqpoint{6.136856in}{4.551789in}}%
\pgfpathlineto{\pgfqpoint{6.136856in}{5.232344in}}%
\pgfpathquadraticcurveto{\pgfqpoint{6.136856in}{5.265677in}}{\pgfqpoint{6.103523in}{5.265677in}}%
\pgfpathlineto{\pgfqpoint{4.378033in}{5.265677in}}%
\pgfpathquadraticcurveto{\pgfqpoint{4.344700in}{5.265677in}}{\pgfqpoint{4.344700in}{5.232344in}}%
\pgfpathlineto{\pgfqpoint{4.344700in}{4.551789in}}%
\pgfpathquadraticcurveto{\pgfqpoint{4.344700in}{4.518456in}}{\pgfqpoint{4.378033in}{4.518456in}}%
\pgfpathclose%
\pgfusepath{stroke,fill}%
\end{pgfscope}%
\begin{pgfscope}%
\pgfsetbuttcap%
\pgfsetroundjoin%
\definecolor{currentfill}{rgb}{1.000000,1.000000,1.000000}%
\pgfsetfillcolor{currentfill}%
\pgfsetlinewidth{1.003750pt}%
\definecolor{currentstroke}{rgb}{0.000000,0.000000,0.000000}%
\pgfsetstrokecolor{currentstroke}%
\pgfsetdash{}{0pt}%
\pgfpathmoveto{\pgfqpoint{4.350255in}{4.546233in}}%
\pgfpathlineto{\pgfqpoint{6.075745in}{4.546233in}}%
\pgfpathquadraticcurveto{\pgfqpoint{6.109078in}{4.546233in}}{\pgfqpoint{6.109078in}{4.579567in}}%
\pgfpathlineto{\pgfqpoint{6.109078in}{5.260121in}}%
\pgfpathquadraticcurveto{\pgfqpoint{6.109078in}{5.293455in}}{\pgfqpoint{6.075745in}{5.293455in}}%
\pgfpathlineto{\pgfqpoint{4.350255in}{5.293455in}}%
\pgfpathquadraticcurveto{\pgfqpoint{4.316922in}{5.293455in}}{\pgfqpoint{4.316922in}{5.260121in}}%
\pgfpathlineto{\pgfqpoint{4.316922in}{4.579567in}}%
\pgfpathquadraticcurveto{\pgfqpoint{4.316922in}{4.546233in}}{\pgfqpoint{4.350255in}{4.546233in}}%
\pgfpathclose%
\pgfusepath{stroke,fill}%
\end{pgfscope}%
\begin{pgfscope}%
\pgfsetbuttcap%
\pgfsetroundjoin%
\pgfsetlinewidth{1.003750pt}%
\definecolor{currentstroke}{rgb}{0.000000,0.000000,1.000000}%
\pgfsetstrokecolor{currentstroke}%
\pgfsetdash{{1.000000pt}{3.000000pt}}{0.000000pt}%
\pgfpathmoveto{\pgfqpoint{4.433589in}{5.168455in}}%
\pgfpathlineto{\pgfqpoint{4.666922in}{5.168455in}}%
\pgfusepath{stroke}%
\end{pgfscope}%
\begin{pgfscope}%
\pgftext[x=4.850255in,y=5.110121in,left,base]{{\rmfamily\fontsize{12.000000}{14.400000}\selectfont Continuous}}%
\end{pgfscope}%
\begin{pgfscope}%
\pgfsetrectcap%
\pgfsetroundjoin%
\pgfsetlinewidth{1.003750pt}%
\definecolor{currentstroke}{rgb}{0.000000,0.500000,0.000000}%
\pgfsetstrokecolor{currentstroke}%
\pgfsetdash{}{0pt}%
\pgfpathmoveto{\pgfqpoint{4.433589in}{4.936048in}}%
\pgfpathlineto{\pgfqpoint{4.666922in}{4.936048in}}%
\pgfusepath{stroke}%
\end{pgfscope}%
\begin{pgfscope}%
\pgftext[x=4.850255in,y=4.877714in,left,base]{{\rmfamily\fontsize{12.000000}{14.400000}\selectfont Trapezoidal}}%
\end{pgfscope}%
\begin{pgfscope}%
\pgfsetrectcap%
\pgfsetroundjoin%
\pgfsetlinewidth{1.003750pt}%
\definecolor{currentstroke}{rgb}{1.000000,0.000000,0.000000}%
\pgfsetstrokecolor{currentstroke}%
\pgfsetdash{}{0pt}%
\pgfpathmoveto{\pgfqpoint{4.433589in}{4.703641in}}%
\pgfpathlineto{\pgfqpoint{4.666922in}{4.703641in}}%
\pgfusepath{stroke}%
\end{pgfscope}%
\begin{pgfscope}%
\pgftext[x=4.850255in,y=4.645307in,left,base]{{\rmfamily\fontsize{12.000000}{14.400000}\selectfont Backward Euler    }}%
\end{pgfscope}%
\begin{pgfscope}%
\pgfsetbuttcap%
\pgfsetroundjoin%
\definecolor{currentfill}{rgb}{1.000000,1.000000,1.000000}%
\pgfsetfillcolor{currentfill}%
\pgfsetlinewidth{0.000000pt}%
\definecolor{currentstroke}{rgb}{0.000000,0.000000,0.000000}%
\pgfsetstrokecolor{currentstroke}%
\pgfsetstrokeopacity{0.000000}%
\pgfsetdash{}{0pt}%
\pgfpathmoveto{\pgfqpoint{0.580766in}{0.532919in}}%
\pgfpathlineto{\pgfqpoint{6.192411in}{0.532919in}}%
\pgfpathlineto{\pgfqpoint{6.192411in}{3.726041in}}%
\pgfpathlineto{\pgfqpoint{0.580766in}{3.726041in}}%
\pgfpathclose%
\pgfusepath{fill}%
\end{pgfscope}%
\begin{pgfscope}%
\pgfpathrectangle{\pgfqpoint{0.580766in}{0.532919in}}{\pgfqpoint{5.611646in}{3.193122in}} %
\pgfusepath{clip}%
\pgfsetbuttcap%
\pgfsetroundjoin%
\pgfsetlinewidth{1.003750pt}%
\definecolor{currentstroke}{rgb}{0.000000,0.000000,1.000000}%
\pgfsetstrokecolor{currentstroke}%
\pgfsetdash{{1.000000pt}{3.000000pt}}{0.000000pt}%
\pgfpathmoveto{\pgfqpoint{0.580766in}{3.726041in}}%
\pgfpathlineto{\pgfqpoint{0.633141in}{3.580452in}}%
\pgfpathlineto{\pgfqpoint{0.685517in}{3.441501in}}%
\pgfpathlineto{\pgfqpoint{0.737892in}{3.308886in}}%
\pgfpathlineto{\pgfqpoint{0.790267in}{3.182317in}}%
\pgfpathlineto{\pgfqpoint{0.842643in}{3.061519in}}%
\pgfpathlineto{\pgfqpoint{0.895018in}{2.946229in}}%
\pgfpathlineto{\pgfqpoint{0.947393in}{2.836195in}}%
\pgfpathlineto{\pgfqpoint{0.999769in}{2.731178in}}%
\pgfpathlineto{\pgfqpoint{1.054015in}{2.627456in}}%
\pgfpathlineto{\pgfqpoint{1.108261in}{2.528627in}}%
\pgfpathlineto{\pgfqpoint{1.162506in}{2.434462in}}%
\pgfpathlineto{\pgfqpoint{1.216752in}{2.344740in}}%
\pgfpathlineto{\pgfqpoint{1.270998in}{2.259251in}}%
\pgfpathlineto{\pgfqpoint{1.325244in}{2.177796in}}%
\pgfpathlineto{\pgfqpoint{1.381361in}{2.097574in}}%
\pgfpathlineto{\pgfqpoint{1.437477in}{2.021265in}}%
\pgfpathlineto{\pgfqpoint{1.493594in}{1.948678in}}%
\pgfpathlineto{\pgfqpoint{1.549710in}{1.879630in}}%
\pgfpathlineto{\pgfqpoint{1.607697in}{1.811817in}}%
\pgfpathlineto{\pgfqpoint{1.665684in}{1.747419in}}%
\pgfpathlineto{\pgfqpoint{1.723671in}{1.686263in}}%
\pgfpathlineto{\pgfqpoint{1.783529in}{1.626363in}}%
\pgfpathlineto{\pgfqpoint{1.843386in}{1.569574in}}%
\pgfpathlineto{\pgfqpoint{1.905114in}{1.514097in}}%
\pgfpathlineto{\pgfqpoint{1.966842in}{1.461590in}}%
\pgfpathlineto{\pgfqpoint{2.030441in}{1.410428in}}%
\pgfpathlineto{\pgfqpoint{2.094040in}{1.362085in}}%
\pgfpathlineto{\pgfqpoint{2.159509in}{1.315101in}}%
\pgfpathlineto{\pgfqpoint{2.226849in}{1.269550in}}%
\pgfpathlineto{\pgfqpoint{2.294188in}{1.226652in}}%
\pgfpathlineto{\pgfqpoint{2.363399in}{1.185164in}}%
\pgfpathlineto{\pgfqpoint{2.434479in}{1.145136in}}%
\pgfpathlineto{\pgfqpoint{2.507431in}{1.106608in}}%
\pgfpathlineto{\pgfqpoint{2.582253in}{1.069609in}}%
\pgfpathlineto{\pgfqpoint{2.658945in}{1.034160in}}%
\pgfpathlineto{\pgfqpoint{2.737508in}{1.000273in}}%
\pgfpathlineto{\pgfqpoint{2.819812in}{0.967227in}}%
\pgfpathlineto{\pgfqpoint{2.903987in}{0.935845in}}%
\pgfpathlineto{\pgfqpoint{2.991903in}{0.905487in}}%
\pgfpathlineto{\pgfqpoint{3.081689in}{0.876843in}}%
\pgfpathlineto{\pgfqpoint{3.175217in}{0.849344in}}%
\pgfpathlineto{\pgfqpoint{3.272485in}{0.823075in}}%
\pgfpathlineto{\pgfqpoint{3.373495in}{0.798102in}}%
\pgfpathlineto{\pgfqpoint{3.480116in}{0.774069in}}%
\pgfpathlineto{\pgfqpoint{3.590478in}{0.751484in}}%
\pgfpathlineto{\pgfqpoint{3.706452in}{0.730027in}}%
\pgfpathlineto{\pgfqpoint{3.828038in}{0.709790in}}%
\pgfpathlineto{\pgfqpoint{3.957106in}{0.690575in}}%
\pgfpathlineto{\pgfqpoint{4.093656in}{0.672515in}}%
\pgfpathlineto{\pgfqpoint{4.239559in}{0.655497in}}%
\pgfpathlineto{\pgfqpoint{4.394814in}{0.639661in}}%
\pgfpathlineto{\pgfqpoint{4.561293in}{0.624946in}}%
\pgfpathlineto{\pgfqpoint{4.740866in}{0.611339in}}%
\pgfpathlineto{\pgfqpoint{4.935403in}{0.598859in}}%
\pgfpathlineto{\pgfqpoint{5.148645in}{0.587449in}}%
\pgfpathlineto{\pgfqpoint{5.382464in}{0.577194in}}%
\pgfpathlineto{\pgfqpoint{5.644341in}{0.567979in}}%
\pgfpathlineto{\pgfqpoint{5.939887in}{0.559862in}}%
\pgfpathlineto{\pgfqpoint{6.190541in}{0.554469in}}%
\pgfpathlineto{\pgfqpoint{6.190541in}{0.554469in}}%
\pgfusepath{stroke}%
\end{pgfscope}%
\begin{pgfscope}%
\pgfpathrectangle{\pgfqpoint{0.580766in}{0.532919in}}{\pgfqpoint{5.611646in}{3.193122in}} %
\pgfusepath{clip}%
\pgfsetrectcap%
\pgfsetroundjoin%
\pgfsetlinewidth{1.003750pt}%
\definecolor{currentstroke}{rgb}{0.000000,0.500000,0.000000}%
\pgfsetstrokecolor{currentstroke}%
\pgfsetdash{}{0pt}%
\pgfpathmoveto{\pgfqpoint{0.580766in}{2.661667in}}%
\pgfpathlineto{\pgfqpoint{1.029698in}{1.952084in}}%
\pgfpathlineto{\pgfqpoint{1.478629in}{1.479029in}}%
\pgfpathlineto{\pgfqpoint{1.927561in}{1.163659in}}%
\pgfpathlineto{\pgfqpoint{2.376492in}{0.953412in}}%
\pgfpathlineto{\pgfqpoint{2.825424in}{0.813247in}}%
\pgfpathlineto{\pgfqpoint{3.274356in}{0.719805in}}%
\pgfpathlineto{\pgfqpoint{3.723287in}{0.657509in}}%
\pgfpathlineto{\pgfqpoint{4.172219in}{0.615979in}}%
\pgfpathlineto{\pgfqpoint{4.621151in}{0.588292in}}%
\pgfpathlineto{\pgfqpoint{5.070082in}{0.569834in}}%
\pgfpathlineto{\pgfqpoint{5.519014in}{0.557529in}}%
\pgfpathlineto{\pgfqpoint{5.967946in}{0.549326in}}%
\pgfusepath{stroke}%
\end{pgfscope}%
\begin{pgfscope}%
\pgfpathrectangle{\pgfqpoint{0.580766in}{0.532919in}}{\pgfqpoint{5.611646in}{3.193122in}} %
\pgfusepath{clip}%
\pgfsetrectcap%
\pgfsetroundjoin%
\pgfsetlinewidth{1.003750pt}%
\definecolor{currentstroke}{rgb}{1.000000,0.000000,0.000000}%
\pgfsetstrokecolor{currentstroke}%
\pgfsetdash{}{0pt}%
\pgfpathmoveto{\pgfqpoint{0.580766in}{2.813720in}}%
\pgfpathlineto{\pgfqpoint{1.029698in}{2.162063in}}%
\pgfpathlineto{\pgfqpoint{1.478629in}{1.696593in}}%
\pgfpathlineto{\pgfqpoint{1.927561in}{1.364114in}}%
\pgfpathlineto{\pgfqpoint{2.376492in}{1.126630in}}%
\pgfpathlineto{\pgfqpoint{2.825424in}{0.956998in}}%
\pgfpathlineto{\pgfqpoint{3.274356in}{0.835832in}}%
\pgfpathlineto{\pgfqpoint{3.723287in}{0.749286in}}%
\pgfpathlineto{\pgfqpoint{4.172219in}{0.687466in}}%
\pgfpathlineto{\pgfqpoint{4.621151in}{0.643310in}}%
\pgfpathlineto{\pgfqpoint{5.070082in}{0.611770in}}%
\pgfpathlineto{\pgfqpoint{5.519014in}{0.589241in}}%
\pgfpathlineto{\pgfqpoint{5.967946in}{0.573149in}}%
\pgfusepath{stroke}%
\end{pgfscope}%
\begin{pgfscope}%
\pgfpathrectangle{\pgfqpoint{0.580766in}{0.532919in}}{\pgfqpoint{5.611646in}{3.193122in}} %
\pgfusepath{clip}%
\pgfsetbuttcap%
\pgfsetroundjoin%
\pgfsetlinewidth{0.501875pt}%
\definecolor{currentstroke}{rgb}{0.000000,0.000000,0.000000}%
\pgfsetstrokecolor{currentstroke}%
\pgfsetdash{{1.000000pt}{3.000000pt}}{0.000000pt}%
\pgfpathmoveto{\pgfqpoint{0.580766in}{0.532919in}}%
\pgfpathlineto{\pgfqpoint{0.580766in}{3.726041in}}%
\pgfusepath{stroke}%
\end{pgfscope}%
\begin{pgfscope}%
\pgfsetbuttcap%
\pgfsetroundjoin%
\definecolor{currentfill}{rgb}{0.000000,0.000000,0.000000}%
\pgfsetfillcolor{currentfill}%
\pgfsetlinewidth{0.501875pt}%
\definecolor{currentstroke}{rgb}{0.000000,0.000000,0.000000}%
\pgfsetstrokecolor{currentstroke}%
\pgfsetdash{}{0pt}%
\pgfsys@defobject{currentmarker}{\pgfqpoint{0.000000in}{0.000000in}}{\pgfqpoint{0.000000in}{0.055556in}}{%
\pgfpathmoveto{\pgfqpoint{0.000000in}{0.000000in}}%
\pgfpathlineto{\pgfqpoint{0.000000in}{0.055556in}}%
\pgfusepath{stroke,fill}%
}%
\begin{pgfscope}%
\pgfsys@transformshift{0.580766in}{0.532919in}%
\pgfsys@useobject{currentmarker}{}%
\end{pgfscope}%
\end{pgfscope}%
\begin{pgfscope}%
\pgfsetbuttcap%
\pgfsetroundjoin%
\definecolor{currentfill}{rgb}{0.000000,0.000000,0.000000}%
\pgfsetfillcolor{currentfill}%
\pgfsetlinewidth{0.501875pt}%
\definecolor{currentstroke}{rgb}{0.000000,0.000000,0.000000}%
\pgfsetstrokecolor{currentstroke}%
\pgfsetdash{}{0pt}%
\pgfsys@defobject{currentmarker}{\pgfqpoint{0.000000in}{-0.055556in}}{\pgfqpoint{0.000000in}{0.000000in}}{%
\pgfpathmoveto{\pgfqpoint{0.000000in}{0.000000in}}%
\pgfpathlineto{\pgfqpoint{0.000000in}{-0.055556in}}%
\pgfusepath{stroke,fill}%
}%
\begin{pgfscope}%
\pgfsys@transformshift{0.580766in}{3.726041in}%
\pgfsys@useobject{currentmarker}{}%
\end{pgfscope}%
\end{pgfscope}%
\begin{pgfscope}%
\pgftext[x=0.580766in,y=0.477363in,,top]{{\rmfamily\fontsize{10.000000}{12.000000}\selectfont \(\displaystyle 0.000\)}}%
\end{pgfscope}%
\begin{pgfscope}%
\pgfpathrectangle{\pgfqpoint{0.580766in}{0.532919in}}{\pgfqpoint{5.611646in}{3.193122in}} %
\pgfusepath{clip}%
\pgfsetbuttcap%
\pgfsetroundjoin%
\pgfsetlinewidth{0.501875pt}%
\definecolor{currentstroke}{rgb}{0.000000,0.000000,0.000000}%
\pgfsetstrokecolor{currentstroke}%
\pgfsetdash{{1.000000pt}{3.000000pt}}{0.000000pt}%
\pgfpathmoveto{\pgfqpoint{1.703095in}{0.532919in}}%
\pgfpathlineto{\pgfqpoint{1.703095in}{3.726041in}}%
\pgfusepath{stroke}%
\end{pgfscope}%
\begin{pgfscope}%
\pgfsetbuttcap%
\pgfsetroundjoin%
\definecolor{currentfill}{rgb}{0.000000,0.000000,0.000000}%
\pgfsetfillcolor{currentfill}%
\pgfsetlinewidth{0.501875pt}%
\definecolor{currentstroke}{rgb}{0.000000,0.000000,0.000000}%
\pgfsetstrokecolor{currentstroke}%
\pgfsetdash{}{0pt}%
\pgfsys@defobject{currentmarker}{\pgfqpoint{0.000000in}{0.000000in}}{\pgfqpoint{0.000000in}{0.055556in}}{%
\pgfpathmoveto{\pgfqpoint{0.000000in}{0.000000in}}%
\pgfpathlineto{\pgfqpoint{0.000000in}{0.055556in}}%
\pgfusepath{stroke,fill}%
}%
\begin{pgfscope}%
\pgfsys@transformshift{1.703095in}{0.532919in}%
\pgfsys@useobject{currentmarker}{}%
\end{pgfscope}%
\end{pgfscope}%
\begin{pgfscope}%
\pgfsetbuttcap%
\pgfsetroundjoin%
\definecolor{currentfill}{rgb}{0.000000,0.000000,0.000000}%
\pgfsetfillcolor{currentfill}%
\pgfsetlinewidth{0.501875pt}%
\definecolor{currentstroke}{rgb}{0.000000,0.000000,0.000000}%
\pgfsetstrokecolor{currentstroke}%
\pgfsetdash{}{0pt}%
\pgfsys@defobject{currentmarker}{\pgfqpoint{0.000000in}{-0.055556in}}{\pgfqpoint{0.000000in}{0.000000in}}{%
\pgfpathmoveto{\pgfqpoint{0.000000in}{0.000000in}}%
\pgfpathlineto{\pgfqpoint{0.000000in}{-0.055556in}}%
\pgfusepath{stroke,fill}%
}%
\begin{pgfscope}%
\pgfsys@transformshift{1.703095in}{3.726041in}%
\pgfsys@useobject{currentmarker}{}%
\end{pgfscope}%
\end{pgfscope}%
\begin{pgfscope}%
\pgftext[x=1.703095in,y=0.477363in,,top]{{\rmfamily\fontsize{10.000000}{12.000000}\selectfont \(\displaystyle 0.002\)}}%
\end{pgfscope}%
\begin{pgfscope}%
\pgfpathrectangle{\pgfqpoint{0.580766in}{0.532919in}}{\pgfqpoint{5.611646in}{3.193122in}} %
\pgfusepath{clip}%
\pgfsetbuttcap%
\pgfsetroundjoin%
\pgfsetlinewidth{0.501875pt}%
\definecolor{currentstroke}{rgb}{0.000000,0.000000,0.000000}%
\pgfsetstrokecolor{currentstroke}%
\pgfsetdash{{1.000000pt}{3.000000pt}}{0.000000pt}%
\pgfpathmoveto{\pgfqpoint{2.825424in}{0.532919in}}%
\pgfpathlineto{\pgfqpoint{2.825424in}{3.726041in}}%
\pgfusepath{stroke}%
\end{pgfscope}%
\begin{pgfscope}%
\pgfsetbuttcap%
\pgfsetroundjoin%
\definecolor{currentfill}{rgb}{0.000000,0.000000,0.000000}%
\pgfsetfillcolor{currentfill}%
\pgfsetlinewidth{0.501875pt}%
\definecolor{currentstroke}{rgb}{0.000000,0.000000,0.000000}%
\pgfsetstrokecolor{currentstroke}%
\pgfsetdash{}{0pt}%
\pgfsys@defobject{currentmarker}{\pgfqpoint{0.000000in}{0.000000in}}{\pgfqpoint{0.000000in}{0.055556in}}{%
\pgfpathmoveto{\pgfqpoint{0.000000in}{0.000000in}}%
\pgfpathlineto{\pgfqpoint{0.000000in}{0.055556in}}%
\pgfusepath{stroke,fill}%
}%
\begin{pgfscope}%
\pgfsys@transformshift{2.825424in}{0.532919in}%
\pgfsys@useobject{currentmarker}{}%
\end{pgfscope}%
\end{pgfscope}%
\begin{pgfscope}%
\pgfsetbuttcap%
\pgfsetroundjoin%
\definecolor{currentfill}{rgb}{0.000000,0.000000,0.000000}%
\pgfsetfillcolor{currentfill}%
\pgfsetlinewidth{0.501875pt}%
\definecolor{currentstroke}{rgb}{0.000000,0.000000,0.000000}%
\pgfsetstrokecolor{currentstroke}%
\pgfsetdash{}{0pt}%
\pgfsys@defobject{currentmarker}{\pgfqpoint{0.000000in}{-0.055556in}}{\pgfqpoint{0.000000in}{0.000000in}}{%
\pgfpathmoveto{\pgfqpoint{0.000000in}{0.000000in}}%
\pgfpathlineto{\pgfqpoint{0.000000in}{-0.055556in}}%
\pgfusepath{stroke,fill}%
}%
\begin{pgfscope}%
\pgfsys@transformshift{2.825424in}{3.726041in}%
\pgfsys@useobject{currentmarker}{}%
\end{pgfscope}%
\end{pgfscope}%
\begin{pgfscope}%
\pgftext[x=2.825424in,y=0.477363in,,top]{{\rmfamily\fontsize{10.000000}{12.000000}\selectfont \(\displaystyle 0.004\)}}%
\end{pgfscope}%
\begin{pgfscope}%
\pgfpathrectangle{\pgfqpoint{0.580766in}{0.532919in}}{\pgfqpoint{5.611646in}{3.193122in}} %
\pgfusepath{clip}%
\pgfsetbuttcap%
\pgfsetroundjoin%
\pgfsetlinewidth{0.501875pt}%
\definecolor{currentstroke}{rgb}{0.000000,0.000000,0.000000}%
\pgfsetstrokecolor{currentstroke}%
\pgfsetdash{{1.000000pt}{3.000000pt}}{0.000000pt}%
\pgfpathmoveto{\pgfqpoint{3.947753in}{0.532919in}}%
\pgfpathlineto{\pgfqpoint{3.947753in}{3.726041in}}%
\pgfusepath{stroke}%
\end{pgfscope}%
\begin{pgfscope}%
\pgfsetbuttcap%
\pgfsetroundjoin%
\definecolor{currentfill}{rgb}{0.000000,0.000000,0.000000}%
\pgfsetfillcolor{currentfill}%
\pgfsetlinewidth{0.501875pt}%
\definecolor{currentstroke}{rgb}{0.000000,0.000000,0.000000}%
\pgfsetstrokecolor{currentstroke}%
\pgfsetdash{}{0pt}%
\pgfsys@defobject{currentmarker}{\pgfqpoint{0.000000in}{0.000000in}}{\pgfqpoint{0.000000in}{0.055556in}}{%
\pgfpathmoveto{\pgfqpoint{0.000000in}{0.000000in}}%
\pgfpathlineto{\pgfqpoint{0.000000in}{0.055556in}}%
\pgfusepath{stroke,fill}%
}%
\begin{pgfscope}%
\pgfsys@transformshift{3.947753in}{0.532919in}%
\pgfsys@useobject{currentmarker}{}%
\end{pgfscope}%
\end{pgfscope}%
\begin{pgfscope}%
\pgfsetbuttcap%
\pgfsetroundjoin%
\definecolor{currentfill}{rgb}{0.000000,0.000000,0.000000}%
\pgfsetfillcolor{currentfill}%
\pgfsetlinewidth{0.501875pt}%
\definecolor{currentstroke}{rgb}{0.000000,0.000000,0.000000}%
\pgfsetstrokecolor{currentstroke}%
\pgfsetdash{}{0pt}%
\pgfsys@defobject{currentmarker}{\pgfqpoint{0.000000in}{-0.055556in}}{\pgfqpoint{0.000000in}{0.000000in}}{%
\pgfpathmoveto{\pgfqpoint{0.000000in}{0.000000in}}%
\pgfpathlineto{\pgfqpoint{0.000000in}{-0.055556in}}%
\pgfusepath{stroke,fill}%
}%
\begin{pgfscope}%
\pgfsys@transformshift{3.947753in}{3.726041in}%
\pgfsys@useobject{currentmarker}{}%
\end{pgfscope}%
\end{pgfscope}%
\begin{pgfscope}%
\pgftext[x=3.947753in,y=0.477363in,,top]{{\rmfamily\fontsize{10.000000}{12.000000}\selectfont \(\displaystyle 0.006\)}}%
\end{pgfscope}%
\begin{pgfscope}%
\pgfpathrectangle{\pgfqpoint{0.580766in}{0.532919in}}{\pgfqpoint{5.611646in}{3.193122in}} %
\pgfusepath{clip}%
\pgfsetbuttcap%
\pgfsetroundjoin%
\pgfsetlinewidth{0.501875pt}%
\definecolor{currentstroke}{rgb}{0.000000,0.000000,0.000000}%
\pgfsetstrokecolor{currentstroke}%
\pgfsetdash{{1.000000pt}{3.000000pt}}{0.000000pt}%
\pgfpathmoveto{\pgfqpoint{5.070082in}{0.532919in}}%
\pgfpathlineto{\pgfqpoint{5.070082in}{3.726041in}}%
\pgfusepath{stroke}%
\end{pgfscope}%
\begin{pgfscope}%
\pgfsetbuttcap%
\pgfsetroundjoin%
\definecolor{currentfill}{rgb}{0.000000,0.000000,0.000000}%
\pgfsetfillcolor{currentfill}%
\pgfsetlinewidth{0.501875pt}%
\definecolor{currentstroke}{rgb}{0.000000,0.000000,0.000000}%
\pgfsetstrokecolor{currentstroke}%
\pgfsetdash{}{0pt}%
\pgfsys@defobject{currentmarker}{\pgfqpoint{0.000000in}{0.000000in}}{\pgfqpoint{0.000000in}{0.055556in}}{%
\pgfpathmoveto{\pgfqpoint{0.000000in}{0.000000in}}%
\pgfpathlineto{\pgfqpoint{0.000000in}{0.055556in}}%
\pgfusepath{stroke,fill}%
}%
\begin{pgfscope}%
\pgfsys@transformshift{5.070082in}{0.532919in}%
\pgfsys@useobject{currentmarker}{}%
\end{pgfscope}%
\end{pgfscope}%
\begin{pgfscope}%
\pgfsetbuttcap%
\pgfsetroundjoin%
\definecolor{currentfill}{rgb}{0.000000,0.000000,0.000000}%
\pgfsetfillcolor{currentfill}%
\pgfsetlinewidth{0.501875pt}%
\definecolor{currentstroke}{rgb}{0.000000,0.000000,0.000000}%
\pgfsetstrokecolor{currentstroke}%
\pgfsetdash{}{0pt}%
\pgfsys@defobject{currentmarker}{\pgfqpoint{0.000000in}{-0.055556in}}{\pgfqpoint{0.000000in}{0.000000in}}{%
\pgfpathmoveto{\pgfqpoint{0.000000in}{0.000000in}}%
\pgfpathlineto{\pgfqpoint{0.000000in}{-0.055556in}}%
\pgfusepath{stroke,fill}%
}%
\begin{pgfscope}%
\pgfsys@transformshift{5.070082in}{3.726041in}%
\pgfsys@useobject{currentmarker}{}%
\end{pgfscope}%
\end{pgfscope}%
\begin{pgfscope}%
\pgftext[x=5.070082in,y=0.477363in,,top]{{\rmfamily\fontsize{10.000000}{12.000000}\selectfont \(\displaystyle 0.008\)}}%
\end{pgfscope}%
\begin{pgfscope}%
\pgfpathrectangle{\pgfqpoint{0.580766in}{0.532919in}}{\pgfqpoint{5.611646in}{3.193122in}} %
\pgfusepath{clip}%
\pgfsetbuttcap%
\pgfsetroundjoin%
\pgfsetlinewidth{0.501875pt}%
\definecolor{currentstroke}{rgb}{0.000000,0.000000,0.000000}%
\pgfsetstrokecolor{currentstroke}%
\pgfsetdash{{1.000000pt}{3.000000pt}}{0.000000pt}%
\pgfpathmoveto{\pgfqpoint{6.192411in}{0.532919in}}%
\pgfpathlineto{\pgfqpoint{6.192411in}{3.726041in}}%
\pgfusepath{stroke}%
\end{pgfscope}%
\begin{pgfscope}%
\pgfsetbuttcap%
\pgfsetroundjoin%
\definecolor{currentfill}{rgb}{0.000000,0.000000,0.000000}%
\pgfsetfillcolor{currentfill}%
\pgfsetlinewidth{0.501875pt}%
\definecolor{currentstroke}{rgb}{0.000000,0.000000,0.000000}%
\pgfsetstrokecolor{currentstroke}%
\pgfsetdash{}{0pt}%
\pgfsys@defobject{currentmarker}{\pgfqpoint{0.000000in}{0.000000in}}{\pgfqpoint{0.000000in}{0.055556in}}{%
\pgfpathmoveto{\pgfqpoint{0.000000in}{0.000000in}}%
\pgfpathlineto{\pgfqpoint{0.000000in}{0.055556in}}%
\pgfusepath{stroke,fill}%
}%
\begin{pgfscope}%
\pgfsys@transformshift{6.192411in}{0.532919in}%
\pgfsys@useobject{currentmarker}{}%
\end{pgfscope}%
\end{pgfscope}%
\begin{pgfscope}%
\pgfsetbuttcap%
\pgfsetroundjoin%
\definecolor{currentfill}{rgb}{0.000000,0.000000,0.000000}%
\pgfsetfillcolor{currentfill}%
\pgfsetlinewidth{0.501875pt}%
\definecolor{currentstroke}{rgb}{0.000000,0.000000,0.000000}%
\pgfsetstrokecolor{currentstroke}%
\pgfsetdash{}{0pt}%
\pgfsys@defobject{currentmarker}{\pgfqpoint{0.000000in}{-0.055556in}}{\pgfqpoint{0.000000in}{0.000000in}}{%
\pgfpathmoveto{\pgfqpoint{0.000000in}{0.000000in}}%
\pgfpathlineto{\pgfqpoint{0.000000in}{-0.055556in}}%
\pgfusepath{stroke,fill}%
}%
\begin{pgfscope}%
\pgfsys@transformshift{6.192411in}{3.726041in}%
\pgfsys@useobject{currentmarker}{}%
\end{pgfscope}%
\end{pgfscope}%
\begin{pgfscope}%
\pgftext[x=6.192411in,y=0.477363in,,top]{{\rmfamily\fontsize{10.000000}{12.000000}\selectfont \(\displaystyle 0.010\)}}%
\end{pgfscope}%
\begin{pgfscope}%
\pgftext[x=3.386589in,y=0.284462in,,top]{{\rmfamily\fontsize{10.000000}{12.000000}\selectfont time (s)}}%
\end{pgfscope}%
\begin{pgfscope}%
\pgfpathrectangle{\pgfqpoint{0.580766in}{0.532919in}}{\pgfqpoint{5.611646in}{3.193122in}} %
\pgfusepath{clip}%
\pgfsetbuttcap%
\pgfsetroundjoin%
\pgfsetlinewidth{0.501875pt}%
\definecolor{currentstroke}{rgb}{0.000000,0.000000,0.000000}%
\pgfsetstrokecolor{currentstroke}%
\pgfsetdash{{1.000000pt}{3.000000pt}}{0.000000pt}%
\pgfpathmoveto{\pgfqpoint{0.580766in}{0.532919in}}%
\pgfpathlineto{\pgfqpoint{6.192411in}{0.532919in}}%
\pgfusepath{stroke}%
\end{pgfscope}%
\begin{pgfscope}%
\pgfsetbuttcap%
\pgfsetroundjoin%
\definecolor{currentfill}{rgb}{0.000000,0.000000,0.000000}%
\pgfsetfillcolor{currentfill}%
\pgfsetlinewidth{0.501875pt}%
\definecolor{currentstroke}{rgb}{0.000000,0.000000,0.000000}%
\pgfsetstrokecolor{currentstroke}%
\pgfsetdash{}{0pt}%
\pgfsys@defobject{currentmarker}{\pgfqpoint{0.000000in}{0.000000in}}{\pgfqpoint{0.055556in}{0.000000in}}{%
\pgfpathmoveto{\pgfqpoint{0.000000in}{0.000000in}}%
\pgfpathlineto{\pgfqpoint{0.055556in}{0.000000in}}%
\pgfusepath{stroke,fill}%
}%
\begin{pgfscope}%
\pgfsys@transformshift{0.580766in}{0.532919in}%
\pgfsys@useobject{currentmarker}{}%
\end{pgfscope}%
\end{pgfscope}%
\begin{pgfscope}%
\pgfsetbuttcap%
\pgfsetroundjoin%
\definecolor{currentfill}{rgb}{0.000000,0.000000,0.000000}%
\pgfsetfillcolor{currentfill}%
\pgfsetlinewidth{0.501875pt}%
\definecolor{currentstroke}{rgb}{0.000000,0.000000,0.000000}%
\pgfsetstrokecolor{currentstroke}%
\pgfsetdash{}{0pt}%
\pgfsys@defobject{currentmarker}{\pgfqpoint{-0.055556in}{0.000000in}}{\pgfqpoint{0.000000in}{0.000000in}}{%
\pgfpathmoveto{\pgfqpoint{0.000000in}{0.000000in}}%
\pgfpathlineto{\pgfqpoint{-0.055556in}{0.000000in}}%
\pgfusepath{stroke,fill}%
}%
\begin{pgfscope}%
\pgfsys@transformshift{6.192411in}{0.532919in}%
\pgfsys@useobject{currentmarker}{}%
\end{pgfscope}%
\end{pgfscope}%
\begin{pgfscope}%
\pgftext[x=0.525210in,y=0.532919in,right,]{{\rmfamily\fontsize{10.000000}{12.000000}\selectfont \(\displaystyle 0\)}}%
\end{pgfscope}%
\begin{pgfscope}%
\pgfpathrectangle{\pgfqpoint{0.580766in}{0.532919in}}{\pgfqpoint{5.611646in}{3.193122in}} %
\pgfusepath{clip}%
\pgfsetbuttcap%
\pgfsetroundjoin%
\pgfsetlinewidth{0.501875pt}%
\definecolor{currentstroke}{rgb}{0.000000,0.000000,0.000000}%
\pgfsetstrokecolor{currentstroke}%
\pgfsetdash{{1.000000pt}{3.000000pt}}{0.000000pt}%
\pgfpathmoveto{\pgfqpoint{0.580766in}{1.171543in}}%
\pgfpathlineto{\pgfqpoint{6.192411in}{1.171543in}}%
\pgfusepath{stroke}%
\end{pgfscope}%
\begin{pgfscope}%
\pgfsetbuttcap%
\pgfsetroundjoin%
\definecolor{currentfill}{rgb}{0.000000,0.000000,0.000000}%
\pgfsetfillcolor{currentfill}%
\pgfsetlinewidth{0.501875pt}%
\definecolor{currentstroke}{rgb}{0.000000,0.000000,0.000000}%
\pgfsetstrokecolor{currentstroke}%
\pgfsetdash{}{0pt}%
\pgfsys@defobject{currentmarker}{\pgfqpoint{0.000000in}{0.000000in}}{\pgfqpoint{0.055556in}{0.000000in}}{%
\pgfpathmoveto{\pgfqpoint{0.000000in}{0.000000in}}%
\pgfpathlineto{\pgfqpoint{0.055556in}{0.000000in}}%
\pgfusepath{stroke,fill}%
}%
\begin{pgfscope}%
\pgfsys@transformshift{0.580766in}{1.171543in}%
\pgfsys@useobject{currentmarker}{}%
\end{pgfscope}%
\end{pgfscope}%
\begin{pgfscope}%
\pgfsetbuttcap%
\pgfsetroundjoin%
\definecolor{currentfill}{rgb}{0.000000,0.000000,0.000000}%
\pgfsetfillcolor{currentfill}%
\pgfsetlinewidth{0.501875pt}%
\definecolor{currentstroke}{rgb}{0.000000,0.000000,0.000000}%
\pgfsetstrokecolor{currentstroke}%
\pgfsetdash{}{0pt}%
\pgfsys@defobject{currentmarker}{\pgfqpoint{-0.055556in}{0.000000in}}{\pgfqpoint{0.000000in}{0.000000in}}{%
\pgfpathmoveto{\pgfqpoint{0.000000in}{0.000000in}}%
\pgfpathlineto{\pgfqpoint{-0.055556in}{0.000000in}}%
\pgfusepath{stroke,fill}%
}%
\begin{pgfscope}%
\pgfsys@transformshift{6.192411in}{1.171543in}%
\pgfsys@useobject{currentmarker}{}%
\end{pgfscope}%
\end{pgfscope}%
\begin{pgfscope}%
\pgftext[x=0.525210in,y=1.171543in,right,]{{\rmfamily\fontsize{10.000000}{12.000000}\selectfont \(\displaystyle 2\)}}%
\end{pgfscope}%
\begin{pgfscope}%
\pgfpathrectangle{\pgfqpoint{0.580766in}{0.532919in}}{\pgfqpoint{5.611646in}{3.193122in}} %
\pgfusepath{clip}%
\pgfsetbuttcap%
\pgfsetroundjoin%
\pgfsetlinewidth{0.501875pt}%
\definecolor{currentstroke}{rgb}{0.000000,0.000000,0.000000}%
\pgfsetstrokecolor{currentstroke}%
\pgfsetdash{{1.000000pt}{3.000000pt}}{0.000000pt}%
\pgfpathmoveto{\pgfqpoint{0.580766in}{1.810167in}}%
\pgfpathlineto{\pgfqpoint{6.192411in}{1.810167in}}%
\pgfusepath{stroke}%
\end{pgfscope}%
\begin{pgfscope}%
\pgfsetbuttcap%
\pgfsetroundjoin%
\definecolor{currentfill}{rgb}{0.000000,0.000000,0.000000}%
\pgfsetfillcolor{currentfill}%
\pgfsetlinewidth{0.501875pt}%
\definecolor{currentstroke}{rgb}{0.000000,0.000000,0.000000}%
\pgfsetstrokecolor{currentstroke}%
\pgfsetdash{}{0pt}%
\pgfsys@defobject{currentmarker}{\pgfqpoint{0.000000in}{0.000000in}}{\pgfqpoint{0.055556in}{0.000000in}}{%
\pgfpathmoveto{\pgfqpoint{0.000000in}{0.000000in}}%
\pgfpathlineto{\pgfqpoint{0.055556in}{0.000000in}}%
\pgfusepath{stroke,fill}%
}%
\begin{pgfscope}%
\pgfsys@transformshift{0.580766in}{1.810167in}%
\pgfsys@useobject{currentmarker}{}%
\end{pgfscope}%
\end{pgfscope}%
\begin{pgfscope}%
\pgfsetbuttcap%
\pgfsetroundjoin%
\definecolor{currentfill}{rgb}{0.000000,0.000000,0.000000}%
\pgfsetfillcolor{currentfill}%
\pgfsetlinewidth{0.501875pt}%
\definecolor{currentstroke}{rgb}{0.000000,0.000000,0.000000}%
\pgfsetstrokecolor{currentstroke}%
\pgfsetdash{}{0pt}%
\pgfsys@defobject{currentmarker}{\pgfqpoint{-0.055556in}{0.000000in}}{\pgfqpoint{0.000000in}{0.000000in}}{%
\pgfpathmoveto{\pgfqpoint{0.000000in}{0.000000in}}%
\pgfpathlineto{\pgfqpoint{-0.055556in}{0.000000in}}%
\pgfusepath{stroke,fill}%
}%
\begin{pgfscope}%
\pgfsys@transformshift{6.192411in}{1.810167in}%
\pgfsys@useobject{currentmarker}{}%
\end{pgfscope}%
\end{pgfscope}%
\begin{pgfscope}%
\pgftext[x=0.525210in,y=1.810167in,right,]{{\rmfamily\fontsize{10.000000}{12.000000}\selectfont \(\displaystyle 4\)}}%
\end{pgfscope}%
\begin{pgfscope}%
\pgfpathrectangle{\pgfqpoint{0.580766in}{0.532919in}}{\pgfqpoint{5.611646in}{3.193122in}} %
\pgfusepath{clip}%
\pgfsetbuttcap%
\pgfsetroundjoin%
\pgfsetlinewidth{0.501875pt}%
\definecolor{currentstroke}{rgb}{0.000000,0.000000,0.000000}%
\pgfsetstrokecolor{currentstroke}%
\pgfsetdash{{1.000000pt}{3.000000pt}}{0.000000pt}%
\pgfpathmoveto{\pgfqpoint{0.580766in}{2.448792in}}%
\pgfpathlineto{\pgfqpoint{6.192411in}{2.448792in}}%
\pgfusepath{stroke}%
\end{pgfscope}%
\begin{pgfscope}%
\pgfsetbuttcap%
\pgfsetroundjoin%
\definecolor{currentfill}{rgb}{0.000000,0.000000,0.000000}%
\pgfsetfillcolor{currentfill}%
\pgfsetlinewidth{0.501875pt}%
\definecolor{currentstroke}{rgb}{0.000000,0.000000,0.000000}%
\pgfsetstrokecolor{currentstroke}%
\pgfsetdash{}{0pt}%
\pgfsys@defobject{currentmarker}{\pgfqpoint{0.000000in}{0.000000in}}{\pgfqpoint{0.055556in}{0.000000in}}{%
\pgfpathmoveto{\pgfqpoint{0.000000in}{0.000000in}}%
\pgfpathlineto{\pgfqpoint{0.055556in}{0.000000in}}%
\pgfusepath{stroke,fill}%
}%
\begin{pgfscope}%
\pgfsys@transformshift{0.580766in}{2.448792in}%
\pgfsys@useobject{currentmarker}{}%
\end{pgfscope}%
\end{pgfscope}%
\begin{pgfscope}%
\pgfsetbuttcap%
\pgfsetroundjoin%
\definecolor{currentfill}{rgb}{0.000000,0.000000,0.000000}%
\pgfsetfillcolor{currentfill}%
\pgfsetlinewidth{0.501875pt}%
\definecolor{currentstroke}{rgb}{0.000000,0.000000,0.000000}%
\pgfsetstrokecolor{currentstroke}%
\pgfsetdash{}{0pt}%
\pgfsys@defobject{currentmarker}{\pgfqpoint{-0.055556in}{0.000000in}}{\pgfqpoint{0.000000in}{0.000000in}}{%
\pgfpathmoveto{\pgfqpoint{0.000000in}{0.000000in}}%
\pgfpathlineto{\pgfqpoint{-0.055556in}{0.000000in}}%
\pgfusepath{stroke,fill}%
}%
\begin{pgfscope}%
\pgfsys@transformshift{6.192411in}{2.448792in}%
\pgfsys@useobject{currentmarker}{}%
\end{pgfscope}%
\end{pgfscope}%
\begin{pgfscope}%
\pgftext[x=0.525210in,y=2.448792in,right,]{{\rmfamily\fontsize{10.000000}{12.000000}\selectfont \(\displaystyle 6\)}}%
\end{pgfscope}%
\begin{pgfscope}%
\pgfpathrectangle{\pgfqpoint{0.580766in}{0.532919in}}{\pgfqpoint{5.611646in}{3.193122in}} %
\pgfusepath{clip}%
\pgfsetbuttcap%
\pgfsetroundjoin%
\pgfsetlinewidth{0.501875pt}%
\definecolor{currentstroke}{rgb}{0.000000,0.000000,0.000000}%
\pgfsetstrokecolor{currentstroke}%
\pgfsetdash{{1.000000pt}{3.000000pt}}{0.000000pt}%
\pgfpathmoveto{\pgfqpoint{0.580766in}{3.087416in}}%
\pgfpathlineto{\pgfqpoint{6.192411in}{3.087416in}}%
\pgfusepath{stroke}%
\end{pgfscope}%
\begin{pgfscope}%
\pgfsetbuttcap%
\pgfsetroundjoin%
\definecolor{currentfill}{rgb}{0.000000,0.000000,0.000000}%
\pgfsetfillcolor{currentfill}%
\pgfsetlinewidth{0.501875pt}%
\definecolor{currentstroke}{rgb}{0.000000,0.000000,0.000000}%
\pgfsetstrokecolor{currentstroke}%
\pgfsetdash{}{0pt}%
\pgfsys@defobject{currentmarker}{\pgfqpoint{0.000000in}{0.000000in}}{\pgfqpoint{0.055556in}{0.000000in}}{%
\pgfpathmoveto{\pgfqpoint{0.000000in}{0.000000in}}%
\pgfpathlineto{\pgfqpoint{0.055556in}{0.000000in}}%
\pgfusepath{stroke,fill}%
}%
\begin{pgfscope}%
\pgfsys@transformshift{0.580766in}{3.087416in}%
\pgfsys@useobject{currentmarker}{}%
\end{pgfscope}%
\end{pgfscope}%
\begin{pgfscope}%
\pgfsetbuttcap%
\pgfsetroundjoin%
\definecolor{currentfill}{rgb}{0.000000,0.000000,0.000000}%
\pgfsetfillcolor{currentfill}%
\pgfsetlinewidth{0.501875pt}%
\definecolor{currentstroke}{rgb}{0.000000,0.000000,0.000000}%
\pgfsetstrokecolor{currentstroke}%
\pgfsetdash{}{0pt}%
\pgfsys@defobject{currentmarker}{\pgfqpoint{-0.055556in}{0.000000in}}{\pgfqpoint{0.000000in}{0.000000in}}{%
\pgfpathmoveto{\pgfqpoint{0.000000in}{0.000000in}}%
\pgfpathlineto{\pgfqpoint{-0.055556in}{0.000000in}}%
\pgfusepath{stroke,fill}%
}%
\begin{pgfscope}%
\pgfsys@transformshift{6.192411in}{3.087416in}%
\pgfsys@useobject{currentmarker}{}%
\end{pgfscope}%
\end{pgfscope}%
\begin{pgfscope}%
\pgftext[x=0.525210in,y=3.087416in,right,]{{\rmfamily\fontsize{10.000000}{12.000000}\selectfont \(\displaystyle 8\)}}%
\end{pgfscope}%
\begin{pgfscope}%
\pgfpathrectangle{\pgfqpoint{0.580766in}{0.532919in}}{\pgfqpoint{5.611646in}{3.193122in}} %
\pgfusepath{clip}%
\pgfsetbuttcap%
\pgfsetroundjoin%
\pgfsetlinewidth{0.501875pt}%
\definecolor{currentstroke}{rgb}{0.000000,0.000000,0.000000}%
\pgfsetstrokecolor{currentstroke}%
\pgfsetdash{{1.000000pt}{3.000000pt}}{0.000000pt}%
\pgfpathmoveto{\pgfqpoint{0.580766in}{3.726041in}}%
\pgfpathlineto{\pgfqpoint{6.192411in}{3.726041in}}%
\pgfusepath{stroke}%
\end{pgfscope}%
\begin{pgfscope}%
\pgfsetbuttcap%
\pgfsetroundjoin%
\definecolor{currentfill}{rgb}{0.000000,0.000000,0.000000}%
\pgfsetfillcolor{currentfill}%
\pgfsetlinewidth{0.501875pt}%
\definecolor{currentstroke}{rgb}{0.000000,0.000000,0.000000}%
\pgfsetstrokecolor{currentstroke}%
\pgfsetdash{}{0pt}%
\pgfsys@defobject{currentmarker}{\pgfqpoint{0.000000in}{0.000000in}}{\pgfqpoint{0.055556in}{0.000000in}}{%
\pgfpathmoveto{\pgfqpoint{0.000000in}{0.000000in}}%
\pgfpathlineto{\pgfqpoint{0.055556in}{0.000000in}}%
\pgfusepath{stroke,fill}%
}%
\begin{pgfscope}%
\pgfsys@transformshift{0.580766in}{3.726041in}%
\pgfsys@useobject{currentmarker}{}%
\end{pgfscope}%
\end{pgfscope}%
\begin{pgfscope}%
\pgfsetbuttcap%
\pgfsetroundjoin%
\definecolor{currentfill}{rgb}{0.000000,0.000000,0.000000}%
\pgfsetfillcolor{currentfill}%
\pgfsetlinewidth{0.501875pt}%
\definecolor{currentstroke}{rgb}{0.000000,0.000000,0.000000}%
\pgfsetstrokecolor{currentstroke}%
\pgfsetdash{}{0pt}%
\pgfsys@defobject{currentmarker}{\pgfqpoint{-0.055556in}{0.000000in}}{\pgfqpoint{0.000000in}{0.000000in}}{%
\pgfpathmoveto{\pgfqpoint{0.000000in}{0.000000in}}%
\pgfpathlineto{\pgfqpoint{-0.055556in}{0.000000in}}%
\pgfusepath{stroke,fill}%
}%
\begin{pgfscope}%
\pgfsys@transformshift{6.192411in}{3.726041in}%
\pgfsys@useobject{currentmarker}{}%
\end{pgfscope}%
\end{pgfscope}%
\begin{pgfscope}%
\pgftext[x=0.525210in,y=3.726041in,right,]{{\rmfamily\fontsize{10.000000}{12.000000}\selectfont \(\displaystyle 10\)}}%
\end{pgfscope}%
\begin{pgfscope}%
\pgftext[x=0.316877in,y=2.129480in,,bottom,rotate=90.000000]{{\rmfamily\fontsize{10.000000}{12.000000}\selectfont voltage (V)}}%
\end{pgfscope}%
\begin{pgfscope}%
\pgfsetbuttcap%
\pgfsetroundjoin%
\pgfsetlinewidth{1.003750pt}%
\definecolor{currentstroke}{rgb}{0.000000,0.000000,0.000000}%
\pgfsetstrokecolor{currentstroke}%
\pgfsetdash{}{0pt}%
\pgfpathmoveto{\pgfqpoint{0.580766in}{3.726041in}}%
\pgfpathlineto{\pgfqpoint{6.192411in}{3.726041in}}%
\pgfusepath{stroke}%
\end{pgfscope}%
\begin{pgfscope}%
\pgfsetbuttcap%
\pgfsetroundjoin%
\pgfsetlinewidth{1.003750pt}%
\definecolor{currentstroke}{rgb}{0.000000,0.000000,0.000000}%
\pgfsetstrokecolor{currentstroke}%
\pgfsetdash{}{0pt}%
\pgfpathmoveto{\pgfqpoint{6.192411in}{0.532919in}}%
\pgfpathlineto{\pgfqpoint{6.192411in}{3.726041in}}%
\pgfusepath{stroke}%
\end{pgfscope}%
\begin{pgfscope}%
\pgfsetbuttcap%
\pgfsetroundjoin%
\pgfsetlinewidth{1.003750pt}%
\definecolor{currentstroke}{rgb}{0.000000,0.000000,0.000000}%
\pgfsetstrokecolor{currentstroke}%
\pgfsetdash{}{0pt}%
\pgfpathmoveto{\pgfqpoint{0.580766in}{0.532919in}}%
\pgfpathlineto{\pgfqpoint{6.192411in}{0.532919in}}%
\pgfusepath{stroke}%
\end{pgfscope}%
\begin{pgfscope}%
\pgfsetbuttcap%
\pgfsetroundjoin%
\pgfsetlinewidth{1.003750pt}%
\definecolor{currentstroke}{rgb}{0.000000,0.000000,0.000000}%
\pgfsetstrokecolor{currentstroke}%
\pgfsetdash{}{0pt}%
\pgfpathmoveto{\pgfqpoint{0.580766in}{0.532919in}}%
\pgfpathlineto{\pgfqpoint{0.580766in}{3.726041in}}%
\pgfusepath{stroke}%
\end{pgfscope}%
\begin{pgfscope}%
\pgftext[x=3.386589in,y=3.795485in,,base]{{\rmfamily\fontsize{12.000000}{14.400000}\selectfont Step-By-Step Approximations of \(\displaystyle v_L(t)\)}}%
\end{pgfscope}%
\begin{pgfscope}%
\pgfsetbuttcap%
\pgfsetroundjoin%
\definecolor{currentfill}{rgb}{0.300000,0.300000,0.300000}%
\pgfsetfillcolor{currentfill}%
\pgfsetfillopacity{0.500000}%
\pgfsetlinewidth{1.003750pt}%
\definecolor{currentstroke}{rgb}{0.300000,0.300000,0.300000}%
\pgfsetstrokecolor{currentstroke}%
\pgfsetstrokeopacity{0.500000}%
\pgfsetdash{}{0pt}%
\pgfpathmoveto{\pgfqpoint{4.378033in}{2.867709in}}%
\pgfpathlineto{\pgfqpoint{6.103523in}{2.867709in}}%
\pgfpathquadraticcurveto{\pgfqpoint{6.136856in}{2.867709in}}{\pgfqpoint{6.136856in}{2.901042in}}%
\pgfpathlineto{\pgfqpoint{6.136856in}{3.581596in}}%
\pgfpathquadraticcurveto{\pgfqpoint{6.136856in}{3.614930in}}{\pgfqpoint{6.103523in}{3.614930in}}%
\pgfpathlineto{\pgfqpoint{4.378033in}{3.614930in}}%
\pgfpathquadraticcurveto{\pgfqpoint{4.344700in}{3.614930in}}{\pgfqpoint{4.344700in}{3.581596in}}%
\pgfpathlineto{\pgfqpoint{4.344700in}{2.901042in}}%
\pgfpathquadraticcurveto{\pgfqpoint{4.344700in}{2.867709in}}{\pgfqpoint{4.378033in}{2.867709in}}%
\pgfpathclose%
\pgfusepath{stroke,fill}%
\end{pgfscope}%
\begin{pgfscope}%
\pgfsetbuttcap%
\pgfsetroundjoin%
\definecolor{currentfill}{rgb}{1.000000,1.000000,1.000000}%
\pgfsetfillcolor{currentfill}%
\pgfsetlinewidth{1.003750pt}%
\definecolor{currentstroke}{rgb}{0.000000,0.000000,0.000000}%
\pgfsetstrokecolor{currentstroke}%
\pgfsetdash{}{0pt}%
\pgfpathmoveto{\pgfqpoint{4.350255in}{2.895486in}}%
\pgfpathlineto{\pgfqpoint{6.075745in}{2.895486in}}%
\pgfpathquadraticcurveto{\pgfqpoint{6.109078in}{2.895486in}}{\pgfqpoint{6.109078in}{2.928820in}}%
\pgfpathlineto{\pgfqpoint{6.109078in}{3.609374in}}%
\pgfpathquadraticcurveto{\pgfqpoint{6.109078in}{3.642708in}}{\pgfqpoint{6.075745in}{3.642708in}}%
\pgfpathlineto{\pgfqpoint{4.350255in}{3.642708in}}%
\pgfpathquadraticcurveto{\pgfqpoint{4.316922in}{3.642708in}}{\pgfqpoint{4.316922in}{3.609374in}}%
\pgfpathlineto{\pgfqpoint{4.316922in}{2.928820in}}%
\pgfpathquadraticcurveto{\pgfqpoint{4.316922in}{2.895486in}}{\pgfqpoint{4.350255in}{2.895486in}}%
\pgfpathclose%
\pgfusepath{stroke,fill}%
\end{pgfscope}%
\begin{pgfscope}%
\pgfsetbuttcap%
\pgfsetroundjoin%
\pgfsetlinewidth{1.003750pt}%
\definecolor{currentstroke}{rgb}{0.000000,0.000000,1.000000}%
\pgfsetstrokecolor{currentstroke}%
\pgfsetdash{{1.000000pt}{3.000000pt}}{0.000000pt}%
\pgfpathmoveto{\pgfqpoint{4.433589in}{3.517708in}}%
\pgfpathlineto{\pgfqpoint{4.666922in}{3.517708in}}%
\pgfusepath{stroke}%
\end{pgfscope}%
\begin{pgfscope}%
\pgftext[x=4.850255in,y=3.459374in,left,base]{{\rmfamily\fontsize{12.000000}{14.400000}\selectfont Continuous}}%
\end{pgfscope}%
\begin{pgfscope}%
\pgfsetrectcap%
\pgfsetroundjoin%
\pgfsetlinewidth{1.003750pt}%
\definecolor{currentstroke}{rgb}{0.000000,0.500000,0.000000}%
\pgfsetstrokecolor{currentstroke}%
\pgfsetdash{}{0pt}%
\pgfpathmoveto{\pgfqpoint{4.433589in}{3.285300in}}%
\pgfpathlineto{\pgfqpoint{4.666922in}{3.285300in}}%
\pgfusepath{stroke}%
\end{pgfscope}%
\begin{pgfscope}%
\pgftext[x=4.850255in,y=3.226967in,left,base]{{\rmfamily\fontsize{12.000000}{14.400000}\selectfont Trapezoidal}}%
\end{pgfscope}%
\begin{pgfscope}%
\pgfsetrectcap%
\pgfsetroundjoin%
\pgfsetlinewidth{1.003750pt}%
\definecolor{currentstroke}{rgb}{1.000000,0.000000,0.000000}%
\pgfsetstrokecolor{currentstroke}%
\pgfsetdash{}{0pt}%
\pgfpathmoveto{\pgfqpoint{4.433589in}{3.052893in}}%
\pgfpathlineto{\pgfqpoint{4.666922in}{3.052893in}}%
\pgfusepath{stroke}%
\end{pgfscope}%
\begin{pgfscope}%
\pgftext[x=4.850255in,y=2.994560in,left,base]{{\rmfamily\fontsize{12.000000}{14.400000}\selectfont Backward Euler    }}%
\end{pgfscope}%
\end{pgfpicture}%
\makeatother%
\endgroup%

    \end{center}
    \caption{Comparison of Approximations at $\Delta{}t_2 = 0.8 ms$}
    \label{approx_comp}
\end{figure}
\section{Conclusion}
Overall this was a good refresher on transient circuit analysis as well as a productive introduction to discretization techniques. Further optimizations and improvements are proposed below:
\begin{itemize}
    \item In order to perform more efficient calculation steps, the program could decide to dynamically modify its solution's step size if it detects the output to be stable. This would allow it to more effectively make use of its compute cycles. A threshold could be implemented to determine the range of change for which it should attempt to modify its step size.
    \item In a similar vein, the program could also attempt changing its discretization rules on the fly to better approximate the transient solution. Perhaps in some scenarios, a circuit exhibits several different stages where certain approximations work better than others.
    \item Using a programming language that supports special compiler hints, a program could be written to make use of special processing units that accelerate an approximation's iteration. Perhaps a program could also run multiple discretization rules in parallel, and through some kind of analysis, dynamically switch between the more accurate solution. 
\end{itemize}
\newpage
\section{Code Listing}
The following is the code written in Python to perform the calculations derived for this homework assignment as well as generate the plots used in this report.
\lstinputlisting[language=Python]{assignment1a.py}
	
\end{document}

