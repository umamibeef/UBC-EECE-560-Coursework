\documentclass[10pt, oneside, letterpaper]{article}
\usepackage[margin=1in]{geometry}
\usepackage[english]{babel}
\usepackage[utf8]{inputenc}
\usepackage{color}
\definecolor{mygreen}{rgb}{0,0.6,0}
\definecolor{mygray}{rgb}{0.5,0.5,0.5}
\definecolor{mymauve}{rgb}{0.58,0,0.82}
\usepackage{listings}
\lstset{
  backgroundcolor=\color{white}, % choose the background color
  basicstyle=\footnotesize\ttfamily, % size of fonts used for the code
  breaklines=true, % automatic line breaking only at whitespace
  frame=single, % add a frame
  captionpos=b, % sets the caption-position to bottom
  commentstyle=\color{mygreen}, % comment style
  escapeinside={\%*}{*)}, % if you want to add LaTeX within your code
  keywordstyle=\color{blue}, % keyword style
  stringstyle=\color{mymauve}, % string literal style
}
\usepackage{enumitem}
\usepackage{blindtext}
\usepackage{datetime2}
\usepackage{fancyhdr}
\usepackage{amsmath}
\usepackage{mathtools}
\usepackage{float}
\usepackage{pgf}
% \usepackage{layouts}
% \printinunitsof{in}\prntlen{\textwidth}
\title{Basic Circuit Discretizations with PSCAD}
\author{Assignment 1b}
\date{Due: 2021/01/29}

\pagestyle{fancy}
\setlength{\headheight}{23pt}
\setlength{\parskip}{1em}
\fancyhf{}
\chead{Assignment 1b \\ Basic Circuit Discretizations with PSCAD}
\rhead{Michel Kakulphimp \\ Student \#63542880}
\lhead{EECE560 \\ UBC MEng}
\cfoot{\thepage}

\begin{document}
\maketitle
\thispagestyle{fancy}

\section{Introduction}
As a follow-up to the previous assignment, we now use a software suite known as PSCAD to simulate the same circuit as before. PSCAD is an electromagnetic transient simulator, allowing the user to model complex power systems and analyze their behaviour.

\section{Setup}
PSCAD provides a master library of components as well as primitives to help model the circuit. The "Single Phase Voltage Source Model" component was used as the source and it was made ideal by setting the source impedance to $0 \Omega{}$ with a ramp-up time of $0 s$ (instantaneous voltage). The ideal resistor and inductors were used to model the passive components and a multimeter was added to the top of the inductor. The multimeter's instantaneous current and voltage readings were connected to plots which were placed directly on the schematic sheet. Figure \ref{pscad-setup} illustrates how the PSCAD schematic sheet was set up for this assignment.

Once all of the components were hooked together and instrumented, the build button was clicked. This generated a Fortran program that models the entire circuit. A listing of this program can be found in Section \ref{code-listing-fortran}.

Clicking the run button ran the simulation and outputted the results on the plots. The data was extracted by right-clicking each plot to save it to the clipboard. It was then formatted and added to the Python program for plotting against the hand-derived approximations.

\begin{figure}[H]
\centering
\includegraphics[width=5in]{pscad.png}
\caption{PSCAD schematic and resulting plots for $\Delta{}t = 100\mu{}s$ circuit analysis.}
\label{pscad-setup}
\end{figure}

\section{Simulation}

Recall that in assignment 1a, the following approximations were derived and plotted:

\begin{enumerate}[label=\alph*)]
  \item Using the trapezoidal rule with $\Delta{}t_1 = 0.1ms$
  \item Using the backward Euler rule with $\Delta{}t_1 = 0.1ms$
  \item Using the trapezoidal rule with $\Delta{}t_2 = 0.8ms$
  \item Using the backward Euler rule with $\Delta{}t_2 = 0.8ms$
\end{enumerate}

In this assignment, we append the PSCAD solutions generated and compare it to the manually derived approximations on the same plots. Two sets of plots are generated, at time steps $\Delta{}t_1 = 0.1ms$ and $\Delta{}t_2 = 0.8ms$. The following two figures, \ref{approx_pscad_comp_0p0001} and \ref{approx_pscad_comp_0p0008}, show PSCAD compared to the manually derived approximations for these two time steps respectively.

\begin{figure}[H]
    \begin{center}
        %% Creator: Matplotlib, PGF backend
%%
%% To include the figure in your LaTeX document, write
%%   \input{<filename>.pgf}
%%
%% Make sure the required packages are loaded in your preamble
%%   \usepackage{pgf}
%%
%% Figures using additional raster images can only be included by \input if
%% they are in the same directory as the main LaTeX file. For loading figures
%% from other directories you can use the `import` package
%%   \usepackage{import}
%% and then include the figures with
%%   \import{<path to file>}{<filename>.pgf}
%%
%% Matplotlib used the following preamble
%%
\begingroup%
\makeatletter%
\begin{pgfpicture}%
\pgfpathrectangle{\pgfpointorigin}{\pgfqpoint{6.500000in}{8.000000in}}%
\pgfusepath{use as bounding box}%
\begin{pgfscope}%
\pgfsetbuttcap%
\pgfsetroundjoin%
\definecolor{currentfill}{rgb}{1.000000,1.000000,1.000000}%
\pgfsetfillcolor{currentfill}%
\pgfsetlinewidth{0.000000pt}%
\definecolor{currentstroke}{rgb}{1.000000,1.000000,1.000000}%
\pgfsetstrokecolor{currentstroke}%
\pgfsetdash{}{0pt}%
\pgfpathmoveto{\pgfqpoint{0.000000in}{0.000000in}}%
\pgfpathlineto{\pgfqpoint{6.500000in}{0.000000in}}%
\pgfpathlineto{\pgfqpoint{6.500000in}{8.000000in}}%
\pgfpathlineto{\pgfqpoint{0.000000in}{8.000000in}}%
\pgfpathclose%
\pgfusepath{fill}%
\end{pgfscope}%
\begin{pgfscope}%
\pgfsetbuttcap%
\pgfsetroundjoin%
\definecolor{currentfill}{rgb}{1.000000,1.000000,1.000000}%
\pgfsetfillcolor{currentfill}%
\pgfsetlinewidth{0.000000pt}%
\definecolor{currentstroke}{rgb}{0.000000,0.000000,0.000000}%
\pgfsetstrokecolor{currentstroke}%
\pgfsetstrokeopacity{0.000000}%
\pgfsetdash{}{0pt}%
\pgfpathmoveto{\pgfqpoint{0.580766in}{4.462900in}}%
\pgfpathlineto{\pgfqpoint{6.192411in}{4.462900in}}%
\pgfpathlineto{\pgfqpoint{6.192411in}{7.656023in}}%
\pgfpathlineto{\pgfqpoint{0.580766in}{7.656023in}}%
\pgfpathclose%
\pgfusepath{fill}%
\end{pgfscope}%
\begin{pgfscope}%
\pgfpathrectangle{\pgfqpoint{0.580766in}{4.462900in}}{\pgfqpoint{5.611646in}{3.193122in}} %
\pgfusepath{clip}%
\pgfsetbuttcap%
\pgfsetroundjoin%
\pgfsetlinewidth{1.003750pt}%
\definecolor{currentstroke}{rgb}{0.000000,0.000000,1.000000}%
\pgfsetstrokecolor{currentstroke}%
\pgfsetdash{{1.000000pt}{3.000000pt}}{0.000000pt}%
\pgfpathmoveto{\pgfqpoint{0.580766in}{4.462900in}}%
\pgfpathlineto{\pgfqpoint{0.627530in}{4.618631in}}%
\pgfpathlineto{\pgfqpoint{0.674293in}{4.766766in}}%
\pgfpathlineto{\pgfqpoint{0.721057in}{4.907677in}}%
\pgfpathlineto{\pgfqpoint{0.767821in}{5.041715in}}%
\pgfpathlineto{\pgfqpoint{0.814584in}{5.169216in}}%
\pgfpathlineto{\pgfqpoint{0.861348in}{5.290499in}}%
\pgfpathlineto{\pgfqpoint{0.909671in}{5.409614in}}%
\pgfpathlineto{\pgfqpoint{0.957993in}{5.522731in}}%
\pgfpathlineto{\pgfqpoint{1.006316in}{5.630153in}}%
\pgfpathlineto{\pgfqpoint{1.054638in}{5.732164in}}%
\pgfpathlineto{\pgfqpoint{1.102961in}{5.829040in}}%
\pgfpathlineto{\pgfqpoint{1.151283in}{5.921037in}}%
\pgfpathlineto{\pgfqpoint{1.199606in}{6.008401in}}%
\pgfpathlineto{\pgfqpoint{1.249487in}{6.093972in}}%
\pgfpathlineto{\pgfqpoint{1.299368in}{6.175099in}}%
\pgfpathlineto{\pgfqpoint{1.349250in}{6.252012in}}%
\pgfpathlineto{\pgfqpoint{1.399131in}{6.324931in}}%
\pgfpathlineto{\pgfqpoint{1.450571in}{6.396164in}}%
\pgfpathlineto{\pgfqpoint{1.502011in}{6.463585in}}%
\pgfpathlineto{\pgfqpoint{1.553451in}{6.527398in}}%
\pgfpathlineto{\pgfqpoint{1.606450in}{6.589576in}}%
\pgfpathlineto{\pgfqpoint{1.659449in}{6.648327in}}%
\pgfpathlineto{\pgfqpoint{1.712448in}{6.703842in}}%
\pgfpathlineto{\pgfqpoint{1.767005in}{6.757797in}}%
\pgfpathlineto{\pgfqpoint{1.821563in}{6.808695in}}%
\pgfpathlineto{\pgfqpoint{1.877680in}{6.858039in}}%
\pgfpathlineto{\pgfqpoint{1.935355in}{6.905762in}}%
\pgfpathlineto{\pgfqpoint{1.993030in}{6.950630in}}%
\pgfpathlineto{\pgfqpoint{2.052264in}{6.993920in}}%
\pgfpathlineto{\pgfqpoint{2.113057in}{7.035587in}}%
\pgfpathlineto{\pgfqpoint{2.175409in}{7.075601in}}%
\pgfpathlineto{\pgfqpoint{2.239319in}{7.113938in}}%
\pgfpathlineto{\pgfqpoint{2.304788in}{7.150587in}}%
\pgfpathlineto{\pgfqpoint{2.371816in}{7.185542in}}%
\pgfpathlineto{\pgfqpoint{2.440403in}{7.218809in}}%
\pgfpathlineto{\pgfqpoint{2.512107in}{7.251076in}}%
\pgfpathlineto{\pgfqpoint{2.585370in}{7.281586in}}%
\pgfpathlineto{\pgfqpoint{2.661751in}{7.310950in}}%
\pgfpathlineto{\pgfqpoint{2.739691in}{7.338541in}}%
\pgfpathlineto{\pgfqpoint{2.820748in}{7.364897in}}%
\pgfpathlineto{\pgfqpoint{2.904922in}{7.389954in}}%
\pgfpathlineto{\pgfqpoint{2.993773in}{7.414067in}}%
\pgfpathlineto{\pgfqpoint{3.085742in}{7.436727in}}%
\pgfpathlineto{\pgfqpoint{3.182387in}{7.458256in}}%
\pgfpathlineto{\pgfqpoint{3.283709in}{7.478561in}}%
\pgfpathlineto{\pgfqpoint{3.391265in}{7.497839in}}%
\pgfpathlineto{\pgfqpoint{3.505057in}{7.515960in}}%
\pgfpathlineto{\pgfqpoint{3.626642in}{7.533034in}}%
\pgfpathlineto{\pgfqpoint{3.756022in}{7.548923in}}%
\pgfpathlineto{\pgfqpoint{3.894754in}{7.563688in}}%
\pgfpathlineto{\pgfqpoint{4.044398in}{7.577340in}}%
\pgfpathlineto{\pgfqpoint{4.206512in}{7.589862in}}%
\pgfpathlineto{\pgfqpoint{4.384215in}{7.601310in}}%
\pgfpathlineto{\pgfqpoint{4.579063in}{7.611600in}}%
\pgfpathlineto{\pgfqpoint{4.797294in}{7.620845in}}%
\pgfpathlineto{\pgfqpoint{5.043583in}{7.628989in}}%
\pgfpathlineto{\pgfqpoint{5.255578in}{7.634472in}}%
\pgfpathlineto{\pgfqpoint{5.255578in}{7.634472in}}%
\pgfusepath{stroke}%
\end{pgfscope}%
\begin{pgfscope}%
\pgfpathrectangle{\pgfqpoint{0.580766in}{4.462900in}}{\pgfqpoint{5.611646in}{3.193122in}} %
\pgfusepath{clip}%
\pgfsetrectcap%
\pgfsetroundjoin%
\pgfsetlinewidth{1.003750pt}%
\definecolor{currentstroke}{rgb}{0.000000,0.500000,0.000000}%
\pgfsetstrokecolor{currentstroke}%
\pgfsetdash{}{0pt}%
\pgfpathmoveto{\pgfqpoint{0.580766in}{4.618662in}}%
\pgfpathlineto{\pgfqpoint{0.627530in}{4.766826in}}%
\pgfpathlineto{\pgfqpoint{0.674293in}{4.907763in}}%
\pgfpathlineto{\pgfqpoint{0.721057in}{5.041824in}}%
\pgfpathlineto{\pgfqpoint{0.767821in}{5.169346in}}%
\pgfpathlineto{\pgfqpoint{0.814584in}{5.290647in}}%
\pgfpathlineto{\pgfqpoint{0.861348in}{5.406031in}}%
\pgfpathlineto{\pgfqpoint{0.908112in}{5.515787in}}%
\pgfpathlineto{\pgfqpoint{0.954876in}{5.620189in}}%
\pgfpathlineto{\pgfqpoint{1.001639in}{5.719498in}}%
\pgfpathlineto{\pgfqpoint{1.048403in}{5.813962in}}%
\pgfpathlineto{\pgfqpoint{1.095167in}{5.903819in}}%
\pgfpathlineto{\pgfqpoint{1.141930in}{5.989292in}}%
\pgfpathlineto{\pgfqpoint{1.188694in}{6.070596in}}%
\pgfpathlineto{\pgfqpoint{1.235458in}{6.147934in}}%
\pgfpathlineto{\pgfqpoint{1.282222in}{6.221499in}}%
\pgfpathlineto{\pgfqpoint{1.328985in}{6.291476in}}%
\pgfpathlineto{\pgfqpoint{1.375749in}{6.358039in}}%
\pgfpathlineto{\pgfqpoint{1.422513in}{6.421356in}}%
\pgfpathlineto{\pgfqpoint{1.469276in}{6.481583in}}%
\pgfpathlineto{\pgfqpoint{1.516040in}{6.538873in}}%
\pgfpathlineto{\pgfqpoint{1.562804in}{6.593368in}}%
\pgfpathlineto{\pgfqpoint{1.609568in}{6.645205in}}%
\pgfpathlineto{\pgfqpoint{1.656331in}{6.694513in}}%
\pgfpathlineto{\pgfqpoint{1.703095in}{6.741416in}}%
\pgfpathlineto{\pgfqpoint{1.749859in}{6.786031in}}%
\pgfpathlineto{\pgfqpoint{1.796622in}{6.828470in}}%
\pgfpathlineto{\pgfqpoint{1.843386in}{6.868838in}}%
\pgfpathlineto{\pgfqpoint{1.890150in}{6.907237in}}%
\pgfpathlineto{\pgfqpoint{1.936914in}{6.943763in}}%
\pgfpathlineto{\pgfqpoint{1.983677in}{6.978508in}}%
\pgfpathlineto{\pgfqpoint{2.030441in}{7.011557in}}%
\pgfpathlineto{\pgfqpoint{2.077205in}{7.042995in}}%
\pgfpathlineto{\pgfqpoint{2.123968in}{7.072898in}}%
\pgfpathlineto{\pgfqpoint{2.170732in}{7.101343in}}%
\pgfpathlineto{\pgfqpoint{2.217496in}{7.128401in}}%
\pgfpathlineto{\pgfqpoint{2.264260in}{7.154139in}}%
\pgfpathlineto{\pgfqpoint{2.311023in}{7.178621in}}%
\pgfpathlineto{\pgfqpoint{2.357787in}{7.201909in}}%
\pgfpathlineto{\pgfqpoint{2.404551in}{7.224061in}}%
\pgfpathlineto{\pgfqpoint{2.451314in}{7.245132in}}%
\pgfpathlineto{\pgfqpoint{2.498078in}{7.265175in}}%
\pgfpathlineto{\pgfqpoint{2.544842in}{7.284241in}}%
\pgfpathlineto{\pgfqpoint{2.591606in}{7.302377in}}%
\pgfpathlineto{\pgfqpoint{2.638369in}{7.319628in}}%
\pgfpathlineto{\pgfqpoint{2.685133in}{7.336037in}}%
\pgfpathlineto{\pgfqpoint{2.731897in}{7.351646in}}%
\pgfpathlineto{\pgfqpoint{2.778660in}{7.366494in}}%
\pgfpathlineto{\pgfqpoint{2.825424in}{7.380617in}}%
\pgfpathlineto{\pgfqpoint{2.872188in}{7.394052in}}%
\pgfpathlineto{\pgfqpoint{2.918952in}{7.406831in}}%
\pgfpathlineto{\pgfqpoint{2.965715in}{7.418986in}}%
\pgfpathlineto{\pgfqpoint{3.012479in}{7.430549in}}%
\pgfpathlineto{\pgfqpoint{3.059243in}{7.441548in}}%
\pgfpathlineto{\pgfqpoint{3.106006in}{7.452010in}}%
\pgfpathlineto{\pgfqpoint{3.152770in}{7.461962in}}%
\pgfpathlineto{\pgfqpoint{3.199534in}{7.471428in}}%
\pgfpathlineto{\pgfqpoint{3.246298in}{7.480433in}}%
\pgfpathlineto{\pgfqpoint{3.293061in}{7.488998in}}%
\pgfpathlineto{\pgfqpoint{3.339825in}{7.497146in}}%
\pgfpathlineto{\pgfqpoint{3.386589in}{7.504896in}}%
\pgfpathlineto{\pgfqpoint{3.433352in}{7.512268in}}%
\pgfpathlineto{\pgfqpoint{3.480116in}{7.519280in}}%
\pgfpathlineto{\pgfqpoint{3.526880in}{7.525951in}}%
\pgfpathlineto{\pgfqpoint{3.573644in}{7.532296in}}%
\pgfpathlineto{\pgfqpoint{3.620407in}{7.538331in}}%
\pgfpathlineto{\pgfqpoint{3.667171in}{7.544072in}}%
\pgfpathlineto{\pgfqpoint{3.713935in}{7.549533in}}%
\pgfpathlineto{\pgfqpoint{3.760698in}{7.554728in}}%
\pgfpathlineto{\pgfqpoint{3.807462in}{7.559669in}}%
\pgfpathlineto{\pgfqpoint{3.854226in}{7.564369in}}%
\pgfpathlineto{\pgfqpoint{3.900990in}{7.568840in}}%
\pgfpathlineto{\pgfqpoint{3.947753in}{7.573093in}}%
\pgfpathlineto{\pgfqpoint{3.994517in}{7.577138in}}%
\pgfpathlineto{\pgfqpoint{4.041281in}{7.580986in}}%
\pgfpathlineto{\pgfqpoint{4.088044in}{7.584646in}}%
\pgfpathlineto{\pgfqpoint{4.134808in}{7.588128in}}%
\pgfpathlineto{\pgfqpoint{4.181572in}{7.591440in}}%
\pgfpathlineto{\pgfqpoint{4.228336in}{7.594591in}}%
\pgfpathlineto{\pgfqpoint{4.275099in}{7.597587in}}%
\pgfpathlineto{\pgfqpoint{4.321863in}{7.600438in}}%
\pgfpathlineto{\pgfqpoint{4.368627in}{7.603149in}}%
\pgfpathlineto{\pgfqpoint{4.415390in}{7.605728in}}%
\pgfpathlineto{\pgfqpoint{4.462154in}{7.608182in}}%
\pgfpathlineto{\pgfqpoint{4.508918in}{7.610515in}}%
\pgfpathlineto{\pgfqpoint{4.555682in}{7.612735in}}%
\pgfpathlineto{\pgfqpoint{4.602445in}{7.614847in}}%
\pgfpathlineto{\pgfqpoint{4.649209in}{7.616855in}}%
\pgfpathlineto{\pgfqpoint{4.695973in}{7.618766in}}%
\pgfpathlineto{\pgfqpoint{4.742736in}{7.620583in}}%
\pgfpathlineto{\pgfqpoint{4.789500in}{7.622312in}}%
\pgfpathlineto{\pgfqpoint{4.836264in}{7.623957in}}%
\pgfpathlineto{\pgfqpoint{4.883027in}{7.625521in}}%
\pgfpathlineto{\pgfqpoint{4.929791in}{7.627009in}}%
\pgfpathlineto{\pgfqpoint{4.976555in}{7.628424in}}%
\pgfpathlineto{\pgfqpoint{5.023319in}{7.629770in}}%
\pgfpathlineto{\pgfqpoint{5.070082in}{7.631051in}}%
\pgfpathlineto{\pgfqpoint{5.116846in}{7.632269in}}%
\pgfpathlineto{\pgfqpoint{5.163610in}{7.633428in}}%
\pgfpathlineto{\pgfqpoint{5.210373in}{7.634530in}}%
\pgfusepath{stroke}%
\end{pgfscope}%
\begin{pgfscope}%
\pgfpathrectangle{\pgfqpoint{0.580766in}{4.462900in}}{\pgfqpoint{5.611646in}{3.193122in}} %
\pgfusepath{clip}%
\pgfsetrectcap%
\pgfsetroundjoin%
\pgfsetlinewidth{1.003750pt}%
\definecolor{currentstroke}{rgb}{1.000000,0.000000,0.000000}%
\pgfsetstrokecolor{currentstroke}%
\pgfsetdash{}{0pt}%
\pgfpathmoveto{\pgfqpoint{0.580766in}{4.614954in}}%
\pgfpathlineto{\pgfqpoint{0.627530in}{4.759766in}}%
\pgfpathlineto{\pgfqpoint{0.674293in}{4.897683in}}%
\pgfpathlineto{\pgfqpoint{0.721057in}{5.029033in}}%
\pgfpathlineto{\pgfqpoint{0.767821in}{5.154128in}}%
\pgfpathlineto{\pgfqpoint{0.814584in}{5.273265in}}%
\pgfpathlineto{\pgfqpoint{0.861348in}{5.386730in}}%
\pgfpathlineto{\pgfqpoint{0.908112in}{5.494792in}}%
\pgfpathlineto{\pgfqpoint{0.954876in}{5.597707in}}%
\pgfpathlineto{\pgfqpoint{1.001639in}{5.695722in}}%
\pgfpathlineto{\pgfqpoint{1.048403in}{5.789070in}}%
\pgfpathlineto{\pgfqpoint{1.095167in}{5.877973in}}%
\pgfpathlineto{\pgfqpoint{1.141930in}{5.962642in}}%
\pgfpathlineto{\pgfqpoint{1.188694in}{6.043279in}}%
\pgfpathlineto{\pgfqpoint{1.235458in}{6.120076in}}%
\pgfpathlineto{\pgfqpoint{1.282222in}{6.193216in}}%
\pgfpathlineto{\pgfqpoint{1.328985in}{6.262874in}}%
\pgfpathlineto{\pgfqpoint{1.375749in}{6.329214in}}%
\pgfpathlineto{\pgfqpoint{1.422513in}{6.392396in}}%
\pgfpathlineto{\pgfqpoint{1.469276in}{6.452568in}}%
\pgfpathlineto{\pgfqpoint{1.516040in}{6.509876in}}%
\pgfpathlineto{\pgfqpoint{1.562804in}{6.564454in}}%
\pgfpathlineto{\pgfqpoint{1.609568in}{6.616434in}}%
\pgfpathlineto{\pgfqpoint{1.656331in}{6.665938in}}%
\pgfpathlineto{\pgfqpoint{1.703095in}{6.713085in}}%
\pgfpathlineto{\pgfqpoint{1.749859in}{6.757986in}}%
\pgfpathlineto{\pgfqpoint{1.796622in}{6.800750in}}%
\pgfpathlineto{\pgfqpoint{1.843386in}{6.841477in}}%
\pgfpathlineto{\pgfqpoint{1.890150in}{6.880265in}}%
\pgfpathlineto{\pgfqpoint{1.936914in}{6.917206in}}%
\pgfpathlineto{\pgfqpoint{1.983677in}{6.952388in}}%
\pgfpathlineto{\pgfqpoint{2.030441in}{6.985894in}}%
\pgfpathlineto{\pgfqpoint{2.077205in}{7.017805in}}%
\pgfpathlineto{\pgfqpoint{2.123968in}{7.048196in}}%
\pgfpathlineto{\pgfqpoint{2.170732in}{7.077140in}}%
\pgfpathlineto{\pgfqpoint{2.217496in}{7.104706in}}%
\pgfpathlineto{\pgfqpoint{2.264260in}{7.130959in}}%
\pgfpathlineto{\pgfqpoint{2.311023in}{7.155962in}}%
\pgfpathlineto{\pgfqpoint{2.357787in}{7.179775in}}%
\pgfpathlineto{\pgfqpoint{2.404551in}{7.202453in}}%
\pgfpathlineto{\pgfqpoint{2.451314in}{7.224052in}}%
\pgfpathlineto{\pgfqpoint{2.498078in}{7.244622in}}%
\pgfpathlineto{\pgfqpoint{2.544842in}{7.264212in}}%
\pgfpathlineto{\pgfqpoint{2.591606in}{7.282870in}}%
\pgfpathlineto{\pgfqpoint{2.638369in}{7.300639in}}%
\pgfpathlineto{\pgfqpoint{2.685133in}{7.317562in}}%
\pgfpathlineto{\pgfqpoint{2.731897in}{7.333679in}}%
\pgfpathlineto{\pgfqpoint{2.778660in}{7.349029in}}%
\pgfpathlineto{\pgfqpoint{2.825424in}{7.363648in}}%
\pgfpathlineto{\pgfqpoint{2.872188in}{7.377570in}}%
\pgfpathlineto{\pgfqpoint{2.918952in}{7.390830in}}%
\pgfpathlineto{\pgfqpoint{2.965715in}{7.403458in}}%
\pgfpathlineto{\pgfqpoint{3.012479in}{7.415485in}}%
\pgfpathlineto{\pgfqpoint{3.059243in}{7.426939in}}%
\pgfpathlineto{\pgfqpoint{3.106006in}{7.437848in}}%
\pgfpathlineto{\pgfqpoint{3.152770in}{7.448237in}}%
\pgfpathlineto{\pgfqpoint{3.199534in}{7.458132in}}%
\pgfpathlineto{\pgfqpoint{3.246298in}{7.467555in}}%
\pgfpathlineto{\pgfqpoint{3.293061in}{7.476530in}}%
\pgfpathlineto{\pgfqpoint{3.339825in}{7.485077in}}%
\pgfpathlineto{\pgfqpoint{3.386589in}{7.493217in}}%
\pgfpathlineto{\pgfqpoint{3.433352in}{7.500970in}}%
\pgfpathlineto{\pgfqpoint{3.480116in}{7.508353in}}%
\pgfpathlineto{\pgfqpoint{3.526880in}{7.515385in}}%
\pgfpathlineto{\pgfqpoint{3.573644in}{7.522082in}}%
\pgfpathlineto{\pgfqpoint{3.620407in}{7.528460in}}%
\pgfpathlineto{\pgfqpoint{3.667171in}{7.534535in}}%
\pgfpathlineto{\pgfqpoint{3.713935in}{7.540320in}}%
\pgfpathlineto{\pgfqpoint{3.760698in}{7.545830in}}%
\pgfpathlineto{\pgfqpoint{3.807462in}{7.551077in}}%
\pgfpathlineto{\pgfqpoint{3.854226in}{7.556074in}}%
\pgfpathlineto{\pgfqpoint{3.900990in}{7.560834in}}%
\pgfpathlineto{\pgfqpoint{3.947753in}{7.565367in}}%
\pgfpathlineto{\pgfqpoint{3.994517in}{7.569683in}}%
\pgfpathlineto{\pgfqpoint{4.041281in}{7.573795in}}%
\pgfpathlineto{\pgfqpoint{4.088044in}{7.577710in}}%
\pgfpathlineto{\pgfqpoint{4.134808in}{7.581440in}}%
\pgfpathlineto{\pgfqpoint{4.181572in}{7.584991in}}%
\pgfpathlineto{\pgfqpoint{4.228336in}{7.588374in}}%
\pgfpathlineto{\pgfqpoint{4.275099in}{7.591595in}}%
\pgfpathlineto{\pgfqpoint{4.321863in}{7.594663in}}%
\pgfpathlineto{\pgfqpoint{4.368627in}{7.597585in}}%
\pgfpathlineto{\pgfqpoint{4.415390in}{7.600368in}}%
\pgfpathlineto{\pgfqpoint{4.462154in}{7.603018in}}%
\pgfpathlineto{\pgfqpoint{4.508918in}{7.605542in}}%
\pgfpathlineto{\pgfqpoint{4.555682in}{7.607946in}}%
\pgfpathlineto{\pgfqpoint{4.602445in}{7.610235in}}%
\pgfpathlineto{\pgfqpoint{4.649209in}{7.612415in}}%
\pgfpathlineto{\pgfqpoint{4.695973in}{7.614492in}}%
\pgfpathlineto{\pgfqpoint{4.742736in}{7.616470in}}%
\pgfpathlineto{\pgfqpoint{4.789500in}{7.618353in}}%
\pgfpathlineto{\pgfqpoint{4.836264in}{7.620147in}}%
\pgfpathlineto{\pgfqpoint{4.883027in}{7.621855in}}%
\pgfpathlineto{\pgfqpoint{4.929791in}{7.623482in}}%
\pgfpathlineto{\pgfqpoint{4.976555in}{7.625032in}}%
\pgfpathlineto{\pgfqpoint{5.023319in}{7.626508in}}%
\pgfpathlineto{\pgfqpoint{5.070082in}{7.627913in}}%
\pgfpathlineto{\pgfqpoint{5.116846in}{7.629252in}}%
\pgfpathlineto{\pgfqpoint{5.163610in}{7.630526in}}%
\pgfpathlineto{\pgfqpoint{5.210373in}{7.631740in}}%
\pgfusepath{stroke}%
\end{pgfscope}%
\begin{pgfscope}%
\pgfpathrectangle{\pgfqpoint{0.580766in}{4.462900in}}{\pgfqpoint{5.611646in}{3.193122in}} %
\pgfusepath{clip}%
\pgfsetrectcap%
\pgfsetroundjoin%
\pgfsetlinewidth{1.003750pt}%
\definecolor{currentstroke}{rgb}{0.000000,0.750000,0.750000}%
\pgfsetstrokecolor{currentstroke}%
\pgfsetdash{}{0pt}%
\pgfpathmoveto{\pgfqpoint{0.580766in}{4.578772in}}%
\pgfpathlineto{\pgfqpoint{0.627530in}{4.728882in}}%
\pgfpathlineto{\pgfqpoint{0.674293in}{4.871669in}}%
\pgfpathlineto{\pgfqpoint{0.721057in}{5.007491in}}%
\pgfpathlineto{\pgfqpoint{0.767821in}{5.136688in}}%
\pgfpathlineto{\pgfqpoint{0.814584in}{5.259582in}}%
\pgfpathlineto{\pgfqpoint{0.861348in}{5.376482in}}%
\pgfpathlineto{\pgfqpoint{0.908112in}{5.487679in}}%
\pgfpathlineto{\pgfqpoint{0.954876in}{5.593452in}}%
\pgfpathlineto{\pgfqpoint{1.001639in}{5.694065in}}%
\pgfpathlineto{\pgfqpoint{1.048403in}{5.789770in}}%
\pgfpathlineto{\pgfqpoint{1.095167in}{5.880807in}}%
\pgfpathlineto{\pgfqpoint{1.141930in}{5.967403in}}%
\pgfpathlineto{\pgfqpoint{1.188694in}{6.049774in}}%
\pgfpathlineto{\pgfqpoint{1.235458in}{6.128128in}}%
\pgfpathlineto{\pgfqpoint{1.282222in}{6.202659in}}%
\pgfpathlineto{\pgfqpoint{1.328985in}{6.273555in}}%
\pgfpathlineto{\pgfqpoint{1.375749in}{6.340993in}}%
\pgfpathlineto{\pgfqpoint{1.422513in}{6.405140in}}%
\pgfpathlineto{\pgfqpoint{1.469276in}{6.466159in}}%
\pgfpathlineto{\pgfqpoint{1.516040in}{6.524201in}}%
\pgfpathlineto{\pgfqpoint{1.562804in}{6.579412in}}%
\pgfpathlineto{\pgfqpoint{1.609568in}{6.631930in}}%
\pgfpathlineto{\pgfqpoint{1.656331in}{6.681885in}}%
\pgfpathlineto{\pgfqpoint{1.703095in}{6.729404in}}%
\pgfpathlineto{\pgfqpoint{1.749859in}{6.774605in}}%
\pgfpathlineto{\pgfqpoint{1.796622in}{6.817601in}}%
\pgfpathlineto{\pgfqpoint{1.843386in}{6.858500in}}%
\pgfpathlineto{\pgfqpoint{1.890150in}{6.897403in}}%
\pgfpathlineto{\pgfqpoint{1.936914in}{6.934409in}}%
\pgfpathlineto{\pgfqpoint{1.983677in}{6.969610in}}%
\pgfpathlineto{\pgfqpoint{2.030441in}{7.003093in}}%
\pgfpathlineto{\pgfqpoint{2.077205in}{7.034944in}}%
\pgfpathlineto{\pgfqpoint{2.123968in}{7.065240in}}%
\pgfpathlineto{\pgfqpoint{2.170732in}{7.094059in}}%
\pgfpathlineto{\pgfqpoint{2.217496in}{7.121472in}}%
\pgfpathlineto{\pgfqpoint{2.264260in}{7.147547in}}%
\pgfpathlineto{\pgfqpoint{2.311023in}{7.172351in}}%
\pgfpathlineto{\pgfqpoint{2.357787in}{7.195945in}}%
\pgfpathlineto{\pgfqpoint{2.404551in}{7.218387in}}%
\pgfpathlineto{\pgfqpoint{2.451314in}{7.239736in}}%
\pgfpathlineto{\pgfqpoint{2.498078in}{7.260042in}}%
\pgfpathlineto{\pgfqpoint{2.544842in}{7.279358in}}%
\pgfpathlineto{\pgfqpoint{2.591606in}{7.297732in}}%
\pgfpathlineto{\pgfqpoint{2.638369in}{7.315210in}}%
\pgfpathlineto{\pgfqpoint{2.685133in}{7.331835in}}%
\pgfpathlineto{\pgfqpoint{2.731897in}{7.347649in}}%
\pgfpathlineto{\pgfqpoint{2.778660in}{7.362691in}}%
\pgfpathlineto{\pgfqpoint{2.825424in}{7.377000in}}%
\pgfpathlineto{\pgfqpoint{2.872188in}{7.390611in}}%
\pgfpathlineto{\pgfqpoint{2.918952in}{7.403558in}}%
\pgfpathlineto{\pgfqpoint{2.965715in}{7.415873in}}%
\pgfpathlineto{\pgfqpoint{3.012479in}{7.427588in}}%
\pgfpathlineto{\pgfqpoint{3.059243in}{7.438731in}}%
\pgfpathlineto{\pgfqpoint{3.106006in}{7.449331in}}%
\pgfpathlineto{\pgfqpoint{3.152770in}{7.459413in}}%
\pgfpathlineto{\pgfqpoint{3.199534in}{7.469004in}}%
\pgfpathlineto{\pgfqpoint{3.246298in}{7.478127in}}%
\pgfpathlineto{\pgfqpoint{3.293061in}{7.486805in}}%
\pgfpathlineto{\pgfqpoint{3.339825in}{7.495059in}}%
\pgfpathlineto{\pgfqpoint{3.386589in}{7.502911in}}%
\pgfpathlineto{\pgfqpoint{3.433352in}{7.510380in}}%
\pgfpathlineto{\pgfqpoint{3.480116in}{7.517484in}}%
\pgfpathlineto{\pgfqpoint{3.526880in}{7.524242in}}%
\pgfpathlineto{\pgfqpoint{3.573644in}{7.530671in}}%
\pgfpathlineto{\pgfqpoint{3.620407in}{7.536785in}}%
\pgfpathlineto{\pgfqpoint{3.667171in}{7.542602in}}%
\pgfpathlineto{\pgfqpoint{3.713935in}{7.548135in}}%
\pgfpathlineto{\pgfqpoint{3.760698in}{7.553397in}}%
\pgfpathlineto{\pgfqpoint{3.807462in}{7.558403in}}%
\pgfpathlineto{\pgfqpoint{3.854226in}{7.563165in}}%
\pgfpathlineto{\pgfqpoint{3.900990in}{7.567695in}}%
\pgfpathlineto{\pgfqpoint{3.947753in}{7.572004in}}%
\pgfpathlineto{\pgfqpoint{3.994517in}{7.576102in}}%
\pgfpathlineto{\pgfqpoint{4.041281in}{7.580001in}}%
\pgfpathlineto{\pgfqpoint{4.088044in}{7.583709in}}%
\pgfpathlineto{\pgfqpoint{4.134808in}{7.587237in}}%
\pgfpathlineto{\pgfqpoint{4.181572in}{7.590592in}}%
\pgfpathlineto{\pgfqpoint{4.228336in}{7.593784in}}%
\pgfpathlineto{\pgfqpoint{4.275099in}{7.596820in}}%
\pgfpathlineto{\pgfqpoint{4.321863in}{7.599708in}}%
\pgfpathlineto{\pgfqpoint{4.368627in}{7.602455in}}%
\pgfpathlineto{\pgfqpoint{4.415390in}{7.605068in}}%
\pgfpathlineto{\pgfqpoint{4.462154in}{7.607553in}}%
\pgfpathlineto{\pgfqpoint{4.508918in}{7.609918in}}%
\pgfpathlineto{\pgfqpoint{4.555682in}{7.612167in}}%
\pgfpathlineto{\pgfqpoint{4.602445in}{7.614306in}}%
\pgfpathlineto{\pgfqpoint{4.649209in}{7.616341in}}%
\pgfpathlineto{\pgfqpoint{4.695973in}{7.618277in}}%
\pgfpathlineto{\pgfqpoint{4.742736in}{7.620118in}}%
\pgfpathlineto{\pgfqpoint{4.789500in}{7.621869in}}%
\pgfpathlineto{\pgfqpoint{4.836264in}{7.623535in}}%
\pgfpathlineto{\pgfqpoint{4.883027in}{7.625120in}}%
\pgfpathlineto{\pgfqpoint{4.929791in}{7.626628in}}%
\pgfpathlineto{\pgfqpoint{4.976555in}{7.628062in}}%
\pgfpathlineto{\pgfqpoint{5.023319in}{7.629425in}}%
\pgfpathlineto{\pgfqpoint{5.070082in}{7.630723in}}%
\pgfpathlineto{\pgfqpoint{5.116846in}{7.631957in}}%
\pgfpathlineto{\pgfqpoint{5.163610in}{7.633131in}}%
\pgfpathlineto{\pgfqpoint{5.210373in}{7.634248in}}%
\pgfpathlineto{\pgfqpoint{5.257137in}{7.635310in}}%
\pgfpathlineto{\pgfqpoint{5.303901in}{7.635310in}}%
\pgfusepath{stroke}%
\end{pgfscope}%
\begin{pgfscope}%
\pgfpathrectangle{\pgfqpoint{0.580766in}{4.462900in}}{\pgfqpoint{5.611646in}{3.193122in}} %
\pgfusepath{clip}%
\pgfsetbuttcap%
\pgfsetroundjoin%
\pgfsetlinewidth{0.501875pt}%
\definecolor{currentstroke}{rgb}{0.000000,0.000000,0.000000}%
\pgfsetstrokecolor{currentstroke}%
\pgfsetdash{{1.000000pt}{3.000000pt}}{0.000000pt}%
\pgfpathmoveto{\pgfqpoint{0.580766in}{4.462900in}}%
\pgfpathlineto{\pgfqpoint{0.580766in}{7.656023in}}%
\pgfusepath{stroke}%
\end{pgfscope}%
\begin{pgfscope}%
\pgfsetbuttcap%
\pgfsetroundjoin%
\definecolor{currentfill}{rgb}{0.000000,0.000000,0.000000}%
\pgfsetfillcolor{currentfill}%
\pgfsetlinewidth{0.501875pt}%
\definecolor{currentstroke}{rgb}{0.000000,0.000000,0.000000}%
\pgfsetstrokecolor{currentstroke}%
\pgfsetdash{}{0pt}%
\pgfsys@defobject{currentmarker}{\pgfqpoint{0.000000in}{0.000000in}}{\pgfqpoint{0.000000in}{0.055556in}}{%
\pgfpathmoveto{\pgfqpoint{0.000000in}{0.000000in}}%
\pgfpathlineto{\pgfqpoint{0.000000in}{0.055556in}}%
\pgfusepath{stroke,fill}%
}%
\begin{pgfscope}%
\pgfsys@transformshift{0.580766in}{4.462900in}%
\pgfsys@useobject{currentmarker}{}%
\end{pgfscope}%
\end{pgfscope}%
\begin{pgfscope}%
\pgfsetbuttcap%
\pgfsetroundjoin%
\definecolor{currentfill}{rgb}{0.000000,0.000000,0.000000}%
\pgfsetfillcolor{currentfill}%
\pgfsetlinewidth{0.501875pt}%
\definecolor{currentstroke}{rgb}{0.000000,0.000000,0.000000}%
\pgfsetstrokecolor{currentstroke}%
\pgfsetdash{}{0pt}%
\pgfsys@defobject{currentmarker}{\pgfqpoint{0.000000in}{-0.055556in}}{\pgfqpoint{0.000000in}{0.000000in}}{%
\pgfpathmoveto{\pgfqpoint{0.000000in}{0.000000in}}%
\pgfpathlineto{\pgfqpoint{0.000000in}{-0.055556in}}%
\pgfusepath{stroke,fill}%
}%
\begin{pgfscope}%
\pgfsys@transformshift{0.580766in}{7.656023in}%
\pgfsys@useobject{currentmarker}{}%
\end{pgfscope}%
\end{pgfscope}%
\begin{pgfscope}%
\pgftext[x=0.580766in,y=4.407345in,,top]{{\rmfamily\fontsize{10.000000}{12.000000}\selectfont \(\displaystyle 0.000\)}}%
\end{pgfscope}%
\begin{pgfscope}%
\pgfpathrectangle{\pgfqpoint{0.580766in}{4.462900in}}{\pgfqpoint{5.611646in}{3.193122in}} %
\pgfusepath{clip}%
\pgfsetbuttcap%
\pgfsetroundjoin%
\pgfsetlinewidth{0.501875pt}%
\definecolor{currentstroke}{rgb}{0.000000,0.000000,0.000000}%
\pgfsetstrokecolor{currentstroke}%
\pgfsetdash{{1.000000pt}{3.000000pt}}{0.000000pt}%
\pgfpathmoveto{\pgfqpoint{1.516040in}{4.462900in}}%
\pgfpathlineto{\pgfqpoint{1.516040in}{7.656023in}}%
\pgfusepath{stroke}%
\end{pgfscope}%
\begin{pgfscope}%
\pgfsetbuttcap%
\pgfsetroundjoin%
\definecolor{currentfill}{rgb}{0.000000,0.000000,0.000000}%
\pgfsetfillcolor{currentfill}%
\pgfsetlinewidth{0.501875pt}%
\definecolor{currentstroke}{rgb}{0.000000,0.000000,0.000000}%
\pgfsetstrokecolor{currentstroke}%
\pgfsetdash{}{0pt}%
\pgfsys@defobject{currentmarker}{\pgfqpoint{0.000000in}{0.000000in}}{\pgfqpoint{0.000000in}{0.055556in}}{%
\pgfpathmoveto{\pgfqpoint{0.000000in}{0.000000in}}%
\pgfpathlineto{\pgfqpoint{0.000000in}{0.055556in}}%
\pgfusepath{stroke,fill}%
}%
\begin{pgfscope}%
\pgfsys@transformshift{1.516040in}{4.462900in}%
\pgfsys@useobject{currentmarker}{}%
\end{pgfscope}%
\end{pgfscope}%
\begin{pgfscope}%
\pgfsetbuttcap%
\pgfsetroundjoin%
\definecolor{currentfill}{rgb}{0.000000,0.000000,0.000000}%
\pgfsetfillcolor{currentfill}%
\pgfsetlinewidth{0.501875pt}%
\definecolor{currentstroke}{rgb}{0.000000,0.000000,0.000000}%
\pgfsetstrokecolor{currentstroke}%
\pgfsetdash{}{0pt}%
\pgfsys@defobject{currentmarker}{\pgfqpoint{0.000000in}{-0.055556in}}{\pgfqpoint{0.000000in}{0.000000in}}{%
\pgfpathmoveto{\pgfqpoint{0.000000in}{0.000000in}}%
\pgfpathlineto{\pgfqpoint{0.000000in}{-0.055556in}}%
\pgfusepath{stroke,fill}%
}%
\begin{pgfscope}%
\pgfsys@transformshift{1.516040in}{7.656023in}%
\pgfsys@useobject{currentmarker}{}%
\end{pgfscope}%
\end{pgfscope}%
\begin{pgfscope}%
\pgftext[x=1.516040in,y=4.407345in,,top]{{\rmfamily\fontsize{10.000000}{12.000000}\selectfont \(\displaystyle 0.002\)}}%
\end{pgfscope}%
\begin{pgfscope}%
\pgfpathrectangle{\pgfqpoint{0.580766in}{4.462900in}}{\pgfqpoint{5.611646in}{3.193122in}} %
\pgfusepath{clip}%
\pgfsetbuttcap%
\pgfsetroundjoin%
\pgfsetlinewidth{0.501875pt}%
\definecolor{currentstroke}{rgb}{0.000000,0.000000,0.000000}%
\pgfsetstrokecolor{currentstroke}%
\pgfsetdash{{1.000000pt}{3.000000pt}}{0.000000pt}%
\pgfpathmoveto{\pgfqpoint{2.451314in}{4.462900in}}%
\pgfpathlineto{\pgfqpoint{2.451314in}{7.656023in}}%
\pgfusepath{stroke}%
\end{pgfscope}%
\begin{pgfscope}%
\pgfsetbuttcap%
\pgfsetroundjoin%
\definecolor{currentfill}{rgb}{0.000000,0.000000,0.000000}%
\pgfsetfillcolor{currentfill}%
\pgfsetlinewidth{0.501875pt}%
\definecolor{currentstroke}{rgb}{0.000000,0.000000,0.000000}%
\pgfsetstrokecolor{currentstroke}%
\pgfsetdash{}{0pt}%
\pgfsys@defobject{currentmarker}{\pgfqpoint{0.000000in}{0.000000in}}{\pgfqpoint{0.000000in}{0.055556in}}{%
\pgfpathmoveto{\pgfqpoint{0.000000in}{0.000000in}}%
\pgfpathlineto{\pgfqpoint{0.000000in}{0.055556in}}%
\pgfusepath{stroke,fill}%
}%
\begin{pgfscope}%
\pgfsys@transformshift{2.451314in}{4.462900in}%
\pgfsys@useobject{currentmarker}{}%
\end{pgfscope}%
\end{pgfscope}%
\begin{pgfscope}%
\pgfsetbuttcap%
\pgfsetroundjoin%
\definecolor{currentfill}{rgb}{0.000000,0.000000,0.000000}%
\pgfsetfillcolor{currentfill}%
\pgfsetlinewidth{0.501875pt}%
\definecolor{currentstroke}{rgb}{0.000000,0.000000,0.000000}%
\pgfsetstrokecolor{currentstroke}%
\pgfsetdash{}{0pt}%
\pgfsys@defobject{currentmarker}{\pgfqpoint{0.000000in}{-0.055556in}}{\pgfqpoint{0.000000in}{0.000000in}}{%
\pgfpathmoveto{\pgfqpoint{0.000000in}{0.000000in}}%
\pgfpathlineto{\pgfqpoint{0.000000in}{-0.055556in}}%
\pgfusepath{stroke,fill}%
}%
\begin{pgfscope}%
\pgfsys@transformshift{2.451314in}{7.656023in}%
\pgfsys@useobject{currentmarker}{}%
\end{pgfscope}%
\end{pgfscope}%
\begin{pgfscope}%
\pgftext[x=2.451314in,y=4.407345in,,top]{{\rmfamily\fontsize{10.000000}{12.000000}\selectfont \(\displaystyle 0.004\)}}%
\end{pgfscope}%
\begin{pgfscope}%
\pgfpathrectangle{\pgfqpoint{0.580766in}{4.462900in}}{\pgfqpoint{5.611646in}{3.193122in}} %
\pgfusepath{clip}%
\pgfsetbuttcap%
\pgfsetroundjoin%
\pgfsetlinewidth{0.501875pt}%
\definecolor{currentstroke}{rgb}{0.000000,0.000000,0.000000}%
\pgfsetstrokecolor{currentstroke}%
\pgfsetdash{{1.000000pt}{3.000000pt}}{0.000000pt}%
\pgfpathmoveto{\pgfqpoint{3.386589in}{4.462900in}}%
\pgfpathlineto{\pgfqpoint{3.386589in}{7.656023in}}%
\pgfusepath{stroke}%
\end{pgfscope}%
\begin{pgfscope}%
\pgfsetbuttcap%
\pgfsetroundjoin%
\definecolor{currentfill}{rgb}{0.000000,0.000000,0.000000}%
\pgfsetfillcolor{currentfill}%
\pgfsetlinewidth{0.501875pt}%
\definecolor{currentstroke}{rgb}{0.000000,0.000000,0.000000}%
\pgfsetstrokecolor{currentstroke}%
\pgfsetdash{}{0pt}%
\pgfsys@defobject{currentmarker}{\pgfqpoint{0.000000in}{0.000000in}}{\pgfqpoint{0.000000in}{0.055556in}}{%
\pgfpathmoveto{\pgfqpoint{0.000000in}{0.000000in}}%
\pgfpathlineto{\pgfqpoint{0.000000in}{0.055556in}}%
\pgfusepath{stroke,fill}%
}%
\begin{pgfscope}%
\pgfsys@transformshift{3.386589in}{4.462900in}%
\pgfsys@useobject{currentmarker}{}%
\end{pgfscope}%
\end{pgfscope}%
\begin{pgfscope}%
\pgfsetbuttcap%
\pgfsetroundjoin%
\definecolor{currentfill}{rgb}{0.000000,0.000000,0.000000}%
\pgfsetfillcolor{currentfill}%
\pgfsetlinewidth{0.501875pt}%
\definecolor{currentstroke}{rgb}{0.000000,0.000000,0.000000}%
\pgfsetstrokecolor{currentstroke}%
\pgfsetdash{}{0pt}%
\pgfsys@defobject{currentmarker}{\pgfqpoint{0.000000in}{-0.055556in}}{\pgfqpoint{0.000000in}{0.000000in}}{%
\pgfpathmoveto{\pgfqpoint{0.000000in}{0.000000in}}%
\pgfpathlineto{\pgfqpoint{0.000000in}{-0.055556in}}%
\pgfusepath{stroke,fill}%
}%
\begin{pgfscope}%
\pgfsys@transformshift{3.386589in}{7.656023in}%
\pgfsys@useobject{currentmarker}{}%
\end{pgfscope}%
\end{pgfscope}%
\begin{pgfscope}%
\pgftext[x=3.386589in,y=4.407345in,,top]{{\rmfamily\fontsize{10.000000}{12.000000}\selectfont \(\displaystyle 0.006\)}}%
\end{pgfscope}%
\begin{pgfscope}%
\pgfpathrectangle{\pgfqpoint{0.580766in}{4.462900in}}{\pgfqpoint{5.611646in}{3.193122in}} %
\pgfusepath{clip}%
\pgfsetbuttcap%
\pgfsetroundjoin%
\pgfsetlinewidth{0.501875pt}%
\definecolor{currentstroke}{rgb}{0.000000,0.000000,0.000000}%
\pgfsetstrokecolor{currentstroke}%
\pgfsetdash{{1.000000pt}{3.000000pt}}{0.000000pt}%
\pgfpathmoveto{\pgfqpoint{4.321863in}{4.462900in}}%
\pgfpathlineto{\pgfqpoint{4.321863in}{7.656023in}}%
\pgfusepath{stroke}%
\end{pgfscope}%
\begin{pgfscope}%
\pgfsetbuttcap%
\pgfsetroundjoin%
\definecolor{currentfill}{rgb}{0.000000,0.000000,0.000000}%
\pgfsetfillcolor{currentfill}%
\pgfsetlinewidth{0.501875pt}%
\definecolor{currentstroke}{rgb}{0.000000,0.000000,0.000000}%
\pgfsetstrokecolor{currentstroke}%
\pgfsetdash{}{0pt}%
\pgfsys@defobject{currentmarker}{\pgfqpoint{0.000000in}{0.000000in}}{\pgfqpoint{0.000000in}{0.055556in}}{%
\pgfpathmoveto{\pgfqpoint{0.000000in}{0.000000in}}%
\pgfpathlineto{\pgfqpoint{0.000000in}{0.055556in}}%
\pgfusepath{stroke,fill}%
}%
\begin{pgfscope}%
\pgfsys@transformshift{4.321863in}{4.462900in}%
\pgfsys@useobject{currentmarker}{}%
\end{pgfscope}%
\end{pgfscope}%
\begin{pgfscope}%
\pgfsetbuttcap%
\pgfsetroundjoin%
\definecolor{currentfill}{rgb}{0.000000,0.000000,0.000000}%
\pgfsetfillcolor{currentfill}%
\pgfsetlinewidth{0.501875pt}%
\definecolor{currentstroke}{rgb}{0.000000,0.000000,0.000000}%
\pgfsetstrokecolor{currentstroke}%
\pgfsetdash{}{0pt}%
\pgfsys@defobject{currentmarker}{\pgfqpoint{0.000000in}{-0.055556in}}{\pgfqpoint{0.000000in}{0.000000in}}{%
\pgfpathmoveto{\pgfqpoint{0.000000in}{0.000000in}}%
\pgfpathlineto{\pgfqpoint{0.000000in}{-0.055556in}}%
\pgfusepath{stroke,fill}%
}%
\begin{pgfscope}%
\pgfsys@transformshift{4.321863in}{7.656023in}%
\pgfsys@useobject{currentmarker}{}%
\end{pgfscope}%
\end{pgfscope}%
\begin{pgfscope}%
\pgftext[x=4.321863in,y=4.407345in,,top]{{\rmfamily\fontsize{10.000000}{12.000000}\selectfont \(\displaystyle 0.008\)}}%
\end{pgfscope}%
\begin{pgfscope}%
\pgfpathrectangle{\pgfqpoint{0.580766in}{4.462900in}}{\pgfqpoint{5.611646in}{3.193122in}} %
\pgfusepath{clip}%
\pgfsetbuttcap%
\pgfsetroundjoin%
\pgfsetlinewidth{0.501875pt}%
\definecolor{currentstroke}{rgb}{0.000000,0.000000,0.000000}%
\pgfsetstrokecolor{currentstroke}%
\pgfsetdash{{1.000000pt}{3.000000pt}}{0.000000pt}%
\pgfpathmoveto{\pgfqpoint{5.257137in}{4.462900in}}%
\pgfpathlineto{\pgfqpoint{5.257137in}{7.656023in}}%
\pgfusepath{stroke}%
\end{pgfscope}%
\begin{pgfscope}%
\pgfsetbuttcap%
\pgfsetroundjoin%
\definecolor{currentfill}{rgb}{0.000000,0.000000,0.000000}%
\pgfsetfillcolor{currentfill}%
\pgfsetlinewidth{0.501875pt}%
\definecolor{currentstroke}{rgb}{0.000000,0.000000,0.000000}%
\pgfsetstrokecolor{currentstroke}%
\pgfsetdash{}{0pt}%
\pgfsys@defobject{currentmarker}{\pgfqpoint{0.000000in}{0.000000in}}{\pgfqpoint{0.000000in}{0.055556in}}{%
\pgfpathmoveto{\pgfqpoint{0.000000in}{0.000000in}}%
\pgfpathlineto{\pgfqpoint{0.000000in}{0.055556in}}%
\pgfusepath{stroke,fill}%
}%
\begin{pgfscope}%
\pgfsys@transformshift{5.257137in}{4.462900in}%
\pgfsys@useobject{currentmarker}{}%
\end{pgfscope}%
\end{pgfscope}%
\begin{pgfscope}%
\pgfsetbuttcap%
\pgfsetroundjoin%
\definecolor{currentfill}{rgb}{0.000000,0.000000,0.000000}%
\pgfsetfillcolor{currentfill}%
\pgfsetlinewidth{0.501875pt}%
\definecolor{currentstroke}{rgb}{0.000000,0.000000,0.000000}%
\pgfsetstrokecolor{currentstroke}%
\pgfsetdash{}{0pt}%
\pgfsys@defobject{currentmarker}{\pgfqpoint{0.000000in}{-0.055556in}}{\pgfqpoint{0.000000in}{0.000000in}}{%
\pgfpathmoveto{\pgfqpoint{0.000000in}{0.000000in}}%
\pgfpathlineto{\pgfqpoint{0.000000in}{-0.055556in}}%
\pgfusepath{stroke,fill}%
}%
\begin{pgfscope}%
\pgfsys@transformshift{5.257137in}{7.656023in}%
\pgfsys@useobject{currentmarker}{}%
\end{pgfscope}%
\end{pgfscope}%
\begin{pgfscope}%
\pgftext[x=5.257137in,y=4.407345in,,top]{{\rmfamily\fontsize{10.000000}{12.000000}\selectfont \(\displaystyle 0.010\)}}%
\end{pgfscope}%
\begin{pgfscope}%
\pgfpathrectangle{\pgfqpoint{0.580766in}{4.462900in}}{\pgfqpoint{5.611646in}{3.193122in}} %
\pgfusepath{clip}%
\pgfsetbuttcap%
\pgfsetroundjoin%
\pgfsetlinewidth{0.501875pt}%
\definecolor{currentstroke}{rgb}{0.000000,0.000000,0.000000}%
\pgfsetstrokecolor{currentstroke}%
\pgfsetdash{{1.000000pt}{3.000000pt}}{0.000000pt}%
\pgfpathmoveto{\pgfqpoint{6.192411in}{4.462900in}}%
\pgfpathlineto{\pgfqpoint{6.192411in}{7.656023in}}%
\pgfusepath{stroke}%
\end{pgfscope}%
\begin{pgfscope}%
\pgfsetbuttcap%
\pgfsetroundjoin%
\definecolor{currentfill}{rgb}{0.000000,0.000000,0.000000}%
\pgfsetfillcolor{currentfill}%
\pgfsetlinewidth{0.501875pt}%
\definecolor{currentstroke}{rgb}{0.000000,0.000000,0.000000}%
\pgfsetstrokecolor{currentstroke}%
\pgfsetdash{}{0pt}%
\pgfsys@defobject{currentmarker}{\pgfqpoint{0.000000in}{0.000000in}}{\pgfqpoint{0.000000in}{0.055556in}}{%
\pgfpathmoveto{\pgfqpoint{0.000000in}{0.000000in}}%
\pgfpathlineto{\pgfqpoint{0.000000in}{0.055556in}}%
\pgfusepath{stroke,fill}%
}%
\begin{pgfscope}%
\pgfsys@transformshift{6.192411in}{4.462900in}%
\pgfsys@useobject{currentmarker}{}%
\end{pgfscope}%
\end{pgfscope}%
\begin{pgfscope}%
\pgfsetbuttcap%
\pgfsetroundjoin%
\definecolor{currentfill}{rgb}{0.000000,0.000000,0.000000}%
\pgfsetfillcolor{currentfill}%
\pgfsetlinewidth{0.501875pt}%
\definecolor{currentstroke}{rgb}{0.000000,0.000000,0.000000}%
\pgfsetstrokecolor{currentstroke}%
\pgfsetdash{}{0pt}%
\pgfsys@defobject{currentmarker}{\pgfqpoint{0.000000in}{-0.055556in}}{\pgfqpoint{0.000000in}{0.000000in}}{%
\pgfpathmoveto{\pgfqpoint{0.000000in}{0.000000in}}%
\pgfpathlineto{\pgfqpoint{0.000000in}{-0.055556in}}%
\pgfusepath{stroke,fill}%
}%
\begin{pgfscope}%
\pgfsys@transformshift{6.192411in}{7.656023in}%
\pgfsys@useobject{currentmarker}{}%
\end{pgfscope}%
\end{pgfscope}%
\begin{pgfscope}%
\pgftext[x=6.192411in,y=4.407345in,,top]{{\rmfamily\fontsize{10.000000}{12.000000}\selectfont \(\displaystyle 0.012\)}}%
\end{pgfscope}%
\begin{pgfscope}%
\pgftext[x=3.386589in,y=4.214443in,,top]{{\rmfamily\fontsize{10.000000}{12.000000}\selectfont time (s)}}%
\end{pgfscope}%
\begin{pgfscope}%
\pgfpathrectangle{\pgfqpoint{0.580766in}{4.462900in}}{\pgfqpoint{5.611646in}{3.193122in}} %
\pgfusepath{clip}%
\pgfsetbuttcap%
\pgfsetroundjoin%
\pgfsetlinewidth{0.501875pt}%
\definecolor{currentstroke}{rgb}{0.000000,0.000000,0.000000}%
\pgfsetstrokecolor{currentstroke}%
\pgfsetdash{{1.000000pt}{3.000000pt}}{0.000000pt}%
\pgfpathmoveto{\pgfqpoint{0.580766in}{4.462900in}}%
\pgfpathlineto{\pgfqpoint{6.192411in}{4.462900in}}%
\pgfusepath{stroke}%
\end{pgfscope}%
\begin{pgfscope}%
\pgfsetbuttcap%
\pgfsetroundjoin%
\definecolor{currentfill}{rgb}{0.000000,0.000000,0.000000}%
\pgfsetfillcolor{currentfill}%
\pgfsetlinewidth{0.501875pt}%
\definecolor{currentstroke}{rgb}{0.000000,0.000000,0.000000}%
\pgfsetstrokecolor{currentstroke}%
\pgfsetdash{}{0pt}%
\pgfsys@defobject{currentmarker}{\pgfqpoint{0.000000in}{0.000000in}}{\pgfqpoint{0.055556in}{0.000000in}}{%
\pgfpathmoveto{\pgfqpoint{0.000000in}{0.000000in}}%
\pgfpathlineto{\pgfqpoint{0.055556in}{0.000000in}}%
\pgfusepath{stroke,fill}%
}%
\begin{pgfscope}%
\pgfsys@transformshift{0.580766in}{4.462900in}%
\pgfsys@useobject{currentmarker}{}%
\end{pgfscope}%
\end{pgfscope}%
\begin{pgfscope}%
\pgfsetbuttcap%
\pgfsetroundjoin%
\definecolor{currentfill}{rgb}{0.000000,0.000000,0.000000}%
\pgfsetfillcolor{currentfill}%
\pgfsetlinewidth{0.501875pt}%
\definecolor{currentstroke}{rgb}{0.000000,0.000000,0.000000}%
\pgfsetstrokecolor{currentstroke}%
\pgfsetdash{}{0pt}%
\pgfsys@defobject{currentmarker}{\pgfqpoint{-0.055556in}{0.000000in}}{\pgfqpoint{0.000000in}{0.000000in}}{%
\pgfpathmoveto{\pgfqpoint{0.000000in}{0.000000in}}%
\pgfpathlineto{\pgfqpoint{-0.055556in}{0.000000in}}%
\pgfusepath{stroke,fill}%
}%
\begin{pgfscope}%
\pgfsys@transformshift{6.192411in}{4.462900in}%
\pgfsys@useobject{currentmarker}{}%
\end{pgfscope}%
\end{pgfscope}%
\begin{pgfscope}%
\pgftext[x=0.525210in,y=4.462900in,right,]{{\rmfamily\fontsize{10.000000}{12.000000}\selectfont \(\displaystyle 0.0\)}}%
\end{pgfscope}%
\begin{pgfscope}%
\pgfpathrectangle{\pgfqpoint{0.580766in}{4.462900in}}{\pgfqpoint{5.611646in}{3.193122in}} %
\pgfusepath{clip}%
\pgfsetbuttcap%
\pgfsetroundjoin%
\pgfsetlinewidth{0.501875pt}%
\definecolor{currentstroke}{rgb}{0.000000,0.000000,0.000000}%
\pgfsetstrokecolor{currentstroke}%
\pgfsetdash{{1.000000pt}{3.000000pt}}{0.000000pt}%
\pgfpathmoveto{\pgfqpoint{0.580766in}{5.101525in}}%
\pgfpathlineto{\pgfqpoint{6.192411in}{5.101525in}}%
\pgfusepath{stroke}%
\end{pgfscope}%
\begin{pgfscope}%
\pgfsetbuttcap%
\pgfsetroundjoin%
\definecolor{currentfill}{rgb}{0.000000,0.000000,0.000000}%
\pgfsetfillcolor{currentfill}%
\pgfsetlinewidth{0.501875pt}%
\definecolor{currentstroke}{rgb}{0.000000,0.000000,0.000000}%
\pgfsetstrokecolor{currentstroke}%
\pgfsetdash{}{0pt}%
\pgfsys@defobject{currentmarker}{\pgfqpoint{0.000000in}{0.000000in}}{\pgfqpoint{0.055556in}{0.000000in}}{%
\pgfpathmoveto{\pgfqpoint{0.000000in}{0.000000in}}%
\pgfpathlineto{\pgfqpoint{0.055556in}{0.000000in}}%
\pgfusepath{stroke,fill}%
}%
\begin{pgfscope}%
\pgfsys@transformshift{0.580766in}{5.101525in}%
\pgfsys@useobject{currentmarker}{}%
\end{pgfscope}%
\end{pgfscope}%
\begin{pgfscope}%
\pgfsetbuttcap%
\pgfsetroundjoin%
\definecolor{currentfill}{rgb}{0.000000,0.000000,0.000000}%
\pgfsetfillcolor{currentfill}%
\pgfsetlinewidth{0.501875pt}%
\definecolor{currentstroke}{rgb}{0.000000,0.000000,0.000000}%
\pgfsetstrokecolor{currentstroke}%
\pgfsetdash{}{0pt}%
\pgfsys@defobject{currentmarker}{\pgfqpoint{-0.055556in}{0.000000in}}{\pgfqpoint{0.000000in}{0.000000in}}{%
\pgfpathmoveto{\pgfqpoint{0.000000in}{0.000000in}}%
\pgfpathlineto{\pgfqpoint{-0.055556in}{0.000000in}}%
\pgfusepath{stroke,fill}%
}%
\begin{pgfscope}%
\pgfsys@transformshift{6.192411in}{5.101525in}%
\pgfsys@useobject{currentmarker}{}%
\end{pgfscope}%
\end{pgfscope}%
\begin{pgfscope}%
\pgftext[x=0.525210in,y=5.101525in,right,]{{\rmfamily\fontsize{10.000000}{12.000000}\selectfont \(\displaystyle 0.2\)}}%
\end{pgfscope}%
\begin{pgfscope}%
\pgfpathrectangle{\pgfqpoint{0.580766in}{4.462900in}}{\pgfqpoint{5.611646in}{3.193122in}} %
\pgfusepath{clip}%
\pgfsetbuttcap%
\pgfsetroundjoin%
\pgfsetlinewidth{0.501875pt}%
\definecolor{currentstroke}{rgb}{0.000000,0.000000,0.000000}%
\pgfsetstrokecolor{currentstroke}%
\pgfsetdash{{1.000000pt}{3.000000pt}}{0.000000pt}%
\pgfpathmoveto{\pgfqpoint{0.580766in}{5.740149in}}%
\pgfpathlineto{\pgfqpoint{6.192411in}{5.740149in}}%
\pgfusepath{stroke}%
\end{pgfscope}%
\begin{pgfscope}%
\pgfsetbuttcap%
\pgfsetroundjoin%
\definecolor{currentfill}{rgb}{0.000000,0.000000,0.000000}%
\pgfsetfillcolor{currentfill}%
\pgfsetlinewidth{0.501875pt}%
\definecolor{currentstroke}{rgb}{0.000000,0.000000,0.000000}%
\pgfsetstrokecolor{currentstroke}%
\pgfsetdash{}{0pt}%
\pgfsys@defobject{currentmarker}{\pgfqpoint{0.000000in}{0.000000in}}{\pgfqpoint{0.055556in}{0.000000in}}{%
\pgfpathmoveto{\pgfqpoint{0.000000in}{0.000000in}}%
\pgfpathlineto{\pgfqpoint{0.055556in}{0.000000in}}%
\pgfusepath{stroke,fill}%
}%
\begin{pgfscope}%
\pgfsys@transformshift{0.580766in}{5.740149in}%
\pgfsys@useobject{currentmarker}{}%
\end{pgfscope}%
\end{pgfscope}%
\begin{pgfscope}%
\pgfsetbuttcap%
\pgfsetroundjoin%
\definecolor{currentfill}{rgb}{0.000000,0.000000,0.000000}%
\pgfsetfillcolor{currentfill}%
\pgfsetlinewidth{0.501875pt}%
\definecolor{currentstroke}{rgb}{0.000000,0.000000,0.000000}%
\pgfsetstrokecolor{currentstroke}%
\pgfsetdash{}{0pt}%
\pgfsys@defobject{currentmarker}{\pgfqpoint{-0.055556in}{0.000000in}}{\pgfqpoint{0.000000in}{0.000000in}}{%
\pgfpathmoveto{\pgfqpoint{0.000000in}{0.000000in}}%
\pgfpathlineto{\pgfqpoint{-0.055556in}{0.000000in}}%
\pgfusepath{stroke,fill}%
}%
\begin{pgfscope}%
\pgfsys@transformshift{6.192411in}{5.740149in}%
\pgfsys@useobject{currentmarker}{}%
\end{pgfscope}%
\end{pgfscope}%
\begin{pgfscope}%
\pgftext[x=0.525210in,y=5.740149in,right,]{{\rmfamily\fontsize{10.000000}{12.000000}\selectfont \(\displaystyle 0.4\)}}%
\end{pgfscope}%
\begin{pgfscope}%
\pgfpathrectangle{\pgfqpoint{0.580766in}{4.462900in}}{\pgfqpoint{5.611646in}{3.193122in}} %
\pgfusepath{clip}%
\pgfsetbuttcap%
\pgfsetroundjoin%
\pgfsetlinewidth{0.501875pt}%
\definecolor{currentstroke}{rgb}{0.000000,0.000000,0.000000}%
\pgfsetstrokecolor{currentstroke}%
\pgfsetdash{{1.000000pt}{3.000000pt}}{0.000000pt}%
\pgfpathmoveto{\pgfqpoint{0.580766in}{6.378774in}}%
\pgfpathlineto{\pgfqpoint{6.192411in}{6.378774in}}%
\pgfusepath{stroke}%
\end{pgfscope}%
\begin{pgfscope}%
\pgfsetbuttcap%
\pgfsetroundjoin%
\definecolor{currentfill}{rgb}{0.000000,0.000000,0.000000}%
\pgfsetfillcolor{currentfill}%
\pgfsetlinewidth{0.501875pt}%
\definecolor{currentstroke}{rgb}{0.000000,0.000000,0.000000}%
\pgfsetstrokecolor{currentstroke}%
\pgfsetdash{}{0pt}%
\pgfsys@defobject{currentmarker}{\pgfqpoint{0.000000in}{0.000000in}}{\pgfqpoint{0.055556in}{0.000000in}}{%
\pgfpathmoveto{\pgfqpoint{0.000000in}{0.000000in}}%
\pgfpathlineto{\pgfqpoint{0.055556in}{0.000000in}}%
\pgfusepath{stroke,fill}%
}%
\begin{pgfscope}%
\pgfsys@transformshift{0.580766in}{6.378774in}%
\pgfsys@useobject{currentmarker}{}%
\end{pgfscope}%
\end{pgfscope}%
\begin{pgfscope}%
\pgfsetbuttcap%
\pgfsetroundjoin%
\definecolor{currentfill}{rgb}{0.000000,0.000000,0.000000}%
\pgfsetfillcolor{currentfill}%
\pgfsetlinewidth{0.501875pt}%
\definecolor{currentstroke}{rgb}{0.000000,0.000000,0.000000}%
\pgfsetstrokecolor{currentstroke}%
\pgfsetdash{}{0pt}%
\pgfsys@defobject{currentmarker}{\pgfqpoint{-0.055556in}{0.000000in}}{\pgfqpoint{0.000000in}{0.000000in}}{%
\pgfpathmoveto{\pgfqpoint{0.000000in}{0.000000in}}%
\pgfpathlineto{\pgfqpoint{-0.055556in}{0.000000in}}%
\pgfusepath{stroke,fill}%
}%
\begin{pgfscope}%
\pgfsys@transformshift{6.192411in}{6.378774in}%
\pgfsys@useobject{currentmarker}{}%
\end{pgfscope}%
\end{pgfscope}%
\begin{pgfscope}%
\pgftext[x=0.525210in,y=6.378774in,right,]{{\rmfamily\fontsize{10.000000}{12.000000}\selectfont \(\displaystyle 0.6\)}}%
\end{pgfscope}%
\begin{pgfscope}%
\pgfpathrectangle{\pgfqpoint{0.580766in}{4.462900in}}{\pgfqpoint{5.611646in}{3.193122in}} %
\pgfusepath{clip}%
\pgfsetbuttcap%
\pgfsetroundjoin%
\pgfsetlinewidth{0.501875pt}%
\definecolor{currentstroke}{rgb}{0.000000,0.000000,0.000000}%
\pgfsetstrokecolor{currentstroke}%
\pgfsetdash{{1.000000pt}{3.000000pt}}{0.000000pt}%
\pgfpathmoveto{\pgfqpoint{0.580766in}{7.017398in}}%
\pgfpathlineto{\pgfqpoint{6.192411in}{7.017398in}}%
\pgfusepath{stroke}%
\end{pgfscope}%
\begin{pgfscope}%
\pgfsetbuttcap%
\pgfsetroundjoin%
\definecolor{currentfill}{rgb}{0.000000,0.000000,0.000000}%
\pgfsetfillcolor{currentfill}%
\pgfsetlinewidth{0.501875pt}%
\definecolor{currentstroke}{rgb}{0.000000,0.000000,0.000000}%
\pgfsetstrokecolor{currentstroke}%
\pgfsetdash{}{0pt}%
\pgfsys@defobject{currentmarker}{\pgfqpoint{0.000000in}{0.000000in}}{\pgfqpoint{0.055556in}{0.000000in}}{%
\pgfpathmoveto{\pgfqpoint{0.000000in}{0.000000in}}%
\pgfpathlineto{\pgfqpoint{0.055556in}{0.000000in}}%
\pgfusepath{stroke,fill}%
}%
\begin{pgfscope}%
\pgfsys@transformshift{0.580766in}{7.017398in}%
\pgfsys@useobject{currentmarker}{}%
\end{pgfscope}%
\end{pgfscope}%
\begin{pgfscope}%
\pgfsetbuttcap%
\pgfsetroundjoin%
\definecolor{currentfill}{rgb}{0.000000,0.000000,0.000000}%
\pgfsetfillcolor{currentfill}%
\pgfsetlinewidth{0.501875pt}%
\definecolor{currentstroke}{rgb}{0.000000,0.000000,0.000000}%
\pgfsetstrokecolor{currentstroke}%
\pgfsetdash{}{0pt}%
\pgfsys@defobject{currentmarker}{\pgfqpoint{-0.055556in}{0.000000in}}{\pgfqpoint{0.000000in}{0.000000in}}{%
\pgfpathmoveto{\pgfqpoint{0.000000in}{0.000000in}}%
\pgfpathlineto{\pgfqpoint{-0.055556in}{0.000000in}}%
\pgfusepath{stroke,fill}%
}%
\begin{pgfscope}%
\pgfsys@transformshift{6.192411in}{7.017398in}%
\pgfsys@useobject{currentmarker}{}%
\end{pgfscope}%
\end{pgfscope}%
\begin{pgfscope}%
\pgftext[x=0.525210in,y=7.017398in,right,]{{\rmfamily\fontsize{10.000000}{12.000000}\selectfont \(\displaystyle 0.8\)}}%
\end{pgfscope}%
\begin{pgfscope}%
\pgfpathrectangle{\pgfqpoint{0.580766in}{4.462900in}}{\pgfqpoint{5.611646in}{3.193122in}} %
\pgfusepath{clip}%
\pgfsetbuttcap%
\pgfsetroundjoin%
\pgfsetlinewidth{0.501875pt}%
\definecolor{currentstroke}{rgb}{0.000000,0.000000,0.000000}%
\pgfsetstrokecolor{currentstroke}%
\pgfsetdash{{1.000000pt}{3.000000pt}}{0.000000pt}%
\pgfpathmoveto{\pgfqpoint{0.580766in}{7.656023in}}%
\pgfpathlineto{\pgfqpoint{6.192411in}{7.656023in}}%
\pgfusepath{stroke}%
\end{pgfscope}%
\begin{pgfscope}%
\pgfsetbuttcap%
\pgfsetroundjoin%
\definecolor{currentfill}{rgb}{0.000000,0.000000,0.000000}%
\pgfsetfillcolor{currentfill}%
\pgfsetlinewidth{0.501875pt}%
\definecolor{currentstroke}{rgb}{0.000000,0.000000,0.000000}%
\pgfsetstrokecolor{currentstroke}%
\pgfsetdash{}{0pt}%
\pgfsys@defobject{currentmarker}{\pgfqpoint{0.000000in}{0.000000in}}{\pgfqpoint{0.055556in}{0.000000in}}{%
\pgfpathmoveto{\pgfqpoint{0.000000in}{0.000000in}}%
\pgfpathlineto{\pgfqpoint{0.055556in}{0.000000in}}%
\pgfusepath{stroke,fill}%
}%
\begin{pgfscope}%
\pgfsys@transformshift{0.580766in}{7.656023in}%
\pgfsys@useobject{currentmarker}{}%
\end{pgfscope}%
\end{pgfscope}%
\begin{pgfscope}%
\pgfsetbuttcap%
\pgfsetroundjoin%
\definecolor{currentfill}{rgb}{0.000000,0.000000,0.000000}%
\pgfsetfillcolor{currentfill}%
\pgfsetlinewidth{0.501875pt}%
\definecolor{currentstroke}{rgb}{0.000000,0.000000,0.000000}%
\pgfsetstrokecolor{currentstroke}%
\pgfsetdash{}{0pt}%
\pgfsys@defobject{currentmarker}{\pgfqpoint{-0.055556in}{0.000000in}}{\pgfqpoint{0.000000in}{0.000000in}}{%
\pgfpathmoveto{\pgfqpoint{0.000000in}{0.000000in}}%
\pgfpathlineto{\pgfqpoint{-0.055556in}{0.000000in}}%
\pgfusepath{stroke,fill}%
}%
\begin{pgfscope}%
\pgfsys@transformshift{6.192411in}{7.656023in}%
\pgfsys@useobject{currentmarker}{}%
\end{pgfscope}%
\end{pgfscope}%
\begin{pgfscope}%
\pgftext[x=0.525210in,y=7.656023in,right,]{{\rmfamily\fontsize{10.000000}{12.000000}\selectfont \(\displaystyle 1.0\)}}%
\end{pgfscope}%
\begin{pgfscope}%
\pgftext[x=0.278296in,y=6.059461in,,bottom,rotate=90.000000]{{\rmfamily\fontsize{10.000000}{12.000000}\selectfont current (A)}}%
\end{pgfscope}%
\begin{pgfscope}%
\pgfsetbuttcap%
\pgfsetroundjoin%
\pgfsetlinewidth{1.003750pt}%
\definecolor{currentstroke}{rgb}{0.000000,0.000000,0.000000}%
\pgfsetstrokecolor{currentstroke}%
\pgfsetdash{}{0pt}%
\pgfpathmoveto{\pgfqpoint{0.580766in}{7.656023in}}%
\pgfpathlineto{\pgfqpoint{6.192411in}{7.656023in}}%
\pgfusepath{stroke}%
\end{pgfscope}%
\begin{pgfscope}%
\pgfsetbuttcap%
\pgfsetroundjoin%
\pgfsetlinewidth{1.003750pt}%
\definecolor{currentstroke}{rgb}{0.000000,0.000000,0.000000}%
\pgfsetstrokecolor{currentstroke}%
\pgfsetdash{}{0pt}%
\pgfpathmoveto{\pgfqpoint{6.192411in}{4.462900in}}%
\pgfpathlineto{\pgfqpoint{6.192411in}{7.656023in}}%
\pgfusepath{stroke}%
\end{pgfscope}%
\begin{pgfscope}%
\pgfsetbuttcap%
\pgfsetroundjoin%
\pgfsetlinewidth{1.003750pt}%
\definecolor{currentstroke}{rgb}{0.000000,0.000000,0.000000}%
\pgfsetstrokecolor{currentstroke}%
\pgfsetdash{}{0pt}%
\pgfpathmoveto{\pgfqpoint{0.580766in}{4.462900in}}%
\pgfpathlineto{\pgfqpoint{6.192411in}{4.462900in}}%
\pgfusepath{stroke}%
\end{pgfscope}%
\begin{pgfscope}%
\pgfsetbuttcap%
\pgfsetroundjoin%
\pgfsetlinewidth{1.003750pt}%
\definecolor{currentstroke}{rgb}{0.000000,0.000000,0.000000}%
\pgfsetstrokecolor{currentstroke}%
\pgfsetdash{}{0pt}%
\pgfpathmoveto{\pgfqpoint{0.580766in}{4.462900in}}%
\pgfpathlineto{\pgfqpoint{0.580766in}{7.656023in}}%
\pgfusepath{stroke}%
\end{pgfscope}%
\begin{pgfscope}%
\pgftext[x=3.386589in,y=7.725467in,,base]{{\rmfamily\fontsize{12.000000}{14.400000}\selectfont Evaluations of \(\displaystyle i(t)\) at \(\displaystyle 0.0001s\)}}%
\end{pgfscope}%
\begin{pgfscope}%
\pgfsetbuttcap%
\pgfsetroundjoin%
\definecolor{currentfill}{rgb}{0.300000,0.300000,0.300000}%
\pgfsetfillcolor{currentfill}%
\pgfsetfillopacity{0.500000}%
\pgfsetlinewidth{1.003750pt}%
\definecolor{currentstroke}{rgb}{0.300000,0.300000,0.300000}%
\pgfsetstrokecolor{currentstroke}%
\pgfsetstrokeopacity{0.500000}%
\pgfsetdash{}{0pt}%
\pgfpathmoveto{\pgfqpoint{4.378033in}{6.565283in}}%
\pgfpathlineto{\pgfqpoint{6.103523in}{6.565283in}}%
\pgfpathquadraticcurveto{\pgfqpoint{6.136856in}{6.565283in}}{\pgfqpoint{6.136856in}{6.598616in}}%
\pgfpathlineto{\pgfqpoint{6.136856in}{7.511578in}}%
\pgfpathquadraticcurveto{\pgfqpoint{6.136856in}{7.544911in}}{\pgfqpoint{6.103523in}{7.544911in}}%
\pgfpathlineto{\pgfqpoint{4.378033in}{7.544911in}}%
\pgfpathquadraticcurveto{\pgfqpoint{4.344700in}{7.544911in}}{\pgfqpoint{4.344700in}{7.511578in}}%
\pgfpathlineto{\pgfqpoint{4.344700in}{6.598616in}}%
\pgfpathquadraticcurveto{\pgfqpoint{4.344700in}{6.565283in}}{\pgfqpoint{4.378033in}{6.565283in}}%
\pgfpathclose%
\pgfusepath{stroke,fill}%
\end{pgfscope}%
\begin{pgfscope}%
\pgfsetbuttcap%
\pgfsetroundjoin%
\definecolor{currentfill}{rgb}{1.000000,1.000000,1.000000}%
\pgfsetfillcolor{currentfill}%
\pgfsetlinewidth{1.003750pt}%
\definecolor{currentstroke}{rgb}{0.000000,0.000000,0.000000}%
\pgfsetstrokecolor{currentstroke}%
\pgfsetdash{}{0pt}%
\pgfpathmoveto{\pgfqpoint{4.350255in}{6.593061in}}%
\pgfpathlineto{\pgfqpoint{6.075745in}{6.593061in}}%
\pgfpathquadraticcurveto{\pgfqpoint{6.109078in}{6.593061in}}{\pgfqpoint{6.109078in}{6.626394in}}%
\pgfpathlineto{\pgfqpoint{6.109078in}{7.539356in}}%
\pgfpathquadraticcurveto{\pgfqpoint{6.109078in}{7.572689in}}{\pgfqpoint{6.075745in}{7.572689in}}%
\pgfpathlineto{\pgfqpoint{4.350255in}{7.572689in}}%
\pgfpathquadraticcurveto{\pgfqpoint{4.316922in}{7.572689in}}{\pgfqpoint{4.316922in}{7.539356in}}%
\pgfpathlineto{\pgfqpoint{4.316922in}{6.626394in}}%
\pgfpathquadraticcurveto{\pgfqpoint{4.316922in}{6.593061in}}{\pgfqpoint{4.350255in}{6.593061in}}%
\pgfpathclose%
\pgfusepath{stroke,fill}%
\end{pgfscope}%
\begin{pgfscope}%
\pgfsetbuttcap%
\pgfsetroundjoin%
\pgfsetlinewidth{1.003750pt}%
\definecolor{currentstroke}{rgb}{0.000000,0.000000,1.000000}%
\pgfsetstrokecolor{currentstroke}%
\pgfsetdash{{1.000000pt}{3.000000pt}}{0.000000pt}%
\pgfpathmoveto{\pgfqpoint{4.433589in}{7.447689in}}%
\pgfpathlineto{\pgfqpoint{4.666922in}{7.447689in}}%
\pgfusepath{stroke}%
\end{pgfscope}%
\begin{pgfscope}%
\pgftext[x=4.850255in,y=7.389356in,left,base]{{\rmfamily\fontsize{12.000000}{14.400000}\selectfont Continuous}}%
\end{pgfscope}%
\begin{pgfscope}%
\pgfsetrectcap%
\pgfsetroundjoin%
\pgfsetlinewidth{1.003750pt}%
\definecolor{currentstroke}{rgb}{0.000000,0.500000,0.000000}%
\pgfsetstrokecolor{currentstroke}%
\pgfsetdash{}{0pt}%
\pgfpathmoveto{\pgfqpoint{4.433589in}{7.215282in}}%
\pgfpathlineto{\pgfqpoint{4.666922in}{7.215282in}}%
\pgfusepath{stroke}%
\end{pgfscope}%
\begin{pgfscope}%
\pgftext[x=4.850255in,y=7.156949in,left,base]{{\rmfamily\fontsize{12.000000}{14.400000}\selectfont Trapezoidal}}%
\end{pgfscope}%
\begin{pgfscope}%
\pgfsetrectcap%
\pgfsetroundjoin%
\pgfsetlinewidth{1.003750pt}%
\definecolor{currentstroke}{rgb}{1.000000,0.000000,0.000000}%
\pgfsetstrokecolor{currentstroke}%
\pgfsetdash{}{0pt}%
\pgfpathmoveto{\pgfqpoint{4.433589in}{6.982875in}}%
\pgfpathlineto{\pgfqpoint{4.666922in}{6.982875in}}%
\pgfusepath{stroke}%
\end{pgfscope}%
\begin{pgfscope}%
\pgftext[x=4.850255in,y=6.924542in,left,base]{{\rmfamily\fontsize{12.000000}{14.400000}\selectfont Backward Euler    }}%
\end{pgfscope}%
\begin{pgfscope}%
\pgfsetrectcap%
\pgfsetroundjoin%
\pgfsetlinewidth{1.003750pt}%
\definecolor{currentstroke}{rgb}{0.000000,0.750000,0.750000}%
\pgfsetstrokecolor{currentstroke}%
\pgfsetdash{}{0pt}%
\pgfpathmoveto{\pgfqpoint{4.433589in}{6.750468in}}%
\pgfpathlineto{\pgfqpoint{4.666922in}{6.750468in}}%
\pgfusepath{stroke}%
\end{pgfscope}%
\begin{pgfscope}%
\pgftext[x=4.850255in,y=6.692135in,left,base]{{\rmfamily\fontsize{12.000000}{14.400000}\selectfont PSCAD}}%
\end{pgfscope}%
\begin{pgfscope}%
\pgfsetbuttcap%
\pgfsetroundjoin%
\definecolor{currentfill}{rgb}{1.000000,1.000000,1.000000}%
\pgfsetfillcolor{currentfill}%
\pgfsetlinewidth{0.000000pt}%
\definecolor{currentstroke}{rgb}{0.000000,0.000000,0.000000}%
\pgfsetstrokecolor{currentstroke}%
\pgfsetstrokeopacity{0.000000}%
\pgfsetdash{}{0pt}%
\pgfpathmoveto{\pgfqpoint{0.580766in}{0.532919in}}%
\pgfpathlineto{\pgfqpoint{6.192411in}{0.532919in}}%
\pgfpathlineto{\pgfqpoint{6.192411in}{3.726041in}}%
\pgfpathlineto{\pgfqpoint{0.580766in}{3.726041in}}%
\pgfpathclose%
\pgfusepath{fill}%
\end{pgfscope}%
\begin{pgfscope}%
\pgfpathrectangle{\pgfqpoint{0.580766in}{0.532919in}}{\pgfqpoint{5.611646in}{3.193122in}} %
\pgfusepath{clip}%
\pgfsetbuttcap%
\pgfsetroundjoin%
\pgfsetlinewidth{1.003750pt}%
\definecolor{currentstroke}{rgb}{0.000000,0.000000,1.000000}%
\pgfsetstrokecolor{currentstroke}%
\pgfsetdash{{1.000000pt}{3.000000pt}}{0.000000pt}%
\pgfpathmoveto{\pgfqpoint{0.580766in}{3.726041in}}%
\pgfpathlineto{\pgfqpoint{0.627530in}{3.570310in}}%
\pgfpathlineto{\pgfqpoint{0.674293in}{3.422175in}}%
\pgfpathlineto{\pgfqpoint{0.721057in}{3.281264in}}%
\pgfpathlineto{\pgfqpoint{0.767821in}{3.147226in}}%
\pgfpathlineto{\pgfqpoint{0.814584in}{3.019725in}}%
\pgfpathlineto{\pgfqpoint{0.861348in}{2.898442in}}%
\pgfpathlineto{\pgfqpoint{0.909671in}{2.779327in}}%
\pgfpathlineto{\pgfqpoint{0.957993in}{2.666210in}}%
\pgfpathlineto{\pgfqpoint{1.006316in}{2.558789in}}%
\pgfpathlineto{\pgfqpoint{1.054638in}{2.456777in}}%
\pgfpathlineto{\pgfqpoint{1.102961in}{2.359901in}}%
\pgfpathlineto{\pgfqpoint{1.151283in}{2.267904in}}%
\pgfpathlineto{\pgfqpoint{1.199606in}{2.180540in}}%
\pgfpathlineto{\pgfqpoint{1.249487in}{2.094969in}}%
\pgfpathlineto{\pgfqpoint{1.299368in}{2.013842in}}%
\pgfpathlineto{\pgfqpoint{1.349250in}{1.936929in}}%
\pgfpathlineto{\pgfqpoint{1.399131in}{1.864010in}}%
\pgfpathlineto{\pgfqpoint{1.450571in}{1.792777in}}%
\pgfpathlineto{\pgfqpoint{1.502011in}{1.725356in}}%
\pgfpathlineto{\pgfqpoint{1.553451in}{1.661543in}}%
\pgfpathlineto{\pgfqpoint{1.606450in}{1.599366in}}%
\pgfpathlineto{\pgfqpoint{1.659449in}{1.540614in}}%
\pgfpathlineto{\pgfqpoint{1.712448in}{1.485099in}}%
\pgfpathlineto{\pgfqpoint{1.767005in}{1.431144in}}%
\pgfpathlineto{\pgfqpoint{1.821563in}{1.380246in}}%
\pgfpathlineto{\pgfqpoint{1.877680in}{1.330902in}}%
\pgfpathlineto{\pgfqpoint{1.935355in}{1.283180in}}%
\pgfpathlineto{\pgfqpoint{1.993030in}{1.238311in}}%
\pgfpathlineto{\pgfqpoint{2.052264in}{1.195022in}}%
\pgfpathlineto{\pgfqpoint{2.113057in}{1.153354in}}%
\pgfpathlineto{\pgfqpoint{2.175409in}{1.113340in}}%
\pgfpathlineto{\pgfqpoint{2.239319in}{1.075003in}}%
\pgfpathlineto{\pgfqpoint{2.304788in}{1.038354in}}%
\pgfpathlineto{\pgfqpoint{2.371816in}{1.003399in}}%
\pgfpathlineto{\pgfqpoint{2.440403in}{0.970132in}}%
\pgfpathlineto{\pgfqpoint{2.512107in}{0.937865in}}%
\pgfpathlineto{\pgfqpoint{2.585370in}{0.907355in}}%
\pgfpathlineto{\pgfqpoint{2.661751in}{0.877991in}}%
\pgfpathlineto{\pgfqpoint{2.739691in}{0.850401in}}%
\pgfpathlineto{\pgfqpoint{2.820748in}{0.824044in}}%
\pgfpathlineto{\pgfqpoint{2.904922in}{0.798987in}}%
\pgfpathlineto{\pgfqpoint{2.993773in}{0.774874in}}%
\pgfpathlineto{\pgfqpoint{3.085742in}{0.752214in}}%
\pgfpathlineto{\pgfqpoint{3.182387in}{0.730685in}}%
\pgfpathlineto{\pgfqpoint{3.283709in}{0.710380in}}%
\pgfpathlineto{\pgfqpoint{3.391265in}{0.691102in}}%
\pgfpathlineto{\pgfqpoint{3.505057in}{0.672981in}}%
\pgfpathlineto{\pgfqpoint{3.626642in}{0.655907in}}%
\pgfpathlineto{\pgfqpoint{3.756022in}{0.640018in}}%
\pgfpathlineto{\pgfqpoint{3.894754in}{0.625253in}}%
\pgfpathlineto{\pgfqpoint{4.044398in}{0.611601in}}%
\pgfpathlineto{\pgfqpoint{4.206512in}{0.599079in}}%
\pgfpathlineto{\pgfqpoint{4.384215in}{0.587631in}}%
\pgfpathlineto{\pgfqpoint{4.579063in}{0.577341in}}%
\pgfpathlineto{\pgfqpoint{4.797294in}{0.568097in}}%
\pgfpathlineto{\pgfqpoint{5.043583in}{0.559952in}}%
\pgfpathlineto{\pgfqpoint{5.255578in}{0.554469in}}%
\pgfpathlineto{\pgfqpoint{5.255578in}{0.554469in}}%
\pgfusepath{stroke}%
\end{pgfscope}%
\begin{pgfscope}%
\pgfpathrectangle{\pgfqpoint{0.580766in}{0.532919in}}{\pgfqpoint{5.611646in}{3.193122in}} %
\pgfusepath{clip}%
\pgfsetrectcap%
\pgfsetroundjoin%
\pgfsetlinewidth{1.003750pt}%
\definecolor{currentstroke}{rgb}{0.000000,0.500000,0.000000}%
\pgfsetstrokecolor{currentstroke}%
\pgfsetdash{}{0pt}%
\pgfpathmoveto{\pgfqpoint{0.580766in}{3.570279in}}%
\pgfpathlineto{\pgfqpoint{0.627530in}{3.422115in}}%
\pgfpathlineto{\pgfqpoint{0.674293in}{3.281179in}}%
\pgfpathlineto{\pgfqpoint{0.721057in}{3.147117in}}%
\pgfpathlineto{\pgfqpoint{0.767821in}{3.019595in}}%
\pgfpathlineto{\pgfqpoint{0.814584in}{2.898294in}}%
\pgfpathlineto{\pgfqpoint{0.861348in}{2.782910in}}%
\pgfpathlineto{\pgfqpoint{0.908112in}{2.673154in}}%
\pgfpathlineto{\pgfqpoint{0.954876in}{2.568752in}}%
\pgfpathlineto{\pgfqpoint{1.001639in}{2.469443in}}%
\pgfpathlineto{\pgfqpoint{1.048403in}{2.374979in}}%
\pgfpathlineto{\pgfqpoint{1.095167in}{2.285122in}}%
\pgfpathlineto{\pgfqpoint{1.141930in}{2.199649in}}%
\pgfpathlineto{\pgfqpoint{1.188694in}{2.118345in}}%
\pgfpathlineto{\pgfqpoint{1.235458in}{2.041007in}}%
\pgfpathlineto{\pgfqpoint{1.282222in}{1.967442in}}%
\pgfpathlineto{\pgfqpoint{1.328985in}{1.897465in}}%
\pgfpathlineto{\pgfqpoint{1.375749in}{1.830902in}}%
\pgfpathlineto{\pgfqpoint{1.422513in}{1.767585in}}%
\pgfpathlineto{\pgfqpoint{1.469276in}{1.707358in}}%
\pgfpathlineto{\pgfqpoint{1.516040in}{1.650068in}}%
\pgfpathlineto{\pgfqpoint{1.562804in}{1.595573in}}%
\pgfpathlineto{\pgfqpoint{1.609568in}{1.543736in}}%
\pgfpathlineto{\pgfqpoint{1.656331in}{1.494428in}}%
\pgfpathlineto{\pgfqpoint{1.703095in}{1.447525in}}%
\pgfpathlineto{\pgfqpoint{1.749859in}{1.402910in}}%
\pgfpathlineto{\pgfqpoint{1.796622in}{1.360472in}}%
\pgfpathlineto{\pgfqpoint{1.843386in}{1.320103in}}%
\pgfpathlineto{\pgfqpoint{1.890150in}{1.281704in}}%
\pgfpathlineto{\pgfqpoint{1.936914in}{1.245178in}}%
\pgfpathlineto{\pgfqpoint{1.983677in}{1.210433in}}%
\pgfpathlineto{\pgfqpoint{2.030441in}{1.177384in}}%
\pgfpathlineto{\pgfqpoint{2.077205in}{1.145947in}}%
\pgfpathlineto{\pgfqpoint{2.123968in}{1.116043in}}%
\pgfpathlineto{\pgfqpoint{2.170732in}{1.087598in}}%
\pgfpathlineto{\pgfqpoint{2.217496in}{1.060540in}}%
\pgfpathlineto{\pgfqpoint{2.264260in}{1.034802in}}%
\pgfpathlineto{\pgfqpoint{2.311023in}{1.010320in}}%
\pgfpathlineto{\pgfqpoint{2.357787in}{0.987032in}}%
\pgfpathlineto{\pgfqpoint{2.404551in}{0.964881in}}%
\pgfpathlineto{\pgfqpoint{2.451314in}{0.943809in}}%
\pgfpathlineto{\pgfqpoint{2.498078in}{0.923766in}}%
\pgfpathlineto{\pgfqpoint{2.544842in}{0.904700in}}%
\pgfpathlineto{\pgfqpoint{2.591606in}{0.886564in}}%
\pgfpathlineto{\pgfqpoint{2.638369in}{0.869313in}}%
\pgfpathlineto{\pgfqpoint{2.685133in}{0.852904in}}%
\pgfpathlineto{\pgfqpoint{2.731897in}{0.837295in}}%
\pgfpathlineto{\pgfqpoint{2.778660in}{0.822447in}}%
\pgfpathlineto{\pgfqpoint{2.825424in}{0.808324in}}%
\pgfpathlineto{\pgfqpoint{2.872188in}{0.794889in}}%
\pgfpathlineto{\pgfqpoint{2.918952in}{0.782110in}}%
\pgfpathlineto{\pgfqpoint{2.965715in}{0.769955in}}%
\pgfpathlineto{\pgfqpoint{3.012479in}{0.758392in}}%
\pgfpathlineto{\pgfqpoint{3.059243in}{0.747393in}}%
\pgfpathlineto{\pgfqpoint{3.106006in}{0.736931in}}%
\pgfpathlineto{\pgfqpoint{3.152770in}{0.726979in}}%
\pgfpathlineto{\pgfqpoint{3.199534in}{0.717513in}}%
\pgfpathlineto{\pgfqpoint{3.246298in}{0.708508in}}%
\pgfpathlineto{\pgfqpoint{3.293061in}{0.699943in}}%
\pgfpathlineto{\pgfqpoint{3.339825in}{0.691795in}}%
\pgfpathlineto{\pgfqpoint{3.386589in}{0.684045in}}%
\pgfpathlineto{\pgfqpoint{3.433352in}{0.676673in}}%
\pgfpathlineto{\pgfqpoint{3.480116in}{0.669661in}}%
\pgfpathlineto{\pgfqpoint{3.526880in}{0.662990in}}%
\pgfpathlineto{\pgfqpoint{3.573644in}{0.656645in}}%
\pgfpathlineto{\pgfqpoint{3.620407in}{0.650610in}}%
\pgfpathlineto{\pgfqpoint{3.667171in}{0.644869in}}%
\pgfpathlineto{\pgfqpoint{3.713935in}{0.639408in}}%
\pgfpathlineto{\pgfqpoint{3.760698in}{0.634213in}}%
\pgfpathlineto{\pgfqpoint{3.807462in}{0.629272in}}%
\pgfpathlineto{\pgfqpoint{3.854226in}{0.624572in}}%
\pgfpathlineto{\pgfqpoint{3.900990in}{0.620101in}}%
\pgfpathlineto{\pgfqpoint{3.947753in}{0.615848in}}%
\pgfpathlineto{\pgfqpoint{3.994517in}{0.611803in}}%
\pgfpathlineto{\pgfqpoint{4.041281in}{0.607955in}}%
\pgfpathlineto{\pgfqpoint{4.088044in}{0.604295in}}%
\pgfpathlineto{\pgfqpoint{4.134808in}{0.600813in}}%
\pgfpathlineto{\pgfqpoint{4.181572in}{0.597501in}}%
\pgfpathlineto{\pgfqpoint{4.228336in}{0.594351in}}%
\pgfpathlineto{\pgfqpoint{4.275099in}{0.591354in}}%
\pgfpathlineto{\pgfqpoint{4.321863in}{0.588503in}}%
\pgfpathlineto{\pgfqpoint{4.368627in}{0.585792in}}%
\pgfpathlineto{\pgfqpoint{4.415390in}{0.583213in}}%
\pgfpathlineto{\pgfqpoint{4.462154in}{0.580759in}}%
\pgfpathlineto{\pgfqpoint{4.508918in}{0.578426in}}%
\pgfpathlineto{\pgfqpoint{4.555682in}{0.576206in}}%
\pgfpathlineto{\pgfqpoint{4.602445in}{0.574094in}}%
\pgfpathlineto{\pgfqpoint{4.649209in}{0.572086in}}%
\pgfpathlineto{\pgfqpoint{4.695973in}{0.570175in}}%
\pgfpathlineto{\pgfqpoint{4.742736in}{0.568358in}}%
\pgfpathlineto{\pgfqpoint{4.789500in}{0.566629in}}%
\pgfpathlineto{\pgfqpoint{4.836264in}{0.564984in}}%
\pgfpathlineto{\pgfqpoint{4.883027in}{0.563420in}}%
\pgfpathlineto{\pgfqpoint{4.929791in}{0.561932in}}%
\pgfpathlineto{\pgfqpoint{4.976555in}{0.560517in}}%
\pgfpathlineto{\pgfqpoint{5.023319in}{0.559171in}}%
\pgfpathlineto{\pgfqpoint{5.070082in}{0.557890in}}%
\pgfpathlineto{\pgfqpoint{5.116846in}{0.556672in}}%
\pgfpathlineto{\pgfqpoint{5.163610in}{0.555513in}}%
\pgfpathlineto{\pgfqpoint{5.210373in}{0.554411in}}%
\pgfusepath{stroke}%
\end{pgfscope}%
\begin{pgfscope}%
\pgfpathrectangle{\pgfqpoint{0.580766in}{0.532919in}}{\pgfqpoint{5.611646in}{3.193122in}} %
\pgfusepath{clip}%
\pgfsetrectcap%
\pgfsetroundjoin%
\pgfsetlinewidth{1.003750pt}%
\definecolor{currentstroke}{rgb}{1.000000,0.000000,0.000000}%
\pgfsetstrokecolor{currentstroke}%
\pgfsetdash{}{0pt}%
\pgfpathmoveto{\pgfqpoint{0.580766in}{3.573987in}}%
\pgfpathlineto{\pgfqpoint{0.627530in}{3.429175in}}%
\pgfpathlineto{\pgfqpoint{0.674293in}{3.291258in}}%
\pgfpathlineto{\pgfqpoint{0.721057in}{3.159908in}}%
\pgfpathlineto{\pgfqpoint{0.767821in}{3.034813in}}%
\pgfpathlineto{\pgfqpoint{0.814584in}{2.915676in}}%
\pgfpathlineto{\pgfqpoint{0.861348in}{2.802211in}}%
\pgfpathlineto{\pgfqpoint{0.908112in}{2.694149in}}%
\pgfpathlineto{\pgfqpoint{0.954876in}{2.591234in}}%
\pgfpathlineto{\pgfqpoint{1.001639in}{2.493219in}}%
\pgfpathlineto{\pgfqpoint{1.048403in}{2.399871in}}%
\pgfpathlineto{\pgfqpoint{1.095167in}{2.310969in}}%
\pgfpathlineto{\pgfqpoint{1.141930in}{2.226299in}}%
\pgfpathlineto{\pgfqpoint{1.188694in}{2.145662in}}%
\pgfpathlineto{\pgfqpoint{1.235458in}{2.068865in}}%
\pgfpathlineto{\pgfqpoint{1.282222in}{1.995725in}}%
\pgfpathlineto{\pgfqpoint{1.328985in}{1.926067in}}%
\pgfpathlineto{\pgfqpoint{1.375749in}{1.859727in}}%
\pgfpathlineto{\pgfqpoint{1.422513in}{1.796545in}}%
\pgfpathlineto{\pgfqpoint{1.469276in}{1.736373in}}%
\pgfpathlineto{\pgfqpoint{1.516040in}{1.679065in}}%
\pgfpathlineto{\pgfqpoint{1.562804in}{1.624487in}}%
\pgfpathlineto{\pgfqpoint{1.609568in}{1.572508in}}%
\pgfpathlineto{\pgfqpoint{1.656331in}{1.523003in}}%
\pgfpathlineto{\pgfqpoint{1.703095in}{1.475856in}}%
\pgfpathlineto{\pgfqpoint{1.749859in}{1.430955in}}%
\pgfpathlineto{\pgfqpoint{1.796622in}{1.388191in}}%
\pgfpathlineto{\pgfqpoint{1.843386in}{1.347464in}}%
\pgfpathlineto{\pgfqpoint{1.890150in}{1.308676in}}%
\pgfpathlineto{\pgfqpoint{1.936914in}{1.271735in}}%
\pgfpathlineto{\pgfqpoint{1.983677in}{1.236553in}}%
\pgfpathlineto{\pgfqpoint{2.030441in}{1.203047in}}%
\pgfpathlineto{\pgfqpoint{2.077205in}{1.171136in}}%
\pgfpathlineto{\pgfqpoint{2.123968in}{1.140745in}}%
\pgfpathlineto{\pgfqpoint{2.170732in}{1.111801in}}%
\pgfpathlineto{\pgfqpoint{2.217496in}{1.084235in}}%
\pgfpathlineto{\pgfqpoint{2.264260in}{1.057982in}}%
\pgfpathlineto{\pgfqpoint{2.311023in}{1.032979in}}%
\pgfpathlineto{\pgfqpoint{2.357787in}{1.009166in}}%
\pgfpathlineto{\pgfqpoint{2.404551in}{0.986488in}}%
\pgfpathlineto{\pgfqpoint{2.451314in}{0.964889in}}%
\pgfpathlineto{\pgfqpoint{2.498078in}{0.944319in}}%
\pgfpathlineto{\pgfqpoint{2.544842in}{0.924729in}}%
\pgfpathlineto{\pgfqpoint{2.591606in}{0.906071in}}%
\pgfpathlineto{\pgfqpoint{2.638369in}{0.888302in}}%
\pgfpathlineto{\pgfqpoint{2.685133in}{0.871379in}}%
\pgfpathlineto{\pgfqpoint{2.731897in}{0.855262in}}%
\pgfpathlineto{\pgfqpoint{2.778660in}{0.839912in}}%
\pgfpathlineto{\pgfqpoint{2.825424in}{0.825293in}}%
\pgfpathlineto{\pgfqpoint{2.872188in}{0.811371in}}%
\pgfpathlineto{\pgfqpoint{2.918952in}{0.798111in}}%
\pgfpathlineto{\pgfqpoint{2.965715in}{0.785483in}}%
\pgfpathlineto{\pgfqpoint{3.012479in}{0.773456in}}%
\pgfpathlineto{\pgfqpoint{3.059243in}{0.762002in}}%
\pgfpathlineto{\pgfqpoint{3.106006in}{0.751093in}}%
\pgfpathlineto{\pgfqpoint{3.152770in}{0.740704in}}%
\pgfpathlineto{\pgfqpoint{3.199534in}{0.730809in}}%
\pgfpathlineto{\pgfqpoint{3.246298in}{0.721386in}}%
\pgfpathlineto{\pgfqpoint{3.293061in}{0.712411in}}%
\pgfpathlineto{\pgfqpoint{3.339825in}{0.703864in}}%
\pgfpathlineto{\pgfqpoint{3.386589in}{0.695724in}}%
\pgfpathlineto{\pgfqpoint{3.433352in}{0.687971in}}%
\pgfpathlineto{\pgfqpoint{3.480116in}{0.680588in}}%
\pgfpathlineto{\pgfqpoint{3.526880in}{0.673556in}}%
\pgfpathlineto{\pgfqpoint{3.573644in}{0.666859in}}%
\pgfpathlineto{\pgfqpoint{3.620407in}{0.660481in}}%
\pgfpathlineto{\pgfqpoint{3.667171in}{0.654406in}}%
\pgfpathlineto{\pgfqpoint{3.713935in}{0.648621in}}%
\pgfpathlineto{\pgfqpoint{3.760698in}{0.643111in}}%
\pgfpathlineto{\pgfqpoint{3.807462in}{0.637864in}}%
\pgfpathlineto{\pgfqpoint{3.854226in}{0.632867in}}%
\pgfpathlineto{\pgfqpoint{3.900990in}{0.628107in}}%
\pgfpathlineto{\pgfqpoint{3.947753in}{0.623575in}}%
\pgfpathlineto{\pgfqpoint{3.994517in}{0.619258in}}%
\pgfpathlineto{\pgfqpoint{4.041281in}{0.615146in}}%
\pgfpathlineto{\pgfqpoint{4.088044in}{0.611231in}}%
\pgfpathlineto{\pgfqpoint{4.134808in}{0.607501in}}%
\pgfpathlineto{\pgfqpoint{4.181572in}{0.603950in}}%
\pgfpathlineto{\pgfqpoint{4.228336in}{0.600567in}}%
\pgfpathlineto{\pgfqpoint{4.275099in}{0.597346in}}%
\pgfpathlineto{\pgfqpoint{4.321863in}{0.594278in}}%
\pgfpathlineto{\pgfqpoint{4.368627in}{0.591356in}}%
\pgfpathlineto{\pgfqpoint{4.415390in}{0.588573in}}%
\pgfpathlineto{\pgfqpoint{4.462154in}{0.585923in}}%
\pgfpathlineto{\pgfqpoint{4.508918in}{0.583399in}}%
\pgfpathlineto{\pgfqpoint{4.555682in}{0.580995in}}%
\pgfpathlineto{\pgfqpoint{4.602445in}{0.578706in}}%
\pgfpathlineto{\pgfqpoint{4.649209in}{0.576526in}}%
\pgfpathlineto{\pgfqpoint{4.695973in}{0.574449in}}%
\pgfpathlineto{\pgfqpoint{4.742736in}{0.572471in}}%
\pgfpathlineto{\pgfqpoint{4.789500in}{0.570588in}}%
\pgfpathlineto{\pgfqpoint{4.836264in}{0.568794in}}%
\pgfpathlineto{\pgfqpoint{4.883027in}{0.567086in}}%
\pgfpathlineto{\pgfqpoint{4.929791in}{0.565459in}}%
\pgfpathlineto{\pgfqpoint{4.976555in}{0.563909in}}%
\pgfpathlineto{\pgfqpoint{5.023319in}{0.562434in}}%
\pgfpathlineto{\pgfqpoint{5.070082in}{0.561028in}}%
\pgfpathlineto{\pgfqpoint{5.116846in}{0.559689in}}%
\pgfpathlineto{\pgfqpoint{5.163610in}{0.558415in}}%
\pgfpathlineto{\pgfqpoint{5.210373in}{0.557201in}}%
\pgfusepath{stroke}%
\end{pgfscope}%
\begin{pgfscope}%
\pgfpathrectangle{\pgfqpoint{0.580766in}{0.532919in}}{\pgfqpoint{5.611646in}{3.193122in}} %
\pgfusepath{clip}%
\pgfsetrectcap%
\pgfsetroundjoin%
\pgfsetlinewidth{1.003750pt}%
\definecolor{currentstroke}{rgb}{0.000000,0.750000,0.750000}%
\pgfsetstrokecolor{currentstroke}%
\pgfsetdash{}{0pt}%
\pgfpathmoveto{\pgfqpoint{0.580766in}{3.610169in}}%
\pgfpathlineto{\pgfqpoint{0.627530in}{3.460059in}}%
\pgfpathlineto{\pgfqpoint{0.674293in}{3.317272in}}%
\pgfpathlineto{\pgfqpoint{0.721057in}{3.181450in}}%
\pgfpathlineto{\pgfqpoint{0.767821in}{3.052253in}}%
\pgfpathlineto{\pgfqpoint{0.814584in}{2.929359in}}%
\pgfpathlineto{\pgfqpoint{0.861348in}{2.812459in}}%
\pgfpathlineto{\pgfqpoint{0.908112in}{2.701262in}}%
\pgfpathlineto{\pgfqpoint{0.954876in}{2.595489in}}%
\pgfpathlineto{\pgfqpoint{1.001639in}{2.494876in}}%
\pgfpathlineto{\pgfqpoint{1.048403in}{2.399171in}}%
\pgfpathlineto{\pgfqpoint{1.095167in}{2.308134in}}%
\pgfpathlineto{\pgfqpoint{1.141930in}{2.221538in}}%
\pgfpathlineto{\pgfqpoint{1.188694in}{2.139167in}}%
\pgfpathlineto{\pgfqpoint{1.235458in}{2.060813in}}%
\pgfpathlineto{\pgfqpoint{1.282222in}{1.986282in}}%
\pgfpathlineto{\pgfqpoint{1.328985in}{1.915386in}}%
\pgfpathlineto{\pgfqpoint{1.375749in}{1.847948in}}%
\pgfpathlineto{\pgfqpoint{1.422513in}{1.783801in}}%
\pgfpathlineto{\pgfqpoint{1.469276in}{1.722782in}}%
\pgfpathlineto{\pgfqpoint{1.516040in}{1.664740in}}%
\pgfpathlineto{\pgfqpoint{1.562804in}{1.609529in}}%
\pgfpathlineto{\pgfqpoint{1.609568in}{1.557011in}}%
\pgfpathlineto{\pgfqpoint{1.656331in}{1.507056in}}%
\pgfpathlineto{\pgfqpoint{1.703095in}{1.459537in}}%
\pgfpathlineto{\pgfqpoint{1.749859in}{1.414336in}}%
\pgfpathlineto{\pgfqpoint{1.796622in}{1.371340in}}%
\pgfpathlineto{\pgfqpoint{1.843386in}{1.330441in}}%
\pgfpathlineto{\pgfqpoint{1.890150in}{1.291538in}}%
\pgfpathlineto{\pgfqpoint{1.936914in}{1.254532in}}%
\pgfpathlineto{\pgfqpoint{1.983677in}{1.219331in}}%
\pgfpathlineto{\pgfqpoint{2.030441in}{1.185848in}}%
\pgfpathlineto{\pgfqpoint{2.077205in}{1.153998in}}%
\pgfpathlineto{\pgfqpoint{2.123968in}{1.123701in}}%
\pgfpathlineto{\pgfqpoint{2.170732in}{1.094882in}}%
\pgfpathlineto{\pgfqpoint{2.217496in}{1.067469in}}%
\pgfpathlineto{\pgfqpoint{2.264260in}{1.041394in}}%
\pgfpathlineto{\pgfqpoint{2.311023in}{1.016590in}}%
\pgfpathlineto{\pgfqpoint{2.357787in}{0.992996in}}%
\pgfpathlineto{\pgfqpoint{2.404551in}{0.970554in}}%
\pgfpathlineto{\pgfqpoint{2.451314in}{0.949206in}}%
\pgfpathlineto{\pgfqpoint{2.498078in}{0.928899in}}%
\pgfpathlineto{\pgfqpoint{2.544842in}{0.909583in}}%
\pgfpathlineto{\pgfqpoint{2.591606in}{0.891209in}}%
\pgfpathlineto{\pgfqpoint{2.638369in}{0.873731in}}%
\pgfpathlineto{\pgfqpoint{2.685133in}{0.857106in}}%
\pgfpathlineto{\pgfqpoint{2.731897in}{0.841292in}}%
\pgfpathlineto{\pgfqpoint{2.778660in}{0.826250in}}%
\pgfpathlineto{\pgfqpoint{2.825424in}{0.811941in}}%
\pgfpathlineto{\pgfqpoint{2.872188in}{0.798330in}}%
\pgfpathlineto{\pgfqpoint{2.918952in}{0.785383in}}%
\pgfpathlineto{\pgfqpoint{2.965715in}{0.773068in}}%
\pgfpathlineto{\pgfqpoint{3.012479in}{0.761353in}}%
\pgfpathlineto{\pgfqpoint{3.059243in}{0.750210in}}%
\pgfpathlineto{\pgfqpoint{3.106006in}{0.739610in}}%
\pgfpathlineto{\pgfqpoint{3.152770in}{0.729528in}}%
\pgfpathlineto{\pgfqpoint{3.199534in}{0.719937in}}%
\pgfpathlineto{\pgfqpoint{3.246298in}{0.710814in}}%
\pgfpathlineto{\pgfqpoint{3.293061in}{0.702136in}}%
\pgfpathlineto{\pgfqpoint{3.339825in}{0.693882in}}%
\pgfpathlineto{\pgfqpoint{3.386589in}{0.686030in}}%
\pgfpathlineto{\pgfqpoint{3.433352in}{0.678561in}}%
\pgfpathlineto{\pgfqpoint{3.480116in}{0.671457in}}%
\pgfpathlineto{\pgfqpoint{3.526880in}{0.664699in}}%
\pgfpathlineto{\pgfqpoint{3.573644in}{0.658270in}}%
\pgfpathlineto{\pgfqpoint{3.620407in}{0.652156in}}%
\pgfpathlineto{\pgfqpoint{3.667171in}{0.646339in}}%
\pgfpathlineto{\pgfqpoint{3.713935in}{0.640807in}}%
\pgfpathlineto{\pgfqpoint{3.760698in}{0.635544in}}%
\pgfpathlineto{\pgfqpoint{3.807462in}{0.630538in}}%
\pgfpathlineto{\pgfqpoint{3.854226in}{0.625776in}}%
\pgfpathlineto{\pgfqpoint{3.900990in}{0.621246in}}%
\pgfpathlineto{\pgfqpoint{3.947753in}{0.616937in}}%
\pgfpathlineto{\pgfqpoint{3.994517in}{0.612839in}}%
\pgfpathlineto{\pgfqpoint{4.041281in}{0.608940in}}%
\pgfpathlineto{\pgfqpoint{4.088044in}{0.605232in}}%
\pgfpathlineto{\pgfqpoint{4.134808in}{0.601704in}}%
\pgfpathlineto{\pgfqpoint{4.181572in}{0.598349in}}%
\pgfpathlineto{\pgfqpoint{4.228336in}{0.595157in}}%
\pgfpathlineto{\pgfqpoint{4.275099in}{0.592121in}}%
\pgfpathlineto{\pgfqpoint{4.321863in}{0.589233in}}%
\pgfpathlineto{\pgfqpoint{4.368627in}{0.586486in}}%
\pgfpathlineto{\pgfqpoint{4.415390in}{0.583873in}}%
\pgfpathlineto{\pgfqpoint{4.462154in}{0.581388in}}%
\pgfpathlineto{\pgfqpoint{4.508918in}{0.579023in}}%
\pgfpathlineto{\pgfqpoint{4.555682in}{0.576774in}}%
\pgfpathlineto{\pgfqpoint{4.602445in}{0.574635in}}%
\pgfpathlineto{\pgfqpoint{4.649209in}{0.572600in}}%
\pgfpathlineto{\pgfqpoint{4.695973in}{0.570664in}}%
\pgfpathlineto{\pgfqpoint{4.742736in}{0.568823in}}%
\pgfpathlineto{\pgfqpoint{4.789500in}{0.567072in}}%
\pgfpathlineto{\pgfqpoint{4.836264in}{0.565406in}}%
\pgfpathlineto{\pgfqpoint{4.883027in}{0.563821in}}%
\pgfpathlineto{\pgfqpoint{4.929791in}{0.562313in}}%
\pgfpathlineto{\pgfqpoint{4.976555in}{0.560880in}}%
\pgfpathlineto{\pgfqpoint{5.023319in}{0.559516in}}%
\pgfpathlineto{\pgfqpoint{5.070082in}{0.558218in}}%
\pgfpathlineto{\pgfqpoint{5.116846in}{0.556984in}}%
\pgfpathlineto{\pgfqpoint{5.163610in}{0.555810in}}%
\pgfpathlineto{\pgfqpoint{5.210373in}{0.554693in}}%
\pgfpathlineto{\pgfqpoint{5.257137in}{0.553631in}}%
\pgfpathlineto{\pgfqpoint{5.303901in}{0.553631in}}%
\pgfusepath{stroke}%
\end{pgfscope}%
\begin{pgfscope}%
\pgfpathrectangle{\pgfqpoint{0.580766in}{0.532919in}}{\pgfqpoint{5.611646in}{3.193122in}} %
\pgfusepath{clip}%
\pgfsetbuttcap%
\pgfsetroundjoin%
\pgfsetlinewidth{0.501875pt}%
\definecolor{currentstroke}{rgb}{0.000000,0.000000,0.000000}%
\pgfsetstrokecolor{currentstroke}%
\pgfsetdash{{1.000000pt}{3.000000pt}}{0.000000pt}%
\pgfpathmoveto{\pgfqpoint{0.580766in}{0.532919in}}%
\pgfpathlineto{\pgfqpoint{0.580766in}{3.726041in}}%
\pgfusepath{stroke}%
\end{pgfscope}%
\begin{pgfscope}%
\pgfsetbuttcap%
\pgfsetroundjoin%
\definecolor{currentfill}{rgb}{0.000000,0.000000,0.000000}%
\pgfsetfillcolor{currentfill}%
\pgfsetlinewidth{0.501875pt}%
\definecolor{currentstroke}{rgb}{0.000000,0.000000,0.000000}%
\pgfsetstrokecolor{currentstroke}%
\pgfsetdash{}{0pt}%
\pgfsys@defobject{currentmarker}{\pgfqpoint{0.000000in}{0.000000in}}{\pgfqpoint{0.000000in}{0.055556in}}{%
\pgfpathmoveto{\pgfqpoint{0.000000in}{0.000000in}}%
\pgfpathlineto{\pgfqpoint{0.000000in}{0.055556in}}%
\pgfusepath{stroke,fill}%
}%
\begin{pgfscope}%
\pgfsys@transformshift{0.580766in}{0.532919in}%
\pgfsys@useobject{currentmarker}{}%
\end{pgfscope}%
\end{pgfscope}%
\begin{pgfscope}%
\pgfsetbuttcap%
\pgfsetroundjoin%
\definecolor{currentfill}{rgb}{0.000000,0.000000,0.000000}%
\pgfsetfillcolor{currentfill}%
\pgfsetlinewidth{0.501875pt}%
\definecolor{currentstroke}{rgb}{0.000000,0.000000,0.000000}%
\pgfsetstrokecolor{currentstroke}%
\pgfsetdash{}{0pt}%
\pgfsys@defobject{currentmarker}{\pgfqpoint{0.000000in}{-0.055556in}}{\pgfqpoint{0.000000in}{0.000000in}}{%
\pgfpathmoveto{\pgfqpoint{0.000000in}{0.000000in}}%
\pgfpathlineto{\pgfqpoint{0.000000in}{-0.055556in}}%
\pgfusepath{stroke,fill}%
}%
\begin{pgfscope}%
\pgfsys@transformshift{0.580766in}{3.726041in}%
\pgfsys@useobject{currentmarker}{}%
\end{pgfscope}%
\end{pgfscope}%
\begin{pgfscope}%
\pgftext[x=0.580766in,y=0.477363in,,top]{{\rmfamily\fontsize{10.000000}{12.000000}\selectfont \(\displaystyle 0.000\)}}%
\end{pgfscope}%
\begin{pgfscope}%
\pgfpathrectangle{\pgfqpoint{0.580766in}{0.532919in}}{\pgfqpoint{5.611646in}{3.193122in}} %
\pgfusepath{clip}%
\pgfsetbuttcap%
\pgfsetroundjoin%
\pgfsetlinewidth{0.501875pt}%
\definecolor{currentstroke}{rgb}{0.000000,0.000000,0.000000}%
\pgfsetstrokecolor{currentstroke}%
\pgfsetdash{{1.000000pt}{3.000000pt}}{0.000000pt}%
\pgfpathmoveto{\pgfqpoint{1.516040in}{0.532919in}}%
\pgfpathlineto{\pgfqpoint{1.516040in}{3.726041in}}%
\pgfusepath{stroke}%
\end{pgfscope}%
\begin{pgfscope}%
\pgfsetbuttcap%
\pgfsetroundjoin%
\definecolor{currentfill}{rgb}{0.000000,0.000000,0.000000}%
\pgfsetfillcolor{currentfill}%
\pgfsetlinewidth{0.501875pt}%
\definecolor{currentstroke}{rgb}{0.000000,0.000000,0.000000}%
\pgfsetstrokecolor{currentstroke}%
\pgfsetdash{}{0pt}%
\pgfsys@defobject{currentmarker}{\pgfqpoint{0.000000in}{0.000000in}}{\pgfqpoint{0.000000in}{0.055556in}}{%
\pgfpathmoveto{\pgfqpoint{0.000000in}{0.000000in}}%
\pgfpathlineto{\pgfqpoint{0.000000in}{0.055556in}}%
\pgfusepath{stroke,fill}%
}%
\begin{pgfscope}%
\pgfsys@transformshift{1.516040in}{0.532919in}%
\pgfsys@useobject{currentmarker}{}%
\end{pgfscope}%
\end{pgfscope}%
\begin{pgfscope}%
\pgfsetbuttcap%
\pgfsetroundjoin%
\definecolor{currentfill}{rgb}{0.000000,0.000000,0.000000}%
\pgfsetfillcolor{currentfill}%
\pgfsetlinewidth{0.501875pt}%
\definecolor{currentstroke}{rgb}{0.000000,0.000000,0.000000}%
\pgfsetstrokecolor{currentstroke}%
\pgfsetdash{}{0pt}%
\pgfsys@defobject{currentmarker}{\pgfqpoint{0.000000in}{-0.055556in}}{\pgfqpoint{0.000000in}{0.000000in}}{%
\pgfpathmoveto{\pgfqpoint{0.000000in}{0.000000in}}%
\pgfpathlineto{\pgfqpoint{0.000000in}{-0.055556in}}%
\pgfusepath{stroke,fill}%
}%
\begin{pgfscope}%
\pgfsys@transformshift{1.516040in}{3.726041in}%
\pgfsys@useobject{currentmarker}{}%
\end{pgfscope}%
\end{pgfscope}%
\begin{pgfscope}%
\pgftext[x=1.516040in,y=0.477363in,,top]{{\rmfamily\fontsize{10.000000}{12.000000}\selectfont \(\displaystyle 0.002\)}}%
\end{pgfscope}%
\begin{pgfscope}%
\pgfpathrectangle{\pgfqpoint{0.580766in}{0.532919in}}{\pgfqpoint{5.611646in}{3.193122in}} %
\pgfusepath{clip}%
\pgfsetbuttcap%
\pgfsetroundjoin%
\pgfsetlinewidth{0.501875pt}%
\definecolor{currentstroke}{rgb}{0.000000,0.000000,0.000000}%
\pgfsetstrokecolor{currentstroke}%
\pgfsetdash{{1.000000pt}{3.000000pt}}{0.000000pt}%
\pgfpathmoveto{\pgfqpoint{2.451314in}{0.532919in}}%
\pgfpathlineto{\pgfqpoint{2.451314in}{3.726041in}}%
\pgfusepath{stroke}%
\end{pgfscope}%
\begin{pgfscope}%
\pgfsetbuttcap%
\pgfsetroundjoin%
\definecolor{currentfill}{rgb}{0.000000,0.000000,0.000000}%
\pgfsetfillcolor{currentfill}%
\pgfsetlinewidth{0.501875pt}%
\definecolor{currentstroke}{rgb}{0.000000,0.000000,0.000000}%
\pgfsetstrokecolor{currentstroke}%
\pgfsetdash{}{0pt}%
\pgfsys@defobject{currentmarker}{\pgfqpoint{0.000000in}{0.000000in}}{\pgfqpoint{0.000000in}{0.055556in}}{%
\pgfpathmoveto{\pgfqpoint{0.000000in}{0.000000in}}%
\pgfpathlineto{\pgfqpoint{0.000000in}{0.055556in}}%
\pgfusepath{stroke,fill}%
}%
\begin{pgfscope}%
\pgfsys@transformshift{2.451314in}{0.532919in}%
\pgfsys@useobject{currentmarker}{}%
\end{pgfscope}%
\end{pgfscope}%
\begin{pgfscope}%
\pgfsetbuttcap%
\pgfsetroundjoin%
\definecolor{currentfill}{rgb}{0.000000,0.000000,0.000000}%
\pgfsetfillcolor{currentfill}%
\pgfsetlinewidth{0.501875pt}%
\definecolor{currentstroke}{rgb}{0.000000,0.000000,0.000000}%
\pgfsetstrokecolor{currentstroke}%
\pgfsetdash{}{0pt}%
\pgfsys@defobject{currentmarker}{\pgfqpoint{0.000000in}{-0.055556in}}{\pgfqpoint{0.000000in}{0.000000in}}{%
\pgfpathmoveto{\pgfqpoint{0.000000in}{0.000000in}}%
\pgfpathlineto{\pgfqpoint{0.000000in}{-0.055556in}}%
\pgfusepath{stroke,fill}%
}%
\begin{pgfscope}%
\pgfsys@transformshift{2.451314in}{3.726041in}%
\pgfsys@useobject{currentmarker}{}%
\end{pgfscope}%
\end{pgfscope}%
\begin{pgfscope}%
\pgftext[x=2.451314in,y=0.477363in,,top]{{\rmfamily\fontsize{10.000000}{12.000000}\selectfont \(\displaystyle 0.004\)}}%
\end{pgfscope}%
\begin{pgfscope}%
\pgfpathrectangle{\pgfqpoint{0.580766in}{0.532919in}}{\pgfqpoint{5.611646in}{3.193122in}} %
\pgfusepath{clip}%
\pgfsetbuttcap%
\pgfsetroundjoin%
\pgfsetlinewidth{0.501875pt}%
\definecolor{currentstroke}{rgb}{0.000000,0.000000,0.000000}%
\pgfsetstrokecolor{currentstroke}%
\pgfsetdash{{1.000000pt}{3.000000pt}}{0.000000pt}%
\pgfpathmoveto{\pgfqpoint{3.386589in}{0.532919in}}%
\pgfpathlineto{\pgfqpoint{3.386589in}{3.726041in}}%
\pgfusepath{stroke}%
\end{pgfscope}%
\begin{pgfscope}%
\pgfsetbuttcap%
\pgfsetroundjoin%
\definecolor{currentfill}{rgb}{0.000000,0.000000,0.000000}%
\pgfsetfillcolor{currentfill}%
\pgfsetlinewidth{0.501875pt}%
\definecolor{currentstroke}{rgb}{0.000000,0.000000,0.000000}%
\pgfsetstrokecolor{currentstroke}%
\pgfsetdash{}{0pt}%
\pgfsys@defobject{currentmarker}{\pgfqpoint{0.000000in}{0.000000in}}{\pgfqpoint{0.000000in}{0.055556in}}{%
\pgfpathmoveto{\pgfqpoint{0.000000in}{0.000000in}}%
\pgfpathlineto{\pgfqpoint{0.000000in}{0.055556in}}%
\pgfusepath{stroke,fill}%
}%
\begin{pgfscope}%
\pgfsys@transformshift{3.386589in}{0.532919in}%
\pgfsys@useobject{currentmarker}{}%
\end{pgfscope}%
\end{pgfscope}%
\begin{pgfscope}%
\pgfsetbuttcap%
\pgfsetroundjoin%
\definecolor{currentfill}{rgb}{0.000000,0.000000,0.000000}%
\pgfsetfillcolor{currentfill}%
\pgfsetlinewidth{0.501875pt}%
\definecolor{currentstroke}{rgb}{0.000000,0.000000,0.000000}%
\pgfsetstrokecolor{currentstroke}%
\pgfsetdash{}{0pt}%
\pgfsys@defobject{currentmarker}{\pgfqpoint{0.000000in}{-0.055556in}}{\pgfqpoint{0.000000in}{0.000000in}}{%
\pgfpathmoveto{\pgfqpoint{0.000000in}{0.000000in}}%
\pgfpathlineto{\pgfqpoint{0.000000in}{-0.055556in}}%
\pgfusepath{stroke,fill}%
}%
\begin{pgfscope}%
\pgfsys@transformshift{3.386589in}{3.726041in}%
\pgfsys@useobject{currentmarker}{}%
\end{pgfscope}%
\end{pgfscope}%
\begin{pgfscope}%
\pgftext[x=3.386589in,y=0.477363in,,top]{{\rmfamily\fontsize{10.000000}{12.000000}\selectfont \(\displaystyle 0.006\)}}%
\end{pgfscope}%
\begin{pgfscope}%
\pgfpathrectangle{\pgfqpoint{0.580766in}{0.532919in}}{\pgfqpoint{5.611646in}{3.193122in}} %
\pgfusepath{clip}%
\pgfsetbuttcap%
\pgfsetroundjoin%
\pgfsetlinewidth{0.501875pt}%
\definecolor{currentstroke}{rgb}{0.000000,0.000000,0.000000}%
\pgfsetstrokecolor{currentstroke}%
\pgfsetdash{{1.000000pt}{3.000000pt}}{0.000000pt}%
\pgfpathmoveto{\pgfqpoint{4.321863in}{0.532919in}}%
\pgfpathlineto{\pgfqpoint{4.321863in}{3.726041in}}%
\pgfusepath{stroke}%
\end{pgfscope}%
\begin{pgfscope}%
\pgfsetbuttcap%
\pgfsetroundjoin%
\definecolor{currentfill}{rgb}{0.000000,0.000000,0.000000}%
\pgfsetfillcolor{currentfill}%
\pgfsetlinewidth{0.501875pt}%
\definecolor{currentstroke}{rgb}{0.000000,0.000000,0.000000}%
\pgfsetstrokecolor{currentstroke}%
\pgfsetdash{}{0pt}%
\pgfsys@defobject{currentmarker}{\pgfqpoint{0.000000in}{0.000000in}}{\pgfqpoint{0.000000in}{0.055556in}}{%
\pgfpathmoveto{\pgfqpoint{0.000000in}{0.000000in}}%
\pgfpathlineto{\pgfqpoint{0.000000in}{0.055556in}}%
\pgfusepath{stroke,fill}%
}%
\begin{pgfscope}%
\pgfsys@transformshift{4.321863in}{0.532919in}%
\pgfsys@useobject{currentmarker}{}%
\end{pgfscope}%
\end{pgfscope}%
\begin{pgfscope}%
\pgfsetbuttcap%
\pgfsetroundjoin%
\definecolor{currentfill}{rgb}{0.000000,0.000000,0.000000}%
\pgfsetfillcolor{currentfill}%
\pgfsetlinewidth{0.501875pt}%
\definecolor{currentstroke}{rgb}{0.000000,0.000000,0.000000}%
\pgfsetstrokecolor{currentstroke}%
\pgfsetdash{}{0pt}%
\pgfsys@defobject{currentmarker}{\pgfqpoint{0.000000in}{-0.055556in}}{\pgfqpoint{0.000000in}{0.000000in}}{%
\pgfpathmoveto{\pgfqpoint{0.000000in}{0.000000in}}%
\pgfpathlineto{\pgfqpoint{0.000000in}{-0.055556in}}%
\pgfusepath{stroke,fill}%
}%
\begin{pgfscope}%
\pgfsys@transformshift{4.321863in}{3.726041in}%
\pgfsys@useobject{currentmarker}{}%
\end{pgfscope}%
\end{pgfscope}%
\begin{pgfscope}%
\pgftext[x=4.321863in,y=0.477363in,,top]{{\rmfamily\fontsize{10.000000}{12.000000}\selectfont \(\displaystyle 0.008\)}}%
\end{pgfscope}%
\begin{pgfscope}%
\pgfpathrectangle{\pgfqpoint{0.580766in}{0.532919in}}{\pgfqpoint{5.611646in}{3.193122in}} %
\pgfusepath{clip}%
\pgfsetbuttcap%
\pgfsetroundjoin%
\pgfsetlinewidth{0.501875pt}%
\definecolor{currentstroke}{rgb}{0.000000,0.000000,0.000000}%
\pgfsetstrokecolor{currentstroke}%
\pgfsetdash{{1.000000pt}{3.000000pt}}{0.000000pt}%
\pgfpathmoveto{\pgfqpoint{5.257137in}{0.532919in}}%
\pgfpathlineto{\pgfqpoint{5.257137in}{3.726041in}}%
\pgfusepath{stroke}%
\end{pgfscope}%
\begin{pgfscope}%
\pgfsetbuttcap%
\pgfsetroundjoin%
\definecolor{currentfill}{rgb}{0.000000,0.000000,0.000000}%
\pgfsetfillcolor{currentfill}%
\pgfsetlinewidth{0.501875pt}%
\definecolor{currentstroke}{rgb}{0.000000,0.000000,0.000000}%
\pgfsetstrokecolor{currentstroke}%
\pgfsetdash{}{0pt}%
\pgfsys@defobject{currentmarker}{\pgfqpoint{0.000000in}{0.000000in}}{\pgfqpoint{0.000000in}{0.055556in}}{%
\pgfpathmoveto{\pgfqpoint{0.000000in}{0.000000in}}%
\pgfpathlineto{\pgfqpoint{0.000000in}{0.055556in}}%
\pgfusepath{stroke,fill}%
}%
\begin{pgfscope}%
\pgfsys@transformshift{5.257137in}{0.532919in}%
\pgfsys@useobject{currentmarker}{}%
\end{pgfscope}%
\end{pgfscope}%
\begin{pgfscope}%
\pgfsetbuttcap%
\pgfsetroundjoin%
\definecolor{currentfill}{rgb}{0.000000,0.000000,0.000000}%
\pgfsetfillcolor{currentfill}%
\pgfsetlinewidth{0.501875pt}%
\definecolor{currentstroke}{rgb}{0.000000,0.000000,0.000000}%
\pgfsetstrokecolor{currentstroke}%
\pgfsetdash{}{0pt}%
\pgfsys@defobject{currentmarker}{\pgfqpoint{0.000000in}{-0.055556in}}{\pgfqpoint{0.000000in}{0.000000in}}{%
\pgfpathmoveto{\pgfqpoint{0.000000in}{0.000000in}}%
\pgfpathlineto{\pgfqpoint{0.000000in}{-0.055556in}}%
\pgfusepath{stroke,fill}%
}%
\begin{pgfscope}%
\pgfsys@transformshift{5.257137in}{3.726041in}%
\pgfsys@useobject{currentmarker}{}%
\end{pgfscope}%
\end{pgfscope}%
\begin{pgfscope}%
\pgftext[x=5.257137in,y=0.477363in,,top]{{\rmfamily\fontsize{10.000000}{12.000000}\selectfont \(\displaystyle 0.010\)}}%
\end{pgfscope}%
\begin{pgfscope}%
\pgfpathrectangle{\pgfqpoint{0.580766in}{0.532919in}}{\pgfqpoint{5.611646in}{3.193122in}} %
\pgfusepath{clip}%
\pgfsetbuttcap%
\pgfsetroundjoin%
\pgfsetlinewidth{0.501875pt}%
\definecolor{currentstroke}{rgb}{0.000000,0.000000,0.000000}%
\pgfsetstrokecolor{currentstroke}%
\pgfsetdash{{1.000000pt}{3.000000pt}}{0.000000pt}%
\pgfpathmoveto{\pgfqpoint{6.192411in}{0.532919in}}%
\pgfpathlineto{\pgfqpoint{6.192411in}{3.726041in}}%
\pgfusepath{stroke}%
\end{pgfscope}%
\begin{pgfscope}%
\pgfsetbuttcap%
\pgfsetroundjoin%
\definecolor{currentfill}{rgb}{0.000000,0.000000,0.000000}%
\pgfsetfillcolor{currentfill}%
\pgfsetlinewidth{0.501875pt}%
\definecolor{currentstroke}{rgb}{0.000000,0.000000,0.000000}%
\pgfsetstrokecolor{currentstroke}%
\pgfsetdash{}{0pt}%
\pgfsys@defobject{currentmarker}{\pgfqpoint{0.000000in}{0.000000in}}{\pgfqpoint{0.000000in}{0.055556in}}{%
\pgfpathmoveto{\pgfqpoint{0.000000in}{0.000000in}}%
\pgfpathlineto{\pgfqpoint{0.000000in}{0.055556in}}%
\pgfusepath{stroke,fill}%
}%
\begin{pgfscope}%
\pgfsys@transformshift{6.192411in}{0.532919in}%
\pgfsys@useobject{currentmarker}{}%
\end{pgfscope}%
\end{pgfscope}%
\begin{pgfscope}%
\pgfsetbuttcap%
\pgfsetroundjoin%
\definecolor{currentfill}{rgb}{0.000000,0.000000,0.000000}%
\pgfsetfillcolor{currentfill}%
\pgfsetlinewidth{0.501875pt}%
\definecolor{currentstroke}{rgb}{0.000000,0.000000,0.000000}%
\pgfsetstrokecolor{currentstroke}%
\pgfsetdash{}{0pt}%
\pgfsys@defobject{currentmarker}{\pgfqpoint{0.000000in}{-0.055556in}}{\pgfqpoint{0.000000in}{0.000000in}}{%
\pgfpathmoveto{\pgfqpoint{0.000000in}{0.000000in}}%
\pgfpathlineto{\pgfqpoint{0.000000in}{-0.055556in}}%
\pgfusepath{stroke,fill}%
}%
\begin{pgfscope}%
\pgfsys@transformshift{6.192411in}{3.726041in}%
\pgfsys@useobject{currentmarker}{}%
\end{pgfscope}%
\end{pgfscope}%
\begin{pgfscope}%
\pgftext[x=6.192411in,y=0.477363in,,top]{{\rmfamily\fontsize{10.000000}{12.000000}\selectfont \(\displaystyle 0.012\)}}%
\end{pgfscope}%
\begin{pgfscope}%
\pgftext[x=3.386589in,y=0.284462in,,top]{{\rmfamily\fontsize{10.000000}{12.000000}\selectfont time (s)}}%
\end{pgfscope}%
\begin{pgfscope}%
\pgfpathrectangle{\pgfqpoint{0.580766in}{0.532919in}}{\pgfqpoint{5.611646in}{3.193122in}} %
\pgfusepath{clip}%
\pgfsetbuttcap%
\pgfsetroundjoin%
\pgfsetlinewidth{0.501875pt}%
\definecolor{currentstroke}{rgb}{0.000000,0.000000,0.000000}%
\pgfsetstrokecolor{currentstroke}%
\pgfsetdash{{1.000000pt}{3.000000pt}}{0.000000pt}%
\pgfpathmoveto{\pgfqpoint{0.580766in}{0.532919in}}%
\pgfpathlineto{\pgfqpoint{6.192411in}{0.532919in}}%
\pgfusepath{stroke}%
\end{pgfscope}%
\begin{pgfscope}%
\pgfsetbuttcap%
\pgfsetroundjoin%
\definecolor{currentfill}{rgb}{0.000000,0.000000,0.000000}%
\pgfsetfillcolor{currentfill}%
\pgfsetlinewidth{0.501875pt}%
\definecolor{currentstroke}{rgb}{0.000000,0.000000,0.000000}%
\pgfsetstrokecolor{currentstroke}%
\pgfsetdash{}{0pt}%
\pgfsys@defobject{currentmarker}{\pgfqpoint{0.000000in}{0.000000in}}{\pgfqpoint{0.055556in}{0.000000in}}{%
\pgfpathmoveto{\pgfqpoint{0.000000in}{0.000000in}}%
\pgfpathlineto{\pgfqpoint{0.055556in}{0.000000in}}%
\pgfusepath{stroke,fill}%
}%
\begin{pgfscope}%
\pgfsys@transformshift{0.580766in}{0.532919in}%
\pgfsys@useobject{currentmarker}{}%
\end{pgfscope}%
\end{pgfscope}%
\begin{pgfscope}%
\pgfsetbuttcap%
\pgfsetroundjoin%
\definecolor{currentfill}{rgb}{0.000000,0.000000,0.000000}%
\pgfsetfillcolor{currentfill}%
\pgfsetlinewidth{0.501875pt}%
\definecolor{currentstroke}{rgb}{0.000000,0.000000,0.000000}%
\pgfsetstrokecolor{currentstroke}%
\pgfsetdash{}{0pt}%
\pgfsys@defobject{currentmarker}{\pgfqpoint{-0.055556in}{0.000000in}}{\pgfqpoint{0.000000in}{0.000000in}}{%
\pgfpathmoveto{\pgfqpoint{0.000000in}{0.000000in}}%
\pgfpathlineto{\pgfqpoint{-0.055556in}{0.000000in}}%
\pgfusepath{stroke,fill}%
}%
\begin{pgfscope}%
\pgfsys@transformshift{6.192411in}{0.532919in}%
\pgfsys@useobject{currentmarker}{}%
\end{pgfscope}%
\end{pgfscope}%
\begin{pgfscope}%
\pgftext[x=0.525210in,y=0.532919in,right,]{{\rmfamily\fontsize{10.000000}{12.000000}\selectfont \(\displaystyle 0\)}}%
\end{pgfscope}%
\begin{pgfscope}%
\pgfpathrectangle{\pgfqpoint{0.580766in}{0.532919in}}{\pgfqpoint{5.611646in}{3.193122in}} %
\pgfusepath{clip}%
\pgfsetbuttcap%
\pgfsetroundjoin%
\pgfsetlinewidth{0.501875pt}%
\definecolor{currentstroke}{rgb}{0.000000,0.000000,0.000000}%
\pgfsetstrokecolor{currentstroke}%
\pgfsetdash{{1.000000pt}{3.000000pt}}{0.000000pt}%
\pgfpathmoveto{\pgfqpoint{0.580766in}{1.171543in}}%
\pgfpathlineto{\pgfqpoint{6.192411in}{1.171543in}}%
\pgfusepath{stroke}%
\end{pgfscope}%
\begin{pgfscope}%
\pgfsetbuttcap%
\pgfsetroundjoin%
\definecolor{currentfill}{rgb}{0.000000,0.000000,0.000000}%
\pgfsetfillcolor{currentfill}%
\pgfsetlinewidth{0.501875pt}%
\definecolor{currentstroke}{rgb}{0.000000,0.000000,0.000000}%
\pgfsetstrokecolor{currentstroke}%
\pgfsetdash{}{0pt}%
\pgfsys@defobject{currentmarker}{\pgfqpoint{0.000000in}{0.000000in}}{\pgfqpoint{0.055556in}{0.000000in}}{%
\pgfpathmoveto{\pgfqpoint{0.000000in}{0.000000in}}%
\pgfpathlineto{\pgfqpoint{0.055556in}{0.000000in}}%
\pgfusepath{stroke,fill}%
}%
\begin{pgfscope}%
\pgfsys@transformshift{0.580766in}{1.171543in}%
\pgfsys@useobject{currentmarker}{}%
\end{pgfscope}%
\end{pgfscope}%
\begin{pgfscope}%
\pgfsetbuttcap%
\pgfsetroundjoin%
\definecolor{currentfill}{rgb}{0.000000,0.000000,0.000000}%
\pgfsetfillcolor{currentfill}%
\pgfsetlinewidth{0.501875pt}%
\definecolor{currentstroke}{rgb}{0.000000,0.000000,0.000000}%
\pgfsetstrokecolor{currentstroke}%
\pgfsetdash{}{0pt}%
\pgfsys@defobject{currentmarker}{\pgfqpoint{-0.055556in}{0.000000in}}{\pgfqpoint{0.000000in}{0.000000in}}{%
\pgfpathmoveto{\pgfqpoint{0.000000in}{0.000000in}}%
\pgfpathlineto{\pgfqpoint{-0.055556in}{0.000000in}}%
\pgfusepath{stroke,fill}%
}%
\begin{pgfscope}%
\pgfsys@transformshift{6.192411in}{1.171543in}%
\pgfsys@useobject{currentmarker}{}%
\end{pgfscope}%
\end{pgfscope}%
\begin{pgfscope}%
\pgftext[x=0.525210in,y=1.171543in,right,]{{\rmfamily\fontsize{10.000000}{12.000000}\selectfont \(\displaystyle 2\)}}%
\end{pgfscope}%
\begin{pgfscope}%
\pgfpathrectangle{\pgfqpoint{0.580766in}{0.532919in}}{\pgfqpoint{5.611646in}{3.193122in}} %
\pgfusepath{clip}%
\pgfsetbuttcap%
\pgfsetroundjoin%
\pgfsetlinewidth{0.501875pt}%
\definecolor{currentstroke}{rgb}{0.000000,0.000000,0.000000}%
\pgfsetstrokecolor{currentstroke}%
\pgfsetdash{{1.000000pt}{3.000000pt}}{0.000000pt}%
\pgfpathmoveto{\pgfqpoint{0.580766in}{1.810167in}}%
\pgfpathlineto{\pgfqpoint{6.192411in}{1.810167in}}%
\pgfusepath{stroke}%
\end{pgfscope}%
\begin{pgfscope}%
\pgfsetbuttcap%
\pgfsetroundjoin%
\definecolor{currentfill}{rgb}{0.000000,0.000000,0.000000}%
\pgfsetfillcolor{currentfill}%
\pgfsetlinewidth{0.501875pt}%
\definecolor{currentstroke}{rgb}{0.000000,0.000000,0.000000}%
\pgfsetstrokecolor{currentstroke}%
\pgfsetdash{}{0pt}%
\pgfsys@defobject{currentmarker}{\pgfqpoint{0.000000in}{0.000000in}}{\pgfqpoint{0.055556in}{0.000000in}}{%
\pgfpathmoveto{\pgfqpoint{0.000000in}{0.000000in}}%
\pgfpathlineto{\pgfqpoint{0.055556in}{0.000000in}}%
\pgfusepath{stroke,fill}%
}%
\begin{pgfscope}%
\pgfsys@transformshift{0.580766in}{1.810167in}%
\pgfsys@useobject{currentmarker}{}%
\end{pgfscope}%
\end{pgfscope}%
\begin{pgfscope}%
\pgfsetbuttcap%
\pgfsetroundjoin%
\definecolor{currentfill}{rgb}{0.000000,0.000000,0.000000}%
\pgfsetfillcolor{currentfill}%
\pgfsetlinewidth{0.501875pt}%
\definecolor{currentstroke}{rgb}{0.000000,0.000000,0.000000}%
\pgfsetstrokecolor{currentstroke}%
\pgfsetdash{}{0pt}%
\pgfsys@defobject{currentmarker}{\pgfqpoint{-0.055556in}{0.000000in}}{\pgfqpoint{0.000000in}{0.000000in}}{%
\pgfpathmoveto{\pgfqpoint{0.000000in}{0.000000in}}%
\pgfpathlineto{\pgfqpoint{-0.055556in}{0.000000in}}%
\pgfusepath{stroke,fill}%
}%
\begin{pgfscope}%
\pgfsys@transformshift{6.192411in}{1.810167in}%
\pgfsys@useobject{currentmarker}{}%
\end{pgfscope}%
\end{pgfscope}%
\begin{pgfscope}%
\pgftext[x=0.525210in,y=1.810167in,right,]{{\rmfamily\fontsize{10.000000}{12.000000}\selectfont \(\displaystyle 4\)}}%
\end{pgfscope}%
\begin{pgfscope}%
\pgfpathrectangle{\pgfqpoint{0.580766in}{0.532919in}}{\pgfqpoint{5.611646in}{3.193122in}} %
\pgfusepath{clip}%
\pgfsetbuttcap%
\pgfsetroundjoin%
\pgfsetlinewidth{0.501875pt}%
\definecolor{currentstroke}{rgb}{0.000000,0.000000,0.000000}%
\pgfsetstrokecolor{currentstroke}%
\pgfsetdash{{1.000000pt}{3.000000pt}}{0.000000pt}%
\pgfpathmoveto{\pgfqpoint{0.580766in}{2.448792in}}%
\pgfpathlineto{\pgfqpoint{6.192411in}{2.448792in}}%
\pgfusepath{stroke}%
\end{pgfscope}%
\begin{pgfscope}%
\pgfsetbuttcap%
\pgfsetroundjoin%
\definecolor{currentfill}{rgb}{0.000000,0.000000,0.000000}%
\pgfsetfillcolor{currentfill}%
\pgfsetlinewidth{0.501875pt}%
\definecolor{currentstroke}{rgb}{0.000000,0.000000,0.000000}%
\pgfsetstrokecolor{currentstroke}%
\pgfsetdash{}{0pt}%
\pgfsys@defobject{currentmarker}{\pgfqpoint{0.000000in}{0.000000in}}{\pgfqpoint{0.055556in}{0.000000in}}{%
\pgfpathmoveto{\pgfqpoint{0.000000in}{0.000000in}}%
\pgfpathlineto{\pgfqpoint{0.055556in}{0.000000in}}%
\pgfusepath{stroke,fill}%
}%
\begin{pgfscope}%
\pgfsys@transformshift{0.580766in}{2.448792in}%
\pgfsys@useobject{currentmarker}{}%
\end{pgfscope}%
\end{pgfscope}%
\begin{pgfscope}%
\pgfsetbuttcap%
\pgfsetroundjoin%
\definecolor{currentfill}{rgb}{0.000000,0.000000,0.000000}%
\pgfsetfillcolor{currentfill}%
\pgfsetlinewidth{0.501875pt}%
\definecolor{currentstroke}{rgb}{0.000000,0.000000,0.000000}%
\pgfsetstrokecolor{currentstroke}%
\pgfsetdash{}{0pt}%
\pgfsys@defobject{currentmarker}{\pgfqpoint{-0.055556in}{0.000000in}}{\pgfqpoint{0.000000in}{0.000000in}}{%
\pgfpathmoveto{\pgfqpoint{0.000000in}{0.000000in}}%
\pgfpathlineto{\pgfqpoint{-0.055556in}{0.000000in}}%
\pgfusepath{stroke,fill}%
}%
\begin{pgfscope}%
\pgfsys@transformshift{6.192411in}{2.448792in}%
\pgfsys@useobject{currentmarker}{}%
\end{pgfscope}%
\end{pgfscope}%
\begin{pgfscope}%
\pgftext[x=0.525210in,y=2.448792in,right,]{{\rmfamily\fontsize{10.000000}{12.000000}\selectfont \(\displaystyle 6\)}}%
\end{pgfscope}%
\begin{pgfscope}%
\pgfpathrectangle{\pgfqpoint{0.580766in}{0.532919in}}{\pgfqpoint{5.611646in}{3.193122in}} %
\pgfusepath{clip}%
\pgfsetbuttcap%
\pgfsetroundjoin%
\pgfsetlinewidth{0.501875pt}%
\definecolor{currentstroke}{rgb}{0.000000,0.000000,0.000000}%
\pgfsetstrokecolor{currentstroke}%
\pgfsetdash{{1.000000pt}{3.000000pt}}{0.000000pt}%
\pgfpathmoveto{\pgfqpoint{0.580766in}{3.087416in}}%
\pgfpathlineto{\pgfqpoint{6.192411in}{3.087416in}}%
\pgfusepath{stroke}%
\end{pgfscope}%
\begin{pgfscope}%
\pgfsetbuttcap%
\pgfsetroundjoin%
\definecolor{currentfill}{rgb}{0.000000,0.000000,0.000000}%
\pgfsetfillcolor{currentfill}%
\pgfsetlinewidth{0.501875pt}%
\definecolor{currentstroke}{rgb}{0.000000,0.000000,0.000000}%
\pgfsetstrokecolor{currentstroke}%
\pgfsetdash{}{0pt}%
\pgfsys@defobject{currentmarker}{\pgfqpoint{0.000000in}{0.000000in}}{\pgfqpoint{0.055556in}{0.000000in}}{%
\pgfpathmoveto{\pgfqpoint{0.000000in}{0.000000in}}%
\pgfpathlineto{\pgfqpoint{0.055556in}{0.000000in}}%
\pgfusepath{stroke,fill}%
}%
\begin{pgfscope}%
\pgfsys@transformshift{0.580766in}{3.087416in}%
\pgfsys@useobject{currentmarker}{}%
\end{pgfscope}%
\end{pgfscope}%
\begin{pgfscope}%
\pgfsetbuttcap%
\pgfsetroundjoin%
\definecolor{currentfill}{rgb}{0.000000,0.000000,0.000000}%
\pgfsetfillcolor{currentfill}%
\pgfsetlinewidth{0.501875pt}%
\definecolor{currentstroke}{rgb}{0.000000,0.000000,0.000000}%
\pgfsetstrokecolor{currentstroke}%
\pgfsetdash{}{0pt}%
\pgfsys@defobject{currentmarker}{\pgfqpoint{-0.055556in}{0.000000in}}{\pgfqpoint{0.000000in}{0.000000in}}{%
\pgfpathmoveto{\pgfqpoint{0.000000in}{0.000000in}}%
\pgfpathlineto{\pgfqpoint{-0.055556in}{0.000000in}}%
\pgfusepath{stroke,fill}%
}%
\begin{pgfscope}%
\pgfsys@transformshift{6.192411in}{3.087416in}%
\pgfsys@useobject{currentmarker}{}%
\end{pgfscope}%
\end{pgfscope}%
\begin{pgfscope}%
\pgftext[x=0.525210in,y=3.087416in,right,]{{\rmfamily\fontsize{10.000000}{12.000000}\selectfont \(\displaystyle 8\)}}%
\end{pgfscope}%
\begin{pgfscope}%
\pgfpathrectangle{\pgfqpoint{0.580766in}{0.532919in}}{\pgfqpoint{5.611646in}{3.193122in}} %
\pgfusepath{clip}%
\pgfsetbuttcap%
\pgfsetroundjoin%
\pgfsetlinewidth{0.501875pt}%
\definecolor{currentstroke}{rgb}{0.000000,0.000000,0.000000}%
\pgfsetstrokecolor{currentstroke}%
\pgfsetdash{{1.000000pt}{3.000000pt}}{0.000000pt}%
\pgfpathmoveto{\pgfqpoint{0.580766in}{3.726041in}}%
\pgfpathlineto{\pgfqpoint{6.192411in}{3.726041in}}%
\pgfusepath{stroke}%
\end{pgfscope}%
\begin{pgfscope}%
\pgfsetbuttcap%
\pgfsetroundjoin%
\definecolor{currentfill}{rgb}{0.000000,0.000000,0.000000}%
\pgfsetfillcolor{currentfill}%
\pgfsetlinewidth{0.501875pt}%
\definecolor{currentstroke}{rgb}{0.000000,0.000000,0.000000}%
\pgfsetstrokecolor{currentstroke}%
\pgfsetdash{}{0pt}%
\pgfsys@defobject{currentmarker}{\pgfqpoint{0.000000in}{0.000000in}}{\pgfqpoint{0.055556in}{0.000000in}}{%
\pgfpathmoveto{\pgfqpoint{0.000000in}{0.000000in}}%
\pgfpathlineto{\pgfqpoint{0.055556in}{0.000000in}}%
\pgfusepath{stroke,fill}%
}%
\begin{pgfscope}%
\pgfsys@transformshift{0.580766in}{3.726041in}%
\pgfsys@useobject{currentmarker}{}%
\end{pgfscope}%
\end{pgfscope}%
\begin{pgfscope}%
\pgfsetbuttcap%
\pgfsetroundjoin%
\definecolor{currentfill}{rgb}{0.000000,0.000000,0.000000}%
\pgfsetfillcolor{currentfill}%
\pgfsetlinewidth{0.501875pt}%
\definecolor{currentstroke}{rgb}{0.000000,0.000000,0.000000}%
\pgfsetstrokecolor{currentstroke}%
\pgfsetdash{}{0pt}%
\pgfsys@defobject{currentmarker}{\pgfqpoint{-0.055556in}{0.000000in}}{\pgfqpoint{0.000000in}{0.000000in}}{%
\pgfpathmoveto{\pgfqpoint{0.000000in}{0.000000in}}%
\pgfpathlineto{\pgfqpoint{-0.055556in}{0.000000in}}%
\pgfusepath{stroke,fill}%
}%
\begin{pgfscope}%
\pgfsys@transformshift{6.192411in}{3.726041in}%
\pgfsys@useobject{currentmarker}{}%
\end{pgfscope}%
\end{pgfscope}%
\begin{pgfscope}%
\pgftext[x=0.525210in,y=3.726041in,right,]{{\rmfamily\fontsize{10.000000}{12.000000}\selectfont \(\displaystyle 10\)}}%
\end{pgfscope}%
\begin{pgfscope}%
\pgftext[x=0.316877in,y=2.129480in,,bottom,rotate=90.000000]{{\rmfamily\fontsize{10.000000}{12.000000}\selectfont voltage (V)}}%
\end{pgfscope}%
\begin{pgfscope}%
\pgfsetbuttcap%
\pgfsetroundjoin%
\pgfsetlinewidth{1.003750pt}%
\definecolor{currentstroke}{rgb}{0.000000,0.000000,0.000000}%
\pgfsetstrokecolor{currentstroke}%
\pgfsetdash{}{0pt}%
\pgfpathmoveto{\pgfqpoint{0.580766in}{3.726041in}}%
\pgfpathlineto{\pgfqpoint{6.192411in}{3.726041in}}%
\pgfusepath{stroke}%
\end{pgfscope}%
\begin{pgfscope}%
\pgfsetbuttcap%
\pgfsetroundjoin%
\pgfsetlinewidth{1.003750pt}%
\definecolor{currentstroke}{rgb}{0.000000,0.000000,0.000000}%
\pgfsetstrokecolor{currentstroke}%
\pgfsetdash{}{0pt}%
\pgfpathmoveto{\pgfqpoint{6.192411in}{0.532919in}}%
\pgfpathlineto{\pgfqpoint{6.192411in}{3.726041in}}%
\pgfusepath{stroke}%
\end{pgfscope}%
\begin{pgfscope}%
\pgfsetbuttcap%
\pgfsetroundjoin%
\pgfsetlinewidth{1.003750pt}%
\definecolor{currentstroke}{rgb}{0.000000,0.000000,0.000000}%
\pgfsetstrokecolor{currentstroke}%
\pgfsetdash{}{0pt}%
\pgfpathmoveto{\pgfqpoint{0.580766in}{0.532919in}}%
\pgfpathlineto{\pgfqpoint{6.192411in}{0.532919in}}%
\pgfusepath{stroke}%
\end{pgfscope}%
\begin{pgfscope}%
\pgfsetbuttcap%
\pgfsetroundjoin%
\pgfsetlinewidth{1.003750pt}%
\definecolor{currentstroke}{rgb}{0.000000,0.000000,0.000000}%
\pgfsetstrokecolor{currentstroke}%
\pgfsetdash{}{0pt}%
\pgfpathmoveto{\pgfqpoint{0.580766in}{0.532919in}}%
\pgfpathlineto{\pgfqpoint{0.580766in}{3.726041in}}%
\pgfusepath{stroke}%
\end{pgfscope}%
\begin{pgfscope}%
\pgftext[x=3.386589in,y=3.795485in,,base]{{\rmfamily\fontsize{12.000000}{14.400000}\selectfont Evaluations of \(\displaystyle v_L(t)\) at \(\displaystyle 0.0001s\)}}%
\end{pgfscope}%
\begin{pgfscope}%
\pgfsetbuttcap%
\pgfsetroundjoin%
\definecolor{currentfill}{rgb}{0.300000,0.300000,0.300000}%
\pgfsetfillcolor{currentfill}%
\pgfsetfillopacity{0.500000}%
\pgfsetlinewidth{1.003750pt}%
\definecolor{currentstroke}{rgb}{0.300000,0.300000,0.300000}%
\pgfsetstrokecolor{currentstroke}%
\pgfsetstrokeopacity{0.500000}%
\pgfsetdash{}{0pt}%
\pgfpathmoveto{\pgfqpoint{4.378033in}{2.635301in}}%
\pgfpathlineto{\pgfqpoint{6.103523in}{2.635301in}}%
\pgfpathquadraticcurveto{\pgfqpoint{6.136856in}{2.635301in}}{\pgfqpoint{6.136856in}{2.668635in}}%
\pgfpathlineto{\pgfqpoint{6.136856in}{3.581596in}}%
\pgfpathquadraticcurveto{\pgfqpoint{6.136856in}{3.614930in}}{\pgfqpoint{6.103523in}{3.614930in}}%
\pgfpathlineto{\pgfqpoint{4.378033in}{3.614930in}}%
\pgfpathquadraticcurveto{\pgfqpoint{4.344700in}{3.614930in}}{\pgfqpoint{4.344700in}{3.581596in}}%
\pgfpathlineto{\pgfqpoint{4.344700in}{2.668635in}}%
\pgfpathquadraticcurveto{\pgfqpoint{4.344700in}{2.635301in}}{\pgfqpoint{4.378033in}{2.635301in}}%
\pgfpathclose%
\pgfusepath{stroke,fill}%
\end{pgfscope}%
\begin{pgfscope}%
\pgfsetbuttcap%
\pgfsetroundjoin%
\definecolor{currentfill}{rgb}{1.000000,1.000000,1.000000}%
\pgfsetfillcolor{currentfill}%
\pgfsetlinewidth{1.003750pt}%
\definecolor{currentstroke}{rgb}{0.000000,0.000000,0.000000}%
\pgfsetstrokecolor{currentstroke}%
\pgfsetdash{}{0pt}%
\pgfpathmoveto{\pgfqpoint{4.350255in}{2.663079in}}%
\pgfpathlineto{\pgfqpoint{6.075745in}{2.663079in}}%
\pgfpathquadraticcurveto{\pgfqpoint{6.109078in}{2.663079in}}{\pgfqpoint{6.109078in}{2.696413in}}%
\pgfpathlineto{\pgfqpoint{6.109078in}{3.609374in}}%
\pgfpathquadraticcurveto{\pgfqpoint{6.109078in}{3.642708in}}{\pgfqpoint{6.075745in}{3.642708in}}%
\pgfpathlineto{\pgfqpoint{4.350255in}{3.642708in}}%
\pgfpathquadraticcurveto{\pgfqpoint{4.316922in}{3.642708in}}{\pgfqpoint{4.316922in}{3.609374in}}%
\pgfpathlineto{\pgfqpoint{4.316922in}{2.696413in}}%
\pgfpathquadraticcurveto{\pgfqpoint{4.316922in}{2.663079in}}{\pgfqpoint{4.350255in}{2.663079in}}%
\pgfpathclose%
\pgfusepath{stroke,fill}%
\end{pgfscope}%
\begin{pgfscope}%
\pgfsetbuttcap%
\pgfsetroundjoin%
\pgfsetlinewidth{1.003750pt}%
\definecolor{currentstroke}{rgb}{0.000000,0.000000,1.000000}%
\pgfsetstrokecolor{currentstroke}%
\pgfsetdash{{1.000000pt}{3.000000pt}}{0.000000pt}%
\pgfpathmoveto{\pgfqpoint{4.433589in}{3.517708in}}%
\pgfpathlineto{\pgfqpoint{4.666922in}{3.517708in}}%
\pgfusepath{stroke}%
\end{pgfscope}%
\begin{pgfscope}%
\pgftext[x=4.850255in,y=3.459374in,left,base]{{\rmfamily\fontsize{12.000000}{14.400000}\selectfont Continuous}}%
\end{pgfscope}%
\begin{pgfscope}%
\pgfsetrectcap%
\pgfsetroundjoin%
\pgfsetlinewidth{1.003750pt}%
\definecolor{currentstroke}{rgb}{0.000000,0.500000,0.000000}%
\pgfsetstrokecolor{currentstroke}%
\pgfsetdash{}{0pt}%
\pgfpathmoveto{\pgfqpoint{4.433589in}{3.285300in}}%
\pgfpathlineto{\pgfqpoint{4.666922in}{3.285300in}}%
\pgfusepath{stroke}%
\end{pgfscope}%
\begin{pgfscope}%
\pgftext[x=4.850255in,y=3.226967in,left,base]{{\rmfamily\fontsize{12.000000}{14.400000}\selectfont Trapezoidal}}%
\end{pgfscope}%
\begin{pgfscope}%
\pgfsetrectcap%
\pgfsetroundjoin%
\pgfsetlinewidth{1.003750pt}%
\definecolor{currentstroke}{rgb}{1.000000,0.000000,0.000000}%
\pgfsetstrokecolor{currentstroke}%
\pgfsetdash{}{0pt}%
\pgfpathmoveto{\pgfqpoint{4.433589in}{3.052893in}}%
\pgfpathlineto{\pgfqpoint{4.666922in}{3.052893in}}%
\pgfusepath{stroke}%
\end{pgfscope}%
\begin{pgfscope}%
\pgftext[x=4.850255in,y=2.994560in,left,base]{{\rmfamily\fontsize{12.000000}{14.400000}\selectfont Backward Euler    }}%
\end{pgfscope}%
\begin{pgfscope}%
\pgfsetrectcap%
\pgfsetroundjoin%
\pgfsetlinewidth{1.003750pt}%
\definecolor{currentstroke}{rgb}{0.000000,0.750000,0.750000}%
\pgfsetstrokecolor{currentstroke}%
\pgfsetdash{}{0pt}%
\pgfpathmoveto{\pgfqpoint{4.433589in}{2.820486in}}%
\pgfpathlineto{\pgfqpoint{4.666922in}{2.820486in}}%
\pgfusepath{stroke}%
\end{pgfscope}%
\begin{pgfscope}%
\pgftext[x=4.850255in,y=2.762153in,left,base]{{\rmfamily\fontsize{12.000000}{14.400000}\selectfont PSCAD}}%
\end{pgfscope}%
\end{pgfpicture}%
\makeatother%
\endgroup%

    \end{center}
    \caption{Comparison of Approximations and PSCAD at $\Delta{}t_1 = 0.1 ms$}
    \label{approx_pscad_comp_0p0001}
\end{figure}

\begin{figure}[H]
    \begin{center}
        %% Creator: Matplotlib, PGF backend
%%
%% To include the figure in your LaTeX document, write
%%   \input{<filename>.pgf}
%%
%% Make sure the required packages are loaded in your preamble
%%   \usepackage{pgf}
%%
%% Figures using additional raster images can only be included by \input if
%% they are in the same directory as the main LaTeX file. For loading figures
%% from other directories you can use the `import` package
%%   \usepackage{import}
%% and then include the figures with
%%   \import{<path to file>}{<filename>.pgf}
%%
%% Matplotlib used the following preamble
%%
\begingroup%
\makeatletter%
\begin{pgfpicture}%
\pgfpathrectangle{\pgfpointorigin}{\pgfqpoint{6.500000in}{8.000000in}}%
\pgfusepath{use as bounding box}%
\begin{pgfscope}%
\pgfsetbuttcap%
\pgfsetroundjoin%
\definecolor{currentfill}{rgb}{1.000000,1.000000,1.000000}%
\pgfsetfillcolor{currentfill}%
\pgfsetlinewidth{0.000000pt}%
\definecolor{currentstroke}{rgb}{1.000000,1.000000,1.000000}%
\pgfsetstrokecolor{currentstroke}%
\pgfsetdash{}{0pt}%
\pgfpathmoveto{\pgfqpoint{0.000000in}{0.000000in}}%
\pgfpathlineto{\pgfqpoint{6.500000in}{0.000000in}}%
\pgfpathlineto{\pgfqpoint{6.500000in}{8.000000in}}%
\pgfpathlineto{\pgfqpoint{0.000000in}{8.000000in}}%
\pgfpathclose%
\pgfusepath{fill}%
\end{pgfscope}%
\begin{pgfscope}%
\pgfsetbuttcap%
\pgfsetroundjoin%
\definecolor{currentfill}{rgb}{1.000000,1.000000,1.000000}%
\pgfsetfillcolor{currentfill}%
\pgfsetlinewidth{0.000000pt}%
\definecolor{currentstroke}{rgb}{0.000000,0.000000,0.000000}%
\pgfsetstrokecolor{currentstroke}%
\pgfsetstrokeopacity{0.000000}%
\pgfsetdash{}{0pt}%
\pgfpathmoveto{\pgfqpoint{0.580766in}{4.462900in}}%
\pgfpathlineto{\pgfqpoint{6.192411in}{4.462900in}}%
\pgfpathlineto{\pgfqpoint{6.192411in}{7.656023in}}%
\pgfpathlineto{\pgfqpoint{0.580766in}{7.656023in}}%
\pgfpathclose%
\pgfusepath{fill}%
\end{pgfscope}%
\begin{pgfscope}%
\pgfpathrectangle{\pgfqpoint{0.580766in}{4.462900in}}{\pgfqpoint{5.611646in}{3.193122in}} %
\pgfusepath{clip}%
\pgfsetbuttcap%
\pgfsetroundjoin%
\pgfsetlinewidth{1.003750pt}%
\definecolor{currentstroke}{rgb}{0.000000,0.000000,1.000000}%
\pgfsetstrokecolor{currentstroke}%
\pgfsetdash{{1.000000pt}{3.000000pt}}{0.000000pt}%
\pgfpathmoveto{\pgfqpoint{0.580766in}{4.462900in}}%
\pgfpathlineto{\pgfqpoint{0.633141in}{4.608489in}}%
\pgfpathlineto{\pgfqpoint{0.685517in}{4.747440in}}%
\pgfpathlineto{\pgfqpoint{0.737892in}{4.880055in}}%
\pgfpathlineto{\pgfqpoint{0.790267in}{5.006624in}}%
\pgfpathlineto{\pgfqpoint{0.842643in}{5.127422in}}%
\pgfpathlineto{\pgfqpoint{0.895018in}{5.242713in}}%
\pgfpathlineto{\pgfqpoint{0.947393in}{5.352746in}}%
\pgfpathlineto{\pgfqpoint{0.999769in}{5.457763in}}%
\pgfpathlineto{\pgfqpoint{1.054015in}{5.561485in}}%
\pgfpathlineto{\pgfqpoint{1.108261in}{5.660314in}}%
\pgfpathlineto{\pgfqpoint{1.162506in}{5.754479in}}%
\pgfpathlineto{\pgfqpoint{1.216752in}{5.844201in}}%
\pgfpathlineto{\pgfqpoint{1.270998in}{5.929690in}}%
\pgfpathlineto{\pgfqpoint{1.325244in}{6.011145in}}%
\pgfpathlineto{\pgfqpoint{1.381361in}{6.091367in}}%
\pgfpathlineto{\pgfqpoint{1.437477in}{6.167676in}}%
\pgfpathlineto{\pgfqpoint{1.493594in}{6.240263in}}%
\pgfpathlineto{\pgfqpoint{1.549710in}{6.309311in}}%
\pgfpathlineto{\pgfqpoint{1.607697in}{6.377124in}}%
\pgfpathlineto{\pgfqpoint{1.665684in}{6.441522in}}%
\pgfpathlineto{\pgfqpoint{1.723671in}{6.502678in}}%
\pgfpathlineto{\pgfqpoint{1.783529in}{6.562578in}}%
\pgfpathlineto{\pgfqpoint{1.843386in}{6.619367in}}%
\pgfpathlineto{\pgfqpoint{1.905114in}{6.674844in}}%
\pgfpathlineto{\pgfqpoint{1.966842in}{6.727352in}}%
\pgfpathlineto{\pgfqpoint{2.030441in}{6.778513in}}%
\pgfpathlineto{\pgfqpoint{2.094040in}{6.826856in}}%
\pgfpathlineto{\pgfqpoint{2.159509in}{6.873840in}}%
\pgfpathlineto{\pgfqpoint{2.226849in}{6.919391in}}%
\pgfpathlineto{\pgfqpoint{2.294188in}{6.962289in}}%
\pgfpathlineto{\pgfqpoint{2.363399in}{7.003777in}}%
\pgfpathlineto{\pgfqpoint{2.434479in}{7.043805in}}%
\pgfpathlineto{\pgfqpoint{2.507431in}{7.082333in}}%
\pgfpathlineto{\pgfqpoint{2.582253in}{7.119332in}}%
\pgfpathlineto{\pgfqpoint{2.658945in}{7.154781in}}%
\pgfpathlineto{\pgfqpoint{2.737508in}{7.188668in}}%
\pgfpathlineto{\pgfqpoint{2.819812in}{7.221714in}}%
\pgfpathlineto{\pgfqpoint{2.903987in}{7.253096in}}%
\pgfpathlineto{\pgfqpoint{2.991903in}{7.283454in}}%
\pgfpathlineto{\pgfqpoint{3.081689in}{7.312098in}}%
\pgfpathlineto{\pgfqpoint{3.175217in}{7.339597in}}%
\pgfpathlineto{\pgfqpoint{3.272485in}{7.365866in}}%
\pgfpathlineto{\pgfqpoint{3.373495in}{7.390839in}}%
\pgfpathlineto{\pgfqpoint{3.480116in}{7.414872in}}%
\pgfpathlineto{\pgfqpoint{3.590478in}{7.437457in}}%
\pgfpathlineto{\pgfqpoint{3.706452in}{7.458914in}}%
\pgfpathlineto{\pgfqpoint{3.828038in}{7.479151in}}%
\pgfpathlineto{\pgfqpoint{3.957106in}{7.498366in}}%
\pgfpathlineto{\pgfqpoint{4.093656in}{7.516426in}}%
\pgfpathlineto{\pgfqpoint{4.239559in}{7.533444in}}%
\pgfpathlineto{\pgfqpoint{4.394814in}{7.549280in}}%
\pgfpathlineto{\pgfqpoint{4.561293in}{7.563995in}}%
\pgfpathlineto{\pgfqpoint{4.740866in}{7.577602in}}%
\pgfpathlineto{\pgfqpoint{4.935403in}{7.590082in}}%
\pgfpathlineto{\pgfqpoint{5.148645in}{7.601492in}}%
\pgfpathlineto{\pgfqpoint{5.382464in}{7.611748in}}%
\pgfpathlineto{\pgfqpoint{5.644341in}{7.620962in}}%
\pgfpathlineto{\pgfqpoint{5.939887in}{7.629079in}}%
\pgfpathlineto{\pgfqpoint{6.190541in}{7.634472in}}%
\pgfpathlineto{\pgfqpoint{6.190541in}{7.634472in}}%
\pgfusepath{stroke}%
\end{pgfscope}%
\begin{pgfscope}%
\pgfpathrectangle{\pgfqpoint{0.580766in}{4.462900in}}{\pgfqpoint{5.611646in}{3.193122in}} %
\pgfusepath{clip}%
\pgfsetrectcap%
\pgfsetroundjoin%
\pgfsetlinewidth{1.003750pt}%
\definecolor{currentstroke}{rgb}{0.000000,0.500000,0.000000}%
\pgfsetstrokecolor{currentstroke}%
\pgfsetdash{}{0pt}%
\pgfpathmoveto{\pgfqpoint{0.580766in}{5.527274in}}%
\pgfpathlineto{\pgfqpoint{1.029698in}{6.236857in}}%
\pgfpathlineto{\pgfqpoint{1.478629in}{6.709912in}}%
\pgfpathlineto{\pgfqpoint{1.927561in}{7.025282in}}%
\pgfpathlineto{\pgfqpoint{2.376492in}{7.235529in}}%
\pgfpathlineto{\pgfqpoint{2.825424in}{7.375694in}}%
\pgfpathlineto{\pgfqpoint{3.274356in}{7.469137in}}%
\pgfpathlineto{\pgfqpoint{3.723287in}{7.531432in}}%
\pgfpathlineto{\pgfqpoint{4.172219in}{7.572962in}}%
\pgfpathlineto{\pgfqpoint{4.621151in}{7.600649in}}%
\pgfpathlineto{\pgfqpoint{5.070082in}{7.619107in}}%
\pgfpathlineto{\pgfqpoint{5.519014in}{7.631412in}}%
\pgfpathlineto{\pgfqpoint{5.967946in}{7.639616in}}%
\pgfusepath{stroke}%
\end{pgfscope}%
\begin{pgfscope}%
\pgfpathrectangle{\pgfqpoint{0.580766in}{4.462900in}}{\pgfqpoint{5.611646in}{3.193122in}} %
\pgfusepath{clip}%
\pgfsetrectcap%
\pgfsetroundjoin%
\pgfsetlinewidth{1.003750pt}%
\definecolor{currentstroke}{rgb}{1.000000,0.000000,0.000000}%
\pgfsetstrokecolor{currentstroke}%
\pgfsetdash{}{0pt}%
\pgfpathmoveto{\pgfqpoint{0.580766in}{5.375221in}}%
\pgfpathlineto{\pgfqpoint{1.029698in}{6.026878in}}%
\pgfpathlineto{\pgfqpoint{1.478629in}{6.492348in}}%
\pgfpathlineto{\pgfqpoint{1.927561in}{6.824827in}}%
\pgfpathlineto{\pgfqpoint{2.376492in}{7.062311in}}%
\pgfpathlineto{\pgfqpoint{2.825424in}{7.231943in}}%
\pgfpathlineto{\pgfqpoint{3.274356in}{7.353109in}}%
\pgfpathlineto{\pgfqpoint{3.723287in}{7.439655in}}%
\pgfpathlineto{\pgfqpoint{4.172219in}{7.501475in}}%
\pgfpathlineto{\pgfqpoint{4.621151in}{7.545631in}}%
\pgfpathlineto{\pgfqpoint{5.070082in}{7.577172in}}%
\pgfpathlineto{\pgfqpoint{5.519014in}{7.599700in}}%
\pgfpathlineto{\pgfqpoint{5.967946in}{7.615792in}}%
\pgfusepath{stroke}%
\end{pgfscope}%
\begin{pgfscope}%
\pgfpathrectangle{\pgfqpoint{0.580766in}{4.462900in}}{\pgfqpoint{5.611646in}{3.193122in}} %
\pgfusepath{clip}%
\pgfsetbuttcap%
\pgfsetroundjoin%
\pgfsetlinewidth{0.501875pt}%
\definecolor{currentstroke}{rgb}{0.000000,0.000000,0.000000}%
\pgfsetstrokecolor{currentstroke}%
\pgfsetdash{{1.000000pt}{3.000000pt}}{0.000000pt}%
\pgfpathmoveto{\pgfqpoint{0.580766in}{4.462900in}}%
\pgfpathlineto{\pgfqpoint{0.580766in}{7.656023in}}%
\pgfusepath{stroke}%
\end{pgfscope}%
\begin{pgfscope}%
\pgfsetbuttcap%
\pgfsetroundjoin%
\definecolor{currentfill}{rgb}{0.000000,0.000000,0.000000}%
\pgfsetfillcolor{currentfill}%
\pgfsetlinewidth{0.501875pt}%
\definecolor{currentstroke}{rgb}{0.000000,0.000000,0.000000}%
\pgfsetstrokecolor{currentstroke}%
\pgfsetdash{}{0pt}%
\pgfsys@defobject{currentmarker}{\pgfqpoint{0.000000in}{0.000000in}}{\pgfqpoint{0.000000in}{0.055556in}}{%
\pgfpathmoveto{\pgfqpoint{0.000000in}{0.000000in}}%
\pgfpathlineto{\pgfqpoint{0.000000in}{0.055556in}}%
\pgfusepath{stroke,fill}%
}%
\begin{pgfscope}%
\pgfsys@transformshift{0.580766in}{4.462900in}%
\pgfsys@useobject{currentmarker}{}%
\end{pgfscope}%
\end{pgfscope}%
\begin{pgfscope}%
\pgfsetbuttcap%
\pgfsetroundjoin%
\definecolor{currentfill}{rgb}{0.000000,0.000000,0.000000}%
\pgfsetfillcolor{currentfill}%
\pgfsetlinewidth{0.501875pt}%
\definecolor{currentstroke}{rgb}{0.000000,0.000000,0.000000}%
\pgfsetstrokecolor{currentstroke}%
\pgfsetdash{}{0pt}%
\pgfsys@defobject{currentmarker}{\pgfqpoint{0.000000in}{-0.055556in}}{\pgfqpoint{0.000000in}{0.000000in}}{%
\pgfpathmoveto{\pgfqpoint{0.000000in}{0.000000in}}%
\pgfpathlineto{\pgfqpoint{0.000000in}{-0.055556in}}%
\pgfusepath{stroke,fill}%
}%
\begin{pgfscope}%
\pgfsys@transformshift{0.580766in}{7.656023in}%
\pgfsys@useobject{currentmarker}{}%
\end{pgfscope}%
\end{pgfscope}%
\begin{pgfscope}%
\pgftext[x=0.580766in,y=4.407345in,,top]{{\rmfamily\fontsize{10.000000}{12.000000}\selectfont \(\displaystyle 0.000\)}}%
\end{pgfscope}%
\begin{pgfscope}%
\pgfpathrectangle{\pgfqpoint{0.580766in}{4.462900in}}{\pgfqpoint{5.611646in}{3.193122in}} %
\pgfusepath{clip}%
\pgfsetbuttcap%
\pgfsetroundjoin%
\pgfsetlinewidth{0.501875pt}%
\definecolor{currentstroke}{rgb}{0.000000,0.000000,0.000000}%
\pgfsetstrokecolor{currentstroke}%
\pgfsetdash{{1.000000pt}{3.000000pt}}{0.000000pt}%
\pgfpathmoveto{\pgfqpoint{1.703095in}{4.462900in}}%
\pgfpathlineto{\pgfqpoint{1.703095in}{7.656023in}}%
\pgfusepath{stroke}%
\end{pgfscope}%
\begin{pgfscope}%
\pgfsetbuttcap%
\pgfsetroundjoin%
\definecolor{currentfill}{rgb}{0.000000,0.000000,0.000000}%
\pgfsetfillcolor{currentfill}%
\pgfsetlinewidth{0.501875pt}%
\definecolor{currentstroke}{rgb}{0.000000,0.000000,0.000000}%
\pgfsetstrokecolor{currentstroke}%
\pgfsetdash{}{0pt}%
\pgfsys@defobject{currentmarker}{\pgfqpoint{0.000000in}{0.000000in}}{\pgfqpoint{0.000000in}{0.055556in}}{%
\pgfpathmoveto{\pgfqpoint{0.000000in}{0.000000in}}%
\pgfpathlineto{\pgfqpoint{0.000000in}{0.055556in}}%
\pgfusepath{stroke,fill}%
}%
\begin{pgfscope}%
\pgfsys@transformshift{1.703095in}{4.462900in}%
\pgfsys@useobject{currentmarker}{}%
\end{pgfscope}%
\end{pgfscope}%
\begin{pgfscope}%
\pgfsetbuttcap%
\pgfsetroundjoin%
\definecolor{currentfill}{rgb}{0.000000,0.000000,0.000000}%
\pgfsetfillcolor{currentfill}%
\pgfsetlinewidth{0.501875pt}%
\definecolor{currentstroke}{rgb}{0.000000,0.000000,0.000000}%
\pgfsetstrokecolor{currentstroke}%
\pgfsetdash{}{0pt}%
\pgfsys@defobject{currentmarker}{\pgfqpoint{0.000000in}{-0.055556in}}{\pgfqpoint{0.000000in}{0.000000in}}{%
\pgfpathmoveto{\pgfqpoint{0.000000in}{0.000000in}}%
\pgfpathlineto{\pgfqpoint{0.000000in}{-0.055556in}}%
\pgfusepath{stroke,fill}%
}%
\begin{pgfscope}%
\pgfsys@transformshift{1.703095in}{7.656023in}%
\pgfsys@useobject{currentmarker}{}%
\end{pgfscope}%
\end{pgfscope}%
\begin{pgfscope}%
\pgftext[x=1.703095in,y=4.407345in,,top]{{\rmfamily\fontsize{10.000000}{12.000000}\selectfont \(\displaystyle 0.002\)}}%
\end{pgfscope}%
\begin{pgfscope}%
\pgfpathrectangle{\pgfqpoint{0.580766in}{4.462900in}}{\pgfqpoint{5.611646in}{3.193122in}} %
\pgfusepath{clip}%
\pgfsetbuttcap%
\pgfsetroundjoin%
\pgfsetlinewidth{0.501875pt}%
\definecolor{currentstroke}{rgb}{0.000000,0.000000,0.000000}%
\pgfsetstrokecolor{currentstroke}%
\pgfsetdash{{1.000000pt}{3.000000pt}}{0.000000pt}%
\pgfpathmoveto{\pgfqpoint{2.825424in}{4.462900in}}%
\pgfpathlineto{\pgfqpoint{2.825424in}{7.656023in}}%
\pgfusepath{stroke}%
\end{pgfscope}%
\begin{pgfscope}%
\pgfsetbuttcap%
\pgfsetroundjoin%
\definecolor{currentfill}{rgb}{0.000000,0.000000,0.000000}%
\pgfsetfillcolor{currentfill}%
\pgfsetlinewidth{0.501875pt}%
\definecolor{currentstroke}{rgb}{0.000000,0.000000,0.000000}%
\pgfsetstrokecolor{currentstroke}%
\pgfsetdash{}{0pt}%
\pgfsys@defobject{currentmarker}{\pgfqpoint{0.000000in}{0.000000in}}{\pgfqpoint{0.000000in}{0.055556in}}{%
\pgfpathmoveto{\pgfqpoint{0.000000in}{0.000000in}}%
\pgfpathlineto{\pgfqpoint{0.000000in}{0.055556in}}%
\pgfusepath{stroke,fill}%
}%
\begin{pgfscope}%
\pgfsys@transformshift{2.825424in}{4.462900in}%
\pgfsys@useobject{currentmarker}{}%
\end{pgfscope}%
\end{pgfscope}%
\begin{pgfscope}%
\pgfsetbuttcap%
\pgfsetroundjoin%
\definecolor{currentfill}{rgb}{0.000000,0.000000,0.000000}%
\pgfsetfillcolor{currentfill}%
\pgfsetlinewidth{0.501875pt}%
\definecolor{currentstroke}{rgb}{0.000000,0.000000,0.000000}%
\pgfsetstrokecolor{currentstroke}%
\pgfsetdash{}{0pt}%
\pgfsys@defobject{currentmarker}{\pgfqpoint{0.000000in}{-0.055556in}}{\pgfqpoint{0.000000in}{0.000000in}}{%
\pgfpathmoveto{\pgfqpoint{0.000000in}{0.000000in}}%
\pgfpathlineto{\pgfqpoint{0.000000in}{-0.055556in}}%
\pgfusepath{stroke,fill}%
}%
\begin{pgfscope}%
\pgfsys@transformshift{2.825424in}{7.656023in}%
\pgfsys@useobject{currentmarker}{}%
\end{pgfscope}%
\end{pgfscope}%
\begin{pgfscope}%
\pgftext[x=2.825424in,y=4.407345in,,top]{{\rmfamily\fontsize{10.000000}{12.000000}\selectfont \(\displaystyle 0.004\)}}%
\end{pgfscope}%
\begin{pgfscope}%
\pgfpathrectangle{\pgfqpoint{0.580766in}{4.462900in}}{\pgfqpoint{5.611646in}{3.193122in}} %
\pgfusepath{clip}%
\pgfsetbuttcap%
\pgfsetroundjoin%
\pgfsetlinewidth{0.501875pt}%
\definecolor{currentstroke}{rgb}{0.000000,0.000000,0.000000}%
\pgfsetstrokecolor{currentstroke}%
\pgfsetdash{{1.000000pt}{3.000000pt}}{0.000000pt}%
\pgfpathmoveto{\pgfqpoint{3.947753in}{4.462900in}}%
\pgfpathlineto{\pgfqpoint{3.947753in}{7.656023in}}%
\pgfusepath{stroke}%
\end{pgfscope}%
\begin{pgfscope}%
\pgfsetbuttcap%
\pgfsetroundjoin%
\definecolor{currentfill}{rgb}{0.000000,0.000000,0.000000}%
\pgfsetfillcolor{currentfill}%
\pgfsetlinewidth{0.501875pt}%
\definecolor{currentstroke}{rgb}{0.000000,0.000000,0.000000}%
\pgfsetstrokecolor{currentstroke}%
\pgfsetdash{}{0pt}%
\pgfsys@defobject{currentmarker}{\pgfqpoint{0.000000in}{0.000000in}}{\pgfqpoint{0.000000in}{0.055556in}}{%
\pgfpathmoveto{\pgfqpoint{0.000000in}{0.000000in}}%
\pgfpathlineto{\pgfqpoint{0.000000in}{0.055556in}}%
\pgfusepath{stroke,fill}%
}%
\begin{pgfscope}%
\pgfsys@transformshift{3.947753in}{4.462900in}%
\pgfsys@useobject{currentmarker}{}%
\end{pgfscope}%
\end{pgfscope}%
\begin{pgfscope}%
\pgfsetbuttcap%
\pgfsetroundjoin%
\definecolor{currentfill}{rgb}{0.000000,0.000000,0.000000}%
\pgfsetfillcolor{currentfill}%
\pgfsetlinewidth{0.501875pt}%
\definecolor{currentstroke}{rgb}{0.000000,0.000000,0.000000}%
\pgfsetstrokecolor{currentstroke}%
\pgfsetdash{}{0pt}%
\pgfsys@defobject{currentmarker}{\pgfqpoint{0.000000in}{-0.055556in}}{\pgfqpoint{0.000000in}{0.000000in}}{%
\pgfpathmoveto{\pgfqpoint{0.000000in}{0.000000in}}%
\pgfpathlineto{\pgfqpoint{0.000000in}{-0.055556in}}%
\pgfusepath{stroke,fill}%
}%
\begin{pgfscope}%
\pgfsys@transformshift{3.947753in}{7.656023in}%
\pgfsys@useobject{currentmarker}{}%
\end{pgfscope}%
\end{pgfscope}%
\begin{pgfscope}%
\pgftext[x=3.947753in,y=4.407345in,,top]{{\rmfamily\fontsize{10.000000}{12.000000}\selectfont \(\displaystyle 0.006\)}}%
\end{pgfscope}%
\begin{pgfscope}%
\pgfpathrectangle{\pgfqpoint{0.580766in}{4.462900in}}{\pgfqpoint{5.611646in}{3.193122in}} %
\pgfusepath{clip}%
\pgfsetbuttcap%
\pgfsetroundjoin%
\pgfsetlinewidth{0.501875pt}%
\definecolor{currentstroke}{rgb}{0.000000,0.000000,0.000000}%
\pgfsetstrokecolor{currentstroke}%
\pgfsetdash{{1.000000pt}{3.000000pt}}{0.000000pt}%
\pgfpathmoveto{\pgfqpoint{5.070082in}{4.462900in}}%
\pgfpathlineto{\pgfqpoint{5.070082in}{7.656023in}}%
\pgfusepath{stroke}%
\end{pgfscope}%
\begin{pgfscope}%
\pgfsetbuttcap%
\pgfsetroundjoin%
\definecolor{currentfill}{rgb}{0.000000,0.000000,0.000000}%
\pgfsetfillcolor{currentfill}%
\pgfsetlinewidth{0.501875pt}%
\definecolor{currentstroke}{rgb}{0.000000,0.000000,0.000000}%
\pgfsetstrokecolor{currentstroke}%
\pgfsetdash{}{0pt}%
\pgfsys@defobject{currentmarker}{\pgfqpoint{0.000000in}{0.000000in}}{\pgfqpoint{0.000000in}{0.055556in}}{%
\pgfpathmoveto{\pgfqpoint{0.000000in}{0.000000in}}%
\pgfpathlineto{\pgfqpoint{0.000000in}{0.055556in}}%
\pgfusepath{stroke,fill}%
}%
\begin{pgfscope}%
\pgfsys@transformshift{5.070082in}{4.462900in}%
\pgfsys@useobject{currentmarker}{}%
\end{pgfscope}%
\end{pgfscope}%
\begin{pgfscope}%
\pgfsetbuttcap%
\pgfsetroundjoin%
\definecolor{currentfill}{rgb}{0.000000,0.000000,0.000000}%
\pgfsetfillcolor{currentfill}%
\pgfsetlinewidth{0.501875pt}%
\definecolor{currentstroke}{rgb}{0.000000,0.000000,0.000000}%
\pgfsetstrokecolor{currentstroke}%
\pgfsetdash{}{0pt}%
\pgfsys@defobject{currentmarker}{\pgfqpoint{0.000000in}{-0.055556in}}{\pgfqpoint{0.000000in}{0.000000in}}{%
\pgfpathmoveto{\pgfqpoint{0.000000in}{0.000000in}}%
\pgfpathlineto{\pgfqpoint{0.000000in}{-0.055556in}}%
\pgfusepath{stroke,fill}%
}%
\begin{pgfscope}%
\pgfsys@transformshift{5.070082in}{7.656023in}%
\pgfsys@useobject{currentmarker}{}%
\end{pgfscope}%
\end{pgfscope}%
\begin{pgfscope}%
\pgftext[x=5.070082in,y=4.407345in,,top]{{\rmfamily\fontsize{10.000000}{12.000000}\selectfont \(\displaystyle 0.008\)}}%
\end{pgfscope}%
\begin{pgfscope}%
\pgfpathrectangle{\pgfqpoint{0.580766in}{4.462900in}}{\pgfqpoint{5.611646in}{3.193122in}} %
\pgfusepath{clip}%
\pgfsetbuttcap%
\pgfsetroundjoin%
\pgfsetlinewidth{0.501875pt}%
\definecolor{currentstroke}{rgb}{0.000000,0.000000,0.000000}%
\pgfsetstrokecolor{currentstroke}%
\pgfsetdash{{1.000000pt}{3.000000pt}}{0.000000pt}%
\pgfpathmoveto{\pgfqpoint{6.192411in}{4.462900in}}%
\pgfpathlineto{\pgfqpoint{6.192411in}{7.656023in}}%
\pgfusepath{stroke}%
\end{pgfscope}%
\begin{pgfscope}%
\pgfsetbuttcap%
\pgfsetroundjoin%
\definecolor{currentfill}{rgb}{0.000000,0.000000,0.000000}%
\pgfsetfillcolor{currentfill}%
\pgfsetlinewidth{0.501875pt}%
\definecolor{currentstroke}{rgb}{0.000000,0.000000,0.000000}%
\pgfsetstrokecolor{currentstroke}%
\pgfsetdash{}{0pt}%
\pgfsys@defobject{currentmarker}{\pgfqpoint{0.000000in}{0.000000in}}{\pgfqpoint{0.000000in}{0.055556in}}{%
\pgfpathmoveto{\pgfqpoint{0.000000in}{0.000000in}}%
\pgfpathlineto{\pgfqpoint{0.000000in}{0.055556in}}%
\pgfusepath{stroke,fill}%
}%
\begin{pgfscope}%
\pgfsys@transformshift{6.192411in}{4.462900in}%
\pgfsys@useobject{currentmarker}{}%
\end{pgfscope}%
\end{pgfscope}%
\begin{pgfscope}%
\pgfsetbuttcap%
\pgfsetroundjoin%
\definecolor{currentfill}{rgb}{0.000000,0.000000,0.000000}%
\pgfsetfillcolor{currentfill}%
\pgfsetlinewidth{0.501875pt}%
\definecolor{currentstroke}{rgb}{0.000000,0.000000,0.000000}%
\pgfsetstrokecolor{currentstroke}%
\pgfsetdash{}{0pt}%
\pgfsys@defobject{currentmarker}{\pgfqpoint{0.000000in}{-0.055556in}}{\pgfqpoint{0.000000in}{0.000000in}}{%
\pgfpathmoveto{\pgfqpoint{0.000000in}{0.000000in}}%
\pgfpathlineto{\pgfqpoint{0.000000in}{-0.055556in}}%
\pgfusepath{stroke,fill}%
}%
\begin{pgfscope}%
\pgfsys@transformshift{6.192411in}{7.656023in}%
\pgfsys@useobject{currentmarker}{}%
\end{pgfscope}%
\end{pgfscope}%
\begin{pgfscope}%
\pgftext[x=6.192411in,y=4.407345in,,top]{{\rmfamily\fontsize{10.000000}{12.000000}\selectfont \(\displaystyle 0.010\)}}%
\end{pgfscope}%
\begin{pgfscope}%
\pgftext[x=3.386589in,y=4.214443in,,top]{{\rmfamily\fontsize{10.000000}{12.000000}\selectfont time (s)}}%
\end{pgfscope}%
\begin{pgfscope}%
\pgfpathrectangle{\pgfqpoint{0.580766in}{4.462900in}}{\pgfqpoint{5.611646in}{3.193122in}} %
\pgfusepath{clip}%
\pgfsetbuttcap%
\pgfsetroundjoin%
\pgfsetlinewidth{0.501875pt}%
\definecolor{currentstroke}{rgb}{0.000000,0.000000,0.000000}%
\pgfsetstrokecolor{currentstroke}%
\pgfsetdash{{1.000000pt}{3.000000pt}}{0.000000pt}%
\pgfpathmoveto{\pgfqpoint{0.580766in}{4.462900in}}%
\pgfpathlineto{\pgfqpoint{6.192411in}{4.462900in}}%
\pgfusepath{stroke}%
\end{pgfscope}%
\begin{pgfscope}%
\pgfsetbuttcap%
\pgfsetroundjoin%
\definecolor{currentfill}{rgb}{0.000000,0.000000,0.000000}%
\pgfsetfillcolor{currentfill}%
\pgfsetlinewidth{0.501875pt}%
\definecolor{currentstroke}{rgb}{0.000000,0.000000,0.000000}%
\pgfsetstrokecolor{currentstroke}%
\pgfsetdash{}{0pt}%
\pgfsys@defobject{currentmarker}{\pgfqpoint{0.000000in}{0.000000in}}{\pgfqpoint{0.055556in}{0.000000in}}{%
\pgfpathmoveto{\pgfqpoint{0.000000in}{0.000000in}}%
\pgfpathlineto{\pgfqpoint{0.055556in}{0.000000in}}%
\pgfusepath{stroke,fill}%
}%
\begin{pgfscope}%
\pgfsys@transformshift{0.580766in}{4.462900in}%
\pgfsys@useobject{currentmarker}{}%
\end{pgfscope}%
\end{pgfscope}%
\begin{pgfscope}%
\pgfsetbuttcap%
\pgfsetroundjoin%
\definecolor{currentfill}{rgb}{0.000000,0.000000,0.000000}%
\pgfsetfillcolor{currentfill}%
\pgfsetlinewidth{0.501875pt}%
\definecolor{currentstroke}{rgb}{0.000000,0.000000,0.000000}%
\pgfsetstrokecolor{currentstroke}%
\pgfsetdash{}{0pt}%
\pgfsys@defobject{currentmarker}{\pgfqpoint{-0.055556in}{0.000000in}}{\pgfqpoint{0.000000in}{0.000000in}}{%
\pgfpathmoveto{\pgfqpoint{0.000000in}{0.000000in}}%
\pgfpathlineto{\pgfqpoint{-0.055556in}{0.000000in}}%
\pgfusepath{stroke,fill}%
}%
\begin{pgfscope}%
\pgfsys@transformshift{6.192411in}{4.462900in}%
\pgfsys@useobject{currentmarker}{}%
\end{pgfscope}%
\end{pgfscope}%
\begin{pgfscope}%
\pgftext[x=0.525210in,y=4.462900in,right,]{{\rmfamily\fontsize{10.000000}{12.000000}\selectfont \(\displaystyle 0.0\)}}%
\end{pgfscope}%
\begin{pgfscope}%
\pgfpathrectangle{\pgfqpoint{0.580766in}{4.462900in}}{\pgfqpoint{5.611646in}{3.193122in}} %
\pgfusepath{clip}%
\pgfsetbuttcap%
\pgfsetroundjoin%
\pgfsetlinewidth{0.501875pt}%
\definecolor{currentstroke}{rgb}{0.000000,0.000000,0.000000}%
\pgfsetstrokecolor{currentstroke}%
\pgfsetdash{{1.000000pt}{3.000000pt}}{0.000000pt}%
\pgfpathmoveto{\pgfqpoint{0.580766in}{5.101525in}}%
\pgfpathlineto{\pgfqpoint{6.192411in}{5.101525in}}%
\pgfusepath{stroke}%
\end{pgfscope}%
\begin{pgfscope}%
\pgfsetbuttcap%
\pgfsetroundjoin%
\definecolor{currentfill}{rgb}{0.000000,0.000000,0.000000}%
\pgfsetfillcolor{currentfill}%
\pgfsetlinewidth{0.501875pt}%
\definecolor{currentstroke}{rgb}{0.000000,0.000000,0.000000}%
\pgfsetstrokecolor{currentstroke}%
\pgfsetdash{}{0pt}%
\pgfsys@defobject{currentmarker}{\pgfqpoint{0.000000in}{0.000000in}}{\pgfqpoint{0.055556in}{0.000000in}}{%
\pgfpathmoveto{\pgfqpoint{0.000000in}{0.000000in}}%
\pgfpathlineto{\pgfqpoint{0.055556in}{0.000000in}}%
\pgfusepath{stroke,fill}%
}%
\begin{pgfscope}%
\pgfsys@transformshift{0.580766in}{5.101525in}%
\pgfsys@useobject{currentmarker}{}%
\end{pgfscope}%
\end{pgfscope}%
\begin{pgfscope}%
\pgfsetbuttcap%
\pgfsetroundjoin%
\definecolor{currentfill}{rgb}{0.000000,0.000000,0.000000}%
\pgfsetfillcolor{currentfill}%
\pgfsetlinewidth{0.501875pt}%
\definecolor{currentstroke}{rgb}{0.000000,0.000000,0.000000}%
\pgfsetstrokecolor{currentstroke}%
\pgfsetdash{}{0pt}%
\pgfsys@defobject{currentmarker}{\pgfqpoint{-0.055556in}{0.000000in}}{\pgfqpoint{0.000000in}{0.000000in}}{%
\pgfpathmoveto{\pgfqpoint{0.000000in}{0.000000in}}%
\pgfpathlineto{\pgfqpoint{-0.055556in}{0.000000in}}%
\pgfusepath{stroke,fill}%
}%
\begin{pgfscope}%
\pgfsys@transformshift{6.192411in}{5.101525in}%
\pgfsys@useobject{currentmarker}{}%
\end{pgfscope}%
\end{pgfscope}%
\begin{pgfscope}%
\pgftext[x=0.525210in,y=5.101525in,right,]{{\rmfamily\fontsize{10.000000}{12.000000}\selectfont \(\displaystyle 0.2\)}}%
\end{pgfscope}%
\begin{pgfscope}%
\pgfpathrectangle{\pgfqpoint{0.580766in}{4.462900in}}{\pgfqpoint{5.611646in}{3.193122in}} %
\pgfusepath{clip}%
\pgfsetbuttcap%
\pgfsetroundjoin%
\pgfsetlinewidth{0.501875pt}%
\definecolor{currentstroke}{rgb}{0.000000,0.000000,0.000000}%
\pgfsetstrokecolor{currentstroke}%
\pgfsetdash{{1.000000pt}{3.000000pt}}{0.000000pt}%
\pgfpathmoveto{\pgfqpoint{0.580766in}{5.740149in}}%
\pgfpathlineto{\pgfqpoint{6.192411in}{5.740149in}}%
\pgfusepath{stroke}%
\end{pgfscope}%
\begin{pgfscope}%
\pgfsetbuttcap%
\pgfsetroundjoin%
\definecolor{currentfill}{rgb}{0.000000,0.000000,0.000000}%
\pgfsetfillcolor{currentfill}%
\pgfsetlinewidth{0.501875pt}%
\definecolor{currentstroke}{rgb}{0.000000,0.000000,0.000000}%
\pgfsetstrokecolor{currentstroke}%
\pgfsetdash{}{0pt}%
\pgfsys@defobject{currentmarker}{\pgfqpoint{0.000000in}{0.000000in}}{\pgfqpoint{0.055556in}{0.000000in}}{%
\pgfpathmoveto{\pgfqpoint{0.000000in}{0.000000in}}%
\pgfpathlineto{\pgfqpoint{0.055556in}{0.000000in}}%
\pgfusepath{stroke,fill}%
}%
\begin{pgfscope}%
\pgfsys@transformshift{0.580766in}{5.740149in}%
\pgfsys@useobject{currentmarker}{}%
\end{pgfscope}%
\end{pgfscope}%
\begin{pgfscope}%
\pgfsetbuttcap%
\pgfsetroundjoin%
\definecolor{currentfill}{rgb}{0.000000,0.000000,0.000000}%
\pgfsetfillcolor{currentfill}%
\pgfsetlinewidth{0.501875pt}%
\definecolor{currentstroke}{rgb}{0.000000,0.000000,0.000000}%
\pgfsetstrokecolor{currentstroke}%
\pgfsetdash{}{0pt}%
\pgfsys@defobject{currentmarker}{\pgfqpoint{-0.055556in}{0.000000in}}{\pgfqpoint{0.000000in}{0.000000in}}{%
\pgfpathmoveto{\pgfqpoint{0.000000in}{0.000000in}}%
\pgfpathlineto{\pgfqpoint{-0.055556in}{0.000000in}}%
\pgfusepath{stroke,fill}%
}%
\begin{pgfscope}%
\pgfsys@transformshift{6.192411in}{5.740149in}%
\pgfsys@useobject{currentmarker}{}%
\end{pgfscope}%
\end{pgfscope}%
\begin{pgfscope}%
\pgftext[x=0.525210in,y=5.740149in,right,]{{\rmfamily\fontsize{10.000000}{12.000000}\selectfont \(\displaystyle 0.4\)}}%
\end{pgfscope}%
\begin{pgfscope}%
\pgfpathrectangle{\pgfqpoint{0.580766in}{4.462900in}}{\pgfqpoint{5.611646in}{3.193122in}} %
\pgfusepath{clip}%
\pgfsetbuttcap%
\pgfsetroundjoin%
\pgfsetlinewidth{0.501875pt}%
\definecolor{currentstroke}{rgb}{0.000000,0.000000,0.000000}%
\pgfsetstrokecolor{currentstroke}%
\pgfsetdash{{1.000000pt}{3.000000pt}}{0.000000pt}%
\pgfpathmoveto{\pgfqpoint{0.580766in}{6.378774in}}%
\pgfpathlineto{\pgfqpoint{6.192411in}{6.378774in}}%
\pgfusepath{stroke}%
\end{pgfscope}%
\begin{pgfscope}%
\pgfsetbuttcap%
\pgfsetroundjoin%
\definecolor{currentfill}{rgb}{0.000000,0.000000,0.000000}%
\pgfsetfillcolor{currentfill}%
\pgfsetlinewidth{0.501875pt}%
\definecolor{currentstroke}{rgb}{0.000000,0.000000,0.000000}%
\pgfsetstrokecolor{currentstroke}%
\pgfsetdash{}{0pt}%
\pgfsys@defobject{currentmarker}{\pgfqpoint{0.000000in}{0.000000in}}{\pgfqpoint{0.055556in}{0.000000in}}{%
\pgfpathmoveto{\pgfqpoint{0.000000in}{0.000000in}}%
\pgfpathlineto{\pgfqpoint{0.055556in}{0.000000in}}%
\pgfusepath{stroke,fill}%
}%
\begin{pgfscope}%
\pgfsys@transformshift{0.580766in}{6.378774in}%
\pgfsys@useobject{currentmarker}{}%
\end{pgfscope}%
\end{pgfscope}%
\begin{pgfscope}%
\pgfsetbuttcap%
\pgfsetroundjoin%
\definecolor{currentfill}{rgb}{0.000000,0.000000,0.000000}%
\pgfsetfillcolor{currentfill}%
\pgfsetlinewidth{0.501875pt}%
\definecolor{currentstroke}{rgb}{0.000000,0.000000,0.000000}%
\pgfsetstrokecolor{currentstroke}%
\pgfsetdash{}{0pt}%
\pgfsys@defobject{currentmarker}{\pgfqpoint{-0.055556in}{0.000000in}}{\pgfqpoint{0.000000in}{0.000000in}}{%
\pgfpathmoveto{\pgfqpoint{0.000000in}{0.000000in}}%
\pgfpathlineto{\pgfqpoint{-0.055556in}{0.000000in}}%
\pgfusepath{stroke,fill}%
}%
\begin{pgfscope}%
\pgfsys@transformshift{6.192411in}{6.378774in}%
\pgfsys@useobject{currentmarker}{}%
\end{pgfscope}%
\end{pgfscope}%
\begin{pgfscope}%
\pgftext[x=0.525210in,y=6.378774in,right,]{{\rmfamily\fontsize{10.000000}{12.000000}\selectfont \(\displaystyle 0.6\)}}%
\end{pgfscope}%
\begin{pgfscope}%
\pgfpathrectangle{\pgfqpoint{0.580766in}{4.462900in}}{\pgfqpoint{5.611646in}{3.193122in}} %
\pgfusepath{clip}%
\pgfsetbuttcap%
\pgfsetroundjoin%
\pgfsetlinewidth{0.501875pt}%
\definecolor{currentstroke}{rgb}{0.000000,0.000000,0.000000}%
\pgfsetstrokecolor{currentstroke}%
\pgfsetdash{{1.000000pt}{3.000000pt}}{0.000000pt}%
\pgfpathmoveto{\pgfqpoint{0.580766in}{7.017398in}}%
\pgfpathlineto{\pgfqpoint{6.192411in}{7.017398in}}%
\pgfusepath{stroke}%
\end{pgfscope}%
\begin{pgfscope}%
\pgfsetbuttcap%
\pgfsetroundjoin%
\definecolor{currentfill}{rgb}{0.000000,0.000000,0.000000}%
\pgfsetfillcolor{currentfill}%
\pgfsetlinewidth{0.501875pt}%
\definecolor{currentstroke}{rgb}{0.000000,0.000000,0.000000}%
\pgfsetstrokecolor{currentstroke}%
\pgfsetdash{}{0pt}%
\pgfsys@defobject{currentmarker}{\pgfqpoint{0.000000in}{0.000000in}}{\pgfqpoint{0.055556in}{0.000000in}}{%
\pgfpathmoveto{\pgfqpoint{0.000000in}{0.000000in}}%
\pgfpathlineto{\pgfqpoint{0.055556in}{0.000000in}}%
\pgfusepath{stroke,fill}%
}%
\begin{pgfscope}%
\pgfsys@transformshift{0.580766in}{7.017398in}%
\pgfsys@useobject{currentmarker}{}%
\end{pgfscope}%
\end{pgfscope}%
\begin{pgfscope}%
\pgfsetbuttcap%
\pgfsetroundjoin%
\definecolor{currentfill}{rgb}{0.000000,0.000000,0.000000}%
\pgfsetfillcolor{currentfill}%
\pgfsetlinewidth{0.501875pt}%
\definecolor{currentstroke}{rgb}{0.000000,0.000000,0.000000}%
\pgfsetstrokecolor{currentstroke}%
\pgfsetdash{}{0pt}%
\pgfsys@defobject{currentmarker}{\pgfqpoint{-0.055556in}{0.000000in}}{\pgfqpoint{0.000000in}{0.000000in}}{%
\pgfpathmoveto{\pgfqpoint{0.000000in}{0.000000in}}%
\pgfpathlineto{\pgfqpoint{-0.055556in}{0.000000in}}%
\pgfusepath{stroke,fill}%
}%
\begin{pgfscope}%
\pgfsys@transformshift{6.192411in}{7.017398in}%
\pgfsys@useobject{currentmarker}{}%
\end{pgfscope}%
\end{pgfscope}%
\begin{pgfscope}%
\pgftext[x=0.525210in,y=7.017398in,right,]{{\rmfamily\fontsize{10.000000}{12.000000}\selectfont \(\displaystyle 0.8\)}}%
\end{pgfscope}%
\begin{pgfscope}%
\pgfpathrectangle{\pgfqpoint{0.580766in}{4.462900in}}{\pgfqpoint{5.611646in}{3.193122in}} %
\pgfusepath{clip}%
\pgfsetbuttcap%
\pgfsetroundjoin%
\pgfsetlinewidth{0.501875pt}%
\definecolor{currentstroke}{rgb}{0.000000,0.000000,0.000000}%
\pgfsetstrokecolor{currentstroke}%
\pgfsetdash{{1.000000pt}{3.000000pt}}{0.000000pt}%
\pgfpathmoveto{\pgfqpoint{0.580766in}{7.656023in}}%
\pgfpathlineto{\pgfqpoint{6.192411in}{7.656023in}}%
\pgfusepath{stroke}%
\end{pgfscope}%
\begin{pgfscope}%
\pgfsetbuttcap%
\pgfsetroundjoin%
\definecolor{currentfill}{rgb}{0.000000,0.000000,0.000000}%
\pgfsetfillcolor{currentfill}%
\pgfsetlinewidth{0.501875pt}%
\definecolor{currentstroke}{rgb}{0.000000,0.000000,0.000000}%
\pgfsetstrokecolor{currentstroke}%
\pgfsetdash{}{0pt}%
\pgfsys@defobject{currentmarker}{\pgfqpoint{0.000000in}{0.000000in}}{\pgfqpoint{0.055556in}{0.000000in}}{%
\pgfpathmoveto{\pgfqpoint{0.000000in}{0.000000in}}%
\pgfpathlineto{\pgfqpoint{0.055556in}{0.000000in}}%
\pgfusepath{stroke,fill}%
}%
\begin{pgfscope}%
\pgfsys@transformshift{0.580766in}{7.656023in}%
\pgfsys@useobject{currentmarker}{}%
\end{pgfscope}%
\end{pgfscope}%
\begin{pgfscope}%
\pgfsetbuttcap%
\pgfsetroundjoin%
\definecolor{currentfill}{rgb}{0.000000,0.000000,0.000000}%
\pgfsetfillcolor{currentfill}%
\pgfsetlinewidth{0.501875pt}%
\definecolor{currentstroke}{rgb}{0.000000,0.000000,0.000000}%
\pgfsetstrokecolor{currentstroke}%
\pgfsetdash{}{0pt}%
\pgfsys@defobject{currentmarker}{\pgfqpoint{-0.055556in}{0.000000in}}{\pgfqpoint{0.000000in}{0.000000in}}{%
\pgfpathmoveto{\pgfqpoint{0.000000in}{0.000000in}}%
\pgfpathlineto{\pgfqpoint{-0.055556in}{0.000000in}}%
\pgfusepath{stroke,fill}%
}%
\begin{pgfscope}%
\pgfsys@transformshift{6.192411in}{7.656023in}%
\pgfsys@useobject{currentmarker}{}%
\end{pgfscope}%
\end{pgfscope}%
\begin{pgfscope}%
\pgftext[x=0.525210in,y=7.656023in,right,]{{\rmfamily\fontsize{10.000000}{12.000000}\selectfont \(\displaystyle 1.0\)}}%
\end{pgfscope}%
\begin{pgfscope}%
\pgftext[x=0.278296in,y=6.059461in,,bottom,rotate=90.000000]{{\rmfamily\fontsize{10.000000}{12.000000}\selectfont current (A)}}%
\end{pgfscope}%
\begin{pgfscope}%
\pgfsetbuttcap%
\pgfsetroundjoin%
\pgfsetlinewidth{1.003750pt}%
\definecolor{currentstroke}{rgb}{0.000000,0.000000,0.000000}%
\pgfsetstrokecolor{currentstroke}%
\pgfsetdash{}{0pt}%
\pgfpathmoveto{\pgfqpoint{0.580766in}{7.656023in}}%
\pgfpathlineto{\pgfqpoint{6.192411in}{7.656023in}}%
\pgfusepath{stroke}%
\end{pgfscope}%
\begin{pgfscope}%
\pgfsetbuttcap%
\pgfsetroundjoin%
\pgfsetlinewidth{1.003750pt}%
\definecolor{currentstroke}{rgb}{0.000000,0.000000,0.000000}%
\pgfsetstrokecolor{currentstroke}%
\pgfsetdash{}{0pt}%
\pgfpathmoveto{\pgfqpoint{6.192411in}{4.462900in}}%
\pgfpathlineto{\pgfqpoint{6.192411in}{7.656023in}}%
\pgfusepath{stroke}%
\end{pgfscope}%
\begin{pgfscope}%
\pgfsetbuttcap%
\pgfsetroundjoin%
\pgfsetlinewidth{1.003750pt}%
\definecolor{currentstroke}{rgb}{0.000000,0.000000,0.000000}%
\pgfsetstrokecolor{currentstroke}%
\pgfsetdash{}{0pt}%
\pgfpathmoveto{\pgfqpoint{0.580766in}{4.462900in}}%
\pgfpathlineto{\pgfqpoint{6.192411in}{4.462900in}}%
\pgfusepath{stroke}%
\end{pgfscope}%
\begin{pgfscope}%
\pgfsetbuttcap%
\pgfsetroundjoin%
\pgfsetlinewidth{1.003750pt}%
\definecolor{currentstroke}{rgb}{0.000000,0.000000,0.000000}%
\pgfsetstrokecolor{currentstroke}%
\pgfsetdash{}{0pt}%
\pgfpathmoveto{\pgfqpoint{0.580766in}{4.462900in}}%
\pgfpathlineto{\pgfqpoint{0.580766in}{7.656023in}}%
\pgfusepath{stroke}%
\end{pgfscope}%
\begin{pgfscope}%
\pgftext[x=3.386589in,y=7.725467in,,base]{{\rmfamily\fontsize{12.000000}{14.400000}\selectfont Step-By-Step Approximations of \(\displaystyle i(t)\)}}%
\end{pgfscope}%
\begin{pgfscope}%
\pgfsetbuttcap%
\pgfsetroundjoin%
\definecolor{currentfill}{rgb}{0.300000,0.300000,0.300000}%
\pgfsetfillcolor{currentfill}%
\pgfsetfillopacity{0.500000}%
\pgfsetlinewidth{1.003750pt}%
\definecolor{currentstroke}{rgb}{0.300000,0.300000,0.300000}%
\pgfsetstrokecolor{currentstroke}%
\pgfsetstrokeopacity{0.500000}%
\pgfsetdash{}{0pt}%
\pgfpathmoveto{\pgfqpoint{4.378033in}{4.518456in}}%
\pgfpathlineto{\pgfqpoint{6.103523in}{4.518456in}}%
\pgfpathquadraticcurveto{\pgfqpoint{6.136856in}{4.518456in}}{\pgfqpoint{6.136856in}{4.551789in}}%
\pgfpathlineto{\pgfqpoint{6.136856in}{5.232344in}}%
\pgfpathquadraticcurveto{\pgfqpoint{6.136856in}{5.265677in}}{\pgfqpoint{6.103523in}{5.265677in}}%
\pgfpathlineto{\pgfqpoint{4.378033in}{5.265677in}}%
\pgfpathquadraticcurveto{\pgfqpoint{4.344700in}{5.265677in}}{\pgfqpoint{4.344700in}{5.232344in}}%
\pgfpathlineto{\pgfqpoint{4.344700in}{4.551789in}}%
\pgfpathquadraticcurveto{\pgfqpoint{4.344700in}{4.518456in}}{\pgfqpoint{4.378033in}{4.518456in}}%
\pgfpathclose%
\pgfusepath{stroke,fill}%
\end{pgfscope}%
\begin{pgfscope}%
\pgfsetbuttcap%
\pgfsetroundjoin%
\definecolor{currentfill}{rgb}{1.000000,1.000000,1.000000}%
\pgfsetfillcolor{currentfill}%
\pgfsetlinewidth{1.003750pt}%
\definecolor{currentstroke}{rgb}{0.000000,0.000000,0.000000}%
\pgfsetstrokecolor{currentstroke}%
\pgfsetdash{}{0pt}%
\pgfpathmoveto{\pgfqpoint{4.350255in}{4.546233in}}%
\pgfpathlineto{\pgfqpoint{6.075745in}{4.546233in}}%
\pgfpathquadraticcurveto{\pgfqpoint{6.109078in}{4.546233in}}{\pgfqpoint{6.109078in}{4.579567in}}%
\pgfpathlineto{\pgfqpoint{6.109078in}{5.260121in}}%
\pgfpathquadraticcurveto{\pgfqpoint{6.109078in}{5.293455in}}{\pgfqpoint{6.075745in}{5.293455in}}%
\pgfpathlineto{\pgfqpoint{4.350255in}{5.293455in}}%
\pgfpathquadraticcurveto{\pgfqpoint{4.316922in}{5.293455in}}{\pgfqpoint{4.316922in}{5.260121in}}%
\pgfpathlineto{\pgfqpoint{4.316922in}{4.579567in}}%
\pgfpathquadraticcurveto{\pgfqpoint{4.316922in}{4.546233in}}{\pgfqpoint{4.350255in}{4.546233in}}%
\pgfpathclose%
\pgfusepath{stroke,fill}%
\end{pgfscope}%
\begin{pgfscope}%
\pgfsetbuttcap%
\pgfsetroundjoin%
\pgfsetlinewidth{1.003750pt}%
\definecolor{currentstroke}{rgb}{0.000000,0.000000,1.000000}%
\pgfsetstrokecolor{currentstroke}%
\pgfsetdash{{1.000000pt}{3.000000pt}}{0.000000pt}%
\pgfpathmoveto{\pgfqpoint{4.433589in}{5.168455in}}%
\pgfpathlineto{\pgfqpoint{4.666922in}{5.168455in}}%
\pgfusepath{stroke}%
\end{pgfscope}%
\begin{pgfscope}%
\pgftext[x=4.850255in,y=5.110121in,left,base]{{\rmfamily\fontsize{12.000000}{14.400000}\selectfont Continuous}}%
\end{pgfscope}%
\begin{pgfscope}%
\pgfsetrectcap%
\pgfsetroundjoin%
\pgfsetlinewidth{1.003750pt}%
\definecolor{currentstroke}{rgb}{0.000000,0.500000,0.000000}%
\pgfsetstrokecolor{currentstroke}%
\pgfsetdash{}{0pt}%
\pgfpathmoveto{\pgfqpoint{4.433589in}{4.936048in}}%
\pgfpathlineto{\pgfqpoint{4.666922in}{4.936048in}}%
\pgfusepath{stroke}%
\end{pgfscope}%
\begin{pgfscope}%
\pgftext[x=4.850255in,y=4.877714in,left,base]{{\rmfamily\fontsize{12.000000}{14.400000}\selectfont Trapezoidal}}%
\end{pgfscope}%
\begin{pgfscope}%
\pgfsetrectcap%
\pgfsetroundjoin%
\pgfsetlinewidth{1.003750pt}%
\definecolor{currentstroke}{rgb}{1.000000,0.000000,0.000000}%
\pgfsetstrokecolor{currentstroke}%
\pgfsetdash{}{0pt}%
\pgfpathmoveto{\pgfqpoint{4.433589in}{4.703641in}}%
\pgfpathlineto{\pgfqpoint{4.666922in}{4.703641in}}%
\pgfusepath{stroke}%
\end{pgfscope}%
\begin{pgfscope}%
\pgftext[x=4.850255in,y=4.645307in,left,base]{{\rmfamily\fontsize{12.000000}{14.400000}\selectfont Backward Euler    }}%
\end{pgfscope}%
\begin{pgfscope}%
\pgfsetbuttcap%
\pgfsetroundjoin%
\definecolor{currentfill}{rgb}{1.000000,1.000000,1.000000}%
\pgfsetfillcolor{currentfill}%
\pgfsetlinewidth{0.000000pt}%
\definecolor{currentstroke}{rgb}{0.000000,0.000000,0.000000}%
\pgfsetstrokecolor{currentstroke}%
\pgfsetstrokeopacity{0.000000}%
\pgfsetdash{}{0pt}%
\pgfpathmoveto{\pgfqpoint{0.580766in}{0.532919in}}%
\pgfpathlineto{\pgfqpoint{6.192411in}{0.532919in}}%
\pgfpathlineto{\pgfqpoint{6.192411in}{3.726041in}}%
\pgfpathlineto{\pgfqpoint{0.580766in}{3.726041in}}%
\pgfpathclose%
\pgfusepath{fill}%
\end{pgfscope}%
\begin{pgfscope}%
\pgfpathrectangle{\pgfqpoint{0.580766in}{0.532919in}}{\pgfqpoint{5.611646in}{3.193122in}} %
\pgfusepath{clip}%
\pgfsetbuttcap%
\pgfsetroundjoin%
\pgfsetlinewidth{1.003750pt}%
\definecolor{currentstroke}{rgb}{0.000000,0.000000,1.000000}%
\pgfsetstrokecolor{currentstroke}%
\pgfsetdash{{1.000000pt}{3.000000pt}}{0.000000pt}%
\pgfpathmoveto{\pgfqpoint{0.580766in}{3.726041in}}%
\pgfpathlineto{\pgfqpoint{0.633141in}{3.580452in}}%
\pgfpathlineto{\pgfqpoint{0.685517in}{3.441501in}}%
\pgfpathlineto{\pgfqpoint{0.737892in}{3.308886in}}%
\pgfpathlineto{\pgfqpoint{0.790267in}{3.182317in}}%
\pgfpathlineto{\pgfqpoint{0.842643in}{3.061519in}}%
\pgfpathlineto{\pgfqpoint{0.895018in}{2.946229in}}%
\pgfpathlineto{\pgfqpoint{0.947393in}{2.836195in}}%
\pgfpathlineto{\pgfqpoint{0.999769in}{2.731178in}}%
\pgfpathlineto{\pgfqpoint{1.054015in}{2.627456in}}%
\pgfpathlineto{\pgfqpoint{1.108261in}{2.528627in}}%
\pgfpathlineto{\pgfqpoint{1.162506in}{2.434462in}}%
\pgfpathlineto{\pgfqpoint{1.216752in}{2.344740in}}%
\pgfpathlineto{\pgfqpoint{1.270998in}{2.259251in}}%
\pgfpathlineto{\pgfqpoint{1.325244in}{2.177796in}}%
\pgfpathlineto{\pgfqpoint{1.381361in}{2.097574in}}%
\pgfpathlineto{\pgfqpoint{1.437477in}{2.021265in}}%
\pgfpathlineto{\pgfqpoint{1.493594in}{1.948678in}}%
\pgfpathlineto{\pgfqpoint{1.549710in}{1.879630in}}%
\pgfpathlineto{\pgfqpoint{1.607697in}{1.811817in}}%
\pgfpathlineto{\pgfqpoint{1.665684in}{1.747419in}}%
\pgfpathlineto{\pgfqpoint{1.723671in}{1.686263in}}%
\pgfpathlineto{\pgfqpoint{1.783529in}{1.626363in}}%
\pgfpathlineto{\pgfqpoint{1.843386in}{1.569574in}}%
\pgfpathlineto{\pgfqpoint{1.905114in}{1.514097in}}%
\pgfpathlineto{\pgfqpoint{1.966842in}{1.461590in}}%
\pgfpathlineto{\pgfqpoint{2.030441in}{1.410428in}}%
\pgfpathlineto{\pgfqpoint{2.094040in}{1.362085in}}%
\pgfpathlineto{\pgfqpoint{2.159509in}{1.315101in}}%
\pgfpathlineto{\pgfqpoint{2.226849in}{1.269550in}}%
\pgfpathlineto{\pgfqpoint{2.294188in}{1.226652in}}%
\pgfpathlineto{\pgfqpoint{2.363399in}{1.185164in}}%
\pgfpathlineto{\pgfqpoint{2.434479in}{1.145136in}}%
\pgfpathlineto{\pgfqpoint{2.507431in}{1.106608in}}%
\pgfpathlineto{\pgfqpoint{2.582253in}{1.069609in}}%
\pgfpathlineto{\pgfqpoint{2.658945in}{1.034160in}}%
\pgfpathlineto{\pgfqpoint{2.737508in}{1.000273in}}%
\pgfpathlineto{\pgfqpoint{2.819812in}{0.967227in}}%
\pgfpathlineto{\pgfqpoint{2.903987in}{0.935845in}}%
\pgfpathlineto{\pgfqpoint{2.991903in}{0.905487in}}%
\pgfpathlineto{\pgfqpoint{3.081689in}{0.876843in}}%
\pgfpathlineto{\pgfqpoint{3.175217in}{0.849344in}}%
\pgfpathlineto{\pgfqpoint{3.272485in}{0.823075in}}%
\pgfpathlineto{\pgfqpoint{3.373495in}{0.798102in}}%
\pgfpathlineto{\pgfqpoint{3.480116in}{0.774069in}}%
\pgfpathlineto{\pgfqpoint{3.590478in}{0.751484in}}%
\pgfpathlineto{\pgfqpoint{3.706452in}{0.730027in}}%
\pgfpathlineto{\pgfqpoint{3.828038in}{0.709790in}}%
\pgfpathlineto{\pgfqpoint{3.957106in}{0.690575in}}%
\pgfpathlineto{\pgfqpoint{4.093656in}{0.672515in}}%
\pgfpathlineto{\pgfqpoint{4.239559in}{0.655497in}}%
\pgfpathlineto{\pgfqpoint{4.394814in}{0.639661in}}%
\pgfpathlineto{\pgfqpoint{4.561293in}{0.624946in}}%
\pgfpathlineto{\pgfqpoint{4.740866in}{0.611339in}}%
\pgfpathlineto{\pgfqpoint{4.935403in}{0.598859in}}%
\pgfpathlineto{\pgfqpoint{5.148645in}{0.587449in}}%
\pgfpathlineto{\pgfqpoint{5.382464in}{0.577194in}}%
\pgfpathlineto{\pgfqpoint{5.644341in}{0.567979in}}%
\pgfpathlineto{\pgfqpoint{5.939887in}{0.559862in}}%
\pgfpathlineto{\pgfqpoint{6.190541in}{0.554469in}}%
\pgfpathlineto{\pgfqpoint{6.190541in}{0.554469in}}%
\pgfusepath{stroke}%
\end{pgfscope}%
\begin{pgfscope}%
\pgfpathrectangle{\pgfqpoint{0.580766in}{0.532919in}}{\pgfqpoint{5.611646in}{3.193122in}} %
\pgfusepath{clip}%
\pgfsetrectcap%
\pgfsetroundjoin%
\pgfsetlinewidth{1.003750pt}%
\definecolor{currentstroke}{rgb}{0.000000,0.500000,0.000000}%
\pgfsetstrokecolor{currentstroke}%
\pgfsetdash{}{0pt}%
\pgfpathmoveto{\pgfqpoint{0.580766in}{2.661667in}}%
\pgfpathlineto{\pgfqpoint{1.029698in}{1.952084in}}%
\pgfpathlineto{\pgfqpoint{1.478629in}{1.479029in}}%
\pgfpathlineto{\pgfqpoint{1.927561in}{1.163659in}}%
\pgfpathlineto{\pgfqpoint{2.376492in}{0.953412in}}%
\pgfpathlineto{\pgfqpoint{2.825424in}{0.813247in}}%
\pgfpathlineto{\pgfqpoint{3.274356in}{0.719805in}}%
\pgfpathlineto{\pgfqpoint{3.723287in}{0.657509in}}%
\pgfpathlineto{\pgfqpoint{4.172219in}{0.615979in}}%
\pgfpathlineto{\pgfqpoint{4.621151in}{0.588292in}}%
\pgfpathlineto{\pgfqpoint{5.070082in}{0.569834in}}%
\pgfpathlineto{\pgfqpoint{5.519014in}{0.557529in}}%
\pgfpathlineto{\pgfqpoint{5.967946in}{0.549326in}}%
\pgfusepath{stroke}%
\end{pgfscope}%
\begin{pgfscope}%
\pgfpathrectangle{\pgfqpoint{0.580766in}{0.532919in}}{\pgfqpoint{5.611646in}{3.193122in}} %
\pgfusepath{clip}%
\pgfsetrectcap%
\pgfsetroundjoin%
\pgfsetlinewidth{1.003750pt}%
\definecolor{currentstroke}{rgb}{1.000000,0.000000,0.000000}%
\pgfsetstrokecolor{currentstroke}%
\pgfsetdash{}{0pt}%
\pgfpathmoveto{\pgfqpoint{0.580766in}{2.813720in}}%
\pgfpathlineto{\pgfqpoint{1.029698in}{2.162063in}}%
\pgfpathlineto{\pgfqpoint{1.478629in}{1.696593in}}%
\pgfpathlineto{\pgfqpoint{1.927561in}{1.364114in}}%
\pgfpathlineto{\pgfqpoint{2.376492in}{1.126630in}}%
\pgfpathlineto{\pgfqpoint{2.825424in}{0.956998in}}%
\pgfpathlineto{\pgfqpoint{3.274356in}{0.835832in}}%
\pgfpathlineto{\pgfqpoint{3.723287in}{0.749286in}}%
\pgfpathlineto{\pgfqpoint{4.172219in}{0.687466in}}%
\pgfpathlineto{\pgfqpoint{4.621151in}{0.643310in}}%
\pgfpathlineto{\pgfqpoint{5.070082in}{0.611770in}}%
\pgfpathlineto{\pgfqpoint{5.519014in}{0.589241in}}%
\pgfpathlineto{\pgfqpoint{5.967946in}{0.573149in}}%
\pgfusepath{stroke}%
\end{pgfscope}%
\begin{pgfscope}%
\pgfpathrectangle{\pgfqpoint{0.580766in}{0.532919in}}{\pgfqpoint{5.611646in}{3.193122in}} %
\pgfusepath{clip}%
\pgfsetbuttcap%
\pgfsetroundjoin%
\pgfsetlinewidth{0.501875pt}%
\definecolor{currentstroke}{rgb}{0.000000,0.000000,0.000000}%
\pgfsetstrokecolor{currentstroke}%
\pgfsetdash{{1.000000pt}{3.000000pt}}{0.000000pt}%
\pgfpathmoveto{\pgfqpoint{0.580766in}{0.532919in}}%
\pgfpathlineto{\pgfqpoint{0.580766in}{3.726041in}}%
\pgfusepath{stroke}%
\end{pgfscope}%
\begin{pgfscope}%
\pgfsetbuttcap%
\pgfsetroundjoin%
\definecolor{currentfill}{rgb}{0.000000,0.000000,0.000000}%
\pgfsetfillcolor{currentfill}%
\pgfsetlinewidth{0.501875pt}%
\definecolor{currentstroke}{rgb}{0.000000,0.000000,0.000000}%
\pgfsetstrokecolor{currentstroke}%
\pgfsetdash{}{0pt}%
\pgfsys@defobject{currentmarker}{\pgfqpoint{0.000000in}{0.000000in}}{\pgfqpoint{0.000000in}{0.055556in}}{%
\pgfpathmoveto{\pgfqpoint{0.000000in}{0.000000in}}%
\pgfpathlineto{\pgfqpoint{0.000000in}{0.055556in}}%
\pgfusepath{stroke,fill}%
}%
\begin{pgfscope}%
\pgfsys@transformshift{0.580766in}{0.532919in}%
\pgfsys@useobject{currentmarker}{}%
\end{pgfscope}%
\end{pgfscope}%
\begin{pgfscope}%
\pgfsetbuttcap%
\pgfsetroundjoin%
\definecolor{currentfill}{rgb}{0.000000,0.000000,0.000000}%
\pgfsetfillcolor{currentfill}%
\pgfsetlinewidth{0.501875pt}%
\definecolor{currentstroke}{rgb}{0.000000,0.000000,0.000000}%
\pgfsetstrokecolor{currentstroke}%
\pgfsetdash{}{0pt}%
\pgfsys@defobject{currentmarker}{\pgfqpoint{0.000000in}{-0.055556in}}{\pgfqpoint{0.000000in}{0.000000in}}{%
\pgfpathmoveto{\pgfqpoint{0.000000in}{0.000000in}}%
\pgfpathlineto{\pgfqpoint{0.000000in}{-0.055556in}}%
\pgfusepath{stroke,fill}%
}%
\begin{pgfscope}%
\pgfsys@transformshift{0.580766in}{3.726041in}%
\pgfsys@useobject{currentmarker}{}%
\end{pgfscope}%
\end{pgfscope}%
\begin{pgfscope}%
\pgftext[x=0.580766in,y=0.477363in,,top]{{\rmfamily\fontsize{10.000000}{12.000000}\selectfont \(\displaystyle 0.000\)}}%
\end{pgfscope}%
\begin{pgfscope}%
\pgfpathrectangle{\pgfqpoint{0.580766in}{0.532919in}}{\pgfqpoint{5.611646in}{3.193122in}} %
\pgfusepath{clip}%
\pgfsetbuttcap%
\pgfsetroundjoin%
\pgfsetlinewidth{0.501875pt}%
\definecolor{currentstroke}{rgb}{0.000000,0.000000,0.000000}%
\pgfsetstrokecolor{currentstroke}%
\pgfsetdash{{1.000000pt}{3.000000pt}}{0.000000pt}%
\pgfpathmoveto{\pgfqpoint{1.703095in}{0.532919in}}%
\pgfpathlineto{\pgfqpoint{1.703095in}{3.726041in}}%
\pgfusepath{stroke}%
\end{pgfscope}%
\begin{pgfscope}%
\pgfsetbuttcap%
\pgfsetroundjoin%
\definecolor{currentfill}{rgb}{0.000000,0.000000,0.000000}%
\pgfsetfillcolor{currentfill}%
\pgfsetlinewidth{0.501875pt}%
\definecolor{currentstroke}{rgb}{0.000000,0.000000,0.000000}%
\pgfsetstrokecolor{currentstroke}%
\pgfsetdash{}{0pt}%
\pgfsys@defobject{currentmarker}{\pgfqpoint{0.000000in}{0.000000in}}{\pgfqpoint{0.000000in}{0.055556in}}{%
\pgfpathmoveto{\pgfqpoint{0.000000in}{0.000000in}}%
\pgfpathlineto{\pgfqpoint{0.000000in}{0.055556in}}%
\pgfusepath{stroke,fill}%
}%
\begin{pgfscope}%
\pgfsys@transformshift{1.703095in}{0.532919in}%
\pgfsys@useobject{currentmarker}{}%
\end{pgfscope}%
\end{pgfscope}%
\begin{pgfscope}%
\pgfsetbuttcap%
\pgfsetroundjoin%
\definecolor{currentfill}{rgb}{0.000000,0.000000,0.000000}%
\pgfsetfillcolor{currentfill}%
\pgfsetlinewidth{0.501875pt}%
\definecolor{currentstroke}{rgb}{0.000000,0.000000,0.000000}%
\pgfsetstrokecolor{currentstroke}%
\pgfsetdash{}{0pt}%
\pgfsys@defobject{currentmarker}{\pgfqpoint{0.000000in}{-0.055556in}}{\pgfqpoint{0.000000in}{0.000000in}}{%
\pgfpathmoveto{\pgfqpoint{0.000000in}{0.000000in}}%
\pgfpathlineto{\pgfqpoint{0.000000in}{-0.055556in}}%
\pgfusepath{stroke,fill}%
}%
\begin{pgfscope}%
\pgfsys@transformshift{1.703095in}{3.726041in}%
\pgfsys@useobject{currentmarker}{}%
\end{pgfscope}%
\end{pgfscope}%
\begin{pgfscope}%
\pgftext[x=1.703095in,y=0.477363in,,top]{{\rmfamily\fontsize{10.000000}{12.000000}\selectfont \(\displaystyle 0.002\)}}%
\end{pgfscope}%
\begin{pgfscope}%
\pgfpathrectangle{\pgfqpoint{0.580766in}{0.532919in}}{\pgfqpoint{5.611646in}{3.193122in}} %
\pgfusepath{clip}%
\pgfsetbuttcap%
\pgfsetroundjoin%
\pgfsetlinewidth{0.501875pt}%
\definecolor{currentstroke}{rgb}{0.000000,0.000000,0.000000}%
\pgfsetstrokecolor{currentstroke}%
\pgfsetdash{{1.000000pt}{3.000000pt}}{0.000000pt}%
\pgfpathmoveto{\pgfqpoint{2.825424in}{0.532919in}}%
\pgfpathlineto{\pgfqpoint{2.825424in}{3.726041in}}%
\pgfusepath{stroke}%
\end{pgfscope}%
\begin{pgfscope}%
\pgfsetbuttcap%
\pgfsetroundjoin%
\definecolor{currentfill}{rgb}{0.000000,0.000000,0.000000}%
\pgfsetfillcolor{currentfill}%
\pgfsetlinewidth{0.501875pt}%
\definecolor{currentstroke}{rgb}{0.000000,0.000000,0.000000}%
\pgfsetstrokecolor{currentstroke}%
\pgfsetdash{}{0pt}%
\pgfsys@defobject{currentmarker}{\pgfqpoint{0.000000in}{0.000000in}}{\pgfqpoint{0.000000in}{0.055556in}}{%
\pgfpathmoveto{\pgfqpoint{0.000000in}{0.000000in}}%
\pgfpathlineto{\pgfqpoint{0.000000in}{0.055556in}}%
\pgfusepath{stroke,fill}%
}%
\begin{pgfscope}%
\pgfsys@transformshift{2.825424in}{0.532919in}%
\pgfsys@useobject{currentmarker}{}%
\end{pgfscope}%
\end{pgfscope}%
\begin{pgfscope}%
\pgfsetbuttcap%
\pgfsetroundjoin%
\definecolor{currentfill}{rgb}{0.000000,0.000000,0.000000}%
\pgfsetfillcolor{currentfill}%
\pgfsetlinewidth{0.501875pt}%
\definecolor{currentstroke}{rgb}{0.000000,0.000000,0.000000}%
\pgfsetstrokecolor{currentstroke}%
\pgfsetdash{}{0pt}%
\pgfsys@defobject{currentmarker}{\pgfqpoint{0.000000in}{-0.055556in}}{\pgfqpoint{0.000000in}{0.000000in}}{%
\pgfpathmoveto{\pgfqpoint{0.000000in}{0.000000in}}%
\pgfpathlineto{\pgfqpoint{0.000000in}{-0.055556in}}%
\pgfusepath{stroke,fill}%
}%
\begin{pgfscope}%
\pgfsys@transformshift{2.825424in}{3.726041in}%
\pgfsys@useobject{currentmarker}{}%
\end{pgfscope}%
\end{pgfscope}%
\begin{pgfscope}%
\pgftext[x=2.825424in,y=0.477363in,,top]{{\rmfamily\fontsize{10.000000}{12.000000}\selectfont \(\displaystyle 0.004\)}}%
\end{pgfscope}%
\begin{pgfscope}%
\pgfpathrectangle{\pgfqpoint{0.580766in}{0.532919in}}{\pgfqpoint{5.611646in}{3.193122in}} %
\pgfusepath{clip}%
\pgfsetbuttcap%
\pgfsetroundjoin%
\pgfsetlinewidth{0.501875pt}%
\definecolor{currentstroke}{rgb}{0.000000,0.000000,0.000000}%
\pgfsetstrokecolor{currentstroke}%
\pgfsetdash{{1.000000pt}{3.000000pt}}{0.000000pt}%
\pgfpathmoveto{\pgfqpoint{3.947753in}{0.532919in}}%
\pgfpathlineto{\pgfqpoint{3.947753in}{3.726041in}}%
\pgfusepath{stroke}%
\end{pgfscope}%
\begin{pgfscope}%
\pgfsetbuttcap%
\pgfsetroundjoin%
\definecolor{currentfill}{rgb}{0.000000,0.000000,0.000000}%
\pgfsetfillcolor{currentfill}%
\pgfsetlinewidth{0.501875pt}%
\definecolor{currentstroke}{rgb}{0.000000,0.000000,0.000000}%
\pgfsetstrokecolor{currentstroke}%
\pgfsetdash{}{0pt}%
\pgfsys@defobject{currentmarker}{\pgfqpoint{0.000000in}{0.000000in}}{\pgfqpoint{0.000000in}{0.055556in}}{%
\pgfpathmoveto{\pgfqpoint{0.000000in}{0.000000in}}%
\pgfpathlineto{\pgfqpoint{0.000000in}{0.055556in}}%
\pgfusepath{stroke,fill}%
}%
\begin{pgfscope}%
\pgfsys@transformshift{3.947753in}{0.532919in}%
\pgfsys@useobject{currentmarker}{}%
\end{pgfscope}%
\end{pgfscope}%
\begin{pgfscope}%
\pgfsetbuttcap%
\pgfsetroundjoin%
\definecolor{currentfill}{rgb}{0.000000,0.000000,0.000000}%
\pgfsetfillcolor{currentfill}%
\pgfsetlinewidth{0.501875pt}%
\definecolor{currentstroke}{rgb}{0.000000,0.000000,0.000000}%
\pgfsetstrokecolor{currentstroke}%
\pgfsetdash{}{0pt}%
\pgfsys@defobject{currentmarker}{\pgfqpoint{0.000000in}{-0.055556in}}{\pgfqpoint{0.000000in}{0.000000in}}{%
\pgfpathmoveto{\pgfqpoint{0.000000in}{0.000000in}}%
\pgfpathlineto{\pgfqpoint{0.000000in}{-0.055556in}}%
\pgfusepath{stroke,fill}%
}%
\begin{pgfscope}%
\pgfsys@transformshift{3.947753in}{3.726041in}%
\pgfsys@useobject{currentmarker}{}%
\end{pgfscope}%
\end{pgfscope}%
\begin{pgfscope}%
\pgftext[x=3.947753in,y=0.477363in,,top]{{\rmfamily\fontsize{10.000000}{12.000000}\selectfont \(\displaystyle 0.006\)}}%
\end{pgfscope}%
\begin{pgfscope}%
\pgfpathrectangle{\pgfqpoint{0.580766in}{0.532919in}}{\pgfqpoint{5.611646in}{3.193122in}} %
\pgfusepath{clip}%
\pgfsetbuttcap%
\pgfsetroundjoin%
\pgfsetlinewidth{0.501875pt}%
\definecolor{currentstroke}{rgb}{0.000000,0.000000,0.000000}%
\pgfsetstrokecolor{currentstroke}%
\pgfsetdash{{1.000000pt}{3.000000pt}}{0.000000pt}%
\pgfpathmoveto{\pgfqpoint{5.070082in}{0.532919in}}%
\pgfpathlineto{\pgfqpoint{5.070082in}{3.726041in}}%
\pgfusepath{stroke}%
\end{pgfscope}%
\begin{pgfscope}%
\pgfsetbuttcap%
\pgfsetroundjoin%
\definecolor{currentfill}{rgb}{0.000000,0.000000,0.000000}%
\pgfsetfillcolor{currentfill}%
\pgfsetlinewidth{0.501875pt}%
\definecolor{currentstroke}{rgb}{0.000000,0.000000,0.000000}%
\pgfsetstrokecolor{currentstroke}%
\pgfsetdash{}{0pt}%
\pgfsys@defobject{currentmarker}{\pgfqpoint{0.000000in}{0.000000in}}{\pgfqpoint{0.000000in}{0.055556in}}{%
\pgfpathmoveto{\pgfqpoint{0.000000in}{0.000000in}}%
\pgfpathlineto{\pgfqpoint{0.000000in}{0.055556in}}%
\pgfusepath{stroke,fill}%
}%
\begin{pgfscope}%
\pgfsys@transformshift{5.070082in}{0.532919in}%
\pgfsys@useobject{currentmarker}{}%
\end{pgfscope}%
\end{pgfscope}%
\begin{pgfscope}%
\pgfsetbuttcap%
\pgfsetroundjoin%
\definecolor{currentfill}{rgb}{0.000000,0.000000,0.000000}%
\pgfsetfillcolor{currentfill}%
\pgfsetlinewidth{0.501875pt}%
\definecolor{currentstroke}{rgb}{0.000000,0.000000,0.000000}%
\pgfsetstrokecolor{currentstroke}%
\pgfsetdash{}{0pt}%
\pgfsys@defobject{currentmarker}{\pgfqpoint{0.000000in}{-0.055556in}}{\pgfqpoint{0.000000in}{0.000000in}}{%
\pgfpathmoveto{\pgfqpoint{0.000000in}{0.000000in}}%
\pgfpathlineto{\pgfqpoint{0.000000in}{-0.055556in}}%
\pgfusepath{stroke,fill}%
}%
\begin{pgfscope}%
\pgfsys@transformshift{5.070082in}{3.726041in}%
\pgfsys@useobject{currentmarker}{}%
\end{pgfscope}%
\end{pgfscope}%
\begin{pgfscope}%
\pgftext[x=5.070082in,y=0.477363in,,top]{{\rmfamily\fontsize{10.000000}{12.000000}\selectfont \(\displaystyle 0.008\)}}%
\end{pgfscope}%
\begin{pgfscope}%
\pgfpathrectangle{\pgfqpoint{0.580766in}{0.532919in}}{\pgfqpoint{5.611646in}{3.193122in}} %
\pgfusepath{clip}%
\pgfsetbuttcap%
\pgfsetroundjoin%
\pgfsetlinewidth{0.501875pt}%
\definecolor{currentstroke}{rgb}{0.000000,0.000000,0.000000}%
\pgfsetstrokecolor{currentstroke}%
\pgfsetdash{{1.000000pt}{3.000000pt}}{0.000000pt}%
\pgfpathmoveto{\pgfqpoint{6.192411in}{0.532919in}}%
\pgfpathlineto{\pgfqpoint{6.192411in}{3.726041in}}%
\pgfusepath{stroke}%
\end{pgfscope}%
\begin{pgfscope}%
\pgfsetbuttcap%
\pgfsetroundjoin%
\definecolor{currentfill}{rgb}{0.000000,0.000000,0.000000}%
\pgfsetfillcolor{currentfill}%
\pgfsetlinewidth{0.501875pt}%
\definecolor{currentstroke}{rgb}{0.000000,0.000000,0.000000}%
\pgfsetstrokecolor{currentstroke}%
\pgfsetdash{}{0pt}%
\pgfsys@defobject{currentmarker}{\pgfqpoint{0.000000in}{0.000000in}}{\pgfqpoint{0.000000in}{0.055556in}}{%
\pgfpathmoveto{\pgfqpoint{0.000000in}{0.000000in}}%
\pgfpathlineto{\pgfqpoint{0.000000in}{0.055556in}}%
\pgfusepath{stroke,fill}%
}%
\begin{pgfscope}%
\pgfsys@transformshift{6.192411in}{0.532919in}%
\pgfsys@useobject{currentmarker}{}%
\end{pgfscope}%
\end{pgfscope}%
\begin{pgfscope}%
\pgfsetbuttcap%
\pgfsetroundjoin%
\definecolor{currentfill}{rgb}{0.000000,0.000000,0.000000}%
\pgfsetfillcolor{currentfill}%
\pgfsetlinewidth{0.501875pt}%
\definecolor{currentstroke}{rgb}{0.000000,0.000000,0.000000}%
\pgfsetstrokecolor{currentstroke}%
\pgfsetdash{}{0pt}%
\pgfsys@defobject{currentmarker}{\pgfqpoint{0.000000in}{-0.055556in}}{\pgfqpoint{0.000000in}{0.000000in}}{%
\pgfpathmoveto{\pgfqpoint{0.000000in}{0.000000in}}%
\pgfpathlineto{\pgfqpoint{0.000000in}{-0.055556in}}%
\pgfusepath{stroke,fill}%
}%
\begin{pgfscope}%
\pgfsys@transformshift{6.192411in}{3.726041in}%
\pgfsys@useobject{currentmarker}{}%
\end{pgfscope}%
\end{pgfscope}%
\begin{pgfscope}%
\pgftext[x=6.192411in,y=0.477363in,,top]{{\rmfamily\fontsize{10.000000}{12.000000}\selectfont \(\displaystyle 0.010\)}}%
\end{pgfscope}%
\begin{pgfscope}%
\pgftext[x=3.386589in,y=0.284462in,,top]{{\rmfamily\fontsize{10.000000}{12.000000}\selectfont time (s)}}%
\end{pgfscope}%
\begin{pgfscope}%
\pgfpathrectangle{\pgfqpoint{0.580766in}{0.532919in}}{\pgfqpoint{5.611646in}{3.193122in}} %
\pgfusepath{clip}%
\pgfsetbuttcap%
\pgfsetroundjoin%
\pgfsetlinewidth{0.501875pt}%
\definecolor{currentstroke}{rgb}{0.000000,0.000000,0.000000}%
\pgfsetstrokecolor{currentstroke}%
\pgfsetdash{{1.000000pt}{3.000000pt}}{0.000000pt}%
\pgfpathmoveto{\pgfqpoint{0.580766in}{0.532919in}}%
\pgfpathlineto{\pgfqpoint{6.192411in}{0.532919in}}%
\pgfusepath{stroke}%
\end{pgfscope}%
\begin{pgfscope}%
\pgfsetbuttcap%
\pgfsetroundjoin%
\definecolor{currentfill}{rgb}{0.000000,0.000000,0.000000}%
\pgfsetfillcolor{currentfill}%
\pgfsetlinewidth{0.501875pt}%
\definecolor{currentstroke}{rgb}{0.000000,0.000000,0.000000}%
\pgfsetstrokecolor{currentstroke}%
\pgfsetdash{}{0pt}%
\pgfsys@defobject{currentmarker}{\pgfqpoint{0.000000in}{0.000000in}}{\pgfqpoint{0.055556in}{0.000000in}}{%
\pgfpathmoveto{\pgfqpoint{0.000000in}{0.000000in}}%
\pgfpathlineto{\pgfqpoint{0.055556in}{0.000000in}}%
\pgfusepath{stroke,fill}%
}%
\begin{pgfscope}%
\pgfsys@transformshift{0.580766in}{0.532919in}%
\pgfsys@useobject{currentmarker}{}%
\end{pgfscope}%
\end{pgfscope}%
\begin{pgfscope}%
\pgfsetbuttcap%
\pgfsetroundjoin%
\definecolor{currentfill}{rgb}{0.000000,0.000000,0.000000}%
\pgfsetfillcolor{currentfill}%
\pgfsetlinewidth{0.501875pt}%
\definecolor{currentstroke}{rgb}{0.000000,0.000000,0.000000}%
\pgfsetstrokecolor{currentstroke}%
\pgfsetdash{}{0pt}%
\pgfsys@defobject{currentmarker}{\pgfqpoint{-0.055556in}{0.000000in}}{\pgfqpoint{0.000000in}{0.000000in}}{%
\pgfpathmoveto{\pgfqpoint{0.000000in}{0.000000in}}%
\pgfpathlineto{\pgfqpoint{-0.055556in}{0.000000in}}%
\pgfusepath{stroke,fill}%
}%
\begin{pgfscope}%
\pgfsys@transformshift{6.192411in}{0.532919in}%
\pgfsys@useobject{currentmarker}{}%
\end{pgfscope}%
\end{pgfscope}%
\begin{pgfscope}%
\pgftext[x=0.525210in,y=0.532919in,right,]{{\rmfamily\fontsize{10.000000}{12.000000}\selectfont \(\displaystyle 0\)}}%
\end{pgfscope}%
\begin{pgfscope}%
\pgfpathrectangle{\pgfqpoint{0.580766in}{0.532919in}}{\pgfqpoint{5.611646in}{3.193122in}} %
\pgfusepath{clip}%
\pgfsetbuttcap%
\pgfsetroundjoin%
\pgfsetlinewidth{0.501875pt}%
\definecolor{currentstroke}{rgb}{0.000000,0.000000,0.000000}%
\pgfsetstrokecolor{currentstroke}%
\pgfsetdash{{1.000000pt}{3.000000pt}}{0.000000pt}%
\pgfpathmoveto{\pgfqpoint{0.580766in}{1.171543in}}%
\pgfpathlineto{\pgfqpoint{6.192411in}{1.171543in}}%
\pgfusepath{stroke}%
\end{pgfscope}%
\begin{pgfscope}%
\pgfsetbuttcap%
\pgfsetroundjoin%
\definecolor{currentfill}{rgb}{0.000000,0.000000,0.000000}%
\pgfsetfillcolor{currentfill}%
\pgfsetlinewidth{0.501875pt}%
\definecolor{currentstroke}{rgb}{0.000000,0.000000,0.000000}%
\pgfsetstrokecolor{currentstroke}%
\pgfsetdash{}{0pt}%
\pgfsys@defobject{currentmarker}{\pgfqpoint{0.000000in}{0.000000in}}{\pgfqpoint{0.055556in}{0.000000in}}{%
\pgfpathmoveto{\pgfqpoint{0.000000in}{0.000000in}}%
\pgfpathlineto{\pgfqpoint{0.055556in}{0.000000in}}%
\pgfusepath{stroke,fill}%
}%
\begin{pgfscope}%
\pgfsys@transformshift{0.580766in}{1.171543in}%
\pgfsys@useobject{currentmarker}{}%
\end{pgfscope}%
\end{pgfscope}%
\begin{pgfscope}%
\pgfsetbuttcap%
\pgfsetroundjoin%
\definecolor{currentfill}{rgb}{0.000000,0.000000,0.000000}%
\pgfsetfillcolor{currentfill}%
\pgfsetlinewidth{0.501875pt}%
\definecolor{currentstroke}{rgb}{0.000000,0.000000,0.000000}%
\pgfsetstrokecolor{currentstroke}%
\pgfsetdash{}{0pt}%
\pgfsys@defobject{currentmarker}{\pgfqpoint{-0.055556in}{0.000000in}}{\pgfqpoint{0.000000in}{0.000000in}}{%
\pgfpathmoveto{\pgfqpoint{0.000000in}{0.000000in}}%
\pgfpathlineto{\pgfqpoint{-0.055556in}{0.000000in}}%
\pgfusepath{stroke,fill}%
}%
\begin{pgfscope}%
\pgfsys@transformshift{6.192411in}{1.171543in}%
\pgfsys@useobject{currentmarker}{}%
\end{pgfscope}%
\end{pgfscope}%
\begin{pgfscope}%
\pgftext[x=0.525210in,y=1.171543in,right,]{{\rmfamily\fontsize{10.000000}{12.000000}\selectfont \(\displaystyle 2\)}}%
\end{pgfscope}%
\begin{pgfscope}%
\pgfpathrectangle{\pgfqpoint{0.580766in}{0.532919in}}{\pgfqpoint{5.611646in}{3.193122in}} %
\pgfusepath{clip}%
\pgfsetbuttcap%
\pgfsetroundjoin%
\pgfsetlinewidth{0.501875pt}%
\definecolor{currentstroke}{rgb}{0.000000,0.000000,0.000000}%
\pgfsetstrokecolor{currentstroke}%
\pgfsetdash{{1.000000pt}{3.000000pt}}{0.000000pt}%
\pgfpathmoveto{\pgfqpoint{0.580766in}{1.810167in}}%
\pgfpathlineto{\pgfqpoint{6.192411in}{1.810167in}}%
\pgfusepath{stroke}%
\end{pgfscope}%
\begin{pgfscope}%
\pgfsetbuttcap%
\pgfsetroundjoin%
\definecolor{currentfill}{rgb}{0.000000,0.000000,0.000000}%
\pgfsetfillcolor{currentfill}%
\pgfsetlinewidth{0.501875pt}%
\definecolor{currentstroke}{rgb}{0.000000,0.000000,0.000000}%
\pgfsetstrokecolor{currentstroke}%
\pgfsetdash{}{0pt}%
\pgfsys@defobject{currentmarker}{\pgfqpoint{0.000000in}{0.000000in}}{\pgfqpoint{0.055556in}{0.000000in}}{%
\pgfpathmoveto{\pgfqpoint{0.000000in}{0.000000in}}%
\pgfpathlineto{\pgfqpoint{0.055556in}{0.000000in}}%
\pgfusepath{stroke,fill}%
}%
\begin{pgfscope}%
\pgfsys@transformshift{0.580766in}{1.810167in}%
\pgfsys@useobject{currentmarker}{}%
\end{pgfscope}%
\end{pgfscope}%
\begin{pgfscope}%
\pgfsetbuttcap%
\pgfsetroundjoin%
\definecolor{currentfill}{rgb}{0.000000,0.000000,0.000000}%
\pgfsetfillcolor{currentfill}%
\pgfsetlinewidth{0.501875pt}%
\definecolor{currentstroke}{rgb}{0.000000,0.000000,0.000000}%
\pgfsetstrokecolor{currentstroke}%
\pgfsetdash{}{0pt}%
\pgfsys@defobject{currentmarker}{\pgfqpoint{-0.055556in}{0.000000in}}{\pgfqpoint{0.000000in}{0.000000in}}{%
\pgfpathmoveto{\pgfqpoint{0.000000in}{0.000000in}}%
\pgfpathlineto{\pgfqpoint{-0.055556in}{0.000000in}}%
\pgfusepath{stroke,fill}%
}%
\begin{pgfscope}%
\pgfsys@transformshift{6.192411in}{1.810167in}%
\pgfsys@useobject{currentmarker}{}%
\end{pgfscope}%
\end{pgfscope}%
\begin{pgfscope}%
\pgftext[x=0.525210in,y=1.810167in,right,]{{\rmfamily\fontsize{10.000000}{12.000000}\selectfont \(\displaystyle 4\)}}%
\end{pgfscope}%
\begin{pgfscope}%
\pgfpathrectangle{\pgfqpoint{0.580766in}{0.532919in}}{\pgfqpoint{5.611646in}{3.193122in}} %
\pgfusepath{clip}%
\pgfsetbuttcap%
\pgfsetroundjoin%
\pgfsetlinewidth{0.501875pt}%
\definecolor{currentstroke}{rgb}{0.000000,0.000000,0.000000}%
\pgfsetstrokecolor{currentstroke}%
\pgfsetdash{{1.000000pt}{3.000000pt}}{0.000000pt}%
\pgfpathmoveto{\pgfqpoint{0.580766in}{2.448792in}}%
\pgfpathlineto{\pgfqpoint{6.192411in}{2.448792in}}%
\pgfusepath{stroke}%
\end{pgfscope}%
\begin{pgfscope}%
\pgfsetbuttcap%
\pgfsetroundjoin%
\definecolor{currentfill}{rgb}{0.000000,0.000000,0.000000}%
\pgfsetfillcolor{currentfill}%
\pgfsetlinewidth{0.501875pt}%
\definecolor{currentstroke}{rgb}{0.000000,0.000000,0.000000}%
\pgfsetstrokecolor{currentstroke}%
\pgfsetdash{}{0pt}%
\pgfsys@defobject{currentmarker}{\pgfqpoint{0.000000in}{0.000000in}}{\pgfqpoint{0.055556in}{0.000000in}}{%
\pgfpathmoveto{\pgfqpoint{0.000000in}{0.000000in}}%
\pgfpathlineto{\pgfqpoint{0.055556in}{0.000000in}}%
\pgfusepath{stroke,fill}%
}%
\begin{pgfscope}%
\pgfsys@transformshift{0.580766in}{2.448792in}%
\pgfsys@useobject{currentmarker}{}%
\end{pgfscope}%
\end{pgfscope}%
\begin{pgfscope}%
\pgfsetbuttcap%
\pgfsetroundjoin%
\definecolor{currentfill}{rgb}{0.000000,0.000000,0.000000}%
\pgfsetfillcolor{currentfill}%
\pgfsetlinewidth{0.501875pt}%
\definecolor{currentstroke}{rgb}{0.000000,0.000000,0.000000}%
\pgfsetstrokecolor{currentstroke}%
\pgfsetdash{}{0pt}%
\pgfsys@defobject{currentmarker}{\pgfqpoint{-0.055556in}{0.000000in}}{\pgfqpoint{0.000000in}{0.000000in}}{%
\pgfpathmoveto{\pgfqpoint{0.000000in}{0.000000in}}%
\pgfpathlineto{\pgfqpoint{-0.055556in}{0.000000in}}%
\pgfusepath{stroke,fill}%
}%
\begin{pgfscope}%
\pgfsys@transformshift{6.192411in}{2.448792in}%
\pgfsys@useobject{currentmarker}{}%
\end{pgfscope}%
\end{pgfscope}%
\begin{pgfscope}%
\pgftext[x=0.525210in,y=2.448792in,right,]{{\rmfamily\fontsize{10.000000}{12.000000}\selectfont \(\displaystyle 6\)}}%
\end{pgfscope}%
\begin{pgfscope}%
\pgfpathrectangle{\pgfqpoint{0.580766in}{0.532919in}}{\pgfqpoint{5.611646in}{3.193122in}} %
\pgfusepath{clip}%
\pgfsetbuttcap%
\pgfsetroundjoin%
\pgfsetlinewidth{0.501875pt}%
\definecolor{currentstroke}{rgb}{0.000000,0.000000,0.000000}%
\pgfsetstrokecolor{currentstroke}%
\pgfsetdash{{1.000000pt}{3.000000pt}}{0.000000pt}%
\pgfpathmoveto{\pgfqpoint{0.580766in}{3.087416in}}%
\pgfpathlineto{\pgfqpoint{6.192411in}{3.087416in}}%
\pgfusepath{stroke}%
\end{pgfscope}%
\begin{pgfscope}%
\pgfsetbuttcap%
\pgfsetroundjoin%
\definecolor{currentfill}{rgb}{0.000000,0.000000,0.000000}%
\pgfsetfillcolor{currentfill}%
\pgfsetlinewidth{0.501875pt}%
\definecolor{currentstroke}{rgb}{0.000000,0.000000,0.000000}%
\pgfsetstrokecolor{currentstroke}%
\pgfsetdash{}{0pt}%
\pgfsys@defobject{currentmarker}{\pgfqpoint{0.000000in}{0.000000in}}{\pgfqpoint{0.055556in}{0.000000in}}{%
\pgfpathmoveto{\pgfqpoint{0.000000in}{0.000000in}}%
\pgfpathlineto{\pgfqpoint{0.055556in}{0.000000in}}%
\pgfusepath{stroke,fill}%
}%
\begin{pgfscope}%
\pgfsys@transformshift{0.580766in}{3.087416in}%
\pgfsys@useobject{currentmarker}{}%
\end{pgfscope}%
\end{pgfscope}%
\begin{pgfscope}%
\pgfsetbuttcap%
\pgfsetroundjoin%
\definecolor{currentfill}{rgb}{0.000000,0.000000,0.000000}%
\pgfsetfillcolor{currentfill}%
\pgfsetlinewidth{0.501875pt}%
\definecolor{currentstroke}{rgb}{0.000000,0.000000,0.000000}%
\pgfsetstrokecolor{currentstroke}%
\pgfsetdash{}{0pt}%
\pgfsys@defobject{currentmarker}{\pgfqpoint{-0.055556in}{0.000000in}}{\pgfqpoint{0.000000in}{0.000000in}}{%
\pgfpathmoveto{\pgfqpoint{0.000000in}{0.000000in}}%
\pgfpathlineto{\pgfqpoint{-0.055556in}{0.000000in}}%
\pgfusepath{stroke,fill}%
}%
\begin{pgfscope}%
\pgfsys@transformshift{6.192411in}{3.087416in}%
\pgfsys@useobject{currentmarker}{}%
\end{pgfscope}%
\end{pgfscope}%
\begin{pgfscope}%
\pgftext[x=0.525210in,y=3.087416in,right,]{{\rmfamily\fontsize{10.000000}{12.000000}\selectfont \(\displaystyle 8\)}}%
\end{pgfscope}%
\begin{pgfscope}%
\pgfpathrectangle{\pgfqpoint{0.580766in}{0.532919in}}{\pgfqpoint{5.611646in}{3.193122in}} %
\pgfusepath{clip}%
\pgfsetbuttcap%
\pgfsetroundjoin%
\pgfsetlinewidth{0.501875pt}%
\definecolor{currentstroke}{rgb}{0.000000,0.000000,0.000000}%
\pgfsetstrokecolor{currentstroke}%
\pgfsetdash{{1.000000pt}{3.000000pt}}{0.000000pt}%
\pgfpathmoveto{\pgfqpoint{0.580766in}{3.726041in}}%
\pgfpathlineto{\pgfqpoint{6.192411in}{3.726041in}}%
\pgfusepath{stroke}%
\end{pgfscope}%
\begin{pgfscope}%
\pgfsetbuttcap%
\pgfsetroundjoin%
\definecolor{currentfill}{rgb}{0.000000,0.000000,0.000000}%
\pgfsetfillcolor{currentfill}%
\pgfsetlinewidth{0.501875pt}%
\definecolor{currentstroke}{rgb}{0.000000,0.000000,0.000000}%
\pgfsetstrokecolor{currentstroke}%
\pgfsetdash{}{0pt}%
\pgfsys@defobject{currentmarker}{\pgfqpoint{0.000000in}{0.000000in}}{\pgfqpoint{0.055556in}{0.000000in}}{%
\pgfpathmoveto{\pgfqpoint{0.000000in}{0.000000in}}%
\pgfpathlineto{\pgfqpoint{0.055556in}{0.000000in}}%
\pgfusepath{stroke,fill}%
}%
\begin{pgfscope}%
\pgfsys@transformshift{0.580766in}{3.726041in}%
\pgfsys@useobject{currentmarker}{}%
\end{pgfscope}%
\end{pgfscope}%
\begin{pgfscope}%
\pgfsetbuttcap%
\pgfsetroundjoin%
\definecolor{currentfill}{rgb}{0.000000,0.000000,0.000000}%
\pgfsetfillcolor{currentfill}%
\pgfsetlinewidth{0.501875pt}%
\definecolor{currentstroke}{rgb}{0.000000,0.000000,0.000000}%
\pgfsetstrokecolor{currentstroke}%
\pgfsetdash{}{0pt}%
\pgfsys@defobject{currentmarker}{\pgfqpoint{-0.055556in}{0.000000in}}{\pgfqpoint{0.000000in}{0.000000in}}{%
\pgfpathmoveto{\pgfqpoint{0.000000in}{0.000000in}}%
\pgfpathlineto{\pgfqpoint{-0.055556in}{0.000000in}}%
\pgfusepath{stroke,fill}%
}%
\begin{pgfscope}%
\pgfsys@transformshift{6.192411in}{3.726041in}%
\pgfsys@useobject{currentmarker}{}%
\end{pgfscope}%
\end{pgfscope}%
\begin{pgfscope}%
\pgftext[x=0.525210in,y=3.726041in,right,]{{\rmfamily\fontsize{10.000000}{12.000000}\selectfont \(\displaystyle 10\)}}%
\end{pgfscope}%
\begin{pgfscope}%
\pgftext[x=0.316877in,y=2.129480in,,bottom,rotate=90.000000]{{\rmfamily\fontsize{10.000000}{12.000000}\selectfont voltage (V)}}%
\end{pgfscope}%
\begin{pgfscope}%
\pgfsetbuttcap%
\pgfsetroundjoin%
\pgfsetlinewidth{1.003750pt}%
\definecolor{currentstroke}{rgb}{0.000000,0.000000,0.000000}%
\pgfsetstrokecolor{currentstroke}%
\pgfsetdash{}{0pt}%
\pgfpathmoveto{\pgfqpoint{0.580766in}{3.726041in}}%
\pgfpathlineto{\pgfqpoint{6.192411in}{3.726041in}}%
\pgfusepath{stroke}%
\end{pgfscope}%
\begin{pgfscope}%
\pgfsetbuttcap%
\pgfsetroundjoin%
\pgfsetlinewidth{1.003750pt}%
\definecolor{currentstroke}{rgb}{0.000000,0.000000,0.000000}%
\pgfsetstrokecolor{currentstroke}%
\pgfsetdash{}{0pt}%
\pgfpathmoveto{\pgfqpoint{6.192411in}{0.532919in}}%
\pgfpathlineto{\pgfqpoint{6.192411in}{3.726041in}}%
\pgfusepath{stroke}%
\end{pgfscope}%
\begin{pgfscope}%
\pgfsetbuttcap%
\pgfsetroundjoin%
\pgfsetlinewidth{1.003750pt}%
\definecolor{currentstroke}{rgb}{0.000000,0.000000,0.000000}%
\pgfsetstrokecolor{currentstroke}%
\pgfsetdash{}{0pt}%
\pgfpathmoveto{\pgfqpoint{0.580766in}{0.532919in}}%
\pgfpathlineto{\pgfqpoint{6.192411in}{0.532919in}}%
\pgfusepath{stroke}%
\end{pgfscope}%
\begin{pgfscope}%
\pgfsetbuttcap%
\pgfsetroundjoin%
\pgfsetlinewidth{1.003750pt}%
\definecolor{currentstroke}{rgb}{0.000000,0.000000,0.000000}%
\pgfsetstrokecolor{currentstroke}%
\pgfsetdash{}{0pt}%
\pgfpathmoveto{\pgfqpoint{0.580766in}{0.532919in}}%
\pgfpathlineto{\pgfqpoint{0.580766in}{3.726041in}}%
\pgfusepath{stroke}%
\end{pgfscope}%
\begin{pgfscope}%
\pgftext[x=3.386589in,y=3.795485in,,base]{{\rmfamily\fontsize{12.000000}{14.400000}\selectfont Step-By-Step Approximations of \(\displaystyle v_L(t)\)}}%
\end{pgfscope}%
\begin{pgfscope}%
\pgfsetbuttcap%
\pgfsetroundjoin%
\definecolor{currentfill}{rgb}{0.300000,0.300000,0.300000}%
\pgfsetfillcolor{currentfill}%
\pgfsetfillopacity{0.500000}%
\pgfsetlinewidth{1.003750pt}%
\definecolor{currentstroke}{rgb}{0.300000,0.300000,0.300000}%
\pgfsetstrokecolor{currentstroke}%
\pgfsetstrokeopacity{0.500000}%
\pgfsetdash{}{0pt}%
\pgfpathmoveto{\pgfqpoint{4.378033in}{2.867709in}}%
\pgfpathlineto{\pgfqpoint{6.103523in}{2.867709in}}%
\pgfpathquadraticcurveto{\pgfqpoint{6.136856in}{2.867709in}}{\pgfqpoint{6.136856in}{2.901042in}}%
\pgfpathlineto{\pgfqpoint{6.136856in}{3.581596in}}%
\pgfpathquadraticcurveto{\pgfqpoint{6.136856in}{3.614930in}}{\pgfqpoint{6.103523in}{3.614930in}}%
\pgfpathlineto{\pgfqpoint{4.378033in}{3.614930in}}%
\pgfpathquadraticcurveto{\pgfqpoint{4.344700in}{3.614930in}}{\pgfqpoint{4.344700in}{3.581596in}}%
\pgfpathlineto{\pgfqpoint{4.344700in}{2.901042in}}%
\pgfpathquadraticcurveto{\pgfqpoint{4.344700in}{2.867709in}}{\pgfqpoint{4.378033in}{2.867709in}}%
\pgfpathclose%
\pgfusepath{stroke,fill}%
\end{pgfscope}%
\begin{pgfscope}%
\pgfsetbuttcap%
\pgfsetroundjoin%
\definecolor{currentfill}{rgb}{1.000000,1.000000,1.000000}%
\pgfsetfillcolor{currentfill}%
\pgfsetlinewidth{1.003750pt}%
\definecolor{currentstroke}{rgb}{0.000000,0.000000,0.000000}%
\pgfsetstrokecolor{currentstroke}%
\pgfsetdash{}{0pt}%
\pgfpathmoveto{\pgfqpoint{4.350255in}{2.895486in}}%
\pgfpathlineto{\pgfqpoint{6.075745in}{2.895486in}}%
\pgfpathquadraticcurveto{\pgfqpoint{6.109078in}{2.895486in}}{\pgfqpoint{6.109078in}{2.928820in}}%
\pgfpathlineto{\pgfqpoint{6.109078in}{3.609374in}}%
\pgfpathquadraticcurveto{\pgfqpoint{6.109078in}{3.642708in}}{\pgfqpoint{6.075745in}{3.642708in}}%
\pgfpathlineto{\pgfqpoint{4.350255in}{3.642708in}}%
\pgfpathquadraticcurveto{\pgfqpoint{4.316922in}{3.642708in}}{\pgfqpoint{4.316922in}{3.609374in}}%
\pgfpathlineto{\pgfqpoint{4.316922in}{2.928820in}}%
\pgfpathquadraticcurveto{\pgfqpoint{4.316922in}{2.895486in}}{\pgfqpoint{4.350255in}{2.895486in}}%
\pgfpathclose%
\pgfusepath{stroke,fill}%
\end{pgfscope}%
\begin{pgfscope}%
\pgfsetbuttcap%
\pgfsetroundjoin%
\pgfsetlinewidth{1.003750pt}%
\definecolor{currentstroke}{rgb}{0.000000,0.000000,1.000000}%
\pgfsetstrokecolor{currentstroke}%
\pgfsetdash{{1.000000pt}{3.000000pt}}{0.000000pt}%
\pgfpathmoveto{\pgfqpoint{4.433589in}{3.517708in}}%
\pgfpathlineto{\pgfqpoint{4.666922in}{3.517708in}}%
\pgfusepath{stroke}%
\end{pgfscope}%
\begin{pgfscope}%
\pgftext[x=4.850255in,y=3.459374in,left,base]{{\rmfamily\fontsize{12.000000}{14.400000}\selectfont Continuous}}%
\end{pgfscope}%
\begin{pgfscope}%
\pgfsetrectcap%
\pgfsetroundjoin%
\pgfsetlinewidth{1.003750pt}%
\definecolor{currentstroke}{rgb}{0.000000,0.500000,0.000000}%
\pgfsetstrokecolor{currentstroke}%
\pgfsetdash{}{0pt}%
\pgfpathmoveto{\pgfqpoint{4.433589in}{3.285300in}}%
\pgfpathlineto{\pgfqpoint{4.666922in}{3.285300in}}%
\pgfusepath{stroke}%
\end{pgfscope}%
\begin{pgfscope}%
\pgftext[x=4.850255in,y=3.226967in,left,base]{{\rmfamily\fontsize{12.000000}{14.400000}\selectfont Trapezoidal}}%
\end{pgfscope}%
\begin{pgfscope}%
\pgfsetrectcap%
\pgfsetroundjoin%
\pgfsetlinewidth{1.003750pt}%
\definecolor{currentstroke}{rgb}{1.000000,0.000000,0.000000}%
\pgfsetstrokecolor{currentstroke}%
\pgfsetdash{}{0pt}%
\pgfpathmoveto{\pgfqpoint{4.433589in}{3.052893in}}%
\pgfpathlineto{\pgfqpoint{4.666922in}{3.052893in}}%
\pgfusepath{stroke}%
\end{pgfscope}%
\begin{pgfscope}%
\pgftext[x=4.850255in,y=2.994560in,left,base]{{\rmfamily\fontsize{12.000000}{14.400000}\selectfont Backward Euler    }}%
\end{pgfscope}%
\end{pgfpicture}%
\makeatother%
\endgroup%

    \end{center}
    \caption{Comparison of Approximations and PSCAD at $\Delta{}t_2 = 0.8 ms$}
    \label{approx_pscad_comp_0p0008}
\end{figure}

\section{Conclusion}
PSCAD offers very similar performance to the manually derived approximations, but with a much easier interface. Much more complex circuits can be simulated in a fraction of the time it would take to manually derive the differential equations. The following observations were made about the results in this assignment.
\begin{itemize}
    \item At a time step of $\Delta{}t_1 = 0.1 ms$, all three solutions are almost indistinguishable from ideal at the plotted time scale, showing how a small time step improves the performance of all three techniques considerably.
    \item The solutions for PSCAD are very close in shape and magnitude to the trapezoidal results that were generated in the previous assignment. This likely means that the underlying mechanism for simulation is likely similar to the trapezoidal approximation.
    \item Conversely, PSCAD does not have the same shape as the backward Euler solution, so it's not likely this technique.
    \item Interestingly enough, PSCAD appears to offer a more precise solution than the manually derived trapezoidal one for the same time step. Perhaps the time step used in PSCAD is different internally (e.g. slightly smaller than what is specified). Or, perhaps PSCAD makes different assumptions about the beginning of time of the circuit. In the PSCAD circuit, at $0^-s$ the voltage is 10 V and undefined in the manually derived equations (times before 0 are not taken into account). Perhaps this value is used somewhere in the PSCAD calculation that wasn't evident.

\end{itemize}

\newpage
\section{Code Listings}

\subsection{Python Code Listing}
\label{code-listing-python}
The following is the code written in Python to perform the calculations derived for this homework assignment as well as generate the plots used in this report. The argument parser was removed in this version due to the same parameters being used for every iteration. Also, the results from PSCAD were copy pasted directly into the Python source to keep things simple. The results are transposed to match the generated results from the manually derivated solutions.
\lstinputlisting[language=Python]{assignment1b.py}

\subsection{Fortran Code Listing}
\label{code-listing-fortran}
The following is the code generated by PSCAD to simulate the circuit for this assignment.
\lstinputlisting[language=Fortran]{pscad.f}
\end{document}

